
\author{John Stuart Mill}
\authdate{1806--1873}
\textdate{1861}
%\chapter[John Stuart Mill -- What Utilitarianism Is]{What
%Utilitarianism Is}
\chapter[John Stuart Mill -- Utilitarianism, chap.
2]{Utilitarianism\\\smaller Chapter 2}

\nfootnote{\fullcite{mill1907.2}}

% Originally published in Fraser's Magazine in 1861.

\page{8}A passing remark is all that needs be given to the ignorant
blunder of supposing that those who stand up for utility as the test
of right and wrong, use the term in that restricted and merely
colloquial sense in which utility is opposed to pleasure. An apology
is due to the philosophical opponents of utilitarianism, for even the
momentary appearance of confounding them with any one capable of so
absurd a misconception; which is the more extraordinary, inasmuch as
the contrary accusation, of referring everything to pleasure, and that
too in its grossest form, is another of the common charges against
utilitarianism: and, as has been pointedly remarked by an able
writer, the same sort of persons, and often the very same persons,
denounce the theory ``as impracticably dry when the word utility
precedes the word pleasure, and as too practicably voluptuous when the
word pleasure precedes the word utility.'' Those who know anything
about the matter are aware that every writer, from Epicurus to
Bentham, who maintained the theory of utility, meant by it, not
something to be contradistinguished from pleasure, but pleasure
itself, together with exemption from pain; and instead of opposing the
useful to the agreeable or the ornamental, have always declared that
the useful \page{9} means these, among other things. Yet the common
herd, including the herd of writers, not only in newspapers and
periodicals, but in books of weight and pretension, are perpetually
falling into this shallow mistake. Having caught up the word
utilitarian, while knowing nothing whatever about it but its sound,
they habitually express by it the rejection, or the neglect, of
pleasure in some of its forms; of beauty, of ornament, or of
amusement. Nor is the term thus ignorantly misapplied solely in
disparagement, but occasionally in compliment; as though it implied
superiority to frivolity and the mere pleasures of the moment. And
this perverted use is the only one in which the word is popularly
known, and the one from which the new generation are acquiring their
sole notion of its meaning. Those who introduced the word, but who had
for many years discontinued it as a distinctive appellation, may well
feel themselves called upon to resume it, if by doing so they can hope
to contribute anything towards rescuing it from this utter
degradation.\footnote{The author of this essay has reason for
believing himself to be the first person who brought the word
utilitarian into use. He did not invent it, but adopted it from a
passing expression in Mr. Galt's \textit{Annals of the Parish}. After
using it as a designation for several years, he and others abandoned
it from a growing dislike to anything resembling a badge or watchword
of sectarian distinction. But as a name for one single opinion, not a
set of o\-pin\-ions---to denote the recognition of utility as a
standard, not any particular way of applying it---the term supplies a
want in the language, and offers, in many cases, a convenient mode of
avoiding tiresome circumlocution.}

The creed which accepts as the foundation of morals, Utility, or the
Greatest Happiness Principle, holds that actions are right in
proportion as they tend to promote happiness, wrong as they tend to
produce the \page{10} reverse of happiness. By happiness is intended
pleasure, and the absence of pain; by unhappiness, pain, and the
privation of pleasure. To give a clear view of the moral standard set
up by the theory, much more requires to be said; in particular, what
things it includes in the ideas of pain and pleasure; and to what
extent this is left an open question. But these supplementary
explanations do not affect the theory of life on which this theory of
morality is ground\-ed---name\-ly, that pleasure, and freedom from
pain, are the only things desirable as ends; and that all desirable
things (which are as numerous in the utilitarian as in any other
scheme) are desirable either for the pleasure inherent in themselves,
or as means to the promotion of pleasure and the prevention of pain.

Now, such a theory of life excites in many minds, and among them in
some of the most estimable in feeling and purpose, inveterate dislike.
To suppose that life has (as they express it) no higher end than
pleas\-ure---no better and nobler object of desire and
pur\-suit---they designate as utterly mean and grovelling; as a
doctrine worthy only of swine, to whom the followers of Epicurus were,
at a very early period, contemptuously likened; and modern holders of
the doctrine are occasionally made the subject of equally polite
comparisons by its German, French, and English assailants.

When thus attacked, the Epicureans have always answered, that it is
not they, but their accusers, who represent human nature in a
degrading light; since the accusation supposes human beings to be
capable of no pleasures except those of which swine are capable. If
this supposition were true, the charge could not be \page{11}
gainsaid, but would then be no longer an imputation: for if the
sources of pleasure were precisely the same to human beings and to
swine, the rule of life which is good enough for the one would be good
enough for the other. The comparison of the Epicurean life to that
of beasts is felt as degrading, precisely because a beast's pleasures
do not satisfy a human being's conceptions of happiness. Human beings
have faculties more elevated than the animal appetites, and when once
made conscious of them, do not regard anything as happiness which
does not include their gratification. I do not, indeed, consider the
Epicureans to have been by any means faultless in drawing out their
scheme of consequences from the utilitarian principle. To do this in
any sufficient manner, many Stoic, as well as Christian elements
require to be included. But there is no known Epicurean theory of life
which does not assign to the pleasures of the intellect, of the
feelings and imagination, and of the moral sentiments, a much higher
value as pleasures than to those of mere sensation. It must be
admitted, however, that utilitarian writers in general have placed the
superiority of mental over bodily pleasures chiefly in the greater
permanency, safety, uncostliness, \&c., of the for\-mer---that is, in
their circumstantial advantages rather than in their intrinsic nature.
And on all these points utilitarians have fully proved their case;
but they might have taken the other, and, as it may be called, higher
ground, with entire consistency. It is quite compatible with the
principle of utility to recognise the fact, that some \textit{kinds}
of pleasure are more desirable and more valuable than others. It would
be absurd that while, in estimating all other things, quality is
\page{12} considered as well as quantity, the estimation of pleasures
should be supposed to depend on quantity alone.

If I am asked, what I mean by difference of quality in pleasures, or
what makes one pleasure more valuable than another, merely as a
pleasure, except its being greater in amount, there is but one
possible answer. Of two pleasures, if there be one to which all or
almost all who have experience of both give a decided preference,
irrespective of any feeling of moral obligation to prefer it, that is
the more desirable pleasure. If one of the two is, by those who are
competently acquainted with both, placed so far above the other that
they prefer it, even though knowing it to be attended with a greater
amount of discontent, and would not resign it for any quantity of the
other pleasure which their nature is capable of, we are justified in
ascribing to the preferred enjoyment a superiority in quality, so far
outweighing quantity as to render it, in comparison, of small account.

Now it is an unquestionable fact that those who are equally acquainted
with, and equally capable of appreciating and enjoying both, do give a
most marked preference to the manner of existence which employs their
higher faculties. Few human creatures would consent to be changed into
any of the lower animals, for a promise of the fullest allowance of a
beast's pleasures; no intelligent human being would consent to be a
fool, no instructed person would be an ignoramus, no person of feeling
and conscience would be selfish and base, even though they should be
persuaded that the fool, the dunce, or the rascal is better satisfied
with his lot than they are with theirs. They would not resign what
they possess more than he, for \page{13} the most complete
satisfaction of all the desires which they have in common with him. If
they ever fancy they would, it is only in cases of unhappiness so
extreme, that to escape from it they would exchange their lot for
almost any other, however undesirable in their own eyes. A being of
higher faculties requires more to make him happy, is capable probably
of more acute suffering, and is certainly accessible to it at more
points, than one of an inferior type; but in spite of these
liabilities, he can never really wish to sink into what he feels to be
a lower grade of existence. We may give what explanation we please of
this unwillingness; we may attribute it to pride, a name which is
given indiscriminately to some of the most and to some of the least
estimable feelings of which mankind are capable; we may refer it to
the love of liberty and personal independence, an appeal to which was
with the Stoics one of the most effective means for the inculcation of
it; to the love of power, or to the love of excitement, both of which
do really enter into and contribute to it: but its most appropriate
appellation is a sense of dignity, which all human beings possess in
one form or other, and in some, though by no means in exact,
proportion to their higher faculties, and which is so essential a part
of the happiness of those in whom it is strong, that nothing which
conflicts with it could be, otherwise than momentarily, an object of
desire to them. Whoever supposes that this preference takes place at a
sacrifice of hap\-pi\-ness---that the superior being, in anything like
equal circumstances, is not happier than the
in\-fe\-ri\-or---con\-founds the two very different ideas, of
happiness, and content. It is indisputable that the being whose
capacities of \page{14} enjoyment are low, has the greatest chance of
having them fully satisfied; and a highly-endowed being will always
feel that any happiness which he can look for, as the world is
constituted, is imperfect. But he can learn to bear its imperfections,
if they are at all bearable; and they will not make him envy the being
who is indeed unconscious of the imperfections, but only because he
feels not at all the good which those imperfections qualify. It is
better to be a human being dissatisfied than a pig satisfied; better
to be Socrates dissatisfied than a fool satisfied. And if the fool, or
the pig, is of a different opinion, it is because they only know their
own side of the question. The other party to the comparison knows both
sides.

It may be objected, that many who are capable of the higher pleasures,
occasionally, under the influence of temptation, postpone them to the
lower. But this is quite compatible with a full appreciation of the
intrinsic superiority of the higher. Men often, from infirmity of
character, make their election for the nearer good, though they know
it to be the less valuable; and this no less when the choice is
between two bodily pleasures, than when it is between bodily and
mental. They pursue sensual indulgences to the injury of health,
though perfectly aware that health is the greater good. It may be
further objected, that many who begin with youthful enthusiasm for
everything noble, as they advance in years sink into indolence and
selfishness. But I do not believe that those who undergo this very
common change, voluntarily choose the lower description of pleasures
in preference to the higher. I believe that before they devote
themselves exclusively to the one, they have already \page{15} become
incapable of the other. Capacity for the nobler feelings is in most
natures a very tender plant, easily killed, not only by hostile
influences, but by mere want of sustenance; and in the majority of
young persons it speedily dies away if the occupations to which their
position in life has devoted them, and the society into which it has
thrown them, are not favourable to keeping that higher capacity in
exercise. Men lose their high aspirations as they lose their
intellectual tastes, because they have not time or opportunity for
indulging them; and they addict themselves to inferior pleasures,
not because they deliberately prefer them, but because they are either
the only ones to which they have access, or the only ones which they
are any longer capable of enjoying. It may be questioned whether any
one who has remained equally susceptible to both classes of pleasures,
ever knowingly and calmly preferred the lower; though many, in all
ages, have broken down in an ineffectual attempt to combine both.

From this verdict of the only competent judges, I apprehend there can
be no appeal. On a question which is the best worth having of two
pleasures, or which of two modes of existence is the most grateful to
the feelings, apart from its moral attributes and from its
consequences, the judgment of those who are qualified by knowledge of
both, or, if they differ, that of the majority among them, must be
admitted as final. And there needs be the less hesitation to accept
this judgment respecting the quality of pleasures, since there is no
other tribunal to be referred to even on the question of quantity.
What means are there of determining which is the acutest of two
\page{16} pains, or the intensest of two pleasurable sensations,
except the general suffrage of those who are familiar with both?
Neither pains nor pleasure are homogeneous, and pain is always
heterogeneous with pleasure. What is there to decide whether a
particular pleasure is worth purchasing at the cost of a particular
pain, except the feelings and judgment of the experienced? When,
therefore, those feelings and judgment declare the pleasures derived
from the higher faculties to be preferable \textit{in kind}, apart
from the question of intensity, to those of which the animal nature,
disjoined from the higher faculties, is susceptible, they are entitled
on this subject to the same regard.

I have dwelt on this point, as being a necessary part of a perfectly
just conception of Utility or Happiness, considered as the directive
rule of human conduct. But it is by no means an indispensable
condition to the acceptance of the utilitarian standard; for that
standard is not the agent's own greatest happiness, but the greatest
amount of happiness altogether; and if it may possibly be doubted
whether a noble character is always the happier for its nobleness,
there can be no doubt that it makes other people happier, and that the
world in general is immensely a gainer by it. Utilitarianism,
therefore, could only attain its end by the general cultivation of
nobleness of character, even if each individual were only benefited by
the nobleness of others, and his own, so far as happiness is
concerned, were a sheer deduction from the benefit. But the bare
enunciation of such an absurdity as this last, renders refutation
superfluous.

\page{17}According to the Greatest Happiness Principle, as above
explained, the ultimate end, with reference to and for the sake of
which all other things are desirable (wheth\-er we are considering our
own good or that of other people), is an existence exempt as far as
possible from pain, and as rich as possible in enjoyments, both in
point of quantity and quality; the test of quality, and the rule for
measuring it against quantity, being the preference felt by those who,
in their opportunities of experience, to which must be added their
habits of self-consciousness and self-observation, are best furnished
with the means of comparison. This, being, according to the
utilitarian opinion, the end of human action, is necessarily also the
standard of morality; which may accordingly be defined, the rules and
precepts for human conduct, by the observance of which an existence
such as has been described might be, to the greatest extent possible,
secured to all mankind; and not to them only, but so far as the nature
of things admits to the whole sentient creation.

Against this doctrine, however, rises another class of objectors, who
say that happiness, in any form, cannot be the rational purpose of
human life and action; because, in the first place, it is
unattainable: and they contemptuously ask, What right hast thou to be
happy? a question which Mr. Carlyle clenches by the addition, What
right, a short time ago, hadst thou even \textit{to be}? Next, they
say, that men can do \textit{without} happiness; that all noble human
beings have felt this, and could not have become noble but by learning
the lesson of Entsagen, or renunciation; which lesson, thoroughly
learnt and \page{18} submitted to, they affirm to be the beginning and
necessary condition of all virtue.

The first of these objections would go to the root of the matter were
it well founded; for if no happiness is to be had at all by human
beings, the attainment of it cannot be the end of morality, or of any
rational conduct. Though, even in that case, something might still be
said for the utilitarian theory; since utility includes not solely the
pursuit of happiness, but the prevention or mitigation of unhappiness;
and if the former aim be chimerical, there will be all the greater
scope and more imperative need for the latter, so long at least as
mankind think fit to live, and do not take refuge in the simultaneous
act of suicide recommended under certain conditions by Novalis. When,
however, it is thus positively asserted to be impossible that human
life should be happy, the assertion, if not something like a verbal
quibble, is at least an exaggeration. If by happiness be meant a
continuity of highly pleasurable excitement, it is evident enough that
this is impossible. A state of exalted pleasure lasts only moments, or
in some cases, and with some intermissions, hours or days, and is the
occasional brilliant flash of enjoyment, not its permanent and steady
flame. Of this the philosophers who have taught that happiness is the
end of life were as fully aware as those who taunt them. The happiness
which they meant was not a life of rapture; but moments of such, in an
existence made up of few and transitory pains, many and various
pleasures, with a decided predominance of the active over the passive,
and having as the foundation of the whole, not to expect more from
life than it is capable of \page{19} bestowing. A life thus composed,
to those who have been fortunate enough to obtain it, has always
appeared worthy of the name of happiness. And such an existence is
even now the lot of many, during some considerable portion of their
lives. The present wretched education, and wretched social
arrangements, are the only real hindrance to its being attainable by
almost all.

The objectors perhaps may doubt whether human beings, if taught to
consider happiness as the end of life, would be satisfied with such a
moderate share of it. But great numbers of mankind have been satisfied
with much less. The main constituents of a satisfied life appear to be
two, either of which by itself is often found sufficient for the
purpose: tranquillity, and excitement. With much tranquillity, many
find that they can be content with very little pleasure: with much
excitement, many can reconcile themselves to a considerable quantity
of pain. There is assuredly no inherent impossibility in enabling even
the mass of mankind to unite both; since the two are so far from being
incompatible that they are in natural alliance, the prolongation of
either being a preparation for, and exciting a wish for, the other. It
is only those in whom indolence amounts to a vice, that do not desire
excitement after an interval of repose; it is only those in whom the
need of excitement is a disease, that feel the tranquillity which
follows excitement dull and insipid, instead of pleasurable in direct
proportion to the excitement which preceded it. When people who are
tolerably fortunate in their outward lot do not find in life
sufficient enjoyment to make it valuable to them, the cause generally
is, caring for \page{} nobody but themselves. To those who have
neither public nor private affections, the excitements of life are
much curtailed, and in any case dwindle in value as the time
approaches when all selfish interests must be terminated by death:
while those who leave after them objects of personal affection, and
especially those who have also cultivated a fellow-feeling with the
collective interests of mankind, retain as lively an interest in life
on the eve of death as in the vigour of youth and health. Next to
selfishness, the principal cause which makes life unsatisfactory, is
want of mental cultivation. A cultivated mind---I do not mean that of
a philosopher, but any mind to which the fountains of knowledge have
been opened, and which has been taught, in any tolerable degree, to
exercise its fac\-ul\-ties---finds sources of inexhaustible interest
in all that surrounds it; in the objects of nature, the achievements
of art, the imaginations of poetry, the incidents of history, the ways
of mankind past and present, and their prospects in the future. It is
possible, indeed, to become indifferent to all this, and that too
without having exhausted a thousandth part of it; but only when one
has had from the beginning no moral or human interest in these things,
and has sought in them only the gratification of curiosity.

Now there is absolutely no reason in the nature of things why an
amount of mental culture sufficient to give an intelligent interest in
these objects of contemplation, should not be the inheritance of every
one born in a civilised country. As little is there an inherent
necessity that any human being should be a selfish egotist, devoid of
every feeling or care but those \page{21} which centre in his own
miserable individuality. Something far superior to this is
sufficiently common even now, to give ample earnest of what the human
species may be made. Genuine private affections, and a sincere
interest in the public good, are possible, though in unequal degrees,
to every rightly brought up human being. In a world in which there is
so much to interest, so much to enjoy, and so much also to correct
and improve, every one who has this moderate amount of moral and
intellectual requisites is capable of an existence which may be called
enviable; and unless such a person, through bad laws, or subjection to
the will of others, is denied the liberty to use the sources of
happiness within his reach, he will not fail to find this enviable
existence, if he escape the positive evils of life, the great sources
of physical and mental suf\-fer\-ing---such as indigence, disease, and
the unkindness, worthlessness, or premature loss of objects of
affection. The main stress of the problem lies, therefore, in the
contest with these calamities, from which it is a rare good fortune
entirely to escape; which, as things now are, cannot be obviated, and
often cannot be in any material degree mitigated. Yet no one whose
opinion deserves a moment's consideration can doubt that most of the
great positive evils of the world are in themselves removable, and
will, if human affairs continue to improve, be in the end reduced
within narrow limits. Poverty, in any sense implying suffering, may be
completely extinguished by the wisdom of society, combined with the
good sense and providence of individuals. Even that most intractable
of enemies, disease, may be indefinitely reduced in dimensions by good
physical and moral education, \page{22} and proper control of noxious
influences; while the progress of science holds out a promise for the
future of still more direct conquests over this detestable foe. And
every advance in that direction relieves us from some, not only of
the chances which cut short our own lives, but, what concerns us still
more, which deprives us of those in whom our happiness is wrapt up. As
for vicissitudes of fortune, and other disappointments connected with
worldly circumstances, these are principally the effect either of
gross imprudence, of ill-regulated desires, or of bad or imperfect
social institutions. All the grand sources, in short, of human
suffering are in a great degree, many of them almost entirely,
conquerable by human care and effort; and though their removal is
grievously slow---though a long succession of generations will perish
in the breach before the conquest is completed, and this world becomes
all that, if will and knowledge were not wanting, it might easily be
made---yet every mind sufficiently intelligent and generous to bear a
part, however small and unconspicuous, in the endeavour, will draw a
noble enjoyment from the contest itself, which he would not for any
bribe in the form of selfish indulgence consent to be without.

And this leads to the true estimation of what is said by the objectors
concerning the possibility, and the obligation, of learning to do
without happiness. Unquestionably it is possible to do without
happiness; it is done involuntarily by nineteen-twentieths of mankind,
even in those parts of our present world which are least deep in
barbarism; and it often has to be done voluntarily by the hero or the
martyr, for the sake of something which he prizes more than his
\page{23} individual happiness. But this something, what is it, unless
the happiness of others, or some of the requisites of happiness? It is
noble to be capable of resigning entirely one's own portion of
happiness or chances of it: but, after all, this self-sacrifice must
be for some end; it is not its own end; and if we are told that its
end is not happiness, but virtue, which is better than happiness, I
ask, would the sacrifice be made if the hero or martyr did not believe
that it would earn for others immunity from similar sacrifices? Would
it be made, if he thought that his renunciation of happiness for
himself would produce no fruit for any of his fellow creatures, but to
make their lot like his, and place them also in the condition of
persons who have renounced happiness? All honour to those who can
abnegate for themselves the personal enjoyment of life, when by such
renunciation they contribute worthily to increase the amount of
happiness in the world; but he who does it, or professes to do it, for
any other purpose, is no more deserving of admiration than the ascetic
mounted on his pillar. He may be an inspiriting proof of what men
\textit{can} do, but assuredly not an example of what they
\textit{should}.

Though it is only in a very imperfect state of the world's
arrangements that any one can best serve the happiness of others by
the absolute sacrifice of his own, yet so long as the world is in that
imperfect state, I fully acknowledge that the readiness to make such a
sacrifice is the highest virtue which can be found in man. I will add,
that in this condition of the world, paradoxical as the assertion may
be, the conscious ability to do without happiness gives the best
prospect of realising such happiness as is attain-\page{24}able. For
nothing except that consciousness can raise a person above the chances
of life, by making him feel that, let fate and fortune do their worst,
they have not power to subdue him: which, once felt, frees him from
excess of anxiety concerning the evils of life, and enables him, like
many a Stoic in the worst times of the Roman Empire, to cultivate in
tranquillity the sources of satisfaction accessible to him, without
concerning himself about the uncertainty of their duration, any more
than about their inevitable end.

Meanwhile, let utilitarians never cease to claim the morality of
self-de\-vo\-tion as a possession which belongs by as good a right to
them, as either to the Stoic or to the Transcendentalist. The
utilitarian morality does recognise in human beings the power of
sacrificing their own greatest good for the good of others. It only
refuses to admit that the sacrifice is itself a good. A sacrifice
which does not increase, or tend to increase, the sum total of
happiness, it considers as wasted. The only self-renunciation which it
applauds, is devotion to the happiness, or to some of the means of
happiness, of others; either of mankind collectively, or of
individuals within the limits imposed by the collective interests of
mankind.

I must again repeat, what the assailants of utilitarianism seldom have
the justice to acknowledge, that the happiness which forms the
utilitarian standard of what is right in conduct, is not the agent's
own happiness, but that of all concerned. As between his own happiness
and that of others, utilitarianism requires him to be as strictly
impartial as a disinterested and benevolent spectator. In the golden
rule of Jesus of Nazareth, we read the complete spirit of the ethics
\page{25} of utility. To do as one would be done by, and to love
fone'sf neighbour as oneself, constitute the ideal perfection of
utilitarian morality. As the means of making the nearest approach to
this ideal, utility would enjoin, first, that laws and social
arrangements should place the happiness, or (as speaking practically
it may be called) the interest, of every individual, as nearly as
possible in harmony with the interest of the whole; and secondly, that
education and opinion, which have so vast a power over human
character, should so use that power as to establish in the mind of
every individual an indissoluble association between his own happiness
and the good of the whole; especially between his own happiness and
the practice of such modes of conduct, negative and positive, as
regard for the universal happiness prescribes: so that not only he may
be unable to conceive the possibility of happiness to himself,
consistently with conduct opposed to the general good, but also that a
direct impulse to promote the general good may be in every individual
one of the habitual motives of action, and the sentiments connected
therewith may fill a large and prominent place in every human being's
sentient existence. If the impugners of the utilitarian morality
represented it to their own minds in this its true character, I know
not what recommendation possessed by any other morality they could
possibly affirm to be wanting to it: what more beautiful or more
exalted developments of human nature any other ethical system can be
supposed to foster, or what springs of action, not accessible to the
utilitarian, such systems rely on for giving effect to their
mandates.

The objectors to utilitarianism cannot always be \page{26} charged
with representing it in a discreditable light. On the contrary, those
among them who entertain anything like a just idea of its
disinterested character, sometimes find fault with its standard as
being too high for humanity. They say it is exacting too much to
require that people shall always act from the inducement of promoting
the general interests of society. But this is to mistake the very
meaning of a standard of morals, and to confound the rule of action
with the motive of it. It is the business of ethics to tell us what
are our duties, or by what test we may know them; but no system of
ethics requires that the sole motive of all we do shall be a feeling
of duty; on the contrary, ninety-nine hundredths of all our actions
are done from other motives, and rightly so done, if the rule of duty
does not condemn them. It is the more unjust to utilitarianism that
this particular misapprehension should be made a ground of objection
to it, inasmuch as utilitarian moralists have gone beyond almost all
others in affirming that the motive has nothing to do with the
morality of the action, though much with the worth of the agent. He
who saves a fellow creature from drowning does what is morally right,
whether his motive be duty, or the hope of being paid for his
trouble: he who betrays the friend that trusts him, is guilty of a
crime, even if his object be to serve another friend to whom he is
under greater obligations.\footnote{An opponent, whose intellectual
and moral fairness it is a pleasure to acknowledge (the Rev. J.
Llewelyn Davies), has objected to this passage, saying, ``Surely the
rightness or wrongness of saving a man from drowning does depend very
much upon the motive with which it is done. Suppose that a tyrant,
when his enemy jumped into the sea to escape from him, saved him from
drowning simply in order that he might inflict upon him more exquisite
tortures, would it tend to clearness to speak of that rescue as `a
morally right action'? Or suppose again, according to one of the stock
illustrations of ethical inquiries, that a man betrayed a trust
received from a friend, because the discharge of it would fatally
injure that friend himself or some one belonging to him, would
utilitarianism compel one to call the betrayal `a crime' as much as if
it had been done from the meanest motive?''

I submit, that he who saves another from drowning in order to kill him
by torture afterwards, does not differ only in motive from him who
does the same thing from duty or benevolence; the act itself is
different. The rescue of the man is, in the case supposed, only the
necessary first step of an act far more atrocious than leaving him to
drown would have been. Had Mr. Davies said, ``The rightness or
wrongness of saving a man from drowning does depend very much''---not
upon the motive, but---``upon the \textit{intention},'' no utilitarian
would have differed from him. Mr. Davies, by an oversight too common
not to be quite venial, has in this case confounded the very different
ideas of Motive and Intention. There is no point which utilitarian
thinkers (and Bentham pre-eminently) have taken more pains to
illustrate than this. The morality of the action depends entirely upon
the intention---that is, upon what the agent \textit{wills to do}. But
the motive, that is, the feeling which makes him will so to do, when
it makes no difference in the act, makes none in the morality: though
it makes a great difference in our moral estimation of the agent,
especially if it indicates a good or a bad habitual
\textit{dis\-po\-si\-tion}---a bent of character from which useful, or
from which hurtful actions are likely to arise.} But to speak only of
actions \page{27} done from the motive of duty, and in direct
obedience to principle: it is a misapprehension of the utilitarian
mode of thought, to conceive it as implying that people should fix
their minds upon so wide a generality as the world, or society at
large. The great majority of good actions are intended, not for the
benefit of the world, but for that of individuals, of which the good
of the world is made up; and the thoughts of the most virtuous man
need not on these occasions travel beyond \page{28} the particular
persons concerned, except so far as is necessary to assure himself
that in benefiting them he is not violating the rights---that is, the
legitimate and authorized ex\-pec\-ta\-tions---of any one else. The
multiplication of happiness is, according to the utilitarian ethics,
the object of virtue: the occasions on which any person (except one in
a thousand) has it in his power to do this on an extended scale, in
other words, to be a public benefactor, are but exceptional; and on
these occasions alone is he called on to consider public utility; in
every other case, private utility, the interest or happiness of some
few persons, is all he has to attend to. Those alone the influence of
whose actions extends to society in general, need concern themselves
habitually about so large an object. In the case of abstinences
in\-deed---of things which people forbear to do, from moral
considerations, though the consequences in the particular case might
be ben\-e\-fi\-cial---it would be unworthy of an intelligent agent not
to be consciously aware that the action is of a class which, if
practised generally, would be generally injurious, and that this is
the ground of the obligation to abstain from it. The amount of regard
for the public interest implied in this recognition, is no greater
than is demanded by every system of morals; for they all enjoin to
abstain from whatever is manifestly pernicious to society.

The same considerations dispose of another reproach against the
doctrine of utility, founded on a still grosser misconception of the
purpose of a standard of morality, and of the very meaning of the
words right and wrong. It is often affirmed that utilitarianism
renders men cold and unsympathizing; that it chills \page{29} their
moral feelings towards individuals; that it makes them regard only the
dry and hard consideration of the consequences of actions, not taking
into their moral estimate the qualities from which those actions
emanate. If the assertion means that they do not allow their judgment
respecting the rightness or wrongness of an action to be influenced by
their opinion of the qualities of the person who does it, this is a
complaint not against utilitarianism, but against having any standard
of morality at all; for certainly no known ethical standard decides an
action to be good or bad because it is done by a good or a bad man,
still less because done by an amiable, a brave, or a benevolent man,
or the contrary. These considerations are relevant, not to the
estimation of actions, but of persons; and there is nothing in the
utilitarian theory inconsistent with the fact that there are other
things which interest us in persons besides the rightness and
wrongness of their actions. The Stoics, indeed, with the paradoxical
misuse of language which was part of their system, and by which they
strove to raise themselves above all concern about anything but
virtue, were fond of saying that he who has that has everything; that
he, and only he, is rich, is beautiful, is a king. But no claim of
this description is made for the virtuous man by the utilitarian
doctrine. Utilitarians are quite aware that there are other desirable
possessions and qualities besides virtue, and are perfectly willing to
allow to all of them their full worth. They are also aware that a
right action does not necessarily indicate a virtuous character, and
that actions which are blameable often proceed from qualities entitled
to praise. When this is apparent in any particular case, it
modi-\page{30}fies their estimation, not certainly of the act, but of
the agent. I grant that they are, notwithstanding, of opinion, that in
the long run the best proof of a good character is good actions; and
resolutely refuse to consider any mental disposition as good, of which
the predominant tendency is to produce bad conduct. This makes them
unpopular with many people; but it is an unpopularity which they must
share with every one who regards the distinction between right and
wrong in a serious light; and the reproach is not one which a
conscientious utilitarian need be anxious to repel.

If no more be meant by the objection than that many utilitarians look
on the morality of actions, as measured by the utilitarian standard,
with too exclusive a regard, and do not lay sufficient stress upon the
other beauties of character which go towards making a human being
loveable or admirable, this may be admitted. Utilitarians who have
cultivated their moral feelings, but not their sympathies nor their
artistic perceptions, do fall into this mistake; and so do all other
moralists under the same conditions. What can be said in excuse for
other moralists is equally available for them, namely, that if there
is to be any error, it is better that it should be on that side. As a
matter of fact, we may affirm that among utilitarians as among
adherents of other systems, there is every imaginable degree of
rigidity and of laxity in the application of their standard: some are
even puritanically rigorous, while others are as indulgent as can
possibly be desired by sinner or by sentimentalist. But on the whole,
a doctrine which brings prominently forward the interest that mankind
have in the re-\page{31}pression and prevention of conduct which
violates the moral law is likely to be inferior to no other in turning
the sanctions of opinion against such violations. It is true, the
question, What does violate the moral law? is one on which those who
recognise different standards of morality are likely now and then to
differ. But difference of opinion on moral questions was not first
introduced into the world by utilitarianism, while that doctrine does
supply, if not always an easy, at all events a tangible and
intelligible mode of deciding such differences.

\vspace{1\baselineskip}

It may not be superfluous to notice a few more of the common
misapprehensions of utilitarian ethics, even those which are so
obvious and gross that it might appear impossible for any person of
candour and intelligence to fall into them: since persons, even of
considerable mental endowments, often give themselves so little
trouble to understand the bearings of any opinion against which they
entertain a prejudice, and men are in general so little conscious of
this voluntary ignorance as a defect, that the vulgarest
misunderstandings of ethical doctrines are continually met with in the
deliberate writings of persons of the greatest pretensions both to
high principle and to philosophy. We not uncommonly hear the doctrine
of utility inveighed against as a \textit{godless} doctrine. If it be
necessary to say anything at all against so mere an assumption, we may
say that the question depends upon what idea we have formed of the
moral character of the Deity. If it be a true belief that God
desires, above all things, the happiness of his creatures, and that
this was his purpose in their creation, utility is \page{32} not only
not a godless doctrine, but more profoundly religious than any other.
If it be meant that utilitarianism does not recognise the revealed
will of God as the supreme law of morals, I answer, that an
utilitarian who believes in the perfect goodness and wisdom of God,
necessarily believes that whatever God has thought fit to reveal on
the subject of morals, must fulfil the requirements of utility in a
supreme degree. But others besides utilitarians have been of opinion
that the Christian revelation was intended, and is fitted, to inform
the hearts and minds of mankind with a spirit which should enable
them to find for themselves what is right, and incline them to do it
when found, rather than to tell them, except in a very general way,
what it is: and that we need a doctrine of ethics, carefully followed
out, to \textit{interpret} to us the will of God. Whether this opinion
is correct or not, it is superfluous here to discuss; since whatever
aid religion, either natural or revealed, can afford to ethical
investigation, is as open to the utilitarian moralist as to any
other. He can use it as the testimony of God to the usefulness or
hurtfulness of any given course of action, by as good a right as
others can use it for the indication of a transcendental law, having
no connexion with usefulness or with happiness.

Again, Utility is often summarily stigmatized as an immoral doctrine
by giving it the name of Expediency, and taking advantage of the
popular use of that term to contrast it with Principle. But the
Expedient, in the sense in which it is opposed to the Right, generally
means that which is expedient for the particular interest of the agent
himself; as when a minister sacrifices the interest of his country to
keep himself \page{33} in place. When it means anything better than
this, it means that which is expedient for some immediate object, some
temporary purpose, but which violates a rule whose observance is
expedient in a much higher degree. The Expedient, in this sense,
instead of being the same thing with the useful, is a branch of the
hurtful. Thus, it would often be expedient, for the purpose of getting
over some momentary embarrassment, or attaining some object
immediately useful to ourselves or others, to tell a lie. But inasmuch
as the cultivation in ourselves of a sensitive feeling on the subject
of veracity, is one of the most useful, and the enfeeblement of that
feeling one of the most hurtful, things to which our conduct can be
instrumental; and inasmuch as any, even unintentional, deviation from
truth, does that much towards weakening the trustworthiness of human
assertion, which is not only the principal support of all present
social well-being, but the insufficiency of which does more than any
one thing that can be named to keep back civilisation, virtue,
everything on which human happiness on the largest scale depends; we
feel that the violation, for a present advantage, of a rule of such
transcendant expediency, is not expedient, and that he who, for the
sake of a convenience to himself or to some other individual, does
what depends on him to deprive mankind of the good, and inflict upon
them the evil, involved in the greater or less reliance which they can
place in each other's word, acts the part of one of their worst
enemies. Yet that even this rule, sacred as it is, admits of possible
exceptions, is acknowledged by all moralists; the chief of which is
when the withholding of some fact (as of information from a
male-\page{34}factor, or of bad news from a person dangerously ill)
would jpreserve some one (especially a person other than oneself) from
great and unmerited evil, and when the withholding can only be
effected by denial. But in order that the exception may not extend
itself beyond the need, and may have the least possible effect in
weakening reliance on veracity, it ought to be recognised, and, if
possible, its limits defined; and if the principle of utility is good
for anything, it must be good for weighing these conflicting utilities
against one another, and marking out the region within which one or
the other preponderates.

Again, defenders of utility often find themselves called upon to reply
to such objections as this---that there is not time, previous to
action, for calculating and weighing the effects of any line of
conduct on the general happiness. This is exactly as if any one were
to say that it is impossible to guide our conduct by Christianity,
because there is not time, on every occasion on which anything has to
be done, to read through the Old and New Testaments. The answer to the
objection is, that there has been ample time, namely, the whole past
duration of the human species. During all that time mankind have been
learning by experience the tendencies of actions; on which experience
all the prudence, as well as all the morality of life, is dependent.
People talk as if the commencement of this course of experience had
hitherto been put off, and as if, at the moment when some man feels
tempted to meddle with the property or life of another, he had to
begin considering for the first time whether murder and theft are
injurious to human happiness. Even then I do not think that he would
find the \page{35} question very puzzling; but, at all events, the
matter is now done to his hand. It is truly a whimsical supposition,
that if mankind were agreed in considering utility to be the test of
morality, they would remain without any agreement as to what
\textit{is} useful, and would take no measures for having their
notions on the subject taught to the young, and enforced by law and
opinion. There is no difficulty in proving any ethical standard
whatever to work ill, if we suppose universal idiocy to be conjoined
with it, but on any hypothesis short of that, mankind must by this
time have acquired positive beliefs as to the effects of some actions
on their happiness; and the beliefs which have thus come down are the
rules of morality for the multitude, and for the philosopher until he
has succeeded in finding better. That philosophers might easily do
this, even now, on many subjects; that the received code of ethics is
by no means of divine right; and that mankind have still much to learn
as to the effects of actions on the general happiness, I admit, or
rather, earnestly maintain. The corollaries from the principle of
utility, like the precepts of every practical art, admit of indefinite
improvement, and, in a progressive state of the human mind, their
improvement is perpetually going on. But to consider the rules of
morality as improvable, is one thing; to pass over the intermediate
generalisations entirely, and endeavour to test each individual action
directly by the first principle, is another. It is a strange notion
that the acknowledgment of a first principle is inconsistent with the
admission of secondary ones. To inform a traveller respecting the
place of his ultimate destination, is not to forbid the \page{36} use
of landmarks and direction-posts on the way. The proposition that
happiness is the end and aim of morality, does not mean that no road
ought to be laid down to that goal, or that persons going thither
should not be advised to take one direction rather than another. Men
really ought to leave off talking a kind of nonsense on this subject,
which they would neither talk nor listen to on other matters of
practical concernment. Nobody argues that the art of navigation is not
founded on astronomy, because sailors cannot wait to calculate the
Nautical Almanack. Being rational creatures, they go to sea with it
ready calculated; and all rational creatures go out upon the sea of
life with their minds made up on the common questions of right and
wrong, as well as on many of the far more difficult questions of
wise and foolish. And this, as long as foresight is a human quality,
it is to be presumed they will continue to do. Whatever we adopt as
the fundamental principle of morality, we require subordinate
principles to apply it by: the impossibility of doing without them,
being common to all systems, can afford no argument against any one in
particular: but gravely to argue as if no such secondary principles
could be had, and as if mankind had remained till now, and always must
remain, without drawing any general conclusions from the experience of
human life, is as high a pitch, I think, as absurdity has ever reached
in philosophical controversy.

The remainder of the stock arguments against utilitarianism mostly
consist in laying to its charge the common infirmities of human
nature, and the general difficulties which embarrass conscientious
\page{37} persons in shaping their course through life. We are told
than an utilitarian will be apt to make his own particular case an
exception to moral rules, and, when under temptation, will see an
utility in the breach of a rule, greater than he will see in its
observance. But is utility the only creed which is able to furnish us
with excuses for evil doing, and means of cheating our own conscience?
They are afforded in abundance by all doctrines which recognise as a
fact in morals the existence of conflicting considerations; which all
doctrines do, that have been believed by sane persons. It is not the
fault of any creed, but of the complicated nature of human affairs,
that rules of conduct cannot be so framed as to require no exceptions,
and that hardly any kind of action can safely be laid down as either
always obligatory or always condemnable. There is no ethical creed
which does not temper the rigidity of its laws, by giving a certain
latitude, under the moral responsibility of the agent, for
accommodation to peculiarities of circumstances; and under every
creed, at the opening thus made, self-deception and dishonest
casuistry get in. There exists no moral system under which there do
not arise unequivocal cases of conflicting obligation. These are the
real difficulties, the knotty points both in the theory of ethics, and
in the conscientious guidance of personal conduct. They are overcome
practically with greater or with less success according to the
intellect and virtue of the individual; but it can hardly be pretended
that any one will be the less qualified for dealing with them, from
possessing an ultimate standard to which conflicting rights and duties
can be referred. If utility is the ultimate source of moral \page{38}
obligations, utility may be invoked to decide between them when their
demands are incompatible. Though the application of the standard may
be difficult, it is better than none at all: while in other systems,
the moral laws all claiming independent authority, there is no common
umpire entitled to interfere between them; their claims to precedence
one over another rest on little better than sophistry, and unless
determined, as they generally are, by the unacknowledged influence of
considerations of utility, afford a free scope for the action of
personal desires and partialities. We must remember that only in
these cases of conflict between secondary principles is it requisite
that first principles should be appealed to. There is no case of moral
obligation in which some secondary principle is not involved; and if
only one, there can seldom be any real doubt which one it is, in the
mind of any person by whom the principle itself is recognised.

