
\author{Plato}
\authdate{ca. 427--348/347 \BCE}
%\textdate{ca. 380 \BCE}
\textdate{ca. early fourth century \BCE}
\addon[Book 2, 358e--60d]{Republic}
\chapter{Glaucon on the Ring of Gyges}
%\source*[bk 2, 358e--60d]{plato1908.1}
\source*{plato1908.1}

% No page numbers in cited text, only Stephanus numbers. However,
% there's no precise way to insert the Stephanus numbers, since the
% line breaks here are different.

They say that to do injustice is, by nature, good; to suffer
injustice, evil; but that the evil is greater than the good. And so
when men have both done and suffered injustice and have had experience
of both, not being able to avoid the one and obtain the other, they
think that they had better agree among themselves to have neither;
hence there arise laws and mutual covenants; and that which is
ordained by law is termed by them lawful and just. This they affirm to
be the origin and nature of jus\-tice;---it is a mean or compromise,
between the best of all, which is to do injustice and not be punished,
and the worst of all, which is to suffer injustice without the power
of retaliation; and justice, being at a middle point between the two,
is tolerated not as a good, but as the lesser evil, and honoured by
reason of the inability of men to do injustice. For no man who is
worthy to be called a man would ever submit to such an agreement if he
were able to resist; he would be mad if he did. Such is the received
account, Socrates, of the nature and origin of justice.

Now that those who practise justice do so involuntarily and because
they have not the power to be unjust will best appear if we imagine
something of this kind: having given both to the just and the unjust
power to do what they will, let us watch and see whither desire will
lead them; then we shall discover in the very act the just and unjust
man to be proceeding along the same road, following their interest,
which all natures deem to be their good, and are only diverted into
the path of justice by the force of law. The liberty which we are
supposing may be most completely given to them in the form of such a
power as is said to have been possessed by Gyges, the ancestor of
Croesus the Lydian. According to the tradition, Gyges was a shepherd
in the service of the king of Lydia; there was a great storm, and an
earthquake made an opening in the earth at the place where he was
feeding his flock. Amazed at the sight, he descended into the opening,
where, among other marvels, he beheld a hollow brazen horse, having
doors, at which he stooping and looking in saw a dead body of stature,
as appeared to him, more than human, and having nothing on but a gold
ring; this he took from the finger of the dead and reascended. Now the
shepherds met together, according to custom, that they might send
their monthly report about the flocks to the king; into their assembly
he came having the ring on his finger, and as he was sitting among
them he chanced to turn the collet of the ring inside his hand, when
instantly he became invisible to the rest of the company and they
began to speak of him as if he were no longer present. He was
astonished at this, and again touching the ring he turned the collet
outwards and reappeared; he made several trials of the ring, and
always with the same result---when he turned the collet inwards he
became invisible, when outwards he reappeared. Whereupon he contrived
to be chosen one of the messengers who were sent to the court; where
as soon as he arrived he seduced the queen, and with her help
conspired against the king and slew him, and took the kingdom. Suppose
now that there were two such magic rings, and the just put on one of
them and the unjust the other; no man can be imagined to be of such an
iron nature that he would stand fast in justice. No man would keep his
hands off what was not his own when he could safely take what he liked
out of the market, or go into houses and lie with any one at his
pleasure, or kill or release from prison whom he would, and in all
respects be like a god among men. Then the actions of the just would
be as the actions of the unjust; they would both come at last to the
same point. And this we may truly affirm to be a great proof that a
man is just, not willingly or because he thinks that justice is any
good to him individually, but of necessity, for wherever any one
thinks that he can safely be unjust, there he is unjust. For all men
believe in their hearts that injustice is far more profitable to the
individual than justice, and he who argues as I have been supposing,
will say that they are right. If you could imagine any one obtaining
this power of becoming invisible, and never doing any wrong or
touching what was another's, he would be thought by the lookers-on to
be a most wretched idiot, although they would praise him to one
another's faces, and keep up appearances with one another from a fear
that they too might suffer injustice.

