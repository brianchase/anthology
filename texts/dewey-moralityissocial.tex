
\author{John Dewey}
\authdate{1859--1952}
\textdate{1922}
\chapter{Morality is Social}
\source{dewey1922.4.4}

\page{314}\noindent Intelligence becomes ours in the degree in which
we use it and accept responsibility for consequences. It is not ours
originally or by production. ``It thinks'' is a truer psychological
statement than ``I think.'' Thoughts sprout and vegetate; ideas
proliferate. They come from deep unconscious sources. ``I think'' is a
statement about voluntary action. Some suggestion surges from the
unknown. Our active body of habits appropriates it. The suggestion
then becomes an assertion. It no longer merely comes to us. It is
accepted and uttered by us. We act upon it and thereby assume, by
implication, its consequences. The stuff of belief and proposition is
not originated by us. It comes to us from others, by education,
tradition and the suggestion of the environment. Our intelligence is
bound up, so far as its materials are concerned, with the community
life of which we are a part. We know what it communicates to us, and
know according to the habits it forms in us. Science is an affair of
civilization not of individual intellect.

So with conscience. When a child acts, those about him re-act. They
shower encouragement upon him, visit him with approval, or they bestow
frowns and rebuke. What others do	to us when we act is as natural a
consequence of our action as what the fire does \page{315} to us when
we plunge our hands in it. The social environment may be as artificial
as you please. But its action in response to ours is natural not
artificial. In language and imagination we rehearse the responses of
others just as we dramatically enact other consequences. We foreknow
how others will act, and the foreknowledge is the beginning of
judgment passed on action. We know \textit{with} them; there is
conscience. An assembly is formed within our breast which discusses
and appraises proposed and performed acts. The community without
becomes a forum and tribunal within, a judgment-seat of charges,
assessments and exculpations. Our thoughts of our own actions are
saturated with the ideas that others entertain about them, ideas which
have been expressed not only in explicit instruction but still more
effectively in reaction to our acts.

Liability is the beginning of responsibility. We are held accountable
by others for the consequences of our acts. They visit their like and
dislike of these consequences upon us. In vain do we claim that these
are not ours; that they are products of ignorance not design, or are
incidents in the execution of a most laudable scheme. Their authorship
is imputed to us. We are disapproved, and disapproval is not an inner
state of mind but a most definite act. Others say to us by their deeds
we do not care a fig whether you did this deliberately or not. We
intend that you \textit{shall} deliberate before you do it again, and
that if possible your deliberation shall prevent a repetition of this
act we object to. The reference in blame and every
unfavor-\page{316}able judgment is prospective, not retrospective.
Theories about responsibility may become confused, but in practice no
one is stupid enough to try to change the past. Approbation and
disapprobation are ways of influencing the formation of habits and
aims; that is, of influencing future acts. The individual is
\textit{held} accountable for what he \textit{has} done in order that
he may be responsive in what he is \textit{going} to do. Gradually
persons learn by dramatic imitation to hold themselves accountable,
and liability becomes a voluntary deliberate acknowledgment that deeds
are our own, that their consequences come from us.

These two facts, that moral judgment and moral responsibility are the
work wrought in us by the social environment, signify that all
morality is social; not because we \textit{ought} to take into account
the effect of our acts upon the welfare of others, but because of
facts. Others \textit{do} take account of what we do, and they respond
accordingly to our acts. Their responses actually \textit{do} affect
the meaning of what we do. The significance thus contributed is as
inevitable as is the effect of interaction with the physical
environment. In fact as civilization advances the physical environment
gets itself more and more humanized, for the meaning of physical
energies and events becomes involved with the part they play in
human activities. Our conduct is socially conditioned whether we
perceive the fact or not.

The effect of custom on habit, and of habit upon thought is enough to
prove this statement. When we \page{317} begin to forecast
consequences, the consequences that most stand out are those which
will proceed from other people. The resistance and the cooperation of
others is the central fact in the furtherance or failure of our
schemes. Connections with our fellows furnish both the opportunities
for action and the instrumentalities by which we take advantage of
opportunity. All of the actions of an individual bear the stamp of his
community as assuredly as does the language he speaks. Difficulty in
reading the stamp is due to variety of impressions in consequence of
membership in many groups. This social saturation is, I repeat, a
matter of fact, not of what should be, not of what is desirable or
undesirable. It does not guarantee the rightness of goodness of an
act; there is no excuse for thinking of evil action as individualistic
and right action as social. Deliberate unscrupulous pursuit of
self-interest is as much conditioned upon social opportunities,
training and assistance as is the course of action prompted by a
beaming benevolence. The difference lies in the quality and degree
of the perception of ties and interdependencies; in the use to which
they are put. Consider the form commonly assumed today by
self-seeking; namely command of money and economic power. Money is a
social institution; property is a legal custom; economic opportunities
are dependent upon the state of society; the objects aimed at, the
rewards sought for, are what they are because of social admiration,
prestige, competition and power. If money-making is morally obnoxious
it is because of the way these \page{318} social facts are handled,
not because a money-making man has withdrawn from society into an
isolated selfhood or turned his back upon society. His
``individualism'' is not found in his original nature but in his
habits acquired under social influences. It is found in his concrete
aims, and these are reflexes of social conditions. Well-grounded
moral objection to a mode of conduct rests upon the kind of social
connections that figure, not upon lack of social aim. A man may
attempt to utilize social relationships for his own advantage in an
inequitable way; he may intentionally or unconsciously try to make
them feed one of his own appetites. Then he is denounced as egoistic.
But both his course of action and the disapproval he is subject to are
facts \textit{within} society. They are social phenomena. He pursues
his unjust advantage as a social asset.

Explicit recognition of this fact is a prerequisite of improvement in
moral education and of an intelligent understanding of the chief ideas
or ``categories'' of morals. Morals is as much a matter of interaction
of a person with his social environment as walking is an interaction
of legs with a physical environment. The character of walking depends
upon the strength and competency of legs. But it also depends upon
whether a man is walking in a bog or on a paved street, upon whether
there is a safeguarded path set aside or whether he has to walk amid
dangerous vehicles. If the standard of morals is low it is because the
education given by the interaction of the individual with his social
en-\page{319}vironment is defective. Of what avail is it to preach
unassuming simplicity and contentment of life when communal admiration
goes to the man who ``suc\-ceeds''---who makes himself conspicuous and
envied because of command of money and other forms of power? If a
child gets on by peevishness or intrigue, then others are his
accomplices who assist in the habits which are built up. The notion
that an abstract ready-made conscience exists in individuals and that
it is only necessary to make an occasional appeal to it and to indulge
in occasional crude rebukes and punishments, is associated with the
causes of lack of definitive and orderly moral advance. For it is
associated with lack of attention to social forces.

There is a peculiar inconsistency in the current idea that morals
\textit{ought} to be social. The introduction of the moral ``ought''
into the idea contains an implicit assertion that morals depend upon
something apart from social relations. Morals \textit{are} social. The
question of ought, should be, is a question of better and worse
\textit{in} social affairs. The extent to which the weight of theories
has been thrown against the perception of the place of social ties and
connections in moral activity is a fair measure of the extent to which
social forces work blindly and develop an accidental morality. The
chief obstacle for example to recognizing the truth of a proposition
frequently set forth in these pages to the effect that all conduct is
potential, if not actual, matter of moral judgment is the habit of
identifying moral judgment with praise and blame. So great is the
in-\page{320}fluence of this habit that it is safe to say that every
professed moralist when he leaves the pages of theory and faces some
actual item of his own or others' behavior, first or ``instinctively''
thinks of acts as moral or non-moral in the degree in which they are
exposed to condemnation or approval. Now this kind of judgment is
certainly not one which could profitably be dispensed with. Its
influence is much needed. But the tendency to equate it with all moral
judgment is largely responsible for the current idea that there is a
sharp line between moral conduct and a larger region of nonmoral
conduct which is a matter of expediency, shrewdness, success or
manners.

Moreover this tendency is a chief reason why the social forces
effective in shaping actual morality work blindly and
unsatisfactorily. Judgment in which the emphasis falls upon blame and
approbation has more heat than light. It is more emotional than
intellectual. It is guided by custom, personal convenience and
resentment rather than by insight into causes and consequences. It
makes toward reducing moral instruction, the educative influence of
social opinion, to an immediate personal matter, that is to say, to an
adjustment of personal likes and dislikes. Fault-finding creates
resentment in the one blamed, and approval, complacency, rather than a
habit of scrutinizing conduct objectively. It puts those who are
sensitive to the judgments of others in a standing defensive attitude,
treating an apologetic, self-accusing and self-exculpating habit of
mind when what is needed is an impersonal \page{321} impartial habit
of observation. ``Moral'' persons get so occupied with defending their
conduct from real and imagined criticism that they have little time
left to see what their acts really amount to, and the habit of
self-blame inevitably extends to include others since it is a habit.

Now it is a wholesome thing for any one to be made aware that
thoughtless, self-centered action on his part exposes him to the
indignation and dislike of others. There is no one who can be safely
trusted to be exempt from immediate reactions of criticism, and there
are few who do not need to be braced by occasional expressions of
approval. But these influences are immensely overdone in comparison
with the assistance that might be given by the influence of social
judgments which operate without accompaniments of praise and blame;
which enable an individual to see for himself what he is doing, and
which put him in command of a method of analyzing the obscure and
usually unavowed forces which move him to act. We need a permeation of
judgments on conduct by the method and materials of a science of human
nature. Without such enlightenment even the best-intentioned attempts
at the moral guidance and improvement of others often eventuate in
tragedies of misunderstanding and division, as is so often seen in the
relations of parents and children.

The development therefore of a more adequate science of human nature
is a matter of first-rate importance. The present revolt against the
notion that psy-\page{322}chology is a science of consciousness may
well turn out in the future to be the beginning of a definitive turn
in thought and action. Historically there are good reasons for the
isolation and exaggeration of the conscious phase of human action, an
isolation which forgot that ``conscious'' is an adjective of some acts
and which erected the resulting abstraction, ``consciousness,'' into a
noun, an existence separate and complete. These reasons are
interesting not only to the student of technical philosophy but also
to the student of the history of culture and even of politics. They
have to do with the attempt to drag realities out of occult essences
and hidden forces and get them into the light of day. They were part
of the general movement called phenomenalism, and of the growing
importance of individual life and private voluntary concerns. But the
effect was to isolate the individual from his connections both with
his fellows and with nature, and thus to create an artificial human
nature, one not capable of being understood and effectively directed
on the basis of analytic understanding. It shut out from view, not to
say from scientific examination, the forces which really move human
nature. It took a few surface phenomena for the whole story of
significant human motive-forces and acts.

As a consequence physical science and its technological applications
were highly developed while the science of man, moral science, is
backward. I believe that it is not possible to estimate how much of
the difficulties of the present world situation are due to the
\page{323} disproportion and unbalance thus introduced into affairs.
It would have seemed absurd to say in the seventeenth century that in
the end the alteration in methods of physical investigation which was
then beginning would prove more important than the religious wars of
that century. Yet the wars marked the end of one era; the dawn of
physical science the beginning of a new one. And a trained imagination
may discover that the nationalistic and economic wars which are the
chief outward mark of the present are in the end to be less
significant than the development of a science of human nature now
inchoate.

It sounds academic to say that substantial bettering of social
relations waits upon the growth of a scientific social psychology. For
the term suggests something specialized and remote. But the formation
of habits of belief, desire and judgment is going on at every instant
under the influence of the conditions set by men's contact,
intercourse and associations with one another. This is the fundamental
fact in social life and in personal character. It is the fact about
which traditional human science gives no enlightenment---a fact which
this traditional science blurs and virtually denies. The enormous
r\^ole played in popular morals by appeal to the supernatural and
quasi-magical is in effect a desperate admission of the futility of
our science. Consequently the whole matter of the formation of the
predispositions which effectively control human relationships is left
to accident, to custom and immediate personal likings, resentments and
ambitions. It is a com-\page{324}monplace that modern industry and
commerce are conditioned upon a control of physical energies due to
proper methods of physical inquiry and analysis. We have no social
arts which are comparable because we have so nearly nothing in the way
of psychological science. Yet through the development of physical
science, and especially of chemistry, biology, physiology, medicine
and anthropology we now have the basis for the development of such a
science of man. Signs of its coming into existence are present in the
movements in clinical, behavioristic and social (in its narrower
sense) psychology.

At present we not only have no assured means of forming character
except crude devices of blame, praise, exhortation and punishment, but
the very meaning of the general notions of moral inquiry is [a] matter
of doubt and dispute. The reason is that these notions are discussed
in isolation from the concrete facts of the interactions of human
beings with one another---an abstraction as fatal as was the old
discussion of phlogiston, gravity and vital force apart from concrete
correlations of changing events with one another. Take for example
such a basic conception as that of Right involving the nature of
authority in conduct. There is no need here to rehearse the multitude
of contending views which give evidence that discussion of this matter
is still in the realm of opinion. We content ourselves with pointing
out that this notion is the last resort of the anti-empirical school
in morals and that it proves the effect of neglect of social
conditions.

% NOTE: The end quotes are not included in the text, so I supply them
% in brackets.

\page{325}In effect its adherents argue as follows: ``Let us concede
that concrete ideas about right and wrong and particular notions of
what is obligatory have grown up within experience. But we cannot
admit this about the idea of Right, of Obligation itself. Why does
moral authority exist at all? Why is the claim of the Right recognized
in conscience even by those who violate it in deed? Our opponents say
that such and such a course is wise, expedient, better. But
\textit{why} act for the wise, or good, or better? Why not follow our
own immediate devices if we are so inclined? There is only one answer:
We have a moral nature, a conscience, call it what you will. And this
nature responds directly in acknowledgment of the supreme authority of
the Right over all claims of inclination and habit. We may not act in
accordance with this acknowledgment, but we still know that the
authority of the moral law, although not its power, is unquestionable.
Men may differ indefinitely according to what their experience has
been as to just \textit{what} is Right, what its contents are. But
they all spontaneously agree in recognizing the supremacy of the
claims of whatever is thought of as Right. Otherwise there would be no
such thing as morality, but merely calculations of how to satisfy
desire.['']

Grant the foregoing argument, and all the apparatus of abstract
moralism follows in its wake. A remote goal of perfection, ideals that
are contrary in a wholesale way to what is actual, a free will of
arbitrary choice; all of these conceptions band themselves together
with that of a non-empirical authority of Right \page{326} and a
non-empirical conscience which acknowledges it. They constitute its
ceremonial or formal train.

Why, indeed, acknowledge the authority of Right? That many persons do
not acknowledge it in fact, in action, and that all persons ignore it
at times, is assumed by the argument. Just what is the significance of
an alleged recognition of a supremacy which is continually denied in
fact? How much would be lost if it were dropped out, and we were left
face to face with actual facts? If a man lived alone in the world
there might be some sense in the question ``Why be moral?'' were it
not for one thing: No such question would then arise. As it is, we
live in a world where other persons live too. Our acts affect them.
They perceive these effects, and react upon us in consequence. Because
they are living beings they make demands upon us for certain things
from us. They approve and con\-demn---not in abstract theory but in
what they do to us. The answer to the question ``Why not put your hand
in the fire?'' is the answer of fact. If you do your hand will be
burnt. The answer to the question why acknowledge the right is of the
same sort. For Right is only an abstract name for the multitude of
concrete demands in action which others impress upon us, and of which
we are obliged, if we would live, to take some account. Its authority
is the exigency of their demands, the efficacy of their insistencies.
There may be good ground for the contention that in theory the idea of
the right is subordinate to that of the good, being a statement of the
curse proper to attain good. But in fact it \page{327} signifies the
totality of social pressures exercised upon us to induce us to think
and desire in certain ways. Hence the right can in fact become the
road to the good only as the elements that compose this unremitting
pressure are enlightened, only as social relationships become
themselves reasonable.

It will be retorted that all pressure is a non-moral affair partaking
of force, not of right; that right must be ideal. Thus we are invited
to enter again the circle in which the ideal has no force and social
actualities no ideal quality. We refuse the invitation because social
pressure is involved in our own lives, as much so as the air we
breathe and the ground we walk upon. If we had desires, judgments,
plans, in short a mind, apart from social connections, then the latter
would be external and their action might be regarded as that of a
nonmoral force. But we live mentally as physically only \textit{in}
and \textit{because} of our environment. Social pressure is but a name
for the interactions which are always going on and in which we
participate, living so far as we partake and dying so far as we do
not. The pressure is not ideal but empirical, yet empirical here means
only actual. It calls attention to the fact that considerations of
right are claims originating not outside of life, but within it. They
are ``ideal'' in precisely the degree in which we intelligently
recognize and act upon them, just as colors and canvas become ideal
when used in ways that give an added meaning to life.

Accordingly failure to recognize the authority of right means defect
in effective apprehension of the real-\page{328}ities of human
association, not an arbitrary exercise of free will. This deficiency
and perversion in apprehension indicates a defect in education---that
is to say, in the operation of actual conditions, in the consequences
upon desire and thought of existing interactions and
interdependencies. It is false that every person has a consciousness
of the supreme authority of right and then misconceives it or ignores
it in action. One has such a sense of the claims of social
relationships as those relationships enforce in one's desires and
observations. The belief in a separate, ideal or transcendental,
practically ineffectual Right is a reflex of the inadequacy with which
existing institutions perform their educative of\-fice---their office
in generating observation of social continuities. It is an endeavor to
``rationalize'' this defect. Like all rationalizations, it operates to
divert attention from the real state of affairs. Thus it helps
maintain the conditions which created it, standing in the way of
effort to make our institutions more humane and equitable. A
theoretical acknowledgment of the supreme authority of Right, of moral
law, gets twisted into an effectual substitute for acts which would
better the customs which now produce vague, dull, halting and evasive
observation of actual social ties. We are not caught in a circle; we
traverse a spiral in which social customs generate some consciousness
of interdependencies, and this consciousness is embodied in acts which
in improving the environment generate new perceptions of social ties,
and so on forever. The relationships, the interactions are
for-\page{329}ever there as fact, but they acquire meaning only in the
desires, judgments and purposes they awaken.

We recur to our fundamental propositions. Morals is connected with
actualities of existence, not with ideals, ends and obligations
independent of concrete actualities. The facts upon which it depends
are those which arise out of active connections of human beings with
one another, the consequences of their mutually intertwined activities
in the life of desire, belief, judgment, satisfaction and
dissatisfaction. In this sense conduct and hence morals are social:
they are not just things which \textit{ought} to be social and which
fail to come up to the scratch. But there are enormous differences of
better and worse in the quality of what is social. Ideal morals begin
with the perception of these differences. Human interaction and ties
are there, are operative in any case. But they can be regulated,
employed in an orderly way for good only as we know how to observe
them. And they cannot be observed aright, they cannot be understood
and utilized, when the mind is left to itself to work without the aid
of science. For the natural unaided mind means precisely the habits of
belief, thought and desire which have been accidentally generated and
confirmed by social institutions or customs. But with all their
admixture of accident and reasonableness we have at last reached a
point where social conditions create a mind capable of scientific
outlook and inquiry. To foster and develop this spirit is the social
obligation of the present because it is its urgent need.

\page{330}Yet the last word is not with obligation nor with the
future. Infinite relationships of man with his fellows and with nature
already exist. The ideal means, as we have seen, a sense of these
encompassing continuities with their infinite reach. This meaning even
now attaches to present activities because they are set in a whole to
which they belong and which belongs to them. Even in the midst of
conflict, struggle and defeat a consciousness is possible of the
enduring and comprehending whole.

To be grasped and held this consciousness needs, like every form of
consciousness, objects, symbols. In the past men have sought many
symbols which no longer serve, especially since men have been
idolators worshiping symbols as things. Yet within these symbols which
have so often claimed to be realities and which have imposed
themselves as dogmas and intolerances, there has rarely been absent
some trace of a vital and enduring reality, that of a community of
life in which continuities of existence are consummated. Consciousness
of the whole has been connected with reverences, affections, and
loyalties which are communal. But special ways of expressing the
communal sense have been established. They have been limited to a
select social group; they have hardened into obligatory rites and been
imposed as conditions of salvation. Religion has lost itself in cults,
dogmas and myths. Consequently the office of religion as sense of
community and one's place in it has been lost. In effect religion has
been distorted into a pos\-ses\-sion---or bur\-den---of a limited part
of \page{331} human nature, of a limited portion of humanity which
finds no way to universalize religion except by imposing its own
dogmas and ceremonies upon others; of a limited class within a partial
group; priests, saints, a church. Thus other gods have been set up
before the one God. Religion as a sense of the whole is the most
individualized of all things, the most spontaneous, undefinable and
varied. For individuality signifies unique connections in the whole.
Yet it has been perverted into something uniform and immutable. It has
been formulated into fixed and defined beliefs expressed in required
acts and ceremonies. Instead of marking the freedom and peace of the
individual as a member of an infinite whole, it has been petrified
into a slavery of thought and sentiment, an intolerant superiority on
the part of the few and an intolerable burden on the part of the many.

Yet every act may carry within itself a consoling and supporting
consciousness of the whole to which it belongs and which in some sense
belongs to it. With responsibility for the intelligent determination
of particular acts may go a joyful emancipation from the burden for
responsibility for the whole which sustains them, giving them their
final outcome and quality. There is a conceit fostered by perversion
of religion which assimilates the universe to our personal desires;
but there is also a conceit of carrying the load of the universe from
which religion liberates us. Within the flickering inconsequential
acts of separate selves dwells a sense of the whole which claims and
dignifies them. \page{332} In its presence we put off mortality and
live in the universal. The life of the community in which we live and
have our being is the fit symbol of this relationship. The acts in
which we express our perception of the ties which bind us to others
are its only rites and ceremonies.

