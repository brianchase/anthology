
%\author{George Edward Moore}
\author{G. E. Moore}
\authdate{1873--1958}
\textdate{1912}
\addon{Ethics, Chapter 5}
\chapter{Results the Test of Right and Wrong}
\source{moore1912.5}

% According to notes at the following link, the edition of Moore's
% Ethics published by Henry Holt and Company, in the series ``The Home
% University Library of Modern Knowledge,'' was published in 1912:
% http://www.publishinghistory.com/home-university-library.html

\page{170}In our last chapter we began considering objections to one
very fundamental principle, which is presupposed by the theory stated
in the first two chapters---a principle which may be summed up in the
two propositions (1) that the question whether an action is right or
wrong always depends upon its \textit{total} consequences, and (2)
that if it is once right to prefer one set of \textit{total}
consequences, A, to another set, B, it must always be right to
prefer any set precisely similar to A to any set precisely similar to
B. The objections to this principle, which we considered in the last
chapter, rested on certain views with regard to the meaning of the
words ``right'' and ``good.'' But there remain several other quite
independent objections, which may be urged against it \page{171} even
if we reject those views. That is to say, there are objections which
may and would be urged against it by many people who accept both of
the two propositions which I was trying to establish in the last
chapter, namely (1) that to call an action ``right'' or ``wrong'' is
not the same thing as to say that any being whatever has towards it
any mental attitude whatever; and (2) that if any given whole is once
intrinsically good or bad, any whole precisely similar to it must
always be intrinsically good or bad in precisely the same degree. And
in the present chapter I wish briefly to consider what seem to me to
be the most important of these remaining objections.

All of them are directed against the view that right and wrong do
always depend upon an action's \textit{actual} consequences or
results. This may be denied for several different reasons; and I shall
try to state fairly the chief among these reasons, and to point out
why they do not seem to be conclusive.

In the first place, it may be said that, by laying down the principle
that right and wrong depend upon consequences, we are doing away with
the distinction between \page{172} what is a \textit{duty} and what is
merely \textit{expedient}; and between what is \textit{wrong} and what
is merely \textit{inexpedient}. People certainly do commonly make a
distinction between duty and expediency. And it may be said that the
very meaning of calling an action ``expedient'' is to say that it will
produce the best consequences possible under the circumstances. If,
therefore, we also say that an action is a \textit{duty}, whenever and
only when it produces the best possible consequences, it may seem
that nothing is left to distinguish duty from expediency.

Now, as against this objection, it is important to point out, first of
all, that, even if we admit that to call an action expedient is the
same thing as to say that it produces the best possible consequences,
our principle still does not compel us to hold that to call an action
expedient is \textit{the same thing} as to call it a duty. All that it
does compel us to hold is that whatever is expedient is always
\textit{also} a duty, and that whatever is a duty is always
\textit{also} expedient. That is to say, it \textit{does} maintain
that duty and expediency \textit{coincide}; but it does \textit{not}
\page{173} maintain that the meaning of the two words is the same. It
is, indeed, quite plain, I think, that the meaning of the two words is
\textit{not} the same; for, if it were, then it would be a mere
tautology to say that it is always our duty to do what will have the
best possible consequences. Our theory does not, therefore, do away
with the distinction between the \textit{meaning} of the words
``duty'' and ``expediency''; it only maintains that both will always
apply to the same actions.

But, no doubt, what is meant by many who urge this objection is to
deny this. What they mean to say is not merely that to call an action
expedient is a different thing from calling it a duty, but also that
sometimes what \textit{is} expedient is \textit{wrong}, and what
\textit{is} a duty is inexpedient. This is a view which is undoubtedly
often held; people often speak as if there often were an actual
conflict between duty and expediency. But many of the cases in which
it would be commonly held that there is such a conflict may, I think,
be explained by supposing that when we call an action ``expedient'' we
do not always mean quite strictly that \page{174} its \textit{total}
consequences, taking absolutely \textit{everything} into account, are
the best possible. It is by no means clear that we do always mean
this. We may, perhaps, sometimes mean merely that the action is
expedient for some particular purpose; and sometimes that it is
expedient in the interests of the agent, though not so on the whole.
But if we only mean this, our theory, of course, does \textit{not}
compel us to maintain that the expedient \textit{is} always a duty,
and duty always expedient. It only compels us to maintain this, if
``expedient'' be understood in the strictest and fullest sense, as
meaning that, when \textit{absolutely all} the consequences are taken
into account, they will be found to be the best possible. And if this
be clearly understood, then most people, I think, will be reluctant to
admit that it can ever be really inexpedient to do our duty, or that
what is really and truly expedient, in this strict sense, can ever be
wrong.

But, no doubt, some people may still maintain that it is or may be
sometimes our duty to do actions which will \textit{not} have the best
possible consequences, and sometimes \page{175} also positively wrong,
to do actions which will. And the chief reason why this is held is, I
think, the following.

It is, in fact, very commonly held indeed that there are certain
specific kinds of action which are absolutely always right, and others
which are absolutely always wrong. Different people will, indeed, take
different views as to exactly what kinds of action have this
character. A rule which will be offered by one set of persons as a
rule to which there is absolutely no exception will be rejected by
others, as obviously admitting of exceptions; but these will
generally, in their turn, maintain that some other rule, which they
can mention, really has no exceptions. Thus there are enormous numbers
of people who would agree that \textit{some rule or other} (and
generally more than one) ought \textit{absolutely always} to be
obeyed; although probably there is not one single rule which
\textit{all} the persons who maintain this would agree upon. Thus, for
instance, some people might maintain that murder (defined in some
particular way) is an act which ought absolutely \textit{never} to be
com-\page{176}mitted; or that to act \textit{justly} is a rule which
ought absolutely always to be obeyed; and similarly it might be
suggested with regard to many other kinds of action, that they are
actions, which it is either \textit{always} our duty, or
\textit{always} wrong to do.

But once we assert with regard to any rule of this kind that it
\textit{is absolutely always} our duty to obey it, it is easy and
natural to take one further step and to say that it \textit{would}
always be our duty to obey it, \textit{whatever} the consequences
might be. Of course, this further step does not necessarily and
logically follow from the mere position that there are some kinds of
action which ought, \textit{in fact}, absolutely always to be done or
avoided. For it is just possible that there are some kinds which do,
as a matter of fact, absolutely always produce the best possible
consequences, and other kinds which absolutely never do so. And there
is a strong tendency among persons who hold the first position to hold
that, as a matter of fact, this is the case: that right actions always
do, as a matter of fact, produce the best possible results, and wrong
actions never. Thus even those who would \page{177} assent to the
maxim that ``Justice should always be done, though the heavens should
falls,'' will generally be disposed to believe that justice never
will, in fact, cause the heavens to fall, but will rather be always
the best means of upholding them. And similarly those who say that
``you should never do evil that good may come,'' though their maxim
seems to imply that good \textit{may} sometimes come from doing wrong,
would yet be very loth to admit that, by doing wrong, you ever would
\textit{really} produce better consequences \textit{on the whole} than
if you had acted rightly instead. Or again, those who say ``that the
end will never justify the means,'' though they certainly imply that
certain ways of acting would be always wrong, \textit{whatever}
advantages might be secured by them, yet, I think, would be inclined
to deny that the advantages to be obtained by acting wrongly ever do
\textit{really} outweigh those to be obtained by acting rightly, if we
take into account absolutely \textit{all} the consequences of each
course.

Those, therefore, who hold that certain specific ways of acting are
absolutely always \page{178} right, and others absolutely always
wrong, do, I think, generally hold that the former do also, as a
matter of fact, absolutely always produce the best results, and the
latter never. But, for the reasons given at the beginning of Chapter
III, it is, I think, very unlikely that this belief can be justified.
The total results of an action always depend, not merely on the
specific nature of the action, but on the circumstances in which it is
done; and the circumstances vary so greatly that it is, in most cases,
extremely unlikely that any particular kind of action will
\textit{absolutely} always, in absolutely all circumstances, either
produce or fail to produce the best possible results. For this reason,
if we do take the view that right and wrong depend upon consequences,
we must, I think, be prepared to doubt whether any particular kind of
action whatever is absolutely always right or absolutely always wrong.
For instance, however we define ``murder,'' it is unlikely that
absolutely \textit{no} case will ever occur in which it would be right
to commit a murder; and, however we define ``justice,'' it is unlikely
that \textit{no} case will ever occur \page{179} in which it would be
right to do an injustice. No doubt it may be possible to define
actions of which it is true that, in an \textit{immense} majority of
cases, it is right or wrong to perform them; and perhaps \textit{some}
rules of this kind might be found to which there are really
\textit{no} exceptions. But in the case of most of the ordinary moral
rules, it seems extremely unlikely that obedience to them will
\textit{absolutely always} produce the best possible results. And most
persons who realise this would, I think, be disposed to give up the
view that they ought absolutely \textit{always} to be obeyed. They
would be content to accept them as \textit{general} rules, to which
there are very few exceptions, without pretending that they are
absolutely universal.

But, no doubt, there may be some persons who will hold, in the case of
some particular rule or set of rules, that even if obedience to it
does in some cases \textit{not} produce the best possible
consequences, yet we ought even in these cases to obey it. It may seem
to them that they really do know certain rules, which ought
\textit{absolutely always} to be obeyed, \textit{whatever} the
consequences may be, and even, \page{180} therefore, if the total
consequences are not the best possible. They may, for instance, take
quite seriously the assertion that justice ought to be done, even
though the heavens should fall, as meaning that, \textit{however} bad
the consequences of doing an act of justice might in some
circumstances be, yet it always would be our duty to do it. And such a
view does necessarily contradict our principle; since, whether it be
true or not that an act of injustice ever actually could in this world
produce the best possible consequences, it is certainly possible to
\textit{conceive} circumstances in which it would do so. I doubt
whether those who believe in the absolute universality of certain
moral rules do generally thus distinguish quite clearly between the
question whether disobedience to the rule ever \textit{could} produce
the best possible consequences, and the question whether, \textit{if}
it did, then disobedience would be wrong. They would generally be
disposed to argue that it never really \textit{could}. But some
persons might perhaps hold that, even if it did, yet disobedience
would be wrong. And if this view be quite clearly held, there
\page{181} is, so far as I can see, absolutely no way of refuting it
except by appealing to the self-evidence of the principle that if we
\textit{knew} that the effect of a given action really would be to
make the world, as a whole, \textit{worse} than it would have been if
we had acted differently, it certainly would be wrong for us to do
that action. Those who say that certain rules ought \textit{absolutely
always} to be obeyed, \textit{whatever} the consequences may be, are
logically bound to deny this; for by saying ``\textit{whatever} the
consequences may be,'' they do imply ``\textit{even if} the world as a
whole were the worse because of our action.'' It seems to me to be
self-evident that knowingly to do an action which would make the
world, on the whole, really and truly \textit{worse} than if we had
acted differently, must always be wrong. And if this be admitted, then
it absolutely disposes of the view that there are any kinds of action
whatever, which it \textit{would} always be our duty to do or to
avoid, \textit{whatever} the consequences might be.

For this reason it seems to me we must reject this particular
objection to the view that right and wrong always depend upon
\page{182} consequences; namely, the objection that there are certain
\textit{kinds} of action which ought absolutely always and quite
unconditionally to be done or avoided. But there still remain two
other objections, which are so commonly held, that it is worth while
to consider them.

The first is the objection that right and wrong depend neither upon
the nature of the action, nor upon its consequences, but partly, or
even entirely, upon the \textit{motive} or \textit{motives} from which
it is done. By the view that it depends \textit{partly} upon the
motives, I mean the view that no action can be \textit{really} right,
unless it be done from some one motive, or some one of a set of
motives, which are supposed to be good; but that the being done from
such a motive is not sufficient, \textit{by itself}, to make an action
right: that the action, if it is to be right, must always
\textit{also} either produce the best possible consequences, or be
distinguished by some other characteristic. And this view, therefore,
will not necessarily contradict our principle so far as it asserts
that no action can be right, \textit{unless} it produces the best
possible consequences: it only contradicts that part of \page{183} it
which asserts that \textit{every} action which does produce them is
right. But the view has sometimes been held, I think, that right and
wrong depend \textit{entirely} upon motives: that is to say, that not
only is no action right, \textit{unless} it be done from a good
motive, but also that \textit{any} action which is done from some one
motive or some one of a set of motives is always right, whatever its
consequences may be and whatever it may be like in other respects. And
this view, of course, will contradict both parts of our principle;
since it not only implies that an action, which produces the best
possible consequences may be wrong, but also that an action may be
right, in spite of failing to produce them.

In favour of both these views it may be urged that in our moral
judgments we actually do, and ought to, take account of motives; and
indeed that it marks a great advance in morality when men do begin to
attach importance to motives and are not guided exclusively in their
praise or blame, by the ``external'' nature of the act done or by its
consequences. And all this may be fully admitted. It is quite certain
that when a \page{184} man does an action which has bad consequences
from a good motive, we do tend to judge him differently from a man who
does a similar action from a bad one; and also that when a man does an
action which has good consequences from a bad motive, we may
nevertheless think badly of him for it. And it may be admitted that,
in some cases at least, it is right and proper that a man's motives
should thus influence our judgment. But the question is: What
\textit{sort} of moral judgment is it right and proper that they
should influence? Should it influence our view as to whether the
action in question is right or wrong? It seems very doubtful whether,
as a rule, it actually does affect our judgment on this particular
point, for we are quite accustomed to judge that a man sometimes acts
\textit{wrongly} from the best of motives; and though we should admit
that the good motive forms some excuse, and that the whole state of
things is better than if he had done the same thing from a bad motive,
it yet does not lead us to deny that the action \textit{is} wrong.
There is, therefore, reason to think that the kind of moral judgments
which a consideration of \page{185} motives actually \textit{does}
affect do not consist of judgments as to whether the action done from
the motive is \textit{right} or \textit{wrong}; but are moral
judgments of \textit{some different kind}; and there is still more
reason to think that it is only judgments of some different kind which
\textit{ought} to be influenced by it.

The fact is that judgments as to the rightness and wrongness of
actions are by no means the only kind of moral judgments which we
make; and it is, I think, solely because some of these other judgments
are confused with judgments of right and wrong that the latter are
ever held to depend upon the motive. There are three other kinds of
judgments which are chiefly concerned in this case. In the first place
it may be held that some motives are \textit{intrinsically good} and
others \textit{intrinsically bad}; and though this is a view which is
inconsistent with the theory of our first two chapters, it is not a
view which we are at present concerned to dispute: for it is not at
all inconsistent with the principle which we are at present
considering---namely, that right and wrong always depend solely upon
consequences. If we held this view, we might still hold that a
\page{186} man may act wrongly from a good motive, and rightly from a
bad one, and that the motive would make no difference whatever to the
rightness or wrongness of the action. What it would make a difference
to is the goodness or badness of the whole state of affairs: for, if
we suppose the same action to be done in one case from a good motive
and in the other from a bad one, then, so far as the consequences of
the action are concerned, the goodness of the whole state of things
will be the same, while the presence of the good motive will mean the
presence of an \textit{additional} good in the one case which is
absent in the other. For this reason alone, therefore, we might
justify the view that motives are relevant to \textit{some} kinds of
moral judgments, though not to judgments of right and wrong.

And there is yet another reason for this view, and this a reason which
may be consistently held even by those who hold the theory of our
first two chapters. It may be held, namely, that good motives have a
\textit{general} tendency to produce right conduct, though they do not
\textit{always} do so, and bad motives to produce wrong conduct; and
this would be another \page{187} reason which would justify us in
regarding right actions done from a good motive differently from right
actions done from a bad one. For though, in the case supposed, the bad
motive would not \textit{actually} have led to wrong action, yet, if
it is true that motives of that kind do \textit{generally} lead to
wrong action, we should be right in passing this judgment upon it; and
judgments to the effect that a motive is of a kind which generally
leads to wrong action are undoubtedly moral judgments of a sort, and
an important sort, though they do not prove that every action done
from such a motive is wrong.

And finally motives seem also to be relevant to a third kind of moral
judgment of great importance---namely, judgments as to whether, and in
what degree, the agent \textit{deserves} moral praise or blame for
acting as he did. This question as to what is deserving of moral
praise or blame is, I think, often confused with the question as to
what is right or wrong. It is very natural, at first sight, to assume
that to call an action morally praiseworthy is the same thing as to
say that it is right, and to call it morally blameworthy the same
thing as to say that it is \page{187} wrong. But yet a very little
reflection suffices to show that the two things are certainly
distinct. When we say that an action \textit{deserves} praise or
blame, we imply that it is \textit{right} to praise or blame it; that
is to say, we are making a judgment \textit{not} about the rightness
of the original action, but about the rightness of the further action
which we should take, if we praised or blamed it. And these two
judgments are certainly not identical; nor is there any reason to
think that what is right \textit{always} also deserves to be praised,
and what is wrong \textit{always} also deserves to be blamed. Even,
therefore, if the motive \textit{is} relevant to the question whether
an action deserves praise or blame, it by no means follows that it is
\textit{also} relevant to the question whether it is right or wrong.
And there is some reason to think that the motive \textit{is} relevant
to judgments of the former kind: that we really \textit{ought}
sometimes to praise an action done from a bad motive less than if it
had been done from a good one, and to blame an action done from a good
motive less than if it had been done from a bad one. For one of the
considerations upon which the question whether it is right to blame an
action depends, \page{189} is that our blame may tend to prevent the
agent from doing similar wrong actions in future; and obviously, if
the agent only acted wrongly from a motive which is not likely to lead
him wrong in the future, there is less need to try to deter him by
blame than if he had acted from a motive which was likely to lead him
to act wrongly again. This is, I think, a very real reason why we
\textit{sometimes} ought to blame a man less when he does wrong from a
good motive. But I do not mean to say that the question whether a man
deserves moral praise or blame, or the degree to which he deserves it,
depends \textit{entirely} or \textit{always} upon his motive. I think
it certainly does not. My point is only that this \textit{question}
does \textit{sometimes} depend on the motive in some degree; whereas
the question whether his action was right or wrong \textit{never}
depends on it at all.

There are, therefore, at least three different kinds of moral
judgments, in making which it is at least plausible to hold that we
ought to take account of motives; and if all these judgments are
carefully distinguished from that particular kind which is solely
concerned with the question whether an action is right \page{190} or
wrong, there ceases, I think, to be any reason to suppose that this
last question ever depends upon the motive at all. At all events the
mere fact that motives are and ought to be taken account of in
\textit{some} moral judgments does not constitute such a reason. And
hence this fact cannot be urged as an objection to the view that right
and wrong depend solely on consequences.

But there remains one last objection to this view, which is, I am
inclined to think, the most serious of all. This is an objection which
will be urged by people who strongly maintain that right and wrong do
\textit{not} depend either upon the nature of the action or upon its
motive, and who will even go so far as to admit as self-evident the
hypothetical proposition that \textit{if} any being absolutely
\textit{knew} that one action would have better total consequences
than another, then it \textit{would} always be his duty to choose the
former rather than the latter. But what such people would point out is
that this hypothetical case is hardly ever, if ever, realised among us
men. We hardly ever, if ever, \textit{know for certain} which among
the \page{191} courses open to us \textit{will} produce the best
consequences. Some accident, which we could not possibly have
foreseen, may always falsify the most careful calculations, and make
an action, which we had every reason to think would have the best
results, \textit{actually} have worse ones than some alternative would
have had. Suppose, then, that a man has taken all possible care to
assure himself that a given course will be the best, and has adopted
it for that reason, but that owing to some subsequent event, which he
could not possibly have foreseen, it turns out \textit{not} to be the
best: are we for that reason to say that his action was wrong? It may
seem outrageous to say so; and yet this is what we must say, if we are
to hold that right and wrong depend upon the \textit{actual}
consequences. Or suppose that a man has deliberately chosen a course,
which he has every reason to suppose will \textit{not} produce the
best consequences, but that some unforeseen accident defeats his
purpose and makes it actually turn out to be the best: are we to say
that such a man, because of this unforeseen accident, has acted
rightly? This also \page{192} may seem an outrageous thing to say;
and yet we must say it, if we are to hold that right and wrong depend
upon the \textit{actual} consequences. For these reasons many people
are strongly inclined to hold that they do \textit{not} depend upon
the \textit{actual} consequences, but only upon those which were
antecedently \textit{probable}, or which the agent had \textit{reason}
to expect, or which it was \textit{possible} for him to
\textit{foresee}. They are inclined to say that an action is
\textit{always} right, whatever its \textit{actual} consequences may
be, provided the agent had reason to expect that they would be the
best possible; and \textit{always} wrong, if he had reason to expect
that they would not.

This, I think, is the most serious objection to the view that right
and wrong depend upon the \textit{actual} consequences. But yet I am
inclined to think that even this objection can be got over by
reference to the distinction between what is right or wrong, on the
one hand, and what is morally praiseworthy or blameworthy on the
other. What we should naturally say of a man whose action turns out
badly owing to some unforseen accident when he had every reason to
expect that it \page{193} would turn out well, is not that his action
was right, but rather that \textit{he is not to blame}. And it may be
fully admitted that in such a case he really \textit{ought} not to be
blamed; since blame cannot possibly serve any good purpose, and would
be likely to do harm. But, even if we admit that he was not to blame,
is that any reason for asserting also that he acted rightly? I cannot
see that it is; and therefore I am inclined to think that in all such
cases the man really did act \textit{wrongly}, although he is not to
blame, and although, perhaps, he even deserves praise for acting as he
did.

But the same difficulty may be put in another form, in which there may
seem an even stronger case against the view that right and wrong
depend on the \textit{actual} consequences. Instead of considering
what judgment we ought to pass on an action \textit{after} it has been
done, and when many of its results are already known, let us consider
what judgment we ought to pass on it \textit{beforehand}, and when the
question is which among several courses still open to a man he
\textit{ought} to choose. It is admitted that he cannot \page{194}
\textit{know for certain} beforehand which of them will actually have
the best results; but let us suppose that he has every reason to think
that one of them will produce decidedly better results than any of the
others---that all probability is in favour of this view. Can we not
say, in such a case, that he absolutely \textit{ought} to choose that
one? that he will be acting very \textit{wrongly} if he chooses any
other? We certainly \textit{should} actually say so; and many people
may be inclined to think that we should be right in saying so, no
matter what the results may subsequently prove to be. There does seem
to be a certain paradox in maintaining the opposite: in maintaining
that, in such a case, it can possibly be true that he \textit{ought}
to choose a course, which he has every reason to think will
\textit{not} be the best. But yet I am inclined to think that even
this difficulty is not fatal to our view. It may be admitted that we
should say, and should be justified in saying, that he absolutely
\textit{ought} to choose the course, which he has reason to think will
be the best. But we may be justified in saying many things, which we
do not \page{195} know to be true, and which in fact are not so,
provided that there is a strong probability that they are. And so in
this case I do not see why we should not hold, that though we should
be justified in saying that he \textit{ought} to choose one course,
yet it may not be really true that he ought. What certainly will be
true is that he will deserve the strongest moral blame if he does not
choose the course in question, even though it may be wrong. And we are
thus committed to the paradox that a man may really deserve the
strongest moral condemnation for choosing an action, which
\textit{actually} is right. But I do not see why we should not accept
this paradox.

I conclude, then, that there is no conclusive reason against the view
that our theory is right, so far as it maintains that the question
whether an action is right or wrong \textit{always} depends on its
\textit{actual} consequences. There seems no sufficient reason for
holding either that it depends on the intrinsic nature of the action,
or that it depends upon the motive, or even that it depends on the
\textit{probable} consequences.

