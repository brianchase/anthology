
\author{Emma Goldman}
\authdate{1869--1940}
\textdate{1906}
\chapter{The Tragedy of Woman's Emancipation}
\source{goldman1917.11}

% First published in Mother Earth in March 1906.

\page{219}\noindent I begin with an admission: Regardless of all
political and economic theories, treating of the fundamental
differences between various groups within the human race, regardless
of class and race distinctions, regardless of all artificial boundary
lines between woman's rights and man's rights, I hold that there is a
point where these differentiations may meet and grow into one perfect
whole.

With this I do not mean to propose a peace treaty. The general social
antagonism which has taken hold of our entire public life today,
brought about through the force of opposing and contradictory
interests, will crumble to pieces when the reorganization of our
social life, based upon the principles of economic justice, shall have
become a reality.

Peace or harmony between the sexes and individuals does not
necessarily depend on a superficial equalization of human beings; nor
does it call for the elimination of individual traits and
peculiarities. The problem that confronts us today, and which the
nearest future is to solve, is how to be one's self and yet \page{220}
in oneness with others, to feel deeply with all human beings and still
retain one's own characteristic qualities. This seems to me to be the
basis upon which the mass and the individual, the true democrat and
the true individuality, man and woman, can meet without antagonism and
opposition. The motto should not be: Forgive one another; rather,
Understand one another. The oft-quoted sentence of Madame de
Sta\"{e}l:  ``To understand everything means to forgive everything,''
has never particularly appealed to me; it has the odor of the
confessional; to forgive one's fellow-being conveys the idea of
pharisaical superiority. To understand one's fellow-being suffices.
The admission partly represents the fundamental aspect of my views on
the emancipation of woman and its effect upon the entire sex.

Emancipation should make it possible for woman to be human in the
truest sense. Everything within her that craves assertion and activity
should reach its fullest expression; all artificial barriers should be
broken, and the road towards greater freedom cleared of every trace of
centuries of submission and slavery.

This was the original aim of the movement for woman's emancipation.
But the results so far achieved have isolated woman and have robbed
her of the fountain springs of that happiness which is so essential to
her. Merely external emancipation has made of the modern woman an
artificial being, who reminds one of the products of French
arboriculture with its arabesque trees and shrubs, pyramids, wheels,
and wreaths; anything, except the forms which would be reached by the
expression of her own inner quali-\page{221}ties. Such artificially
grown plants of the female sex are to be found in large numbers,
especially in the so-called intellectual sphere of our life.

Liberty and equality for woman! What hopes and aspirations these words
awakened when they were first uttered by some of the noblest and
bravest souls of those days. The sun in all his light and glory was to
rise upon a new world; in this world woman was to be free to direct
her own destiny---an aim certainly worthy of the great enthusiasm,
courage, perseverance, and ceaseless effort of the tremendous host of
pioneer men and women, who staked everything against a world of
prejudice and ignorance.

% Without '\linebreak', an overfull hbox warning:

My hopes also move towards that goal, but I hold that the emancipation
of \linebreak[4] woman, as interpreted and practically applied today,
has failed to reach that great end. Now, woman is confronted with the
necessity of emancipating herself from emancipation, if she really
desires to be free. This may sound paradoxical, but is, nevertheless,
only too true.

What has she achieved through her emancipation? Equal suffrage in a
few States. Has that purified our political life, as many well-meaning
advocates predicted? Certainly not. Incidentally, it is really time
that persons with plain, sound judgment should cease to talk about
corruption in politics in a boarding-school tone. Corruption of
politics has nothing to do with the morals, or the laxity of morals,
of various political personalities. Its cause is altogether a material
one. Politics is the reflex of the business and industrial world, the
mottos of which are:  ``To take is more blessed than to give'';  ``buy
cheap and sell \page{222} dear'';  ``one soiled hand washes the
other.'' There is no hope even that woman, with her right to vote,
will ever purify politics.

Emancipation has brought woman economic equality with man; that is,
she can choose her own profession and trade; but as her past and
present physical training has not equipped her with the necessary
strength to compete with man, she is often compelled to exhaust all
her energy, use up her vitality, and strain every nerve in order to
reach the market value. Very few ever succeed, for it is a fact that
women teachers, doctors, lawyers, architects, and engineers are
neither met with the same confidence as their male colleagues, nor
receive equal remuneration. And those that do reach that enticing
equality, generally do so at the expense of their physical and
psychical well-being. As to the great mass of working girls and women,
how much independence is gained if the narrowness and lack of freedom
of the home is exchanged for the narrowness and lack of freedom of the
factory, sweat-shop, department store, or office? In addition is the
burden which is laid on many women of looking after a  ``home, sweet
home''---cold, dreary, disorderly, uninviting---after a day's hard
work. Glorious independence! No wonder that hundreds of girls are so
willing to accept the first offer of marriage, sick and tired of their
``independence'' behind the counter, at the sewing or typewriting
machine. They are just as ready to marry as girls of the middle class,
who long to throw off the yoke of parental supremacy. A so-called
independence which leads only to earning the merest subsistence is not
so \page{223} enticing, not so ideal, that one could expect woman to
sacrifice everything for it. Our highly praised independence is, after
all, but a slow process of dulling and stifling woman's nature, her
love instinct, and her mother instinct.

Nevertheless, the position of the working girl is far more natural and
human than that of her seemingly more fortunate sister in the more
cultured professional walks of life---teachers, physicians, lawyers,
engineers, etc., who have to make a dignified, proper appearance,
while the inner life is growing empty and dead.

The narrowness of the existing conception of woman's independence and
emancipation; the dread of love for a man who is not her social equal;
the fear that love will rob her of her freedom and independence; the
horror that love or the joy of motherhood will only hinder her in the
full exercise of her profession---all these together make of the
emancipated modern woman a compulsory vestal, before whom life, with
its great clarifying sorrows and its deep, entrancing joys, rolls on
without touching or gripping her soul.

Emancipation, as understood by the majority of its adherents and
exponents, is of too narrow a scope to permit the boundless love and
ecstasy contained in the deep emotion of the true woman, sweetheart,
mother, in freedom.

The tragedy of the self-supporting or economically free woman does not
lie in too many, but in too few experiences. True, she surpasses her
sister of past generations in knowledge of the world and human
\page{224} nature; it is just because of this that she feels deeply
the lack of life's essence, which alone can enrich the human soul, and
without which the majority of women have become mere professional
automatons.

That such a state of affairs was bound to come was foreseen by those
who realized that, in the domain of ethics, there still remained many
decaying ruins of the time of the undisputed superiority of man; ruins
that are still considered useful. And, what is more important, a
goodly number of the emancipated are unable to get along without them.
Every movement that aims at the destruction of existing institutions
and the replacement thereof with something more advanced, more
perfect, has followers who in theory stand for the most radical ideas,
but who, nevertheless, in their every-day practice, are like the
average Philistine, feigning respectability and clamoring for the good
opinion of their opponents. There are, for example, Socialists, and
even Anarchists, who stand for the idea that property is robbery, yet
who will grow indignant if anyone owe them the value of a half-dozen
pins.

The same Philistine can be found in the movement for woman's
emancipation. Yellow journalists and milk-and-water litterateurs have
painted pictures of the e\-man\-ci\-pat\-ed woman that make the hair
of the good citizen and his dull companion stand up on end. Every
member of the woman's rights movement was pictured as a George Sand in
her absolute disregard of morality. Nothing was sacred to her. She had
no respect for the ideal relation between man and woman. In short,
emancipation stood only for a reck-\page{225}less life of lust and
sin; regardless of society, religion, and morality. The exponents of
woman's rights were highly indignant at such misrepresentation, and,
lacking humor, they exerted all their energy to prove that they were
not at all as bad as they were painted, but the very reverse. Of
course, as long as woman was the slave of man, she could not be good
and pure, but now that she was free and independent she would prove
how good she could be and that her influence would have a purifying
effect on all institutions in society. True, the movement for woman's
rights has broken many old fetters, but it has also forged new ones.
The great movement of \textit{true} emancipation has not met with a
great race of women who could look liberty in the face. Their narrow,
Puritanical vision banished man, as a disturber and doubtful
character, out of their eniotional life. Man was not to be tolerated
at any price, except perhaps as the father of a child, since a child
could not very well come to life without a father. Fortunately, the
most rigid Puritans never will be strong enough to kill the innate
craving for motherhood. But woman's freedom is closely allied with
man's freedom, and many of my so-called emancipated sisters seem to
overlook the fact that a child born in freedom needs the love and
devotion of each human being about him, man as well as woman.
Unfortunately, it is this narrow conception of human relations that
has brought about a great tragedy in the lives of the modern man and
woman.

About fifteen years ago appeared a work from the pen of the brilliant
Norwegian Laura Marholm, called \textit{Woman, a Character Study}. She
was one of \page{226} the first to call attention to the emptiness and
narrowness of the existing conception of woman's emancipation, and its
tragic effect upon the inner life of woman. In her work Laura Marholm
speaks of the fate of several gifted women of international fame: the
genius Eleonora Duse; the great mathematician and writer Sonya
Kovalevskaia; the artist and poet-nature Marie Bashkirtzeff, who died
so young. Through each description of the lives of these women of such
extraordinary mentality runs a marked trail of unsatisfied craving for
a full, rounded, complete, and beautiful life, and the unrest and
loneliness resulting from the lack of it. Through these masterly
psychological sketches one cannot help but see that the higher the
mental development of woman, the less possible it is for her to meet a
congenial mate who will see in her, not only sex, but also the human
being, the friend, the comrade and strong individuality, who cannot
and ought not lose a single trait of her character.

The average man with his self-sufficiency, his ridiculously superior
airs of patronage towards the female sex, is an impossibility for
woman as depicted in the \textit{Character Study} by Laura Marholm.
Equally impossible for her is the man who can see in her nothing more
than her mentality and her genius, and who fails to awaken her woman
nature.

A rich intellect and a fine soul are usually considered necessary
attributes of a deep and beautiful personality. In the case of the
modern woman, these attributes serve as a hindrance to the complete
assertion of her being. For over a hundred years the old \page{227}
form of marriage, based on the Bible,  ``till death doth part,'' has
been denounced as an institution that stands for the sovereignty of
the man over the woman, of her complete submission to his whims and
commands, and absolute dependence on his name and support. Time and
again it has been conclusively proved that the old matrimonial
relation restricted woman to the function of man's servant and the
bearer of his children. And yet we find many emancipated women who
prefer marriage, with all its deficiencies, to the narrowness of an
unmarried life: narrow and unendurable because of the chains of moral
and social prejudice that cramp and bind her nature.

The explanation of such inconsistency on the part of many advanced
women is to be found in the fact that they never truly understood the
meaning of emancipation. They thought that all that was needed was
independence from external tyrannies; the internal tyrants, far more
harmful to life and growth---ethical and social conventions---were
left to take care of themselves; and they have taken care of
themselves. They seem to get along as beautifully in the heads and
hearts of the most active exponents of woman's emancipation, as in the
heads and hearts of our grandmothers.

These internal tyrants, whether they be in the form of public opinion
or what will mother say, or brother, father, aunt, or relative of any
sort; what will Mrs. Grundy, Mr. Comstock, the employer, the Board of
Education say? All these busybodies, moral detectives, jailers of the
human spirit, what \page{228} will they say? Until woman has learned
to defy them all, to stand firmly on her own ground and to insist upon
her own unrestricted freedom, to listen to the voice of her nature,
whether it call for life's greatest treasure, love for a man, or her
most glorious privilege, the right to give birth to a child, she
cannot call herself emancipated. How many emancipated women are brave
enough to acknowledge that the voice of love is calling, wildly
beating against their breasts, demanding to be heard, to be satisfied.

The French writer Jean Reibrach, in one of his novels, \textit{New
Beauty}, attempts to picture the ideal, beautiful, emancipated woman.
This ideal is embodied in a young girl, a physician. She talks very
cleverly and wisely of how to feed infants; she is kind, and
administers medicines free to poor mothers. She converses with a young
man of her acquaintance about the sanitary conditions of the future,
and how various bacilli and germs shall be exterminated by the use of
stone walls and floors, and by the doing away with rugs and hangings.
She is, of course, very plainly and practically dressed, mostly in
black. The young man, who, at their first meeting, was overawed by the
wisdom of his emancipated friend, gradually learns to understand her,
and recognizes one fine day that he loves her. They are young, and she
is kind and beautiful, and though always in rigid attire, her
appearance is softened by a spotlessly clean white collar and cuffs.
One would expect that he would tell her of his love, but he is not one
to commit romantic absurdities. Poetry and the enthusiasm of love
cover their blushing faces before the pure beauty \page{229} of the
lady. He silences the voice of his nature, and remains correct. She,
too, is always exact, always rational, always well behaved. I fear if
they had formed a union, the young man would have risked freezing to
death. I must confess that I can see nothing beautiful in this new
beauty, who is as cold as the stone walls and floors she dreams of.
Rather would I have the love songs of romantic ages, rather Don Juan
and Madame Venus, rather an elopement by ladder and rope on a
moonlight night, followed by the father's curse, mother's moans, and
the moral comments of neighbors, than correctness and propriety
measured by yardsticks. If love does not know how to give and take
without restrictions, it is not love, but a transaction that never
fails to lay stress on a plus and a minus.

The greatest shortcoming of the emancipation of the present day lies
in its artificial stiffness and its narrow respectabilities, which
produce an emptiness in wo\-man's soul that will not let her drink
from the fountain of life. I once remarked that there seemed to be a
deeper relationship between the old-fashioned mother and hostess, ever
on the alert for the happiness of her little ones and the comfort of
those she loved, and the truly new woman, than between the latter and
her average emancipated sister. The disciples of emancipation pure and
simple declared me a heathen, fit only for the stake. Their blind zeal
did not let them see that my comparison between the old and the new
was merely to prove that a goodly number of our grandmothers had more
blood in their veins, far more humor and wit, and certainly \page{230}
a greater amount of naturalness, kind-heartedness, and simplicity,
than the majority of our emancipated professional women who fill the
colleges, halls of learning, and various offices. This does not mean a
wish to return to the past, nor does it condemn woman to her old
sphere, the kitchen and the nursery.

Salvation lies in an energetic march onward towards a brighter and
clearer future. We are in need of unhampered growth out of old
traditions and habits. The movement for woman's emancipation has so
far made but the first step in that direction. It is to be hoped that
it will gather strength to make another. The right to vote, or equal
civil rights, may be good demands, but true emancipation begins
neither at the polls nor in courts. It begins in woman's soul. History
tells us that every oppressed class gained true liberation from its
masters through its own efforts. It is necessary that woman learn that
lesson, that she realize that her freedom will reach as far as her
power to achieve her freedom reaches. It is, therefore, far more
important for her to begin with her inner regeneration, to cut loose
from the weight of prejudices, traditions, and customs. The demand for
equal rights in every vocation of life is just and fair; but, after
all, the most vital right is the right to love and be loved. Indeed,
if partial emancipation is to become a complete and true emancipation
of woman, it will have to do away with the ridiculous notion that to
be loved, to be sweetheart and mother, is synonymous with being slave
or subordinate. It will have to do away with the absurd \page{231}
notion of the dualism of the sexes, or that man and woman represent
two antagonistic worlds.

Pettiness separates; breadth unites. Let us be broad and big. Let us
not overlook vital things because of the bulk of trifles confronting
us. A true conception of the relation of the sexes will not admit of
conqueror and conquered; it knows of but one great thing: to give of
one's self boundlessly, in order to find one's self richer, deeper,
better. That alone can fill the emptiness, and transform the tragedy
of woman's emancipation into joy, limitless joy.

