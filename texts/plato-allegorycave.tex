
\author{Plato}
\authdate{ca. 427--348/347 \BCE}
%\textdate{ca. 380 \BCE}
\textdate{ca. early fourth century \BCE}
\addon{Republic, Book 7, 514a--19b}
\chapter{Socrates' Allegory of the Cave}
%\source*[bk 7, 514a--19b]{plato1908.2}
\source*{plato1908.2}

% No page numbers in cited text, only Stephanus numbers. However,
% there's no precise way to insert the Stephanus numbers, since the
% line breaks here are different.

And now, I said, let me show in a figure how far our nature is
enlightened or un\-en\-light\-ened:---Behold! human beings living in
an underground den, which has a mouth open towards the light and
reaching all along the den; here they have been from their childhood,
and have their legs and necks chained so that they cannot move, and
can only see before them, being prevented by the chains from turning
round their heads. Above and behind them a fire is blazing at a
distance, and between the fire and the prisoners there is a raised
way; and you will see, if you look, a low wall built along the way,
like the screen which marionette players have in front of them, over
which they show the puppets.

I see.

And do you see, I said, men passing along the wall carrying all sorts
of vessels, and statues and figures of animals made of wood and stone
and various materials, which appear over the wall? Some of them are
talking, others silent.

You have shown me a strange image, and they are strange prisoners.

Like ourselves, I replied; and they see only their own shadows, or the
shadows of one another, which the fire throws on the opposite wall of
the cave?

True, he said; how could they see anything but the shadows if they
were never allowed to move their heads?

And of the objects which are being carried in like manner they would
only see the shadows?

Yes, he said.

And if they were able to converse with one another, would they not
suppose that they were naming what was actually before them?

Very true.

And suppose further that the prison had an echo which came from the
other side, would they not be sure to fancy when one of the passers-by
spoke that the voice which they heard came from the passing shadow?

No question, he replied.

To them, I said, the truth would be literally nothing but the shadows
of the images.

That is certain.

And now look again, and see what will naturally follow if the
prisoners are released and disabused of their error. At first, when
any of them is liberated and compelled suddenly to stand up and turn
his neck round and walk and look towards the light, he will suffer
sharp pains; the glare will distress him, and he will be unable to see
the realities of which in his former state he had seen the shadows;
and then conceive some one saying to him, that what he saw before was
an illusion, but that now, when he is approaching nearer to being and
his eye is turned towards more real existence, he has a clearer
vi\-sion,---what will be his reply? And you may further imagine that
his instructor is pointing to the objects as they pass and requiring
him to name them,---will he not be perplexed? Will he not fancy that
the shadows which he formerly saw are truer than the objects which are
now shown to him?

Far truer.

And if he is compelled to look straight at the light, will he not have
a pain in his eyes which will make him turn away to take refuge in the
objects of vision which he can see, and which he will conceive to be
in reality clearer than the things which are now being shown to him?

True, he said.

And suppose once more, that he is reluctantly dragged up a steep and
rugged ascent, and held fast until he is forced into the presence of
the sun himself, is he not likely to be pained and irritated? When he
approaches the light his eyes will be dazzled, and he will not be able
to see anything at all of what are now called realities.

Not all in a moment, he said.

He will require to grow accustomed to the sight of the upper world.
And first he will see the shadows best, next the reflections of men
and other objects in the water, and then the objects themselves; then
he will gaze upon the light of the moon and the stars and the spangled
heaven; and he will see the sky and the stars by night better than the
sun or the light of the sun by day?

Certainly.

Last of all he will be able to see the sun, and not mere reflections
of him in the water, but he will see him in his own proper place, and
not in another; and he will contemplate him as he is.

Certainly.

He will then proceed to argue that this is he who gives the season and
the years, and is the guardian of all that is in the visible world,
and in a certain way the cause of all things which he and his fellows
have been accustomed to behold?

Clearly, he said, he would first see the sun and then reason about
him.

And when he remembered his old habitation, and the wisdom of the den
and his fellow-prisoners, do you not suppose that he would felicitate
himself on the change, and pity him?

Certainly, he would.

And if they were in the habit of conferring honours among themselves
on those who were quickest to observe the passing shadows and to
remark which of them went before, and which followed after, and which
were together; and who were therefore best able to draw conclusions as
to the future, do you think that he would care for such honours and
glories, or envy the possessors of them? Would he not say with Homer,

\begin{quote} Better to be the poor servant of a poor master,
\end{quote}

\noindent and to endure anything, rather than think as they do and
live after their manner?

Yes, he said, I think that he would rather suffer anything than
entertain these false notions and live in this miserable manner.

Imagine once more, I said, such a one coming suddenly out of the sun
to be replaced in his old situation; would he not be certain to have
his eyes full of darkness?

To be sure, he said.

And if there were a contest, and he had to compete in measuring the
shadows with the prisoners who had never moved out of the den, while
his sight was still weak, and before his eyes had become steady (and
the time which would be needed to acquire this new habit of sight
might be very considerable), would he not be ridiculous? Men would say
of him that up he went and down he came without his eyes; and that it
was better not even to think of ascending; and if anyone tried to
loose another and lead him up to the light, let them only catch the
offender, and they would put him to death.

No question, he said.

This entire allegory, I said, you may now append, dear Glaucon, to the
previous argument; the prison-house is the world of sight, the light
of the fire is the sun, and you will not misapprehend me if you
interpret the journey upwards to be the ascent of the soul into the
intellectual world according to my poor belief, which, at your desire,
I have ex\-pressed---whether rightly or wrongly God knows. But,
whether true or false, my opinion is that in the world of knowledge
the idea of good appears last of all, and is seen only with an effort;
and, when seen, is also inferred to be the universal author of all
things beautiful and right, parent of light and of the lord of light
in this visible world, and the immediate source of reason and truth in
the intellectual; and that this is the power upon which he who would
act rationally either in public or private life must have his eye
fixed.

I agree, he said, as far as I am able to understand you.

Moreover, I said, you must not wonder that those who attain to this
beatific vision are unwilling to descend to human affairs; for their
souls are ever hastening into the upper world where they desire to
dwell; which desire of theirs is very natural, if our allegory may be
trusted.

Yes, very natural.

And is there anything surprising in one who passes from divine
contemplations to the evil state of man, misbehaving himself in a
ridiculous manner; if, while his eyes are blinking and before he has
become accustomed to the surrounding darkness, he is compelled to
fight in courts of law, or in other places, about the images or the
shadows of images of justice, and is endeavouring to meet the
conceptions of those who have never yet seen absolute justice?

Anything but surprising, he replied.

Any one who has common sense will remember that the bewilderments of
the eyes are of two kinds, and arise from two causes, either from
coming out of the light or from going into the light, which is true of
the mind's eye, quite as much as of the bodily eye; and he who
remembers this when he sees anyone whose vision is perplexed and weak,
will not be too ready to laugh; he will first ask whether that soul of
man has come out of the brighter life, and is unable to see because
unaccustomed to the dark, or having turned from darkness to the day is
dazzled by excess of light. And he will count the one happy in his
condition and state of being, and he will pity the other; or, if he
have a mind to laugh at the soul which comes from below into the
light, there will be more reason in this than in the laugh which
greets him who returns from above out of the light into the den.

That, he said, is a very just distinction.

But then, if I am right, certain professors of education must be wrong
when they say that they can put a knowledge into the soul which was
not there before, like sight into blind eyes.

They undoubtedly say this, he replied.

Whereas, our argument shows that the power and capacity of learning
exists in the soul already; and that just as the eye was unable to
turn from darkness to light without the whole body, so too the
instrument of knowledge can only by the movement of the whole soul be
turned from the world of becoming into that of being, and learn by
degrees to endure the sight of being, and of the brightest and best of
being, or, in other words, of the good.

Very true.

And must there not be some art which will effect conversion in the
easiest and quickest manner; not implanting the faculty of sight, for
that exists already, but has been turned in the wrong direction, and
is looking away from the truth?

Yes, he said, such an art may be presumed.

And whereas the other so-called virtues of the soul seem to be akin to
bodily qualities, for even when they are not originally innate they
can be implanted later by habit and exercise, the virtue of wisdom
more than anything else contains a divine element which always
remains, and by this conversion is rendered useful and profitable; or,
on the other hand, hurtful and useless. Did you never observe the
narrow intelligence flashing from the keen eye of a clever rogue---how
eager he is, how clearly his paltry soul sees the way to his end; he
is the reverse of blind, but his keen eyesight is forced into the
service of evil, and he is mischievous in proportion to his
cleverness?

Very true, he said.

But what if there had been a circumcision of such natures in the days
of their youth; and they had been severed from those sensual
pleasures, such as eating and drinking, which, like leaden weights,
were attached to them at their birth, and which drag them down and
turn the vision of their souls upon the things that are be\-low---if,
I say, they had been released from these impediments and turned in the
opposite direction, the very same faculty in them would have seen the
truth as keenly as they see what their eyes are turned to now.

Very likely.

