
\author{G. E. Moore}
\authdate{1873--1958}
\textdate{1903}
\addon{Principia Ethica, Chapter 1, excerpt}
\chapter{The Naturalistic Fallacy}
\source{moore1922}

% FIX: Proofread. Look especially for missing or misplaced punctuation
% and for missing italics.

% p. 39 in PDF

\page{5}5. But our question `What is good?' may have still another
meaning. We may, in the third place, mean to ask, not what thing or
things are good, but how `good' is to be defined. This is an enquiry
which belongs only to Ethics, not to Casuistry; and this is the
enquiry which will occupy us first.

It is an enquiry to which most special attention should be directed;
since this question, how `good' is to be defined, is the most
fundamental in all Ethics. That which is meant by `good' is, in fact,
except its converse `bad,' the \textit{only} simple object of thought
which is peculiar to Ethics. Its definition is, therefore, the most
essential point in the definition of Ethics; and moreover a mistake
with regard to it entails a far larger number of erroneous ethical
judgments than any other. Unless this first question be fully
understood, and its true answer clearly recognised, the rest of Ethics
is as good as useless from the point of view of systematic knowledge.
True ethical judgments, of the two kinds last dealt with, may indeed
be made by those who do not know the answer to this question as well
as by those who do; and it goes without saying that the two classes of
people may lead equally good lives. But it is extremely unlikely that
the \textit{most general} ethical judgments will be equally valid, in
the absence of a true answer to this question: I shall presently try
to shew that the gravest errors have been largely due to \page{6}
beliefs in a false answer. And, in any case, it is impossible that,
till the answer to this question be known, any one should know
\textit{what is the evidence} for any ethical judgment whatsoever. But
the main object of Ethics, as a systematic science, is to give correct
\textit{reasons} for thinking that this or that is good; and, unless
this question be answered, such reasons cannot be given. Even,
therefore, apart from the fact that a false answer leads to false
conclusions, the present enquiry is a most necessary and important
part of the science of Ethics.

6. What, then, is good? How is good to be defined? Now, it may be
thought that this is a verbal question. A definition does indeed often
mean the expressing of one word's meaning in other words. But this is
not the sort of definition I am asking for. Such a definition can
never be of ultimate importance in any study except lexicography. If I
wanted that kind of definition I should have to consider in the first
place how people generally used the word `good'; but my business is
not with its proper usage, as established by custom. I should, indeed,
be foolish, if I tried to use it for something which it did not
usually denote: if, for instance, I were to announce that, whenever I
used the word `good,' I must be understood to be thinking of that
object which is usually denoted by the word `table.' I shall,
therefore, use the word in the sense in which I think it is ordinarily
used; but at the same time I am not anxious to discuss whether I am
right in thinking that it is so used. My business is solely with that
object or idea, which I hold, rightly or wrongly, that the word is
generally used to stand for. What I want to discover is the nature of
that object or idea, and about this I am extremely anxious to arrive
at an agreement.

But, if we understand the question in this sense, my answer to it may
seem a very disappointing one. If I am asked `What is good?' my answer
is that good is good, and that is the end of the matter. Or if I am
asked `How is good to be defined?' my answer is that it cannot be
defined, and that is all I have to say about it. But disappointing as
these answers may appear, they are of the very last importance. To
readers who are familiar with philosophic terminology, I can express
their im-\page{7}portance by saying that they amount to this: That
propositions about the good are all of them synthetic and never
analytic; and that is plainly no trivial matter. And the same thing
may be expressed more popularly, by saying that, if I am right, then
nobody can foist upon us such an axiom as that `Pleasure is the only
good' or that `The good is the desired' on the pretence that this is
`the very meaning of the word.'

7. Let us, then, consider this position. My point is that `good' is a
simple notion, just as `yellow' is a simple notion; that, just as you
cannot, by any manner of means, explain to any one who does not
already know it, what yellow is, so you cannot explain what good is.
Definitions of the kind that I was asking for, definitions which
describe the real nature of the object or notion denoted by a word,
and which do not merely tell us what the word is used to mean, are
only possible when the object or notion in question is something
complex. You can give a definition of a horse, because a horse has
many different properties and qualities, all of which you can
enumerate. But when you have enumerated them all, when you have
reduced a horse to his simplest terms, then you can no longer define
those terms. They are simply something which you think of or perceive,
and to any one who cannot think of or perceive them, you can never, by
any definition, make their nature known. It may perhaps be objected to
this that we are able to describe to others, objects which they have
never seen or thought of. We can, for instance, make a man understand
what a chimaera is, although he has never heard of one or seen one.
You can tell him that it is an animal with a lioness's head and body,
with a goat's head growing from the middle of its back, and with a
snake in place of a tail. But here the object which you are describing
is a complex object; it is entirely composed of parts, with which we
are all perfectly familiar---a snake, a goat, a lioness; and we know,
too, the manner in which those parts are to be put together, because
we know what is meant by the middle of a lioness's back, and where her
tail is wont to grow. And so it is with all objects, not previously
known, which we are able to define: they are all complex; all composed
of parts, which may themselves, in the \page{8} first instance, be
capable of similar definition, but which must in the end be reducible
to simplest parts, which can no longer be defined. But yellow and
good, we say, are not complex: they are notions of that simple kind,
out of which definitions are composed and with which the power of
further defining ceases.

8. When we say, as Webster says, `The definition of horse is ``A
hoofed quadruped of the genus Equus,''' we may, in fact, mean three
different things. (1) We may mean merely: `When I say ``horse,'' you
are to understand that I am talking about a hoofed quadruped of the
genus Equus.' This might be called the arbitrary verbal definition:
and I do not mean that good is indefinable in that sense. (2) We may
mean, as Webster ought to mean: `When most English people say
``horse,'' they mean a hoofed quadruped of the genus Equus.' This may
be called the verbal definition proper, and I do not say that good is
indefinable in this sense either; for it is certainly possible to
discover how people use a word: otherwise, we could never have known
that `good' may be translated by `gut' in German and by `bon' in
French. But (3) we may, when we define horse, mean something much more
important. We may mean that a certain object, which we all of us know,
is composed in a certain manner: that it has four legs, a head, a
heart, a liver, etc., etc., all of them arranged in definite relations
to one another. It is in this sense that I deny good to be definable.
I say that it is not composed of any parts, which we can substitute
for it in our minds when we are thinking of it. We might think just as
clearly and correctly about a horse, if we thought of all its parts
and their arrangement instead of thinking of the whole: we could, I
say, think how a horse differed from a donkey just as well, just as
truly, in this way, as now we do, only not so easily; but there is
nothing whatsoever which we could so substitute for good; and that is
what I mean, when I say that good is indefinable.

9. But I am afraid I have still not removed the chief difficulty which
may prevent acceptance of the proposition that good is indefinable. I
do not mean to say that \textit{the} good, that which is good, is thus
indefinable; if I did think so, I should not \page{9} be writing on
Ethics, for my main object is to help towards discovering that
definition. It is just because I think there will be less risk of
error in our search for a definition of `the good,' that I am now
insisting that \textit{good} is indefinable. I must try to explain the
difference between these two. I suppose it may be granted that `good'
is an adjective. Well `the good,' `that which is good,' must therefore
be the substantive to which the adjective `good' will apply: it must
be the whole of that to which the adjective will apply, and the
adjective must \textit{always} truly apply to it. But if it is that to
which the adjective will apply, it must be something different from
that adjective itself; and the whole of that something different,
whatever it is, will be our definition of \textit{the} good. Now it
may be that this something will have other adjectives, beside `good,'
that will apply to it. It may be full of pleasure, for example; it may
be intelligent: and if these two adjectives are really part of its
definition, then it will certainly be true, that pleasure and
intelligence are good. And many people appear to think that, if we say
`Pleasure and intelligence are good,' or if we say `Only pleasure and
intelligence are good,' we are defining `good.' Well, I cannot deny
that propositions of this nature may sometimes be called definitions;
I do not know well enough how the word is generally used to decide
upon this point. I only wish it to be understood that that is not what
I mean when I say there is no possible definition of good, and that I
shall not mean this if I use the word again. I do most fully believe
that some true proposition of the form `Intelligence is good and
intelligence alone is good' can be found; if none could be found, our
definition of \textit{the} good would be impossible. As it is, I
believe \textit{the} good to be definable; and yet I still say that
good itself is indefinable.

10. `Good,' then, if we mean by it that quality which we assert to
belong to a thing, when we say that the thing is good, is incapable of
any definition, in the most important sense of that word. The most
important sense of `definition' is that in which a definition states
what are the parts which invariably compose a certain whole; and in
this sense `good' has no definition because it is simple and has no
parts. It is one of \page{10} those innumerable objects of thought
which are themselves incapable of definition, because they are the
ultimate terms by reference to which whatever \textit{is} capable of
definition must be defined. That there must be an indefinite number of
such terms is obvious, on reflection; since we cannot define anything
except by an analysis, which, when carried as far as it will go,
refers us to something, which is simply different from anything else,
and which by that ultimate difference explains the peculiarity of the
whole which we are defining: for every whole contains some parts which
are common to other wholes also. There is, therefore, no intrinsic
difficulty in the contention that `good' denotes a simple and
indefinable quality. There are many other instances of such qualities.

Consider yellow, for example. We may try to define it, by describing
its physical equivalent; we may state what kind of light-vibrations
must stimulate the normal eye, in order that we may perceive it. But a
moment's reflection is sufficient to shew that those light-vibrations
are not themselves what we mean by yellow. \textit{They} are not what
we perceive. Indeed we should never have been able to discover their
existence, unless we had first been struck by the patent difference of
quality between the different colours. The most we can be entitled to
say of those vibrations is that they are what corresponds in space to
the yellow which we actually perceive.

Yet a mistake of this simple kind has commonly been made about `good.'
It may be true that all things which are good are \textit{also}
something else, just as it is true that all things which are yellow
produce a certain kind of vibration in the light. And it is a fact,
that Ethics aims at discovering what are those other properties
belonging to all things which are good. But far too many philosophers
have thought that when they named those other properties they were
actually defining good; that these properties, in fact, were simply
not `other,' but absolutely and entirely the same with goodness. This
view I propose to call the `naturalistic fallacy' and of it I shall
now endeavour to dispose.

11. Let us consider what it is such philosophers say. And first it is
to be noticed that they do not agree among themselves. \page{11} They
not only say that they are right as to what good is, but they
endeavour to prove that other people who say that it is something
else, are wrong. One, for instance, will affirm that good is pleasure,
another, perhaps, that good is that which is desired; and each of
these will argue eagerly to prove that the other is wrong. But how is
that possible? One of them says that good is nothing but the object of
desire, and at the same time tries to prove that it is not pleasure.
But from his first assertion, that good just means the object of
desire, one of two things must follow as regards his proof:

(1) He may be trying to prove that the object of desire is not
pleasure. But, if this be all, where is his Ethics? The position he is
maintaining is merely a psychological one. Desire is something which
occurs in our minds, and pleasure is something else which so occurs;
and our would-be ethical philosopher is merely holding that the latter
is not the object of the former. But what has that to do with the
question in dispute? His opponent held the ethical proposition that
pleasure was the good, and although he should prove a million times
over the psychological proposition that pleasure is not the object of
desire, he is no nearer proving his opponent to be wrong. The position
is like this. One man says a triangle is a circle: another replies `A
triangle is a straight line, and I will prove to you that I am right:
\textit{for}' (this is the only argument) `a straight line is not a
circle.' `That is quite true,' the other may reply; `but nevertheless
a triangle is a circle, and you have said nothing whatever to prove
the contrary. What is proved is that one of us is wrong, for we agree
that a triangle cannot be both a straight line and a circle: but which
is wrong, there can be no earthly means of proving, since you define
triangle as straight line and I define it as circle.'---Well, that is
one alternative which any naturalistic Ethics has to face; if good is
\textit{defined} as something else, it is then impossible either to
prove that any other definition is wrong or even to deny such
definition.

(2) The other alternative will scarcely be more welcome. It is that
the discussion is after all a verbal one. When A says `Good means
pleasant' and B says `Good means desired,' they may merely wish to
assert that most people have used the word \page{12} for what is
pleasant and for what is desired respectively. And this is quite an
interesting subject for discussion: only it is not a whit more an
ethical discussion than the last was. Nor do I think that any exponent
of naturalistic Ethics would be willing to allow that this was all he
meant. They are all so anxious to persuade us that what they call the
good is what we really ought to do. `Do, pray, act so, because the
word ``good'' is generally used to denote actions of this nature':
such, on this view, would be the substance of their teaching. And in
so far as they tell us how we ought to act, their teaching is truly
ethical, as they mean it to be. But how perfectly absurd is the reason
they would give for it! `You are to do this, because most people use a
certain word to denote conduct such as this.' `You are to say the
thing which is not, because most people call it lying.' That is an
argument just as good!---My dear sirs, what we want to know from you
as ethical teachers, is not how people use a word; it is not even,
what kind of actions they approve, which the use of this word `good'
may certainly imply: what we want to know is simply what \textit{is}
good. We may indeed agree that what most people do think good, is
actually so; we shall at all events be glad to know their opinions:
but when we say their opinions about what \textit{is} good, we do mean
what we say; we do not care whether they call that thing which they
mean `horse' or `table' or `chair,' `gut' or `bon' or `\grk{ἀγαθός}';
we want to know what it is that they so call. When they say `Pleasure
is good,' we cannot believe that they merely mean `Pleasure is
pleasure' and nothing more than that.

12. Suppose a man says `I am pleased'; and suppose that is not a lie
or a mistake but the truth. Well, if it is true, what does that mean?
It means that his mind, a certain definite mind, distinguished by
certain definite marks from all others, has at this moment a certain
definite feeling called pleasure. `Pleased' \textit{means} nothing but
having pleasure, and though we may be more pleased or less pleased,
and even, we may admit for the present, have one or another kind of
pleasure; yet in so far as it is pleasure we have, whether there be
more or less of it, and whether it be of one kind or another, what we
have is \page{13} one definite thing, absolutely indefinable, some one
thing that is the same in all the various degrees and in all the
various kinds of it that there may be. We may be able to say how it is
related to other things: that, for example, it is in the mind, that it
causes desire, that we are conscious of it, etc., etc. We can, I say,
describe its relations to other things, but define it we can
\textit{not}. And if anybody tried to define pleasure for us as being
any other natural object; if anybody were to say, for instance, that
pleasure \textit{means} the sensation of red, and were to proceed to
deduce from that that pleasure is a colour, we should be entitled to
laugh at him and to distrust his future statements about pleasure.
Well, that would be the same fallacy which I have called the
naturalistic fallacy. That `pleased' does not mean `having the
sensation of red,' or anything else whatever, does not prevent us from
understanding what it does mean. It is enough for us to know that
`pleased' does mean `having the sensation of pleasure,' and though
pleasure is absolutely indefinable, though pleasure is pleasure and
nothing else whatever, yet we feel no difficulty in saying that we are
pleased. The reason is, of course, that when I say `I am pleased,' I
do \textit{not} mean that `I' am the same thing as `having pleasure.'
And similarly no difficulty need be found in my saying that `pleasure
is good' and yet not meaning that `pleasure' is the same thing as
`good,' that pleasure \textit{means} good, and that good
\textit{means} pleasure. If I were to imagine that when I said `I am
pleased,' I meant that I was exactly the same thing as `pleased,' I
should not indeed call that a naturalistic fallacy, although it would
be the same fallacy as I have called naturalistic with reference to
Ethics. The reason of this is obvious enough. When a man confuses
two natural objects with one another, defining the one by the other,
if for instance, he confuses himself, who is one natural object, with
`pleased' or with `pleasure' which are others, then there is no reason
to call the fallacy naturalistic. But if he confuses `good,' which
is not in the same sense a natural object, with any natural object
whatever, then there is a reason for calling that a naturalistic
fallacy; its being made with regard to `good' marks it as something
quite specific, and this specific mistake deserves a name because it
is so common. \page{14} As for the reasons why good is not to be
considered a natural object, they may be reserved for discussion in
another place. But, for the present, it is sufficient to notice
this: Even if it were a natural object, that would not alter the
nature of the fallacy nor diminish its importance one whit. All that I
have said about it would remain quite equally true: only the name
which I have called it would not be so appropriate as I think it is.
And I do not care about the name: what I do care about is the fallacy.
It does not matter what we call it, provided we recognise it when we
meet with it. It is to be met with in almost every book on Ethics; and
yet it is not recognised: and that is why it is necessary to multiply
illustrations of it, and convenient to give it a name. It is a very
simple fallacy indeed. When we say that an orange is yellow, we do not
think our statement binds us to hold that `orange' means nothing else
than `yellow,' or that nothing can be yellow but an orange. Supposing
the orange is also sweet! Does that bind us to say that `sweet' is
exactly the same thing as `yellow,' that `sweet' must be defined as
`yellow'? And supposing it be recognised that `yellow' just means
`yellow' and nothing else whatever, does that make it any more
difficult to hold that oranges are yellow? Most certainly it does not:
on the contrary, it would be absolutely meaningless to say that
oranges were yellow, unless yellow did in the end mean just `yellow'
and nothing else whatever---unless it was absolutely indefinable. We
should not get any very clear notion about things, which are
yellow---we should not get very far with our science, if we were bound
to hold that everything which was yellow, \textit{meant} exactly the
same thing as yellow. We should find we had to hold that an orange was
exactly the same thing as a stool, a piece of paper, a lemon, anything
you like. We could prove any number of absurdities; but should we be
the nearer to the truth? Why, then, should it be different with
`good'? Why, if good is good and indefinable, should I be held to deny
that pleasure is good? Is there any difficulty in holding both to be
true at once? On the contrary, there is no meaning in saying that
pleasure is good, unless good is something different from pleasure. It
is absolutely useless, so far as Ethics is concerned, to prove, as Mr
Spencer \page{15} tries to do, that increase of pleasure coincides
with increase of life, unless good \textit{means} something different
from either life or pleasure. He might just as well try to prove that
an orange is yellow by shewing that it always is wrapped up in
paper.

13. In fact, if it is not the case that `good' denotes something
simple and indefinable, only two alternatives are possible: either it
is a complex, a given whole, about the correct analysis of which there
may be disagreement; or else it means nothing at all, and there is no
such subject as Ethics. In general, however, ethical philosophers have
attempted to define good, without recognising what such an attempt
must mean. They actually use arguments which involve one or both of
the absurdities considered in \S11. We are, therefore, justified in
concluding that the attempt to define good is chiefly due to want of
clearness as to the possible nature of definition. There are, in fact,
only two serious alternatives to be considered, in order to establish
the conclusion that `good' does denote a simple and indefinable
notion. It might possibly denote a complex, as `horse' does; or it
might have no meaning at all. Neither of these possibilities has,
however, been clearly conceived and seriously maintained, as such, by
those who presume to define good; and both may be dismissed by a
simple appeal to facts.

(1) The hypothesis that disagreement about the meaning of good is
disagreement with regard to the correct analysis of a given whole, may
be most plainly seen to be incorrect by consideration of the fact
that, whatever definition be offered, it may be always asked, with
significance, of the complex so defined, whether it is itself good. To
take, for instance, one of the more plausible, because one of the more
complicated, of such proposed definitions, it may easily be thought,
at first sight, that to be good may mean to be that which we desire to
desire. Thus if we apply this definition to a particular instance and
say `When we think that A is good, we are thinking that A is one of
the things which we desire to desire,' our proposition may seem quite
plausible. But, if we carry the investigation further, and ask
ourselves `Is it good to desire to desire A?' it is apparent, on a
little reflection, that this question is itself as intelligible, as
the original question `Is A good?'---that we are, \page{16} in fact,
now asking for exactly the same information about the desire to desire
A, for which we formerly asked with regard to A itself. But it is also
apparent that the meaning of this second question cannot be correctly
analysed into `Is the desire to desire A one of the things which we
desire to desire?': we have not before our minds anything so
complicated as the question `Do we desire to desire to desire to
desire A?' Moreover any one can easily convince himself by inspection
that the predicate of this proposition---`good'---is positively
different from the notion of `desiring to desire' which enters into
its subject: `That we should desire to desire A is good' is
\textit{not} merely equivalent to `That A should be good is good.' It
may indeed be true that what we desire to desire is always also good;
perhaps, even the converse may be true: but it is very doubtful
whether this is the case, and the mere fact that we understand very
well what is meant by doubting it, shews clearly that we have two
different notions before our minds.

(2) And the same consideration is sufficient to dismiss the hypothesis
that `good' has no meaning whatsoever. It is very natural to make the
mistake of supposing that what is universally true is of such a nature
that its negation would be self-contradictory: the importance which
has been assigned to analytic propositions in the history of
philosophy shews how easy such a mistake is. And thus it is very easy
to conclude that what seems to be a universal ethical principle is in
fact an identical proposition; that, if, for example, whatever is
called `good' seems to be pleasant, the proposition `Pleasure is the
good' does not assert a connection between two different notions, but
involves only one, that of pleasure, which is easily recognised as a
distinct entity. But whoever will attentively consider with himself
what is actually before his mind when he asks the question `Is
pleasure (or whatever it may be) after all good?' can easily satisfy
himself that he is not merely wondering whether pleasure is pleasant.
And if he will try this experiment with each suggested definition in
succession, he may become expert enough to recognise that in every
case he has before his mind a unique object, with regard to the
connection of which with any other object, a distinct question may be
asked. Every \page{17} one does in fact understand the question `Is
this good?' When he thinks of it, his state of mind is different from
what it would be, were he asked `Is this pleasant, or desired, or
approved?' It has a distinct meaning for him, even though he may not
recognise in what respect it is distinct. Whenever he thinks of
`intrinsic value,' or `intrinsic worth,' or says that a thing `ought
to exist,' he has before his mind the unique object---the unique
property of things---which I mean by `good.' Everybody is constantly
aware of this notion, although he may never become aware at all that
it is different from other notions of which he is also aware. But, for
correct ethical reasoning, it is extremely important that he should
become aware of this fact; and, as soon as the nature of the problem
is clearly understood, there should be little difficulty in advancing
so far in analysis.

14. `Good,' then, is indefinable; and yet, so far as I know, there is
only one ethical writer, Prof. Henry Sidgwick, who has clearly
recognised and stated this fact. We shall see, indeed, how far many of
the most reputed ethical systems fall short of drawing the conclusions
which follow from such a recognition. At present I will only quote one
instance, which will serve to illustrate the meaning and importance of
this principle that `good' is indefinable, or, as Prof. Sidgwick says,
an `unanalysable notion.' It is an instance to which Prof. Sidgwick
himself refers in a note on the passage, in which he argues that
`ought' is unanalysable\footnote{\textit{Methods of Ethics}, Bk. I,
Chap. iii, \S1 (6th edition).}.

`Bentham,' says Sidgwick, `explains that his fundamental principle
``states the greatest happiness of all those whose interest is in
question as being the right and proper end of human action'''; and yet
`his language in other passages of the same chapter would seem to
imply' that he \textit{means} by the word ``right'' ``conducive to the
general happiness.'' Prof. Sidgwick sees that, if you take these two
statements together, you get the absurd result that `greatest
happiness is the end of human action, which is conducive to the
general happiness'; and so absurd does it seem to him to call this
result, as Bentham calls it, `the fundamental principle of a moral
system,' that he suggests that Bentham cannot have meant it. Yet Prof.
Sidgwick \page{18} himself states elsewhere\footnote{\textit{Methods
of Ethics}, Bk. I, Chap. iv, \S1.} that Psychological Hedonism is `not
seldom confounded with Egoistic Hedonism'; and that confusion, as we
shall see, rests chiefly on that same fallacy, the naturalistic
fallacy, which is implied in Bentham's statements. Prof. Sidgwick
admits therefore that this fallacy is sometimes committed, absurd as
it is; and I am inclined to think that Bentham may really have been
one of those who committed it. Mill, as we shall see, certainly did
commit it. In any case, whether Bentham committed it or not, his
doctrine, as above quoted, will serve as a very good illustration of
this fallacy, and of the importance of the contrary proposition that
good is indefinable.

Let us consider this doctrine. Bentham seems to imply, so Prof.
Sidgwick says, that the word `right' \textit{means} `conducive to
general happiness.' Now this, by itself, need not necessarily involve
the naturalistic fallacy. For the word `right' is very commonly
appropriated to actions which lead to the attainment of what is good;
which are regarded as \textit{means} to the ideal and not as
ends-in-themselves. This use of `right,' as denoting what is good as a
means, whether or not it be also good as an end, is indeed the use to
which I shall confine the word. Had Bentham been using `right' in this
sense, it might be perfectly consistent for him to \textit{define}
right as `conducive to the general happiness,' \textit{provided only}
(and notice this proviso) he had already proved, or laid down as an
axiom, that general happiness was \textit{the} good, or (what is
equivalent to this) that general happiness alone was good. For in that
case he would have already defined \textit{the} good as general
happiness (a position perfectly consistent, as we have seen, with the
contention that `good' is indefinable), and, since right was to be
defined as `conducive to \textit{the} good,' it would actually
\textit{mean} `conducive to general happiness.' But this method of
escape from the charge of having committed the naturalistic fallacy
has been closed by Bentham himself. For his fundamental principle is,
we see, that the greatest happiness of all concerned is the
\textit{right} and proper \textit{end} of human action. He applies the
word `right,' therefore, to the end, as such, not only to the means
which are \page{19} conducive to it; and, that being so, right can no
longer be defined as `conducive to the general happiness,' without
involving the fallacy in question. For now it is obvious that the
definition of right as conducive to general happiness can be used by
him in support of the fundamental principle that general happiness is
the right end; instead of being itself derived from that principle. If
right, by definition, means conducive to general happiness, then it is
obvious that general happiness is the right end. It is not necessary
now first to prove or assert that general happiness is the right end,
before right is defined as conducive to general happiness---a
perfectly valid procedure; but on the contrary the definition of right
as conducive to general happiness proves general happiness to be the
right end---a perfectly invalid procedure, since in this case the
statement that `general happiness is the right end of human action' is
not an ethical principle at all, but either, as we have seen, a
proposition about the meaning of words, or else a proposition about
the \textit{nature} of general happiness, not about its rightness or
goodness.

Now, I do not wish the importance I assign to this fallacy to be
misunderstood. The discovery of it does not at all refute Bentham's
contention that greatest happiness is the proper end of human action,
if that be understood as an ethical proposition, as he undoubtedly
intended it. That principle may be true all the same; we shall
consider whether it is so in succeeding chapters. Bentham might have
maintained it, as Professor Sidgwick does, even if the fallacy had
been pointed out to him. What I am maintaining is that the
\textit{reasons} which he actually gives for his ethical proposition
are fallacious ones so far as they consist in a definition of right.
What I suggest is that he did not perceive them to be fallacious;
that, if he had done so, he would have been led to seek for other
reasons in support of his Utilitarianism; and that, had he sought for
other reasons, he \textit{might} have found none which he thought to
be sufficient. In that case he would have changed his whole system---a
most important consequence. It is undoubtedly also possible that he
would have thought other reasons to be sufficient, and in that case
his ethical system, \page{20} in its main results, would still have
stood. But, even in this latter case, his use of the fallacy would be
a serious objection to him as an ethical philosopher. For it is the
business of Ethics, I must insist, not only to obtain true results,
but also to find valid reasons for them. The direct object of Ethics
is knowledge and not practice; and any one who uses the naturalistic
fallacy has certainly not fulfilled this first object, however correct
his practical principles may be.

My objections to Naturalism are then, in the first place, that it
offers no reason at all, far less any valid reason, for any ethical
principle whatever; and in this it already fails to satisfy the
requirements of Ethics, as a scientific study. But in the second place
I contend that, though it gives a reason for no ethical principle, it
is a \textit{cause} of the acceptance of false principles---it deludes
the mind into accepting ethical principles, which are false; and in
this it is contrary to every aim of Ethics. It is easy to see that if
we start with a definition of right conduct as conduct conducive to
general happiness; then, knowing that right conduct is universally
conduct conducive to the good, we very easily arrive at the result
that the good is general happiness. If, on the other hand, we once
recognise that we must start our Ethics without a definition, we shall
be much more apt to look about us, before we adopt any ethical
principle whatever; and the more we look about us, the less likely are
we to adopt a false one. It may be replied to this: Yes, but we shall
look about us just as much, before we settle on our definition, and
are therefore just as likely to be right. But I will try to shew that
this is not the case. If we start with the conviction that a
definition of good can be found, we start with the conviction that
good \textit{can mean} nothing else than some one property of things;
and our only business will then be to discover what that property is.
But if we recognise that, so far as the meaning of good goes, anything
whatever may be good, we start with a much more open mind. Moreover,
apart from the fact that, when we think we have a definition, we
cannot logically defend our ethical principles in any way whatever, we
shall also be much less apt to defend them well, even if illogically.
For we shall start with the conviction that good \page{21} must mean
so and so, and shall therefore be inclined either to misunderstand our
opponent's arguments or to cut them short with the reply, `This is not
an open question: the very meaning of the word decides it; no one can
think otherwise except through confusion.'

