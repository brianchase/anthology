
%\author{Georg Wilhelm Friedrich Hegel}
\author{G. W. F. Hegel}
\authdate{1770--1831}
\textdate{ca. 1822--1830}
\addon{Introduction, excerpt}
\chapter[Lectures on the Philosophy of History, excerpt]{Lectures on
the Philosophy of History}
\source{hegel1914.1}

\page{9}The only Thought which Philosophy brings with it to the
contemplation of History, is the simple conception of \textit{Reason};
that Reason is the Sovereign of the World; that the history of the
world, therefore, presents us with a rational process. This
conviction and intuition is a hypothesis in the domain of history as
such. In that of Philosophy it is no hypothesis. It is there proved by
speculative cognition, that Reason---and this term may here suffice
us, without investigating the relation sustained by the Universe to
the Divine Being,---is \textit{Substance}, as well as \textit{Infinite
Power}; its own \textit{Infinite Material} underlying all the natural
and spiritual life which it originates, as also the \textit{Infinite
Form},---that which sets this Material in motion. On the one hand,
Reason is the \textit{substance} of the Universe; viz. that by which
and in which all reality has its \page{10} being and subsistence. On
the other hand, it is the \textit{Infinite Energy} of the Universe;
since Reason is not so powerless as to be incapable of producing
anything but a mere ideal, a mere intention---having its place
outside reality, nobody knows where; something separate and abstract,
in the heads of certain human beings. It is \textit{the infinite
complex of things}, their entire Essence and Truth. It is its own
material which it commits to its own Active Energy to work up; not
needing, as finite action does, the conditions of an external material
of given means from which it may obtain its support, and the objects
of its activity. It supplies its own nourishment, and is the object of
its own operations. While it is exclusively its own basis of
existence, and absolute final aim, it is also the energizing power
realizing this aim; developing it not only in the phenomena of the
Natural, but also of the Spiritual Universe---the History of the
World. That this ``Idea'' or ``Reason'' is the \textit{True}, the
\textit{Eternal}, the absolutely \textit{powerful} essence; that it
reveals itself in the World, and that in that World nothing else is
revealed but this and its honour and glory---is the thesis which, as
we have said, has been proved in Philosophy, and is here regarded as
demonstrated.

In those of my hearers who are not acquainted with Philosophy, I may
fairly presume, at least, the existence of a \textit{belief} in
Reason, a desire, a thirst for acquaintance with it, in entering upon
this course of Lectures. It is, in fact, the wish for rational
insight, not the ambition to amass a mere heap of acquirements, that
should be presupposed in every case as possessing the mind of the
learner in the study of science. If the clear idea of Reason is not
already developed in our minds, in beginning the study of Universal
History, we should at least have the firm, unconquerable faith that
Reason \textit{does} exist there; and that the World of intelligence
and conscious volition is not abandoned to chance, but must shew
itself in the light of the self-cognizant Idea. Yet I am not obliged
to make any such preliminary demand upon your faith. What I have said
thus provisionally, and what I shall have further to say, is, even in
reference to \textit{our} branch of science, not to be regarded as
hypothetical, but as a summary view of the whole; the \textit{result
of the investigation} we are about to pur-\page{11}sue; a result which
happens to be known to \textit{me}, because I have traversed the
entire field. It is only an inference from the history of the World,
that its development has been a rational process; that the history in
question has constituted the rational necessary course of the
World-Spirit---that Spirit whose nature is always one and the same,
but which unfolds this its one nature in the phenomena of the World's
existence. This must, as before stated, present itself as the ultimate
\textit{result} of History. But we have to take the latter as it is.
We must proceed historically---empirically. Among other precautions we
must take care not to be misled by professed historians who
(especially among the Germans, and enjoying a considerable authority),
are chargeable with the very procedure of which they accuse the
Philosopher---introducing \textit{\`a priori} inventions of their own
into the records of the Past. It is, for example, a widely current
fiction, that there was an original prim\ae val people, taught
immediately by God, endowed with perfect insight and wisdom,
possessing a thorough knowledge of all natural laws and spiritual
truth; that there have been such or such sacerdotal peoples; or, to
mention a more specific averment, that there was a Roman Epos, from
which the Roman historians derived the early annals of their city,
\&c. Authorities of this kind we leave to those talented historians by
profession, among whom (in Germany at least) their use is not
uncommon.---We might then announce it as the first condition to be
observed, that we should faithfully adopt all that is historical. But
in such general expressions themselves, as ``faithfully'' and
``adopt,'' lies the ambiguity. Even the ordinary, the ``impartial''
historiographer, who believes and professes that he maintains a simply
receptive attitude; surrendering himself only to the data supplied
him---is by no means passive as regards the exercise of his thinking
powers. He brings his categories with him, and sees the phenomena
presented to his mental vision, exclusively through these media. And,
especially in all that pretends to the name of science, it is
indispensable that Reason should not sleep---that reflection should be
in full play. To him who looks upon the world rationally, the world in
its turn, presents a rational aspect. The relation is mutual. But the
various exercises of reflection---the dif-\page{12}ferent points of
view---the modes of deciding the simple question of the relative
importance of events (the first category that occupies the attention
of the historian), do not belong to this place.

I will only mention two phases and points of view that concern the
generally diffused conviction that Reason has ruled, and is still
ruling in the world, and consequently in the world's history; because
they give us, at the same time, an opportunity for more closely
investigating the question that presents the greatest difficulty, and
for indicating a branch of the subject, which will have to be enlarged
on in the sequel.

% NOTE: the em-dash below, after 'I.', is in the source but none
% follows the subsequent 'II.'

I.---One of these points is, that passage in history, which informs us
that the Greek Anaxagoras was the first to enunciate the doctrine that
\grk{νοῦς}, Understanding generally, or Reason, governs the world. It
is not intelligence as self-conscious Reason,---not a Spirit as such
that is meant; and we must clearly distinguish these from each other.
The movement of the solar system takes place according to unchangeable
laws. These laws are Reason, implicit in the phenomena in question.
But neither the sun nor the planets, which revolve around it according
to these laws, can be said to have any consciousness of them.

A thought of this kind,---that Nature is an embodiment of Reason; that
it is unchangeably subordinate to universal laws, appears nowise
striking or strange to us. We are accustomed to such conceptions, and
find nothing extraordinary in them. And I have mentioned this
extraordinary occurrence, partly to shew how history teaches, that
ideas of this kind, which may seem trivial to us, have not always been
in the world; that on the contrary, such a thought makes an epoch in
the annals of human intelligence. Aristotle says of Anaxagoras, as the
originator of the thought in question, that he appeared as a sober man
among the drunken. Socrates adopted the doctrine from Anaxagoras, and
it forthwith became the ruling idea in Philosophy,---except in the
school of Epicurus, who ascribed all events to chance. ``I was
delighted with the sentiment,''---Plato makes Socrates say,---``and
hoped I had found a teacher who would shew me Nature in harmony with
Reason, who would demonstrate in each particular phenomenon its
specific aim, and in the whole, \page{13} the grand object of the
Universe. I would not have surrendered this hope for a great deal. But
how very much was I disappointed, when, having zealously applied
myself to the writings of Anaxagoras, I found that he adduces only
external causes, such as Atmosphere, Ether, Water, and the like.'' It
is evident that the defect which Socrates complains of respecting
Anaxagoras's doctrine, does not concern the principle itself, but the
shortcoming of the propounder in applying it to Nature in the
concrete. Nature is not deduced from that principle: the latter
remains in fact a mere abstraction, inasmuch as the former is not
comprehended and exhibited as a development of it,---an organisation
produced by and from Reason. I wish, at the very outset, to call your
attention to the important difference between a conception, a
principle, a truth limited to an \textit{abstract} form and its
determinate application, and concrete development. This distinction
affects the whole fabric of philosophy; and among other bearings of it
there is one to which we shall have to revert at the close of our view
of Universal History, in investigating the aspect of political affairs
in the most recent period.

% NOTE: the source has 'realized' (p. 16) but, consistent with British
% spelling, also 'realise', 'realised', etc.

We have next to notice the rise of this idea---that Reason directs the
World---in connection with a further application of it, well known to
us,---in the form, viz. of the \textit{religious truth}, that the
world is not abandoned to chance and external contingent causes, but
that a \textit{Providence} controls it. I stated above, that I would
not make a demand on your faith, in regard to the principle announced.
Yet I might appeal to your belief in it, \textit{in this religious
aspect}, if, as a general rule, the nature of philosophical science
allowed it to attach authority to presuppositions. To put it in
another shape,---this appeal is forbidden, because the science of
which we have to treat, proposes itself to furnish the proof (not
indeed of the abstract \textit{Truth} of the doctrine, but) of its
correctness as compared with facts. The truth, then, that a Providence
(that of God) presides over the events of the World---consorts with
the proposition in question; for \textit{Divine} Providence is Wisdom,
endowed with an infinite Power, which realises its aim, viz. the
absolute rational design of the World. Reason is Thought conditioning
itself with perfect freedom. But a difference---rather a
contra-\page{14}diction---will manifest itself, between this belief
and our principle, just as was the case in reference to the demand
made by Socrates in the case of Anaxagoras's dictum. For that belief
is similarly indefinite; it is what is called a belief in a general
Providence, and is not followed out into definite application, or
displayed in its bearing on the grand total---the entire course of
human history. But to \textit{explain} History is to depict the
passions of mankind, the genius, the active powers, that play their
part on the great stage; and the providentially determined process
which these exhibit, constitutes what is generally called the ``plan''
of Providence. Yet it is this very plan which is supposed to be
concealed from our view: which it is deemed presumption, even to wish
to recognise. The ignorance of Anaxagoras, as to how intelligence
reveals itself in actual existence, was ingenuous. Neither in his
consciousness, nor in that of Greece at large, had that thought been
farther expanded. He had not attained the power to apply his general
principle to the concrete, so as to deduce the latter from the former.
It was Socrates who took the first step in comprehending the union of
the Concrete with the Universal. Anaxagoras, then, did not take up a
\textit{hostile} position towards such an application. The common
belief in Providence \textit{does}; at least it opposes the use of the
principle on the large scale, and denies the possibility of discerning
the plan of Providence. In isolated cases this plan is supposed to be
manifest. Pious persons are encouraged to recognise in particular
circumstances, something more than mere chance; to acknowledge the
guiding hand of God; \textit{e.g.} when help has unexpectedly come to
an individual in great perplexity and need. But these instances of
providential design are of a limited kind, and concern the
accomplishment of nothing more than the desires of the individual in
question. But in the history of the World, the \textit{Individuals} we
have to do with are \textit{Peoples}; Totalities that are States. We
cannot, therefore, be satisfied with what we may call this
``peddling'' view of Providence, to which the belief alluded to limits
itself. Equally unsatisfactory is the merely abstract, undefined
belief in a Providence, when that belief is not brought to bear upon
the details of the process which it conducts. On the contrary our
earnest endeavour must be directed to the recognition \page{15} of the
ways of Providence, the means it uses, and the historical phenomena in
which it manifests itself; and we must shew their connection with the
general principle above mentioned. But in noticing the recognition of
the plan of Divine Providence generally, I have implicitly touched
upon a prominent question of the day; viz. that of the possibility of
knowing God: or rather---since public opinion has ceased to allow it
to be a matter of \textit{question}---the \textit{doctrine} that it is
impossible to know God. In direct contravention of what is commanded
in holy Scripture as the highest duty,---that we should not merely
love, but \textit{know} God,---the prevalent dogma involves the denial
of what is there said; viz. that it is the Spirit (der Geist) that
leads into Truth, knows all things, penetrates even into the deep
things of the Godhead. While the Divine Being is thus placed beyond
our knowledge, and outside the limit of all human things, we have the
convenient licence of wandering as far as we list, in the direction of
our own fancies. We are freed from the obligation to refer our
knowledge to the Divine and True. On the other hand, the vanity and
egotism which characterise it, find, in this false position, ample
justification; and the pious modesty which puts far from it the
knowledge of God, can well estimate how much furtherance thereby
accrues to its own wayward and vain strivings. I have been unwilling
to leave out of sight the connection between our thesis---that Reason
governs and has governed the World---and the question of the
possibility of a knowledge of God, chiefly that I might not lose the
opportunity of mentioning the imputation against Philosophy of being
shy of noticing religious truths, or of having occasion to be so; in
which is insinuated the suspicion that it has anything but a clear
conscience in the presence of these truths. So far from this being the
case, the fact is, that in recent times Philosophy has been obliged to
defend the domain of religion against the attacks of several
theological systems. In the Christian religion God has revealed
Himself,---that is, he has given us to understand what He is; so that
He is no longer a concealed or secret existence. And this possibility
of knowing Him, thus afforded us, renders such knowledge a duty. God
wishes no narrow-hearted souls or empty heads for his children; but
those whose spirit is of itself indeed, poor, but rich in the
knowledge of Him; and who regard this knowledge of \page{16} God as
the only valuable possession. That development of the thinking spirit,
which has resulted from the revelation of the Divine Being as its
original basis, must ultimately advance to the \textit{intellectual}
comprehension of what was presented in the first instance, to
\textit{feeling} and \textit{imagination}. The time must eventually
come for understanding that rich product of active Reason, which the
History of the World offers to us. It was for a while the fashion to
profess admiration for the wisdom of God, as displayed in animals,
plants, and isolated occurrences. But, if it be allowed that
Providence manifests itself in such objects and forms of existence,
why not also in Universal History. This is deemed too great a matter
to be thus regarded. But Divine Wisdom, \textit{i.e.} Reason, is one
and the same in the great as in the little; and we must not imagine
God to be too weak to exercise his wisdom on the grand scale. Our
intellectual striving aims at realizing the conviction that what was
\textit{intended} by eternal wisdom, is actually \textit{accomplished}
in the domain of existent, active Spirit, as well as in that of mere
Nature. Our mode of treating the subject is, in this aspect, a
Theodic\ae a,---a justification of the ways of God,---which Leibnitz
attempted metaphysically, in his method, \textit{i.e.} in indefinite
abstract categories,---so that the ill that is found in the World may
be comprehended, and the thinking Spirit reconciled with the fact of
the existence of evil. Indeed, nowhere is such a harmonising view more
pressingly demanded than in Universal History; and it can be attained
only by recognising the \textit{positive} existence, in which that
negative element is a subordinate, and vanquished nullity. On the one
hand, the ultimate design of the World must be perceived; and, on the
other hand, the fact that this design has been actually realised in
it, and that evil has not been able permanently to assert a competing
position. But this conviction involves much more than the mere belief
in a superintending \grk{νοῦς}, or in ``Providence.'' ``Reason,''
whose sovereignty over the World has been maintained, is as indefinite
a term as ``Providence,'' supposing the term to be used by those who
are unable to characterise it distinctly,---to shew wherein it
consists, so as to enable us to decide whether a thing is rational or
irrational. An adequate definition of Reason is the first desideratum;
and whatever \page{17} boast may be made of strict adherence to it in
explaining phenomena,---without such a definition we get no farther
than mere words. With these observations we may proceed to the second
point of view that has to be considered in this Introduction.

II. The enquiry into the \textit{essential destiny} of Reason---as far
as it is considered in reference to the World---is identical with the
question, \textit{what is the ultimate design of the World?} And the
expression implies that that design is destined to be realised. Two
points of consideration suggest themselves: first, the \textit{import}
of this design---its abstract definition; and secondly, its
\textit{realization}.

It must be observed at the outset, that the phenomenon we
in\-ves\-ti\-gate---Universal History---belongs to the realm of
\textit{Spirit}. The term ``\textit{World},'' includes both physical
and psychical Nature. Physical Nature also plays its part in the
World's History, and attention will have to be paid to the fundamental
natural relations thus involved. But Spirit, and the course of its
development, is our substantial object. Our task does not require us
to contemplate Nature as a Rational System in itself---though in its
own proper domain it proves itself such---but simply in its relation
to \textit{Spirit}. On the stage on which we are observing
it,---Universal History---Spirit displays itself in its most concrete
reality. Notwithstanding this (or rather for the very purpose of
comprehending the \textit{general} principles which this, its form of
\textit{concrete reality}, embodies) we must premise some abstract
characteristics of the \textit{nature of Spirit}. Such an explanation,
however, cannot be given here under any other form than that of bare
assertion. The present is not the occasion for unfolding the idea of
Spirit speculatively; for whatever has a place in an Introduction,
must, as already observed, be taken as simply historical; something
assumed as having been explained and proved elsewhere; or whose
demonstration awaits the sequel of the Science of History itself.

\snip

% NOTE: changed 'organization' (p. 19) to 'organisation'

\page{18}\ldots The nature of Spirit may be understood by a glance at
its direct opposite---\textit{Matter}. As the essence of Matter is
Gravity, so, on the other hand, we may affirm that the substance, the
essence of Spirit is Freedom. All will readily assent to the
doctrine that Spirit, among other properties, is also endowed with
Freedom; but philosophy teaches that all the qualities of Spirit exist
only through Freedom; that all are but means for attaining Freedom;
that all seek and produce this and this alone. It is a result of
speculative Philosophy, that Freedom is the sole truth of Spirit.
Matter possesses gravity in virtue of its tendency towards a central
point. It is essentially composite; consisting of parts that
\textit{exclude} each other. It seeks its Unity; and therefore
exhibits itself as self-destructive, as verging towards its opposite
[an indivisible point]. If it could attain this, it would be Matter no
longer, it would have perished. It strives after the realization of
its Idea; for in Unity it exists \textit{ideally}. Spirit, on the
contrary, may be defined as that which has its centre in itself. It
has not a unity outside itself, but has already found it; it exists
\textit{in} and \textit{with itself}. Matter has its essence out of
itself; Spirit is \textit{self-contained existence}
(Bei-sich-selbst-seyn). Now this is Freedom, exactly. For if I am
dependent, my being is referred to something else which I am not; I
cannot exist independently of something external. I am free, on the
contrary, when my existence depends upon myself. This self-contained
existence of Spirit is none other than
self-consciousness---consciousness of one's own being. Two things must
be distinguished in consciousness; first, the fact \textit{that I
know}; secondly, \textit{what I know}. In \textit{self}
consciousness these are merged in one; for Spirit \textit{knows
itself}. It involves an appreciation of its own nature, as also an
energy enabling it to realise itself; to make itself \textit{actually}
that which it is \textit{potentially}. According to this abstract
definition it may be said of Universal History, that it is the
exhibition of Spirit in the process of working out the knowledge of
that which it is potentially. And as the germ bears in itself the
whole nature of the tree, and the taste and form of its fruits, so do
the first traces of Spirit virtually contain the whole of that
History. The Orientals have not attained the knowledge that
Spirit---Man \textit{as such}---is free; and because \page{19} they do
not know this, they are not free. They only know that \textit{one is
free}. But on this very account, the freedom of that one is only
caprice; ferocity---brutal recklessness of passion, or a mildness and
tameness of the desires, which is itself only an accident of
Nature---mere caprice like the former.---That \textit{one} is
therefore only a Despot; not a \textit{free man}. The consciousness of
Freedom first arose among the Greeks, and therefore they were free;
but they, and the Romans likewise, knew only that \textit{some} are
free,---not man as such. Even Plato and Aristotle did not know this.
The Greeks, therefore, had slaves; and their whole life and the
maintenance of their splendid liberty, was implicated with the
institution of slavery: a fact moreover, which made that liberty on
the one hand only an accidental, transient and limited growth; on the
other hand, constituted it a rigorous thraldom of our common
nature---of the Human. The German nations, under the influence of
Christianity, were the first to attain the consciousness, that man, as
man, is free: that it is the \textit{freedom} of Spirit which
constitutes its essence. This consciousness arose first in religion,
the inmost region of Spirit; but to introduce the principle into the
various relations of the actual world, involves a more extensive
problem than its simple implantation; a problem whose solution and
application require a severe and lengthened process of culture. In
proof of this, we may note that slavery did not cease immediately on
the reception of Christianity. Still less did liberty predominate in
States; or Governments and Constitutions adopt a rational
organisation, or recognise freedom as their basis. That application
of the principle to political relations; the thorough moulding and
interpenetration of the constitution of society by it, is a process
identical with history itself. I have already directed attention to
the distinction here involved, between a principle as such, and its
\textit{application}; \textit{i.e.} its introduction and carrying out
in the actual phenomena of Spirit and Life. This is a point of
fundamental importance in our science, and one which must be
constantly respected as essential. And in the same way as this
distinction has attracted attention in view of the \textit{Christian}
principle of selfconsciousness---Freedom; it also shews itself as an
essential one, in view of the principle of Freedom \textit{generally}.
The History of the world is none other \page{20} than the progress of
the consciousness of Freedom; a progress whose development according
to the necessity of its nature, it is our business to investigate.

The general statement given above, of the various grades in the
consciousness of Freedom---and which we applied in the first instance
to the fact that the Eastern nations knew only that \textit{one} is
free; the Greek and Roman world only that \textit{some} are free;
while \textit{we} know that all men absolutely (man \textit{as man})
are free,---supplies us with the natural division of Universal
History, and suggests the mode of its discussion. This is remarked,
however, only incidentally and anticipatively; some other ideas must
be first explained.

% NOTE: changed 'realized' to 'realised'

The destiny of the spiritual World, and,---since this is the
\textit{substantial World}, while the physical remains subordinate to
it, or, in the language of speculation, has no truth \textit{as
against} the spiritual,---the \textit{final cause of the World at
large}, we allege to be the \textit{consciousness} of its own freedom
on the part of Spirit, and \textit{ipso facto}, the \textit{reality}
of that freedom. But that this term ``Freedom,'' without further
qualification, is an indefinite, and incalculable ambiguous term; and
that while that which it represents is the \textit{ne plus ultra} of
attainment, it is liable to an infinity of misunderstandings,
confusions and errors, and to become the occasion for all imaginable
excesses,---has never been more clearly known and felt than in modern
times. Yet, for the present, we must content ourselves with the term
itself without farther definition. Attention was also directed to the
importance of the infinite difference between a principle in the
abstract, and its realization in the concrete. In the process before
us, the essential nature of freedom,---which involves in it absolute
necessity,---is to be displayed as coming to a consciousness of itself
(for it is in its very nature, self-consciousness) and thereby
realizing its existence. Itself is its own object of attainment, and
the sole aim of Spirit. This result it is, at which the process of the
World's History has been continually aiming; and to which the
sacrifices that have ever and anon been laid on the vast altar of the
earth, through the long lapse of ages, have been offered. This is the
only aim that sees itself realised and fulfilled; the only pole of
repose amid the ceaseless change of events and conditions, and the
sole efficient principle that pervades them. This final aim is God's
purpose with the \page{21} world; but God is the absolutely perfect
Being, and can, therefore, will nothing other than himself---his own
Will. The Nature of His Will---that is, His Nature itself---is what we
here call the Idea of Freedom; translating the language of Religion
into that of Thought\ldots

