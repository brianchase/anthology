
\author{Thomas Hobbes}
\authdate{1588--1679}
\textdate{1651}
\addon{Chapters 13 and 17}
\chapter[Leviathan, chaps. 13 and 17]{Leviathan}
\source{hobbes1904}

\page{81}\section{Chap. XIII. Of the \textsc{Naturall Condition} of
Mankind, as concerning their Felicity, and Misery.}

Nature hath made men so equall, in the faculties of body, and mind; as
that though there bee found one man sometimes manifestly stronger in
body, or of quicker mind then another; yet when all is reckoned
together, the difference between man, and man, is not so considerable,
as that one man can thereupon claim to himselfe any benefit, to which
another may not pretend, as well as he. For as to the strength of
body, the weakest has strength enough to kill the strongest, either by
secret machination, or by confederacy with others, that are in the
same danger with himselfe.

\page{82}And as to the faculties of the mind, (setting aside the arts
grounded upon words, and especially that skill of proceeding upon
generall, and infallible rules, called Science; which very few have,
and but in few things; as being not a native faculty, born with us;
nor attained, (as Prudence,) while we look after somewhat els,) I find
yet a greater equality amongst men, than that of strength. For
Prudence, is but Experience; which equall time, equally bestowes on
all men, in those things they equally apply themselves unto. That
which may perhaps make such equality incredible, is but a vain
conceipt of ones owne wisdome, which almost all men think they have
in a greater degree, than the Vulgar; that is, than all men but
themselves, and a few others, whom by Fame, or for concurring with
themselves, they approve. For such is the nature of men, that
howsoever they may acknowledge many others to be more witty, or more
eloquent, or more learned; Yet they will hardly believe there be many
so wise as themselves: For they see their own wit at hand, and other
mens at a distance. But this proveth rather that men are in that
point equall, than unequall. For there is not ordinarily a greater
signe of the equall distribution of any thing, than that every man is
contented with his share.

From this equality of ability, ariseth equality of hope in the
attaining of our Ends. And therefore if any two men desire the same
thing, which neverthelesse they cannot both enjoy, they become
enemies; and in the way to their End, (which is principally their owne
conservation, and sometimes their delectation only,) endeavour to
destroy, or subdue one an other. And from hence it comes to passe,
that where an Invader hath no more to feare, than an other mans single
power; if one plant, sow, build, or possesse a convenient Seat, others
may probably be expected to come prepared with forces united, to
dispossesse, and deprive him, not only of the fruit of his labour, but
also of his life, or liberty. And the Invader again is in the like
danger of another.

And from this diffidence of one another, there is no way for any man
to secure himselfe, so reasonable, as Anticipation; that is, by force,
or wiles, to master the persons of all men he can, so long, till he
see \page{83} no other power great enough to endanger him: And this is
no more than his own conservation requireth, and is generally allowed.
Also because there be some, that taking pleasure in contemplating
their own power in the acts of conquest, which they pursue farther
than their security requires; if others, that otherwise would be glad
to be at ease within modest bounds, should not by invasion increase
their power, they would not be able, long time, by standing only on
their defence, to subsist. And by consequence, such augmentation of
dominion over men, being necessary to a mans conservation, it ought to
be allowed him.

Againe, men have no pleasure, (but on the contrary a great deale of
griefe) in keeping company, where there is no power able to over-awe
them all. For every man looketh that his companion should value him,
at the same rate he sets upon himselfe: And upon all signes of
contempt, or undervaluing, naturally endeavours, as far as he dares
(which amongst them that have no common power to keep them in quiet,
is far enough to make them destroy each other,) to extort a greater
value from his contemners, by dommage; and from others, by the
example.

So that in the nature of man, we find three principall causes of
quarrel. First, Competition; Secondly, Diffidence; Thirdly, Glory.

The first, maketh men invade for Gain; the second, for Safety; and the
third, for Reputation. The first use Violence, to make themselves
Masters of other mens persons, wives, children, and cattell; the
second, to defend them; the third, for trifles, as a word, a smile, a
different opinion, and any other signe of undervalue, either direct in
their Persons, or by reflexion in their Kindred, their Friends, their
Nation, their Profession, or their Name.

Hereby it is manifest, that during the time men live without a common
Power to keep them all in awe, they are in that condition which is
called Warre; and such a warre, as is of every man, against every man.
For \textsc{Warre}, consisteth not in Battell onely, or the act of
fighting; but in a tract of time, wherein the Will to contend by
Battell is sufficiently known: and therefore the \page{84} notion of
\textit{Time}, is to be considered in the nature of Warre; as it is in
the nature of Weather. For as the nature of Foule weather, lyeth not
in a showre or two of rain; but in an inclination thereto of many
dayes together: So the nature of War, consisteth not in actual
fighting; but in the known disposition thereto, during all the time
there is no assurance to the contrary. All other time is
\textsc{Peace}.

Whatsoever therefore is consequent to a time of Warre, where every man
is Enemy to every man; the same is consequent to the time, wherein men
live without other security, than what their own strength, and their
own invention shall furnish them withall. In such condition, there is
no place for Industry; because the fruit thereof is uncertain: and
consequently no Culture of the Earth; no Navigation, nor use of the
commodities that may be imported by Sea; no commodious Building; no
Instruments of moving, and removing such things as require much force;
no Knowledge of the face of the Earth; no account of Time; no Arts; no
Letters; no Society; and which is worst of all, continuall feare, and
danger of violent death; And the life of man, solitary, poore, nasty,
brutish, and short.

It may seem strange to some man, that has not well weighed these
things; that Nature should thus dissociate, and render men apt to
invade, and destroy one another: and he may therefore, not trusting to
this Inference, made from the Passions, desire perhaps to have the
same confirmed by Experience. Let him therefore consider with
himselfe, when taking a journey, he armes himselfe, and seeks to go
well accompanied; when going to sleep, he locks his dores; when even
in his house he locks his chests; and this when he knowes there bee
Lawes, and publike Officers, armed, to revenge all injuries shall bee
done him; what opinion he has of his fellow subjects, when he rides
armed; of his fellow Citizens, when he locks his dores; and of his
children, and servants, when he locks his chests. Does he not there as
much accuse mankind by his actions, as I do by my words? But neither
of us accuse mans nature in it. The Desires, and other Passions of
man, are in themselves no Sin. No more are the Actions, that proceed
from those Passions, till they know a Law that forbids them: which
till Lawes be made they cannot know: nor can any \page{85} Law be
made, till they have agreed upon the Person that shall make it.

It may peradventure be thought, there was never such a time, nor
condition of warre as this; and I believe it was never generally so,
over all the world: but there are many places, where they live so now.
For the savage people in many places of \textit{America}, except the
government of small Families, the concord whereof dependeth on
naturall lust, have no government at all; and live at this day in that
brutish manner, as I said before. Howsoever, it may be perceived what
manner of life there would be, where there were no common Power to
feare; by the manner of life, which men that have formerly lived under
a peacefull government, use to degenerate into, in a civill Warre.

But though there had never been any time, wherein particular men were
in a condition of warre one against another; yet in all times, Kings,
and Persons of Soveraigne authority, because of their Independency,
are in continuall jealousies, and in the state and posture of
Gladiators; having their weapons pointing, and their eyes fixed on one
another; that is, their Forts, Garrisons, and Guns upon the Frontiers
of their Kingdomes; and continuall Spyes upon their neighbours; which
is a posture of War. But because they uphold thereby, the Industry of
their Subjects; there does not follow from it, that misery, which
accompanies the Liberty of particular men.

To this warre of every man against every man, this also is consequent;
that nothing can be Unjust. The notions of Right and Wrong, Justice
and Injustice have there no place. Where there is no common Power,
there is no Law: where no Law, no Injustice. Force, and Fraud, are in
warre, the two Cardinall vertues. Justice, and Injustice are none of
the Faculties neither of the Body, nor Mind. If they were, they might
be in a man that were alone in the world, as well as his Senses, and
Passions. They are Qualities, that relate to men in Society, not in
Solitude. It is consequent also to the same condition, that there be
no Propriety, no Dominion, no \textit{Mine} and \textit{Thine}
distinct; but onely that to be every mans, that he can get; and for so
long, as he can keep it. And thus much for the ill condition, which
man by meer Nature is actually \page{86} placed in; though with a
possibility to come out of it, consisting partly in the Passions,
partly in his Reason.

The Passions that encline men to Peace, are Feare of Death; Desire of
such things as are necessary to commodious living; and a Hope by their
Industry to obtain them. And Reason suggesteth convenient Articles of
Peace, upon which men may be drawn to agreement. These Articles, are
they, which otherwise are called the Lawes of Nature: whereof I shall
speak more particularly, in the two following Chapters.

% Part II. Of Commom-wealth

\page{115}\section{Chap. XVII. Of the Causes, Generation, and
Definition of a \textsc{Common-Wealth}.}

The finall Cause, End, or Designe of men, (who naturally love Liberty,
and Dominion over others,) in the introduction of that restraint upon
themselves, (in which wee see them live in Common-wealths,) is the
foresight of their own preservation, and of a more contented life
thereby; that is to say, of getting themselves out from that miserable
condition of Warre, which is necessarily consequent (as hath been
shewn) to the naturall Passions of men, when there is no visible Power
to keep them in awe, and tye them by feare of punishment to the
performance of their Covenants, and observation of these Lawes of
Nature set down in the fourteenth and fifteenth Chapters.

For the Lawes of Nature (as \textit{Justice, Equity, Modesty, Mercy},
and (in summe) \textit{doing to others, as wee would be done to},) of
themselves, without the terrour of some Power, to cause them to be
observed, are contrary to our naturall Passions, that carry us to
Partiality, Pride, Revenge, and the like. And Covenants, without the
Sword, are but Words, and of no strength to secure a man at all.
Therefore notwithstanding the Lawes of Nature, (which every one hath
then kept, when he has the will to keep them, when he can do it
safely,) if there be no Power erected, or not great enough for our
security; every man will, and may lawfully rely on his own strength
and art, for caution against all other men. And in all places,
\page{116} where men have lived by small Families, to robbe and spoyle
one another, has been a Trade, and so farre from being reputed against
the Law of Nature, that the greater spoyles they gained, the greater
was their honour; and men observed no other Lawes therein, but the
Lawes of Honour; that is, to abstain from cruelty, leaving to men
their lives, and instruments of husbandry. And as small Familyes did
then; so now do Cities and Kingdomes which are but greater Families
(for their own security) enlarge their Dominions, upon all pretences
of danger, and fear of Invasion, or assistance that may be given to
Invaders, endeavour as much as they can, to subdue, or weaken their
neighbours, by open force, and secret arts, for want of other Caution,
justly; and are rememdbred for it in after ages with honour.

Nor is it the joyning together of a small number of men, that gives
them this security; because in small numbers, small additions on the
one side or the other, make the advantage of strength so great, as is
sufficient to carry the Victory; and therefore gives encouragement to
an Invasion. The Multitude sufficient to confide in for our Security,
is not determined by any certain number, but by comparison with the
Enemy we feare; and is then sufficient, when the odds of the Enemy is
not of so visible and conspicuous moment, to determine the event of
warre, as to move him to attempt.

And be there never so great a Multitude; yet if their actions be
directed according to their particular judgements, and particular
appetites, they can expect thereby no defence, nor protection, neither
against a Common enemy, nor against the injuries of one another. For
being distracted in opinions concerning the best use and application
of their strength, they do not help, but hinder one another; and
reduce their strength by mutuall opposition to nothing: whereby they
are easily, not onely subdued by a very few that agree together; but
also when there is no common enemy, they make warre upon each other,
for their particular interests. For if we could suppose a great
Multitude of men to consent in the observation of Justice, and other
Lawes of Nature, without a common Power to keep them all in awe; we
might \page{117} as well suppose all Man-kind to do the same; and then
there neither would be, nor need to be any Civill Government, or
Common-wealth at all; because there would be Peace without subjection.

Nor is it enough for the security, which men desire should last all
the time of their life, that they be governed, and directed by one
judgement, for a limited time; as in one Battell, or one Warre. For
though they obtain a Victory by their unanimous endeavour against a
forraign enemy; yet afterwards, when either they have no common enemy,
or he that by one part is held for an enemy, is by another part held
for a friend, they must needs by the difference of their interests
dissolve, and fall again into a Warre amongst themselves.

It is true, that certain living creatures, as Bees, and Ants, live
sociably one with another, (which are therefore by \textit{Aristotle}
numbred amongst Politicall creatures;) and yet have no other
direction, than their particular judgements and appetites; nor speech,
whereby one of them can signifie to another, what he thinks expedient
for the common benefit: and therefore some man may perhaps desire to
know, why Man-kind cannot do the same. To which I answer,

First, that men are continually in competition for Honour and Dignity,
which these creatures are not; and consequently amongst men there
ariseth on that ground, Envy and Hatred, and finally Warre; but
amongst these not so.

Secondly, that amongst these creatures, the Common good differeth not
from the Private; and being by nature enclined to their private, they
procure thereby the common benefit. But man, whose Joy consisteth in
comparing himselfe with other men, can relish nothing but what is
eminent.

Thirdly, that these creatures, having not (as man) the use of reason,
do not see, nor think they see any fault, in the administration of
their common businesse: whereas amongst men, there are very many, that
thinke themselves wiser, and abler to govern the Publique, better than
the rest; and these strive to reforme and innovate, one this way,
another that way; and thereby bring it into Distraction and Civill
warre.

\page{118}Fourthly, that these creatures, though they have some use of
voice, in making knowne to one another their desires, and other
affections; yet they want that art of words, by which some men can
represent to others, that which is Good, in the likenesse of Evill;
and Evill, in the likenesse of Good; and augment, or diminish the
apparent greatnesse of Good and Evill; discontenting men, and
troubling their Peace at their pleasure.

Fiftly, irrationall creatures cannot distinguish betweene
\textit{Injury}, and \textit{Damm\-age}; and therefore as long as they
be at ease, they are not offended with their fellowes: whereas Man is
then most troublesome, when he is most at ease: for then it is that he
loves to shew his Wisdome, and controule the Actions of them that
governe the Common-wealth.

Lastly, the agreement of these creatures is Naturall; that of men, is
by Covenant only, which is Artificiall: and therefore it is no wonder
if there be somewhat else required (besides Covenant) to make their
Agreement constant and lasting; which is a Common Power, to keep them
in awe, and to direct their actions to the Common Benefit.

The only way to erect such a Common Power, as may be able to defend
them from the invasion of Forraigners, and the injuries of one
another, and thereby to secure them in such sort, as that by their
owne industrie, and by the fruites of the Earth, they may nourish
themselves and live contentedly; is, to conferre all their power and
strength upon one Man, or upon one Assembly of men, that may reduce
all their Wills, by plurality of voices, unto one Will: which is as
much as to say, to appoint one Man, or Assembly of men, to beare their
Person; and every one to owne, and acknowledge himselfe to be Author
of whatsoever he that so beareth their Person, shall Act, or cause to
be Acted, in those things which concerne the Common Peace and Safetie;
and therein to submit their Wills, every one to his Will, and their
Judgements, to his Judgment. This is more than Consent, or Concord; it
is a reall Unitie of them all, in one and the same Person, made by
Covenant of every man with every man, in such manner, as if every man
should say to every man, \textit{I Authorise and give up my Right of
Governing my selfe, to this Man, or to this Assembly of men, on
\page{119} this condition, that thou give up thy Right to him, and
Authorise all his Actions in like manner}. This done, the Multitude so
united in one Person, is called a \textsc{Common-wealth}, in latine
\textsc{Civitas}. This is the Generation of that great
\textsc{Leviathan}, or rather (to speake more reverently) of that
\textit{Mortall God}, to which wee owe under the \textit{Immortall
God}, our peace and defence. For by this Authoritie, given him by
every particular man in the Common-Wealth, he hath the use of so much
Power and Strength conferred on him, that by terror thereof, he is
inabled to forme the wills of them all, to Peace at home, and mutuall
ayd against their enemies abroad. And in him consisteth the Essence of
the Common-wealth; which (to define it,) is \textit{One Person, of
whose Acts a great Multitude, by mutuall Covenants one with another,
have made themselves every one the Author, to the end he may use the
strength and means of them all, as he shall think expedient, for their
Peace and Common Defence}.

And he that carryeth this Person, is called \textsc{Soveraigne}, and
said to have \textit{Sov\-er\-aigne Power}; and every one besides, his
\textsc{Subject}.

The attaining to this Soveraigne Power, is by two wayes. One, by
Naturall force; as when a man maketh his children, to submit
themselves, and their children to his government, as being able to
destroy them if they refuse; or by Warre subdueth his enemies to his
will, giving them their lives on that condition. The other, is when
men agree amongst themselves, to submit to some Man, or Assembly of
men, voluntarily, on confidence to be protected by him against all
others. This later, may be called a Politicall Common-wealth, or
Common-wealth by \textit{Institution}; and the former, a Common-wealth
by \textit{Acquisition}. And first, I shall speak of a Common-wealth
by Institution.

