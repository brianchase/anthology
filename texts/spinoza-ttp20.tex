
\author{Benedict de Spinoza}
\authdate{1632--1677}
\textdate{1670}
\chapter[Benedict de Spinoza -- A Theological-Political Treatise,
chap. 20]{A Theological-Political Treatise, chap. 20}

%\page{257}\section*{Chapter XX. That in a Free State Every Man May
%Think What He Likes, and Say What He Thinks}

\nfootnote{\fullcite{spinoza1900.1.1.20}}

\page{257}If men's minds were as easily controlled as their tongues,
every king would sit safely on his throne, and government by
compulsion would cease; for every subject would shape his life
according to the intentions of his rulers, and would esteem a thing
true or false, good or evil, just or unjust, in obedience to their
dictates. However, we have shown already (Chapter XVII.) that no man's
mind can possibly lie wholly at the disposition of another, for no
one can willingly transfer his natural right of free reason and
judgment, or be compelled so to do. For this reason government which
attempts to control minds is accounted tyrannical, and it is
considered an abuse of sovereignty and a usurpation of the rights of
subjects, to seek to prescribe what shall be accepted as true, or
rejected as false, or what opinions should actuate men in their
worship of God. All these questions fall within a man's natural right,
which he cannot abdicate even with his own consent.

I admit that the judgment can be biassed in many ways, and to an
almost incredible degree, so that while exempt from direct external
control it may be so dependent on another man's words, that it may
fitly be said to be ruled by him; but although this influence is
carried to great lengths, it has never gone so far as to invalidate
the statement, that every man's understanding is his own, and that
brains are as diverse as palates.

Moses, not by fraud, but by Divine virtue, gained such a hold over the
popular judgment that he was accounted superhuman, and believed to
speak and act through the inspiration of the Deity; nevertheless, even
he could not escape murmurs and evil interpretations. How much less
then can other monarchs avoid them! Yet such unlimited power, if it
exists at all, must belong to a monarch, and \page{258} least of all
to a democracy, where the whole or a great part of the people wield
authority collectively. This is a fact which I think everyone can
explain for himself.

However unlimited, therefore, the power of a sovereign may be, however
implicitly it is trusted as the exponent of law and religion, it can
never prevent men from forming judgments according to their intellect,
or being influenced by any given emotion. It is true that it has the
right to treat as enemies all men whose opinions do not, on all
subjects, entirely coincide with its own; but we are not discussing
its strict rights, but its proper course of action. I grant that it
has the right to rule in the most violent manner, and to put citizens
to death for very trivial causes, but no one supposes it can do this
with the approval of sound judgment. Nay, inasmuch as such things
cannot be done without extreme peril to itself, we may even deny that
it has the absolute power to do them, or, consequently, the absolute
right; for the rights of the sovereign are limited by his power.

Since, therefore, no one can abdicate his freedom of judgment and
feeling; since every man is by indefeasible natural right the master
of his own thoughts, it follows that men thinking in diverse and
contradictory fashions, cannot, without disastrous results, be
compelled to speak only according to the dictates of the supreme
power. Not even the most experienced, to say nothing of the multitude,
know how to keep silence. Men's common failing is to confide their
plans to others, though there be need for secrecy, so that a
government would be most harsh which deprived the individual of his
freedom of saying and teaching what he thought; and would be moderate
if such freedom were granted. Still we cannot deny that authority may
be as much injured by words as by actions; hence, although the freedom
we are discussing cannot be entirely denied to subjects, its unlimited
concession would be most baneful; we must, therefore, now inquire, how
far such freedom can and ought to be conceded without danger to the
peace of the state, or the power of the rulers; and this, as I said at
the beginning of Chapter XVI., is my principal object.

It follows, plainly, from the explanation given above, of the
foundations of a state, that the ultimate aim of govern-\page{259}ment
is not to rule, or restrain, by fear, nor to exact obedience, but
contrariwise, to free every man from fear, that he may live in all
possible security; in other words, to strengthen his natural right
to exist and work without injury to himself or others.

No, the object of government is not to change men from rational beings
into beasts or puppets, but to enable them to develope their minds and
bodies in security, and to employ their reason unshackled; neither
showing hatred, anger, or deceit, nor watched with the eyes of
jealousy and injustice. In fact, the true aim of government is
liberty.

Now we have seen that in forming a state the power of making laws must
either be vested in the body of the citizens, or in a portion of them,
or in one man. For, although men's free judgments are very diverse,
each one thinking that he alone knows everything, and although
complete unanimity of feeling and speech is out of the question, it is
impossible to preserve peace, unless individuals abdicate their right
of acting entirely on their own judgment. Therefore, the individual
justly cedes the right of free action, though not of free reason and
judgment; no one can act against the authorities without danger to the
state, though his feelings and judgment may be at variance therewith;
he may even speak against them, provided that he does so from rational
conviction, not from fraud, anger, or hatred, and provided that he
does not attempt to introduce any change on his private authority.

For instance, supposing a man shows that a law is repugnant to sound
reason, and should therefore be repealed; if he submits his opinion to
the judgment of the authorities (who, alone, have the right of making
and repealing laws), and meanwhile acts in nowise contrary to that
law, he has deserved well of the state, and has behaved as a good
citizen should; but if he accuses the authorities of injustice, and
stirs up the people against them, or if he seditiously strives to
abrogate the law without their consent, he is a mere agitator and
rebel.

Thus we see how an individual may declare and teach what he believes,
without injury to the authority of his rulers, or to the public peace;
namely, by leaving in their hands the entire power of legislation as
it affects action, \page{260} and by doing nothing against their laws,
though he be compelled often to act in contradiction to what he
believes, and openly feels, to be best.

Such a course can be taken without detriment to justice and
dutifulness, nay, it is the one which a just and dutiful man would
adopt. We have shown that justice is dependent on the laws of the
authorities, so that no one who contravenes their accepted decrees can
be just, while the highest regard for duty, as we have pointed out in
the preceding chapter, is exercised in maintaining public peace and
tranquillity; these could not be preserved if every man were to live
as he pleased; therefore it is no less than undutiful for a man to act
contrary to his country's laws, for if the practice became universal
the ruin of states would necessarily follow.

Hence, so long as a man acts in obedience to the laws of his rulers,
he in nowise contravenes his reason, for in obedience to reason he
transferred the right of controlling his actions from his own hands to
theirs. This doctrine we can confirm from actual custom, for in a
conference of great and small powers, schemes are seldom carried
unanimously, yet all unite in carrying out what is decided on, whether
they voted for or against. But I return to my proposition.

From the fundamental notions of a state, we have discovered how a man
may exercise free judgment without detriment to the supreme power:
from the same premises we can no less easily determine what opinions
would be seditious. Evidently those which by their very nature nullify
the compact by which the right of free action was ceded. For instance,
a man who holds that the supreme power has no rights over him, or that
promises ought not to be kept, or that everyone should live as he
pleases, or other doctrines of this nature in direct opposition to the
above-mentioned contract, is seditious, not so much from his actual
opinions and judgment, as from the deeds which they involve; for he
who maintains such theories abrogates the contract which tacitly, or
openly, he made with his rulers. Other opinions which do not involve
acts violating the contract, such as revenge, anger, and the like, are
not seditious, unless it be in some corrupt state, where superstitious
and ambitious persons, unable to endure men of \page{261} learning,
are so popular with the multitude that their word is more valued than
the law.

However, I do not deny that there are some doctrines which, while they
are apparently only concerned with abstract truths and falsehoods, are
yet propounded and published with unworthy motives. This question we
have discussed in Chapter XV., and shown that reason should
nevertheless remain unshackled. If we hold to the principle that a
man's loyalty to the state should be judged, like his loyalty to God,
from his actions on\-ly---name\-ly, from his charity towards his
neighbours; we cannot doubt that the best government will allow
freedom of philosophical speculation no less than of religious belief.
I confess that from such freedom inconveniences may sometimes arise,
but what question was ever settled so wisely that no abuses could
possibly spring therefrom? He who seeks to regulate everything by law,
is more likely to arouse vices than to reform them. It is best to
grant what cannot be abolished, even though it be in itself harmful.
How many evils spring from luxury, envy, avarice, drunkenness, and the
like, yet these are tol\-er\-at\-ed---vices as they are---be\-cause
they cannot be prevented by legal enactments. How much more then
should free thought be granted, seeing that it is in itself a virtue
and that it cannot be crushed! Besides, the evil results can easily be
checked, as I will show, by the secular authorities, not to mention
that such freedom is absolutely necessary for progress in science and
the liberal arts: for no man follows such pursuits to advantage unless
his judgment be entirely free and unhampered.

But let it be granted that freedom may be crushed, and men be so bound
down, that they do not dare to utter a whisper, save at the bidding of
their rulers; nevertheless this can never be carried to the pitch of
making them think according to authority, so that the necessary
consequences would be that men would daily be thinking one thing and
saying another, to the corruption of good faith, that mainstay of
government, and to the fostering of hateful flattery and perfidy,
whence spring stratagems, and the corruption of every good art.

It is far from possible to impose uniformity of speech, for the more
rulers strive to curtail freedom of speech, the \page{262} more
obstinately are they resisted; not indeed by the avaricious, the
flatterers, and other numskulls, who think supreme salvation consists
in filling their stomachs and gloating over their money-bags, but by
those whom good education, sound morality, and virtue have rendered
more free. Men, as generally constituted, are most prone to resent the
branding as criminal of opinions which they believe to be true, and
the proscription as wicked of that which inspires them with piety
towards God and man; hence they are ready to forswear the laws and
conspire against the authorities, thinking it not shameful but
honourable to stir up seditions and perpetuate any sort of crime with
this end in view. Such being the constitution of human nature, we see
that laws directed against opinions affect the generous-minded rather
than the wicked, and are adapted less for coercing criminals than for
irritating the upright; so that they cannot be maintained without
great peril to the state.

Moreover, such laws are almost always useless, for those who hold that
the opinions proscribed are sound, cannot possibly obey the law;
whereas those who already reject them as false, accept the law as a
kind of privilege, and make such boast of it, that authority is
powerless to repeal it, even if such a course be subsequently desired.

To these considerations may be added what we said in Chapter XVIII. in
treating of the history of the Hebrews. And, lastly, how many schisms
have arisen in the Church from the attempt of the authorities to
decide by law the intricacies of theological controversy! If men were
not allured by the hope of getting the law and the authorities on
their side, of triumphing over their adversaries in the sight of an
applauding multitude, and of acquiring honourable distinctions, they
would not strive so maliciously, nor would such fury sway their minds.
This is taught not only by reason but by daily examples, for laws of
this kind prescribing what every man shall believe and forbidding
anyone to speak or write to the contrary, have often been passed, as
sops or concessions to the anger of those who cannot tolerate men of
enlightenment, and who, by such harsh and crooked enactments, can
easily turn the devotion of the masses into fury and direct it against
whom they will.

\page{263}How much better would it be to restrain popular anger and
fury, instead of passing useless laws, which can only be broken by
those who love virtue and the liberal arts, thus paring down the state
till it is too small to harbour men of talent. What greater
misfortune for a state can be conceived than that honourable men
should be sent like criminals into exile, because they hold diverse
opinions which they cannot disguise? What, I say, can be more hurtful
than that men who have committed no crime or wickedness should, simply
because they are enlightened, be treated as enemies and put to death,
and that the scaffold, the terror of evil-doers, should become the
arena where the highest examples of tolerance and virtue are displayed
to the people with all the marks of ignominy that authority can
devise?

He that knows himself to be upright does not fear the death of a
criminal, and shrinks from no punishment; his mind is not wrung with
remorse for any disgraceful deed: he holds that death in a good cause
is no punishment, but an honour, and that death for freedom is glory.

What purpose then is served by the death of such men, what example is
proclaimed? the cause for which they die is unknown to the idle and
the foolish, hateful to the turbulent, loved by the upright. The only
lesson we can draw from such scenes is to flatter the persecutor, or
else to imitate the victim.

If formal assent is not to be esteemed above conviction, and if
governments are to retain a firm hold of authority and not be
compelled to yield to agitators, it is imperative that freedom of
judgment should be granted, so that men may live together in harmony,
however diverse, or even openly contradictory their opinions may be.
We cannot doubt that such is the best system of government and open to
the fewest objections, since it is the one most in harmony with human
nature. In a democracy (the most natural form of government, as we
have shown in Chapter XVI.) everyone submits to the control of
authority over his actions, but not over his judgment and reason; that
is, seeing that all cannot think alike, the voice of the majority has
the force of law, subject to repeal if circumstances bring about a
change of opinion. In proportion as the \page{264} power of free
judgment is withheld we depart from the natural condition of mankind,
and consequently the government becomes more tyrannical.

In order to prove that from such freedom no inconvenience arises,
which cannot easily be checked by the exercise of the sovereign power,
and that men's actions can easily be kept in bounds, though their
opinions be at open variance, it will be well to cite an example. Such
an one is not very far to seek. The city of Amsterdam reaps the fruit
of this freedom in its own great prosperity and in the admiration of
all other people. For in this most flourishing state, and most
splendid city, men of every nation and religion live together in the
greatest harmony, and ask no questions before trusting their goods to
a fellow-citizen, save whether he be rich or poor, and whether he
generally acts honestly, or the reverse. His religion and sect is
considered of no importance: for it has no effect before the judges in
gaining or losing a cause, and there is no sect so despised that its
followers, provided that they harm no one, pay every man his due, and
live uprightly, are deprived of the protection of the magisterial
authority.

On the other hand, when the religious controversy between Remonstrants
and Counter-Remonstrants began to be taken up by politicians and the
States, it grew into a schism, and abundantly showed that laws dealing
with religion and seeking to settle its controversies are much more
calculated to irritate than to reform, and that they give rise to
extreme licence: further, it was seen that schisms do not originate in
a love of truth, which is a source of courtesy and gentleness, but
rather in an inordinate desire for supremacy. From all these
considerations it is clearer than the sun at noonday, that the true
schismatics are those who condemn other men's writings, and
seditiously stir up the quarrelsome masses against their authors,
rather than those authors themselves, who generally write only for the
learned, and appeal solely to reason. In fact, the real disturbers of
the peace are those who, in a free state, seek to curtail the liberty
of judgment which they are unable to tyrannize over.

I have thus shown:---I. That it is impossible to deprive men of the
liberty of saying what they think. II. That \page{265} such liberty
can be conceded to every man without injury to the rights and
authority of the sovereign power, and that every man may retain it
without injury to such rights, provided that he does not presume upon
it to the extent of introducing any new rights into the state, or
acting in any way contrary to the existing laws. III. That every man
may enjoy this liberty without detriment to the public peace, and that
no inconveniences arise therefrom which cannot easily be checked. IV.
That every man may enjoy it without injury to his allegiance. V. That
laws dealing with speculative problems are entirely useless. VI.
Lastly, that not only may such liberty be granted without prejudice to
the public peace, to loyalty, and to the rights of rulers, but that it
is even necessary for their preservation. For when people try to take
it away, and bring to trial, not only the acts which alone are capable
of offending, but also the opinions of mankind, they only succeed in
surrounding their victims with an appearance of martyrdom, and raise
feelings of pity and revenge rather than of terror. Uprightness and
good faith are thus corrupted, flatterers and traitors are encouraged,
and sectarians triumph, inasmuch as concessions have been made to
their animosity, and they have gained the state sanction for the
doctrines of which they are the interpreters. Hence they arrogate to
themselves the state authority and rights, and do not scruple to
assert that they have been directly chosen by God, and that their laws
are Divine, whereas the laws of the state are human, and should
therefore yield obedience to the laws of God---in other words, to
their own laws. Everyone must see that this is not a state of affairs
conducive to public welfare. Wherefore, as we have shown in Chapter
XVIII., the safest way for a state is to lay down the rule that
religion is comprised solely in the exercise of charity and justice,
and that the rights of rulers in sacred, no less than in secular
matters, should merely have to do with actions, but that every man
should think what he likes and say what he thinks.

I have thus fulfilled the task I set myself in this treatise. It
remains only to call attention to the fact that I have written nothing
which I do not most willingly submit to the examination and approval
of my country's rulers; and \page{266} that I am willing to retract
anything which they shall decide to be repugnant to the laws, or
prejudicial to the public good. I know that I am a man, and as a man
liable to error, but against error I have taken scrupulous care, and
have striven to keep in entire accordance with the laws of my country,
with loyalty, and with morality.

