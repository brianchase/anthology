
\author{G. W. Leibniz}
\authdate{1646--1716}
\textdate{1714}
\chapter{The Monadology}
\source{leibniz1918.3}

% NOTE: the source has 'means' below

\page{249}1. The \textit{monad}, of which we will speak here, is
nothing else than a simple substance which goes to make up composites;
by simple, we mean without parts.

2. There must be simple substances because there are composites; for a
composite is nothing else than a collection or \textit{aggregatum} of
simple substances.

3. Now where there are no constituent parts there is possible neither
extension, nor form, nor divisibility. These monads are the true atoms
of nature, and, in fact, the elements of things.

4. Their dissolution, therefore, is not to be feared, and there is no
way conceivable by which a simple substance can perish through natural
means.

5. For the same reason there is no way conceivable by which a simple
substance might, through natural means, come into existence, since it
can not be formed by composition.

% NOTE: added comma after 'however'

6. We may say then, that the existence of monads can begin or end only
all at once, that is to say, the monad can begin only through creation
and end only through annihilation. Composites, however, begin or end
gradually.

7. There is also no way of explaining how a monad can be altered or
changed in its inner being by any \page{250} other created thing,
since there is no possibility of transposition within it, nor can we
conceive of any internal movement which can be produced, directed,
increased or diminished there within the substance, such as can take
place in the case of composites where a change can occur among the
parts. The monads have no windows through which anything may come in
or go out. The attributes are not liable to detach themselves and make
an excursion outside the substance, as could \textit{sensible species}
of the Schoolmen. In the same way neither substance nor attribute can
enter from without into a monad.

8. Still monads must needs have some qualities, otherwise they would
not even be existences. And if simple substances did not differ at all
in their qualities, there would be no means of perceiving any change
in things. Whatever is in a composite can come into it only through
its simple elements, and if the monads were without qualities they would be
indistinguishable one from another since they do not differ at all in
quantity. For instance, if we imagine a \textit{plenum} or completely
filled space, where each part receives only the equivalent of its own
previous motion, one state of things would not be distinguishable from
another.

9. Each monad, indeed, must be different from every other. For there
are never in nature two beings which are exactly alike, and in which
it is not possible to find a difference either internal or based on an
intrinsic property.

10. I assume it as admitted that every created being, and consequently
the created monad, is subject to change, and indeed that this change
is continuous in each.

11. It follows from what has just been said, that \page{251} the
natural changes of the monad come from an internal principle, because
an external cause can have no influence upon its inner being.

12. Now besides this principle of change there must also be in the
monad a manifoldness which changes. This manifoldness constitutes, so
to speak, the specific nature and the variety of the simple
substances.

13. This manifoldness must involve a multiplicity in the unity or in
that which is simple. For since every natural change takes place by
degrees, there must be something which changes and something which
remains unchanged, and consequently there must be in the simple
substance a plurality of conditions and relations, even though it has
no parts.

14. The passing condition which involves and represents a multiplicity
in the unity, or in the simple substance, is nothing else than what is
called perception. This should be carefully distinguished from
apperception or consciousness, as will appear in what follows. In this
matter the Cartesians have fallen into a serious error, in that they
treat as non-existent those perceptions of which we are not conscious.
It is this also which has led them to believe that spirits alone are
monads and that there are no souls of animals or other entelechies,
and it has led them to make the common confusion between a protracted
period of unconsciousness and actual death. They have thus adopted the
scholastic error that souls can exist entirely separated from bodies,
and have even confirmed ill-balanced minds in the belief that souls
are mortal.

15. The action of the internal principle which brings about the change
or the passing from one perception to another may be called
appetition. It is true that the desire (\textit{l'app\'etit}) is not
always able to attain to \page{252} the whole of the perception which
it strives for, but it always attains a portion of it and reaches new
perceptions.

16. We, ourselves, experience a multiplicity in a simple substance,
when we find that the most trifling thought of which we are conscious
involves a variety in the object. Therefore all those who acknowledge
that the soul is a simple substance ought to grant this multiplicity
in the monad, and M. Bayle should have found no difficulty in it, as
he has done in his dictionary, article ``Rorarius.''

17. It must be confessed, however, that perception, and that which
depends upon it, are inexplicable by mechanical causes, that is to
say, by figures and motions. Supposing that there were a machine whose
structure produced thought, sensation, and perception, we could
conceive of it as increased in size with the same proportions until
one was able to enter into its interior as he would into a mill. Now,
on going into it he would find only pieces working upon one another,
but never would he find anything to explain perception. It is
accordingly in the simple substance, and not in the composite nor in a
machine that the perception is to be sought. Furthermore, there is
nothing besides perceptions and their changes to be found in the
simple substance. And it is in these alone that all the internal
activities of the simple substance can consist.

18. All simple substances or created monads may be called
\textit{entelechies}, because they have in themselves a certain
perfection (\grk{ἔχουσι τὸ ἐντελές}). There is in them a sufficiency
(\grk{αὐτάρκεια}) which makes them the source of their internal
activities, and renders them, so to speak, incorporeal automatons.

\page{253}19. If we wish to designate as soul everything which has
perceptions and desires in the general sense that I have just
explained, all simple substances or created monads could be called
souls. But since feeling is something more than a mere perception I
think that the general name of monad or entelechy should suffice for
simple substances which have only perception, while we may reserve the
term \textit{soul} for those whose perception is more distinct and is
accompanied by memory.

20. We experience in ourselves a state where we remember nothing and
where we have no distinct perception, as in periods of fainting, or
when we are overcome by a profound, dreamless sleep. In such a state
the soul does not sensibly differ at all from a simple monad. As this
state, however, is not permanent and the soul can recover from it, the
soul is something more.

21. Nevertheless it does not follow at all that the simple substance
is in such a state without perception. This is so because of the
reasons given above; for it cannot perish, nor on the other hand would
it exist without some affection and the affection is nothing else than
its perception. When, however, there are a great number of little
perceptions where nothing stands out distinctively, we are stunned; as
when one turns around and around in the same direction, a dizziness
comes on which makes him swoon and makes him able to distinguish
nothing. Among animals death can occasion this state for quite a
period.

22. Every present state of a simple substance is a natural consequence
of its preceding state, in such a way that its present is pregnant
with its future.

23. Therefore, since on awakening after a period of \page{254}
unconsciousness we become conscious of our perceptions, we must have
had perceptions immediately before without having been conscious of
them; for one perception can come in a natural way only from another
perception, just as a motion can come in a natural way only from a
motion.

24. It is evident from this that if we were to have nothing
distinctive, or so to speak prominent and of a higher flavor in our
perceptions, we should be in a continual state of stupor. This is the
condition of monads which are wholly bare.

25. We see that nature has given to animals heightened perceptions,
having provided them with organs which collect numerous rays of light
or numerous waves of air and thus make them more effective in their
combination. Something similar to this takes place in the case of
smell, in that of taste and of touch, and perhaps in many other senses
which are unknown to us. I shall have occasion very soon to explain
how that which occurs in the soul represents that which goes on in the
sense-organs.

26. The memory furnishes souls with a sort of consecutiveness which
imitates reason but is to be distinguished from it. We see that when
animals have the perception of something which strikes their attention
and of which they have had a similar previous perception, they are led
by the representation of their memory to expect that which was
associated in the preceding perception, and they come to have feelings
like those which they had before. For instance, if a stick be shown to
a dog, he remembers the pain which it has caused him and he whines or
runs away.

27. The vividness of the picture, which comes to him or moves him, is
derived either from the magni-\page{255}tude or from the number of the
previous perceptions. For often a strong impression brings about all
at once the same effect as a long-continued habit or as a great many
re-iterated, moderate perceptions.

28. Men act in like manner as animals in so far as the sequence of
their perceptions is determined only by the law of memory, resembling
the \textit{empirical physicians} who practice simply without any
theory, and we are empiricists in three-fourths of our actions. For
instance, when we expect that there will be day-light to-morrow, we do
so empirically, because it has always happened so up to the present
time. It is only the astronomer who uses his reason in making such an
affirmation.

29. But the knowledge of eternal and necessary truths is that which
distinguishes us from mere animals and gives us reason and the
sciences, thus raising us to a knowledge of ourselves and of God. This
is what is called in us the \textit{rational soul} or the
\textit{spirit}.

30. It is also through the knowledge of necessary truths and through
the abstractions they involve that we come to perform reflective acts,
which cause us to think of what is called the I, and to decide that
this or that is within us. It is thus that in thinking upon ourselves
we think of \textit{being}, of \textit{substance}, of the
\textit{simple} and \textit{composite}, of the \textit{immaterial},
and of \textit{God} himself, conceiving that what is limited in us is
in him without limits. These reflective acts furnish the principal
objects about which we reason.

31. Our reasoning is based upon two great principles: first, that of
\textit{contradiction}, by means of which we decide that to be false
which involves contradiction and that to be true which contradicts or
is opposed to the false.

32. And second, the principle of \textit{sufficient reason},
\page{256} in virtue of which we believe that no fact can be real or
existing and no statement true unless it has a sufficient reason why
it should be thus and not otherwise. Most frequently, however, these
reasons cannot be known by us.

33. There are also two kinds of truths: those of reasoning and those
of fact. The \textit{truths of reasoning} are necessary and their
opposite is impossible. Those of \textit{fact}, however, are
contingent, and their opposite is possible. When a truth is necessary,
the reason can be found by analysis in resolving it into simpler ideas
and into simpler truths until we reach those which are primary.

34. It is thus that with mathematicians the speculative theorems and
the practical canons are reduced by analysis to definitions, axioms,
and postulates.

35. There are, finally, simple ideas of which no definition can be
given. There are also the axioms and postulates or, in a word, the
primary principles which cannot be proved and, indeed, have no need of
proof. These are identical propositions whose opposites involve
express contradictions.

36. But there must be also a sufficient reason for contingent truths
or truths of fact; that is to say, for the sequence of the things
which extend throughout the universe of created beings, where the
analysis into more particular reasons can be continued into greater
detail without limit because of the immense variety of the things in
nature and because of the infinite division of bodies. There is an
infinity of figures and of movements, present and past, which enter
into the efficient cause of my present writing, and in its final cause
there are an infinity of slight tendencies and dispositions of my
soul, present and past.

\page{257}37. And as all this detail again involves other and more
detailed contingencies, each of which again has need of a similar
analysis in order to find its explanation, no real advance has been
made. Therefore, the sufficient or ultimate reason must needs be
outside of the sequence or series of this manifold of contingencies,
however infinite they may be.

38. It is thus that the ultimate reason for things must be found in a
necessary substance, in which the details of the changes shall be
present merely eminently, as in the fountain-head, and this substance
we call God.

39. Now since this substance is a sufficient reason for all the
above-mentioned details which are linked together throughout,
\textit{there is but one God, and this God is sufficient}.

40. We may hold that the supreme substance, which is unique, universal
and necessary with nothing independent outside of it, and which is a
direct consequence of possible being, must be incapable of limitation
and must contain as much reality as possible.

41. Whence it follows that God is absolutely perfect, perfection being
understood as the magnitude of positive reality in the strict sense,
when the limitations or the bounds of those things which have them are
removed. Where there are no limits, that is to say, in God, perfection
is absolutely infinite.

42. It follows also that created things derive their perfections
through the influence of God, but their imperfections come from their
own natures, which cannot be unlimited. It is in this latter that they
are distinguished from God. An example of this original imperfection
of created things is to be found in the natural inertia of bodies.

\page{258}43. It is true, furthermore, that in God is found not only
the source of existences but also that of essences, in so far as they
are real. In other words, he is the source of whatever there is real
in the possible. This is because the understanding of God constitutes
the region of eternal truths or of the ideas upon which they depend,
and because without him there would be nothing real in the
possibilities of things, and not only would nothing be existent,
nothing would be even possible.

44. For it must needs be that if there is a reality in essences or in
possibilities or indeed in the eternal truths, this reality is based
upon something existent and actual, and consequently in the existence
of the necessary Being in whom essence includes existence or in whom
possibility is sufficient to produce actuality.

45. Therefore God alone (or the Necessary Being) has this prerogative,
that if he be possible he must necessarily exist and, as nothing is
able to prevent the possibility of that which involves no bounds, no
negation, and consequently no contradiction, this alone is sufficient
to establish \textit{a priori} his existence. We have, therefore,
proved his existence through the reality of eternal truths. But a
little while ago we also proved it \textit{a posteriori}, because
contingent beings exist which can have their ultimate and sufficient
reason only in the necessary being which, in turn, has the reason for
existence in itself.

% NOTE: changed 'necessarily truths' to 'necessary truths'

46. Yet we must not think that the eternal truths being dependent upon
God are therefore arbitrary and depend upon his will, as Descartes
seems to have held, and after him M. Poiret. This is the case only
with contingent truths which depend upon fitness or \page{259} the
choice of the greatest good; necessary truths on the other hand depend
solely upon his understanding and are the inner objects of it.

47. God alone is the ultimate unity or the original simple substance,
of which all created or derivative monads are the products, and arise,
so to speak, through the continual outflashings (fulgurations) of the
divinity from moment to moment, limited by the receptivity of the
creature to whom limitation is an essential.

48. In God are present: power, which is the source of everything;
knowledge, which contains the details of the ideas; and, finally,
will, which changes or produces things in accordance with the
principle of the greatest good. To these correspond in the created
monad, the subject or basis, the faculty of perception, and the
faculty of appetition. In God these attributes are absolutely infinite
or perfect, while in the created monads or in the entelechies
(\textit{perfectihabies}, as Hermolaus Barbarus translates this
word), they are imitations approaching him in proportion to their
perfection.

49. A created thing is said to act outwardly in so far as it has
perfection, and to be acted upon by another in so far as it is
imperfect. Thus action is attributed to the monad in so far as it has
distinct perceptions, and passion or passivity is attributed in so far
as it has confused perceptions.

50. One created thing is more perfect than another when we find in the
first that which gives an \textit{a priori} reason for what occurs in
the second. This is why we say that one acts upon the other.

51. In the case of simple substances, the influence which one monad
has upon another is only ideal. It \page{260} can have its effect only
through the mediation of God, in so far as in the ideas of God each
monad can rightly demand that God, in regulating the others from the
beginning of things, should have regarded it also. For since one
created monad cannot have a physical influence upon the inner being of
another, it is only through this primal regulation that one can have
dependence upon another.

52. It is thus that among created things action and passion are
reciprocal. For God, in comparing two simple substances, finds in each
one reasons obliging him to adapt the other to it; and consequently
what is active in certain respects is passive from another point of
view, active in so far as what we distinctly know in it serves to give
a reason for that which occurs in another, and passive in so far as
the reason for what occurs in it is found in what is distinctly known
in another.

53. Now as there are an infinity of possible universes in the ideas of
God, and but one of them can exist, there must be a sufficient reason
for the choice of God which determines him to select one rather than
another.

54. And this reason is to be found only in the fitness or in the
degree of perfection which these worlds possess, each possible thing
having the right to claim existence in proportion to the perfection
which it involves.

55. This is the cause for the existence of the greatest good; namely,
that the wisdom of God permits him to know it, his goodness causes him
to choose it, and his power enables him to produce it.

56. Now this interconnection, relationship, or this adaptation of all
things to each particular one, and \page{261} of each one to all the
rest, brings it about that every simple substance has relations which
express all the others and that it is consequently a perpetual living
mirror of the universe.

% NOTE: no comma after 'as it were'

57. And as the same city regarded from different sides appears
entirely different, and is, as it were multiplied respectively, so,
because of the infinite number of simple substances, there are a
similar infinite number of universes which are, nevertheless, only the
aspects of a single one as seen from the special point of view of each
monad.

58. Through this means has been obtained the greatest possible
variety, together with the greatest order that may be; that is to say,
through this means has been obtained the greatest possible
perfection.

59. This hypothesis, moreover, which I venture to call demonstrated,
is the only one which fittingly gives proper prominence to the
greatness of God. M. Bayle recognized this when in his dictionary
(article ``Rorarius'') he raised objections to it; indeed, he was
inclined to believe that I attributed too much to God, and more than
it is possible to attribute to him. But he was unable to bring forward
any reason why this universal harmony which causes every substance to
express exactly all others through the relation which it has with them
is impossible.

60. Besides, in what has just been said can be seen the \textit{a
priori} reasons why things cannot be otherwise than they are. It is
because God, in ordering the whole, has had regard to every part and
in particular to each monad; and since the monad is by its very nature
\textit{representative}, nothing can limit it to represent merely a
part of things. It is nevertheless true that this representation is,
as regards the details of the \page{262} whole universe, only a
confused representation, and is distinct only as regards a small part
of them, that is to say, as regards those things which are nearest or
greatest in relation to each monad. If the representation were
distinct as to the details of the entire universe, each monad would be
a Deity. It is not in the object represented that the monads are
limited, but in the modifications of their knowledge of the object. In
a confused way they reach out to infinity or to the whole, but are
limited and differentiated in the degree of their distinct
perceptions.

61. In this respect composites are like simple substances, for all
space is filled up; therefore, all matter is connected. And in a
plenum or filled space every movement has an effect upon bodies in
proportion to their distance, so that not only is every body affected
by those which are in contact with it and responds in some way to
whatever happens to them, but also by means of them the body responds
to those bodies adjoining them, and their intercommunication reaches
to any distance whatsoever. Consequently every body responds to all
that happens in the universe, so that he who saw all could read in
each one what is happening everywhere, and even what has happened and
what will happen. He can discover in the present what is distant both
as regards space and as regards time; \grk{σύμπνοια
πάντα},\footnote{``All things conspire'' is what Leibniz means. See
note in Latta's edition.---A. R. C.} as Hippocrates said. A soul can,
however, read in itself only what is there represented distinctly. It
cannot all at once open up all its folds, because they extend to
infinity.

62. Thus although each created monad represents the whole universe, it
represents more distinctly the \page{263} body which specially
pertains to it and of which it constitutes the entelechy. And as this
body expresses all the universe through the interconnection of all
matter in the plenum, the soul also represents the whole universe in
representing this body, which belongs to it in a particular way.

63. The body belonging to a monad, which is its entelechy or soul,
constitutes together with the entelechy what may be called a
\textit{living being}, and with a soul what is called an
\textit{animal}. Now this body of a living being or of an animal is
always organic, because every monad is a mirror of the universe
according to its own fashion, and, since the universe is regulated
with perfect order there must needs be order also in what represents
it, that is to say in the perceptions of the soul and consequently in
the body through which the universe is represented in the soul.

64. Therefore every organic body of a living being is a kind of divine
machine, or natural automaton, infinitely surpassing all artificial
automatons. Because a machine constructed by man's skill is not a
machine in each of its parts; for instance, the teeth of a brass wheel
have parts or bits which to us are not artificial products and contain
nothing in themselves to show the use to which the wheel was destined
in the machine. The machines of nature, however, that is to say,
living bodies, are still machines in their smallest parts \textit{ad
infinitum}. Such is the difference between nature and art, that is to
say, between divine art and ours.

65. The author of nature has been able to employ this divine and
infinitely marvelous artifice, because each portion of matter is not
only, as the ancients \page{264} rec\-og\-nized, infinitely divisible,
but also because it is really divided without end, every part into
other parts, each one of which has its own proper motion. Otherwise it
would be impossible for each portion of matter to express all the
universe.

66. Whence we see that there is a world of created things, of living
beings, of animals, of entelechies, of souls, in the minutest particle
of matter.

67. Every portion of matter may be conceived as like a garden full of
plants and like a pond full of fish. But every branch of a plant,
every member of an animal, and every drop of the fluids within it, is
also such a garden or such a pond.

68. And although the ground and the air which lies between the plants
of the garden, and the water which is between the fish in the pond,
are not themselves plant or fish, yet they nevertheless contain these,
usually so small however as to be imperceptible to us.

69. There is, therefore, nothing uncultivated, or sterile or dead in
the universe, no chaos, no confusion, save in appearance; somewhat as
a pond would appear at a distance when we could see in it a confused
movement, and so to speak, a swarming of the fish, without however
discerning the fish themselves.

70. It is evident, then, that every living body has a dominating
entelechy, which in animals is the soul. The parts, however, of this
living body are full of other living beings, plants and animals, which
in turn have each one its entelechy or dominating soul.

71. This does not mean, as some who have misunderstood my thought have
imagined, that each soul has a quantity or portion of matter
appropriated to it or attached to itself for ever, and that it
consequently \page{265} owns other inferior living beings destined to
serve it always; because all bodies are in a state of perpetual flux
like rivers, and the parts are continually entering in and passing
out.

72. The soul, therefore, changes its body only gradually and by
degrees, so that it is never deprived all at once of all its organs.
There is frequently a metamorphosis in animals, but never
metempsychosis or a transmigration of souls. Neither are there souls
wholly separate from bodies, nor bodiless spirits. God alone is
without body.

73. This is also why there is never absolute generation or perfect
death in the strict sense, consisting in the separation of the soul
from the body. What we call generation is development and growth, and
what we call death is envelopment and diminution.

74. Philosophers have been much perplexed in accounting for the origin
of forms, entelechies, or souls. To-day, however, when it has been
learned through careful investigations made in plant, insect and
animal life, that the organic bodies of nature are never the product
of chaos or putrefaction, but always come from seeds in which there
was without doubt some preformation, it has been decided that not
only is the organic body already present before conception, but also a
soul in this body, in a word, the animal itself; and it has been
decided that, by means of conception the animal is merely made ready
for a great transformation, so as to become an animal of another sort.
We can see cases somewhat similar outside of generation when grubs
become flies and caterpillars butterflies.

75. These little animals, some of which by conception become large
animals, may be called spermatic. Those \page{266} among them which
remain in their species, that is to say, the greater part, are born,
multiply, and are destroyed, like the larger animals. There are only a
few chosen ones which come out upon a greater stage.

76. This, however, is only half the truth. I believe, therefore, that
if the animal never actually commences in nature, no more does it by
natural means come to an end. Not only is there no generation, but
also there is no entire destruction or absolute death. These
reasonings, carried on \textit{a posteriori} and drawn from
experience, accord perfectly with the principles which I have above
deduced \textit{a priori}.

77. Therefore we may say that not only the soul (the mirror of an
indestructible universe) is indestructible, but also the animal itself
is, although its mechanism is frequently destroyed in parts and
although it puts off and takes on organic coatings.

78. These principles have furnished me the means of explaining on
natural \llb grounds the union, or, rather the conformity between the
soul and the organic body. The soul follows its own laws, and the body
likewise follows its own laws. They are fitted to each other in virtue
of the preestablished harmony between all substances, since they are
all representations of one and the same universe.

79. Souls act in accordance with the laws of final causes through
their desires, ends and means. Bodies act in accordance with the laws
of efficient causes or of motion. The two realms, that of efficient
causes and that of final causes, are in harmony, each with the other.

80. Descartes saw that souls cannot at all impart force to bodies,
because there is always the same quantity of force in matter. Yet he
thought that the soul \page{267} could change the direction of bodies.
This was, however, because at that time the law of nature which
affirms also the conservation of the same total direction in the
motion of matter was not known. If he had known that law, he would
have fallen upon my system of preestablished harmony.

81. According to this system bodies act as if (to suppose the
impossible) there were no souls at all, and souls act as if there were
no bodies, and yet both body and soul act as if the one were
influencing the other.

82. Although I find that essentially the same thing is true of all
living things and animals, which we have just said (namely, that
animals and souls begin from the very commencement of the world and
that they no more come to an end than does the world) nevertheless,
rational animals have this peculiarity, that their little spermatic
animals, as long as they remain such, have only ordinary or sensuous
souls, but those of them which are, so to speak, elevated, attain by
actual conception to human nature, and their sensuous souls are
raised to the rank of reason and to the prerogative of spirits.

83. Among the differences that there are between ordinary souls and
spirits, some of which I have already instanced, there is also this,
that while souls in general are living mirrors or images of the
universe of created things, spirits are also images of the Deity
himself or of the author of nature. They are capable of knowing the
system of the universe, and of imitating some features of it by means
of artificial models, each spirit being like a small divinity in its
sphere.

84. Therefore, spirits are able to enter into a sort of social
relationship with God, and with respect to \page{268} them he is not
only what an inventor is to his machine (as is his relation to the
other created things), but he is also what a prince is to his
subjects, and even what a father is to his children.

85. Whence it is easy to conclude that the totality of all spirits
must compose the city of God, that is to say, the most perfect state
that is possible under the most perfect monarch.

86. This city of God, this truly universal monarchy, is a moral world
within the natural world. It is what is noblest and most divine among
the works of God. And in it consists in reality the glory of God,
because he would have no glory were not his greatness and goodness
known and wondered at by spirits. It is also in relation to this
divine city that God properly has goodness. His wisdom and his power
are shown everywhere.

87. As we established above that there is a perfect harmony between
the two natural realms of efficient and final causes, it will be in
place here to point out another harmony which appears between the
physical realm of nature and the moral realm of grace, that is to say,
between God considered as the architect of the mechanism of the world
and God considered as the monarch of the divine city of spirits.

88. This harmony brings it about that things progress of themselves
toward grace along natural lines, and that this earth, for example,
must be destroyed and restored by natural means at those times when
the proper government of spirits demands it, for chastisement in the
one case and for a reward in the other.

89. We can say also that God, the Architect, satisfies in all respects
God the Law-Giver, that therefore \page{269} sins will bring their own
penalty with them through the order of nature, and because of the very
structure of things, mechanical though it is. And in the same way the
good actions will attain their rewards in mechanical ways through
their relation to bodies, although this cannot and ought not always to
take place without delay.

90. Finally, under this perfect government, there will be no good
action unrewarded and no evil action unpunished; everything must turn
out for the well-being of the good; that is to say, of those who are
not disaffected in this great state, who, after having done their
duty, trust in Providence and who love and imitate, as is meet, the
Author of all Good, delighting in the contemplation of his perfections
according to the nature of that genuine, pure love which finds
pleasure in the happiness of those who are loved. It is for this
reason that wise and virtuous persons work in behalf of everything
which seems conformable to the presumptive or antecedent will of God,
and are, nevertheless, content with what God actually brings to pass
through his secret, consequent and determining will, recognizing that
if we were able to understand sufficiently well the order of the
universe, we should find that it surpasses all the desires of the
wisest of us, and that it is impossible to render it better than it
is, not only for all in general, but also for each one of us in
particular, provided that we have the proper attachment for the author
of all, not only as the Architect and the efficient cause of our
being, but also as our Lord and the Final Cause, who ought to be the
whole goal of our will, and who alone can make us happy.

