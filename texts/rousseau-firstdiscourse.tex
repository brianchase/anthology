
\author{Jean-Jacques Rousseau}
\authdate{1712--1778}
\textdate{1751}
\chapter{A Discourse on the Arts and Sciences}
\source{rousseau1913.2}

\page{129}\begin{center}
\textit{Decipimur specie recti}.---\textsc{Horace}.
\end{center}

\noindent The question before me is, ``Whether the Restoration of the
arts and sciences has had the effect of purifying or corrupting
morals.'' Which side am I to take? That, gentlemen, which becomes an
honest man, who is sensible of his own ignorance, and thinks himself
none the worse for it.

I feel the difficulty of treating this subject fittingly, before the
tribunal which is to judge of what I advance. How can I presume to
belittle the sciences before one of the most learned assemblies in
Europe, to commend ignorance in a famous Academy, and reconcile my
contempt for study with the respect due to the truly learned?

I was aware of these inconsistencies, but not discouraged by them. It
is not science, I said to myself, that I am attacking; it is virtue
that I am defending, and that before virtuous men---and goodness is
even dearer to the good than learning to the learned.

What then have I to fear? The sagacity of the assembly before which I
am pleading? That, I acknowledge, is to be feared; but rather on
account of faults of construction than of the views I hold. Just
sovereigns have never hesitated to decide against themselves in
doubtful cases; and indeed the most advantageous situation in which a
just claim can be, is that of being laid before a just and enlightened
arbitrator, who is judge in his own case.

To this motive, which encouraged me, I may add another which finally
decided me. And this is, that as I have upheld the cause of truth to
the best of my natural abilities, whatever my apparent success, there
is one reward which cannot fail me. That reward I shall find in the
bottom of my heart.

\page{130}\section*{The First Part}

It is a noble and beautiful spectacle to see man raising himself, so
to speak, from nothing by his own exertions; dissipating, by the light
of reason, all the thick clouds in which he was by nature enveloped;
mounting above himself; soaring in thought even to the celestial
regions; like the sun, encompassing with giant strides the vast extent
of the universe; and, what is still grander and more wonderful, going
back into himself, there to study man and get to know his own nature,
his duties and his end. All these miracles we have seen renewed within
the last few generations.

Europe had relapsed into the barbarism of the earliest ages; the
inhabitants of this part of the world, which is at present so highly
enlightened, were plunged, some centuries ago, in a state still worse
than ignorance. A scientific jargon, more despicable than mere
ignorance, had usurped the name of knowledge, and opposed an almost
invincible obstacle to its restoration.

Things had come to such a pass, that it required a complete revolution
to bring men back to common sense. This came at last from the quarter
from which it was least to be expected. It was the stupid Mussulman,
the eternal scourge of letters, who was the immediate cause of their
revival among us. The fall of the throne of Constantine brought to
Italy the relics of ancient Greece; and with these precious spoils
France in turn was enriched. The sciences soon followed literature,
and the art of thinking joined that of writing: an order which may
seem strange, but is perhaps only too natural. The world now began to
perceive the principal advantage of an intercourse with the Muses,
that of rendering mankind more sociable by inspiring them with the
desire to please one another with performances worthy of their mutual
approbation.

The mind, as well as the body, has its needs: those of the body are
the basis of society, those of the mind its ornaments.

So long as government and law provide for the security and well-being
of men in their common life, the arts, literature and the sciences,
less despotic though perhaps more \page{131} powerful, fling garlands
of flowers over the chains which weigh them down. They stifle in men's
breasts that sense of original liberty, for which they seem to have
been born; cause them to love their own slavery, and so make of them
what is called a civilised people.

Necessity raised up thrones; the arts and sciences have made them
strong. Powers of the earth, cherish all talents and protect those who
cultivate them.\footnote{Sovereigns always see with pleasure a taste
for the arts of amusement and superfluity, which do not result in the
exportation of bullion, increase among their subjects. They very well
know that, besides nourishing that littleness of mind which is proper
to slavery, the increase of artificial wants only binds so many more
chains upon the people. Alexander, wishing to keep the Ichthyophages
in a state of dependence, compelled them to give up fishing, and
subsist on the customary food of civilised nations. The American
savages, who go naked, and live entirely on the products of the chase,
have been always impossible to subdue. What yoke, indeed, can be
imposed on men who stand in need of nothing?} Civilised peoples,
cultivate such pursuits: to them, happy slaves, you owe that delicacy
and exquisiteness of taste, which is so much your boast, that
sweetness of disposition and urbanity of manners which make
intercourse so easy and agreeable among you---in a word, the
appearance of all the virtues, without being in possession of one of
them.

It was for this sort of accomplishment, which is by so much the more
captivating as it seems less affected, that Athens and Rome were so
much distinguished in the boasted times of their splendour and
magnificence: and it is doubtless in the same respect that our own age
and nation will excel all periods and peoples. An air of philosophy
without pedantry; an address at once natural and engaging, distant
equally from Teutonic clumsiness and Italian pantomime; these are the
effects of a taste acquired by liberal studies and improved by
conversation with the world. What happiness would it be for those who
live among us, if our external appearance were always a true mirror of
our hearts; if decorum were but virtue; if the maxims we professed
were the rules of our conduct; and if real philosophy were inseparable
from the title of a philosopher! But so many good qualities too seldom
go together; virtue rarely appears in so much pomp and state.

Richness of apparel may proclaim the man of fortune, \page{132} and
elegance the man of taste; but true health and manliness are known by
different signs. It is under the homespun of the labourer, and not
beneath the gilt and tinsel of the courtier, that we should look for
strength and vigour of body.

External ornaments are no less foreign to virtue, which is the
strength and activity of the mind. The honest man is an athlete, who
loves to wrestle stark naked; he scorns all those vile trappings,
which prevent the exertion of his strength, and were, for the most
part, invented only to conceal some deformity.

Before art had moulded our behaviour, and taught our passions to speak
an artificial language, our morals were rude but natural; and the
different ways in which we behaved proclaimed at the first glance the
difference of our dispositions. Human nature was not at bottom better
then than now; but men found their security in the ease with which
they could see through one another, and this advantage, of which we no
longer feel the value, prevented their having many vices.

In our day, now that more subtle study and a more refined taste have
reduced the art of pleasing to a system, there prevails in modern
manners a servile and deceptive conformity; so that one would think
every mind had been cast in the same mould. Politeness requires this
thing; decorum that; ceremony has its forms, and fashion its laws, and
these we must always follow, never the promptings of our own nature.

We no longer dare seem what we really are, but lie under a perpetual
restraint; in the meantime the herd of men, which we call society, all
act under the same circumstances exactly alike, unless very particular
and powerful motives prevent them. Thus we never know with whom we
have to deal; and even to know our friends we must wait for some
critical and pressing occasion; that is, till it is too late; for it
is on those very occasions that such knowledge is of use to us.

What a train of vices must attend this uncertainty! Sincere
friendship, real esteem, and perfect confidence are banished from
among men. Jealousy, suspicion, fear, coldness, reserve, hate and
fraud lie constantly concealed under that uniform and deceitful veil
of politeness; that \page{133} boasted candour and urbanity, for which
we are indebted to the light and leading of this age. We shall no
longer take in vain by our oaths the name of our Creator; but we shall
insult Him with our blasphemies, and our scrupulous ears will take no
offence. We have grown too modest to brag of our own deserts; but we
do not scruple to decry those of others. We do not grossly outrage
even our enemies, but artfully calumniate them. Our hatred of other
nations diminishes, but patriotism dies with it. Ignorance is held in
contempt; but a dangerous scepticism has succeeded it. Some vices
indeed are condemned and others grown dishonourable; but we have still
many that are honoured with the names of virtues, and it is become
necessary that we should either have, or at least pretend to have
them. Let who will extol the moderation of our modern sages, I see
nothing in it but a refinement of intemperance as unworthy of my
commendation as their artificial simplicity.\footnote{``I love,'' said
Montaigne, ``to converse and hold an argument; but only with very few
people, and that for my own gratification. For to do so, by way of
affording amusement for the great, or of making a parade of one's
talents, is, in my opinion, a trade very ill-becoming a man of
honour.'' It is the trade of all our intellectuals, save one.}

Such is the purity to which our morals have attained; this is the
virtue we have made our own. Let the arts and sciences claim the share
they have had in this salutary work. I shall add but one reflection
more; suppose an inhabitant of some distant country should endeavour
to form an idea of European morals from the state of the sciences, the
perfection of the arts, the propriety of our public entertainments,
the politeness of our behaviour, the affability of our conversation,
our constant professions of benevolence, and from those tumultuous
assemblies of people of all ranks, who seem, from morning till night,
to have no other care than to oblige one another. Such a stranger, I
maintain, would arrive at a totally false view of our morality.

Where there is no effect, it is idle to look for a cause: but here the
effect is certain and the depravity actual; our minds have been
corrupted in proportion as the arts and sciences have improved. Will
it be said, that this is a misfortune peculiar to the present age? No,
gentlemen, \page{134} the evils resulting from our vain curiosity are
as old as the world. The daily ebb and flow of the tides are not more
regularly influenced by the moon, than the morals of a people by the
progress of the arts and sciences. As their light has risen above our
horizon, virtue has taken flight, and the same phenomenon has been
constantly observed in all times and places.

Take Egypt, the first school of mankind, that ancient country, famous
for its fertility under a brazen sky; the spot from which Sesostris
once set out to conquer the world. Egypt became the mother of
philosophy and the fine arts; soon she was conquered by Cambyses, and
then successively by the Greeks, the Romans, the Arabs, and finally
the Turks.

Take Greece, once peopled by heroes, who twice vanquished Asia.
Letters, as yet in their infancy, had not corrupted the disposition of
its inhabitants; but the progress of the sciences soon produced a
dissoluteness of manners, and the imposition of the Macedonian yoke:
from which time Greece, always learned, always voluptuous and always a
slave, has experienced amid all its revolutions no more than a change
of masters. Not all the eloquence of Demosthenes could breathe life
into a body which luxury and the arts had once enervated.

It was not till the days of Ennius and Terence that Rome, founded by a
shepherd, and made illustrious by peasants, began to degenerate. But
after the appearance of an Ovid, a Catullus, a Martial, and the rest
of those numerous obscene authors, whose very names are enough to put
modesty to the blush, Rome, once the shrine of virtue, became the
theatre of vice, a scorn among the nations, and an object of derision
even to barbarians. Thus the capital of the world at length submitted
to the yoke of slavery it had imposed on others, and the very day of
its fall was the eve of that on which it conferred on one of its
citizens the title of Arbiter of Good Taste.

What shall I say of that metropolis of the Eastern Empire, which, by
its situation, seemed destined to be the capital of the world; that
refuge of the arts and sciences, when they were banished from the rest
of Europe, more perhaps by wisdom than barbarism? The most profligate
debaucheries, the most abandoned villainies, the most atrocious
crimes, plots, murders and assassinations form \page{135} the warp and
woof of the history of Constantinople. Such is the pure source from
which have flowed to us the floods of knowledge on which the present
age so prides itself.

But wherefore should we seek, in past ages, for proofs of a truth, of
which the present affords us ample evidence? There is in Asia a vast
empire, where learning is held in honour, and leads to the highest
dignities in the state. If the sciences improved our morals, if they
inspired us with courage and taught us to lay down our lives for the
good of our country, the Chinese should be wise, free and invincible.
But, if there be no vice they do not practise, no crime with which
they are not familiar; if the sagacity of their ministers, the
supposed wisdom of their laws, and the multitude of inhabitants who
people that vast empire, have alike failed to preserve them from the
yoke of the rude and ignorant Tartars, of what use were their men of
science and literature? What advantage has that country reaped from
the honours bestowed on its learned men? Can it be that of being
peopled by a race of scoundrels and slaves?

Contrast with these instances the morals of those few nations which,
being preserved from the contagion of useless knowledge, have by their
virtues become happy in themselves and afforded an example to the rest
of the world. Such were the first inhabitants of Persia, a nation so
singular that virtue was taught among them in the same manner as the
sciences are with us. They very easily subdued Asia, and possess the
exclusive glory of having had the history of their political
institutions regarded as a philosophical romance. Such were the
Scythians, of whom such wonderful eulogies have come down to us. Such
were the Germans, whose simplicity, innocence and virtue, afforded a
most delightful contrast to the pen of an historian, weary of
describing the baseness and villainies of an enlightened, opulent and
voluptuous nation. Such had been even Rome in the days of its poverty
and ignorance. And such has shown itself to be, even in our own times,
that rustic nation, whose justly renowned courage not even adversity
could conquer, and whose fidelity no example could corrupt.\footnote{I
dare not speak of those happy nations, who did not even know the name
of many vices, which we find it difficult to suppress; the savages of
America, whose simple and natural mode of government Montaigne
preferred, without hesitation, not only to the laws of Plato, but to
the most perfect visions of government philosophy can ever suggest. He
cites many examples, striking for those who are capable of
appreciating them. But, what of all that, says he, they can't run to a
pair of breeches!}

\page{136}It is not through stupidity that the people have preferred
other activities to those of the mind. They were not ignorant that in
other countries there were men who spent their time in disputing idly
about the sovereign good, and about vice and virtue. They knew that
these useless thinkers were lavish in their own praises, and
stigmatised other nations contemptuously as barbarians. But they noted
the morals of these people, and so learnt what to think of their
learning.\footnote{What are we to think was the real opinion of the
Athenians themselves about eloquence, when they were so very careful
to banish declamation from that upright tribunal, against whose
decision even their gods made no appeal? What did the Romans think of
physicians, when they expelled medicine from the republic? And when
the relics of humanity left among the Spaniards induced them to
forbid their lawyers to set foot in America, what must they have
thought of jurisprudence? May it not be said that they thought, by
this single expedient, to make reparation for all the outrages they
had committed against the unhappy Indians?}

Can it be forgotten that, in the very heart of Greece, there arose a
city as famous for the happy ignorance of its inhabitants, as for the
wisdom of its laws; a republic of demi-gods rather than of men, so
greatly superior their virtues seemed to those of mere humanity?
Sparta, eternal proof of the vanity of science, while the vices, under
the conduct of the fine arts, were being introduced into Athens, even
while its tyrant was carefully collecting together the works of the
prince of poets, was driving from her walls artists and the arts, the
learned and their learning!

The difference was seen in the outcome. Athens became the seat of
politeness and taste, the country of orators and philosophers. The
elegance of its buildings equalled that of its language; on every side
might be seen marble and canvas, animated by the hands of the most
skilful artists. From Athens we derive those astonishing performances,
which will serve as models to every corrupt age. The picture of
Laced\ae mon is not so highly coloured. There, the neighbouring nations
used to say, ``men were born virtuous, their native air seeming
\page{137} to inspire them with virtue.'' But its inhabitants have
left us nothing but the memory of their heroic actions: monuments that
should not count for less in our eyes than the most curious relics of
Athenian marble.

It is true that, among the Athenians, there were some few wise men who
withstood the general torrent, and preserved their integrity even in
the company of the muses. But hear the judgment which the principal,
and most unhappy of them, passed on the artists and learned men of his
day.

``I have considered the poets,'' says he, ``and I look upon them as
people whose talents impose both on themselves and on others; they
give themselves out for wise men, and are taken for such; but in
reality they are anything sooner than that.''

``From the poets,'' continues Socrates, ``I turned to the artists.
Nobody was more ignorant of the arts than myself; nobody was more
fully persuaded that the artists were possessed of amazing knowledge.
I soon discovered, however, that they were in as bad a way as the
poets, and that both had fallen into the same misconception. Because
the most skilful of them excel others in their particular jobs, they
think themselves wiser than all the rest of mankind. This arrogance
spoilt all their skill in my eyes, so that, putting myself in the
place of the oracle, and asking myself whether I would rather be what
I am or what they are, know what they know, or know that I know
nothing, I very readily answered, for myself and the god, that I had
rather remain as I am.

``None of us, neither the sophists, nor the poets, nor the orators,
nor the artists, nor I, know what is the nature of the \textit{true},
the \textit{good}, or the \textit{beautiful}. But there is this
difference between us; that, though none of these people know
anything, they all think they know something; whereas for my part, if
I know nothing, I am at least in no doubt of my ignorance. So the
superiority of wisdom, imputed to me by the oracle, is reduced merely
to my being fully convinced that I am ignorant of what I do not
know.''

Thus we find Socrates, the wisest of men in the judgment of the god,
and the most learned of all the Athenians in the opinion of all
Greece, speaking in praise of ignor-\page{138}ance. Were he alive now,
there is little reason to think that our modern scholars and artists
would induce him to change his mind. No, gentlemen, that honest man
would still persist in despising our vain sciences. He would lend no
aid to swell the flood of books that flows from every quarter: he
would leave to us, as he did to his disciples, only the example and
memory of his virtues; that is the noblest method of instructing
mankind.

Socrates had begun at Athens, and the elder Cato proceeded at Rome, to
inveigh against those seductive and subtle Greeks, who corrupted the
virtue and destroyed the courage of their fellow-citizens: culture,
however, prevailed. Rome was filled with philosophers and orators,
military discipline was neglected, agriculture was held in contempt,
men formed sects, and forgot their country. To the sacred names of
liberty, disinterestedness and obedience to law, succeeded those of
Epicurus, Zeno and Arcesilaus. It was even a saying among their own
philosophers that since learned men appeared among them, honest men
had been in eclipse. Before that time the Romans were satisfied with
the practice of virtue; they were undone when they began to study it.

What would the great soul of Fabricius have felt, if it had been his
misfortune to be called back to life, when he saw the pomp and
magnificence of that Rome, which his arm had saved from ruin, and his
honourable name made more illustrious than all its conquests. ``Ye
gods!'' he would have said, ``what has become of those thatched roofs
and rustic hearths, which were formerly the habitations of temperance
and virtue? What fatal splendour has succeeded the ancient Roman
simplicity? What is this foreign language, this effeminacy of manners?
What is the meaning of these statues, paintings and buildings? Fools,
what have you done? You, the lords of the earth, have made yourselves
the slaves of the frivolous nations you have subdued. You are governed
by rhetoricians, and it has been only to enrich architects, painters,
sculptors and stage-players that you have watered Greece and Asia with
your blood. Even the spoils of Carthage are the prize of a
flute-player. Romans! Romans! make haste to demolish those
amphitheatres, break to pieces those \page{139} statues, burn those
paintings; drive from among you those slaves who keep you in
subjection, and whose fatal arts are corrupting your morals. Let other
hands make themselves illustrious by such vain talents; the only
talent worthy of Rome is that of conquering the world and making
virtue its ruler. When Cyneas took the Roman senate for an assembly of
kings, he was not struck by either useless pomp or studied elegance.
He heard there none of that futile eloquence, which is now the study
and the charm of frivolous orators. What then was the majesty that
Cyneas beheld? Fellow citizens, he saw the noblest sight that ever
existed under heaven, a sight which not all your riches or your arts
can show; an assembly of two hundred virtuous men, worthy to command
in Rome, and to govern the world.''

But let pass the distance of time and place, and let us see what has
happened in our own time and country; or rather let us banish odious
descriptions that might offend our delicacy, and spare ourselves the
pains of repeating the same things under different names. It was not
for nothing that I invoked the Manes of Fabricius; for what have I put
into his mouth, that might not have come with as much propriety from
Louis the Twelfth or Henry the Fourth? It is true that in France
Socrates would not have drunk the hemlock, but he would have drunk of
a potion infinitely more bitter, of insult, mockery and contempt a
hundred times worse than death.

Thus it is that luxury, profligacy and slavery, have been, in all
ages, the scourge of the efforts of our pride to emerge from that
happy state of ignorance, in which the wisdom of providence had placed
us. That thick veil with which it has covered all its operations seems
to be a sufficient proof that it never designed us for such fruitless
researches. But is there, indeed, one lesson it has taught us, by
which we have rightly profited, or which we have neglected with
impunity? Let men learn for once that nature would have preserved them
from science, as a mother snatches a dangerous weapon from the hands
of her child. Let them know that all the secrets she hides are so many
evils from which she protects them, and that the very difficulty they
find in acquiring knowledge is not the least of her bounty towards
them. Men are per-\page{140}verse; but they would have been far worse,
if they had had the misfortune to be born learned.

How humiliating are these reflections to humanity, and how mortified
by them our pride should be! What! it will be asked, is uprightness
the child of ignorance? Is virtue inconsistent with learning? What
consequences might not be drawn from such suppositions? But to
reconcile these apparent contradictions, we need only examine closely
the emptiness and vanity of those pompous titles, which are so
liberally bestowed on human knowledge, and which so blind our
judgment. Let us consider, therefore, the arts and sciences in
themselves. Let us see what must result from their advancement, and
let us not hesitate to admit the truth of all those points on which
our arguments coincide with the inductions we can make from history.

\section*{The Second Part}

An ancient tradition passed out of Egypt into Greece, that some god,
who was an enemy to the repose of mankind, was the inventor of the
sciences.\footnote{It is easy to see the allegory in the fable of
Prometheus: and it does not appear that the Greeks, who chained him to
the Caucasus, had a better opinion of him than the Egyptians had of
their god Theutus. The Satyr, says an ancient fable, the first time he
saw a fire, was going to kiss and embrace it; but Prometheus cried out
to him to forbear, or his beard would rue it. It burns, says he,
everything that touches it.} What must the Egyptians, among whom the
sciences first arose, have thought of them? And they beheld, near at
hand, the sources from which they sprang. In fact, whether we turn to
the annals of the world, or eke out with philosophical investigations
the uncertain chronicles of history, we shall not find for human
knowledge an origin answering to the idea we are pleased to entertain
of it at present. Astronomy was born of superstition, eloquence of
ambition, hatred, falsehood and flattery; geometry of avarice; physics
of an idle curiosity; and even moral philosophy of human pride. Thus
the arts and sciences owe their birth to our vices; we should be less
doubtful of their advantages, if they had sprung from our virtues.

\page{141}Their evil origin is, indeed, but too plainly reproduced in
their objects. What would become of the arts, were they not cherished
by luxury? If men were not unjust, of what use were jurisprudence?
What would become of history, if there were no tyrants, wars, or
conspiracies? In a word, who would pass his life in barren
speculations, if everybody, attentive only to the obligations of
humanity and the necessities of nature, spent his whole life in
serving his country, obliging his friends, and relieving the unhappy?
Are we then made to live and die on the brink of that well at the
bottom of which Truth lies hid? This reflection alone is, in my
opinion, enough to discourage at first setting out every man who
seriously endeavours to instruct himself by the study of philosophy.

What a variety of dangers surrounds us! What a number of wrong paths
present themselves in the investigation of the sciences! Through how
many errors, more perilous than truth itself is useful, must we not
pass to arrive at it? The disadvantages we lie under are evident; for
falsehood is capable of an infinite variety of combinations; but the
truth has only one manner of being. Besides, where is the man who
sincerely desires to find it? Or even admitting his good will, by what
characteristic marks is he sure of knowing it? Amid the infinite
diversity of opinions where is the criterion\footnote{The less we
know, the more we think we know. The peripatetics doubted of nothing.
Did not Descartes construct the universe with cubes and vortices? And
is there in all Europe one single physicist who does not boldly
explain the inexplicable mysteries of electricity, which will,
perhaps, be for ever the despair of real philosophers?} by which we
may certainly judge of it? Again, what is still more difficult, should
we even be fortunate enough to discover it, who among us will know how
to make right use of it?

If our sciences are futile in the objects they propose, they are no
less dangerous in the effects they produce. Being the effect of
idleness, they generate idleness in their turn; and an irreparable
loss of time is the first prejudice which they must necessarily cause
to society. To live without doing some good is a great evil as well in
the political as in the moral world; and hence every useless citizen
should be regarded as a pernicious person. Tell me then, illustrious
philosophers, of whom we learn the \page{142} ratios in which
attraction acts in vacuo; and in the revolution of the planets, the
relations of spaces traversed in equal times; by whom we are taught
what curves have conjugate points, points of inflexion, and cusps; how
the soul and body correspond, like two clocks, without actual
communication; what planets may be inhabited; and what insects
reproduce in an extraordinary manner. Answer me, I say, you from whom
we receive all this sublime information, whether we should have been
less numerous, worse governed, less formidable, less flourishing, or
more perverse, supposing you had taught us none of all these fine
things.

Reconsider therefore the importance of your productions; and, since
the labours of the most enlightened of our learned men and the best of
our citizens are of so little utility, tell us what we ought to think
of that numerous herd of obscure writers and useless litt\'erateurs, who
devour without any return the substance of the State.

Useless, do I say? Would God they were! Society would be more
peaceful, and morals less corrupt. But these vain and futile
declaimers go forth on all sides, armed with their fatal paradoxes, to
sap the foundations of our faith, and nullify virtue. They smile
contemptuously at such old names as patriotism and religion, and
consecrate their talents and philosophy to the destruction and
defamation of all that men hold sacred. Not that they bear any real
hatred to virtue or dogma; they are the enemies of public opinion
alone; to bring them to the foot of the altar, it would be enough to
banish them to a land of atheists. What extravagancies will not the
rage of singularity induce men to commit!

The waste of time is certainly a great evil; but still greater evils
attend upon literature and the arts. One is luxury, produced like them
by indolence and vanity. Luxury is seldom unattended by the arts and
sciences; and they are always attended by luxury. I know that our
philosophy, fertile in paradoxes, pretends, in contradiction to the
experience of all ages, that luxury contributes to the splendour of
States. But, without insisting on the necessity of sumptuary laws, can
it be denied that rectitude of morals is essential to the duration of
empires, and that luxury is diametrically opposed to such rectitude?
\page{143} Let it be admitted that luxury is a certain indication of
wealth; that it even serves, if you will, to increase such wealth:
what conclusion is to be drawn from this paradox, so worthy of the
times? And what will become of virtue if riches are to be acquired at
any cost? The politicians of the ancient world were always talking of
morals and virtue; ours speak of nothing but commerce and money. One
of them will tell you that in such a country a man is worth just as
much as he will sell for at Algiers: another, pursuing the same mode
of calculation, finds that in some countries a man is worth nothing,
and in others still less than nothing; they value men as they do
droves of oxen. According to them, a man is worth no more to the
State, than the amount he consumes; and thus a Sybarite would be worth
at least thirty Laced\ae monians. Let these writers tell me, however,
which of the two republics, Sybaris or Sparta, was subdued by a
handful of peasants, and which became the terror of Asia.

The monarchy of Cyrus was conquered by thirty thousand men, led by a
prince poorer than the meanest of Persian Satraps: in like manner the
Scythians, the poorest of all nations, were able to resist the most
powerful monarchs of the universe. When two famous republics contended
for the empire of the world, the one rich and the other poor, the
former was subdued by the latter. The Roman empire in its turn, after
having engulfed all the riches of the universe, fell a prey to peoples
who knew not even what riches were. The Franks conquered the Gauls,
and the Saxons England, without any other treasures than their bravery
and their poverty. A band of poor mountaineers, whose whole cupidity
was confined to the possession of a few sheep-skins, having first
given a check to the arrogance of Austria, went on to crush the
opulent and formidable house of Burgundy, which at that time made the
potentates of Europe tremble. In short, all the power and wisdom of
the heir of Charles the Fifth, backed by all the treasures of the
Indies, broke before a few herring-fishers. Let our politicians
condescend to lay aside their calculations for a moment, to reflect on
these examples; let them learn for once that money, though it buys
everything else, cannot buy morals and citizens. What then is the
precise point in dispute about luxury? \page{144} It is to know which
is most advantageous to empires, that their existence should be
brilliant and momentary, or virtuous and lasting? I say brilliant, but
with what lustre! A taste for ostentation never prevails in the same
minds as a taste for honesty. No, it is impossible that
understandings, degraded by a multitude of futile cares, should ever
rise to what is truly great and noble; even if they had the strength,
they would want the courage.

Every artist loves applause. The praise of his contemporaries is the
most valuable part of his recompense. What then will he do to obtain
it, if he have the misfortune to be born among a people, and at a
time, when learning is in vogue, and the superficiality of youth is in
a position to lead the fashion; when men have sacrificed their taste
to those who tyrannise over their liberty, and one sex dare not
approve anything but what is proportionate to the pusillanimity of the
other;\footnote{I am far from thinking that the ascendancy which women
have obtained over men is an evil in itself. It is a present which
nature has made them for the good of mankind. If better directed, it
might be productive of as much good, as it is now of evil. We are not
sufficiently sensible of what advantage it would be to society to give
a better education to that half of our species which governs the
other. Men will always be what women choose to make them. If you wish
then that they should be noble and virtuous, let women be taught what
greatness of soul and virtue are. The reflections which this subject
arouses, and which Plato formerly made, deserve to be more fully
developed by a pen worthy of following so great a master, and
defending so great a cause.} when the greatest masterpieces of
dramatic poetry are condemned, and the noblest of musical productions
neglected? This is what he will do. He will lower his genius to the
level of the age, and will rather submit to compose mediocre works,
that will be admired during his life-time, than labour at sublime
achievements which will not be admired till long after he is dead. Let
the famous Voltaire tell us how many nervous and masculine beauties he
has sacrificed to our false delicacy, and how much that is great and
noble, that spirit of gallantry, which delights in what is frivolous
and petty, has cost him.

It is thus that the dissolution of morals, the necessary consequence
of luxury, brings with it in its turn the corruption of taste.
Further, if by chance there be found \page{145} among men of average
ability, an individual with enough strength of mind to refuse to
comply with the spirit of the age, and to debase himself by puerile
productions, his lot will be hard. He will die in indigence and
oblivion. This is not so much a prediction, as a fact already
confirmed by experience! Yes, Carle and Pierre Vanloo, the time is
already come when your pencils, destined to increase the majesty of
our temples by sublime and holy images, must fall from your hands, or
else be prostituted to adorn the panels of a coach with lascivious
paintings. And you, inimitable Pigal, rival of Phidias and Praxiteles,
whose chisel the ancients would have employed to carve them gods,
whose images almost excuse their idolatry in our eyes; even your hand
must condescend to fashion the belly of an ape, or else remain idle.

We cannot reflect on the morality of mankind without contemplating
with pleasure the picture of the simplicity which prevailed in the
earliest times. This image may be justly compared to a beautiful
coast, adorned only by the hands of nature; towards which our eyes are
constantly turned, and which we see receding with regret. While men
were innocent and virtuous and loved to have the gods for witnesses of
their actions, they dwelt together in the same huts; but when they
became vicious, they grew tired of such inconvenient onlookers, and
banished them to magnificent temples. Finally, they expelled their
deities even from these, in order to dwell there themselves; or at
least the temples of the gods were no longer more magnificent than the
palaces of the citizens. This was the height of degeneracy; nor could
vice ever be carried to greater lengths than when it was seen,
supported, as it were, at the doors of the great, on columns of
marble, and graven on Corinthian capitals.

As the conveniences of life increase, as the arts are brought to
perfection, and luxury spreads, true courage flags, the virtues
disappear; and all this is the effect of the sciences and of those
arts which are exercised in the privacy of men's dwellings. When the
Goths ravaged Greece, the libraries only escaped the flames owing to
an opinion that was set on foot among them, that it was best to leave
the enemy with a possession so calculated to divert their attention
from military exercises, \page{146} and keep them engaged in indolent
and sedentary occupations.

Charles the Eighth found himself master of Tuscany and the kingdom of
Naples, almost without drawing sword; and all his court attributed
this unexpected success to the fact that the princes and nobles of
Italy applied themselves with greater earnestness to the cultivation
of their understandings than to active and martial pursuits. In fact,
says the sensible person who records these characteristics, experience
plainly tells us, that in military matters and all that resemble them
application to the sciences tends rather to make men effeminate and
cowardly than resolute and vigorous.

The Romans confessed that military virtue was extinguished among them,
in proportion as they became connoisseurs in the arts of the painter,
the engraver and the goldsmith, and began to cultivate the fine arts.
Indeed, as if this famous country was to be for ever an example to
other nations, the rise of the Medici and the revival of letters has
once more destroyed, this time perhaps for ever, the martial
reputation which Italy seemed a few centuries ago to have recovered.

The ancient republics of Greece, with that wisdom which was so
conspicuous in most of their institutions, forbade their citizens to
pursue all those inactive and sedentary occupations, which by
enervating and corrupting the body diminish also the vigour of the
mind. With what courage, in fact, can it be thought that hunger and
thirst, fatigues, dangers and death, can be faced by men whom the
smallest want overwhelms and the slightest difficulty repels? With
what resolution can soldiers support the excessive toils of war, when
they are entirely unaccustomed to them? With what spirits can they
make forced marches under officers who have not even the strength to
travel on horseback? It is no answer to cite the reputed valour of all
the modern warriors who are so scientifically trained. I hear much of
their bravery in a day's battle; but I am told nothing of how they
support excessive fatigue, how they stand the severity of the seasons
and the inclemency of the weather. A little sunshine or snow, or the
want of a few superfluities, is enough to cripple and destroy one of
our finest armies in a few days. Intrepid \page{147} warriors! permit
me for once to tell you the truth, which you seldom hear. Of your
bravery I am fully satisfied. I have no doubt that you would have
triumphed with Hannibal at Cann\ae, and at Trasimene: that you would
have passed the Rubicon with C\ae sar, and enabled him to enslave his
country; but you never would have been able to cross the Alps with the
former, or with the latter to subdue your own ancestors, the Gauls.

A war does not always depend on the events of battle: there is in
generalship an art superior to that of gaining victories. A man may
behave with great intrepidity under fire, and yet be a very bad
officer. Even in the common soldier, a little more strength and vigour
would perhaps be more useful than so much courage, which after all is
no protection from death. And what does it matter to the State whether
its troops perish by cold and fever, or by the sword of the enemy?

If the cultivation of the sciences is prejudicial to military
qualities, it is still more so to moral qualities. Even from our
infancy an absurd system of education serves to adorn our wit and
corrupt our judgment. We see, on every side, huge institutions, where
our youth are educated at great expense, and instructed in everything
but their duty. Your children will be ignorant of their own language,
when they can talk others which are not spoken anywhere. They will be
able to compose verses which they can hardly understand; and, without
being capable of distinguishing truth from error, they will possess
the art of making them unrecognisable by specious arguments. But
magnanimity, equity, temperance, humanity and courage will be words of
which they know not the meaning. The dear name of country will never
strike on their ears; and if they ever hear speak of
God,\footnote{Pens\'ees philosophiques (Diderot).} it will be less to
fear, than to be frightened of, Him. I would as soon, said a wise man,
that my pupil had spent his time in the tennis court as in this
manner; for there his body at least would have got exercise.

I well know that children ought to be kept employed, and that idleness
is for them the danger most to be feared. But what should they be
taught? This is undoubtedly an \page{148} important question. Let them
be taught what they are to practise when they come to be
men;\footnote{Such was the education of the Spartans with regard to
one of the greatest of their kings. It is well worthy of notice, says
Montaigne, that the excellent institutions of Lycurgus, which were in
truth miraculously perfect, paid as much attention to the bringing up
of youth as if this were their principal object, and yet, at the very
seat of the Muses, they make so little mention of learning that it
seems as if their generous-spirited youth disdained every other
restraint, and required, instead of masters of the sciences,
instructors in valour, prudence and justice alone.

Let us hear next what the same writer says of the ancient Persians.
Plato, says he, relates that the heir to the throne was thus brought
up. At his birth he was committed, not to the care of women, but to
eunuchs in the highest authority and near the person of the king, on
account of their virtue. These undertook to render his body beautiful
and healthy. At seven years of age they taught him to ride and go
hunting. At fourteen he was placed in the hands of four, the wisest,
the most just, the most temperate and the bravest persons in the
kingdom. The first instructed him in religion, the second taught him
to adhere inviolably to truth, the third to conquer his passions, and
the fourth to be afraid of nothing. All, I may add, taught him to be a
good man; but not one taught him to be learned.

Astyages, in Xenophon, desires Cyrus to give him an account of his
last lesson. It was this, answered Cyrus, one of the big boys of the
school having a small coat, gave it to a little boy and took away from
him his coat, which was larger. Our master having appointed me arbiter
in the dispute, I ordered that matters should stand as they were, as
each boy seemed to be better suited than before. The master, however,
remonstrated with me, saying that I considered only convenience,
whereas justice ought to have been the first concern, and justice
teaches that no one should suffer forcible interference with what
belongs to him. He added that he was punished for his wrong decision,
just as boys are punished in our country schools when they forget the
first aorist of \grk{τύπτω}. My tutor must make me a fine harangue,
\textit{in genere demonstrativo}, before he will persuade me that his
school is as good as this.} not what they ought to forget.

Our gardens are adorned with statues and our galleries with pictures.
What would you imagine these masterpieces of art, thus exhibited to
public admiration, represent? The great men, who have defended their
country, or the still greater men who have enriched it by their
virtues? Far from it. They are the images of every perversion of heart
and mind, carefully selected from ancient mythology, and presented to
the early curiosity of our children, doubtless that they may have
before their eyes the representations of vicious actions, even before
they are able to read.

\page{149}Whence arise all those abuses, unless it be from that fatal
inequality introduced among men by the difference of talents and the
cheapening of virtue? This is the most evident effect of all our
studies, and the most dangerous of all their consequences. The
question is no longer whether a man is honest, but whether he is
clever. We do not ask whether a book is useful, but whether it is
well-written. Rewards are lavished on wit and ingenuity, while virtue
is left unhonoured. There are a thousand prizes for fine discourses,
and none for good actions. I should be glad, however, to know whether
the honour attaching to the best discourse that ever wins the prize in
this Academy is comparable with the merit of having founded the prize.

A wise man does not go in chase of fortune; but he is by no means
insensible to glory, and when he sees it so ill distributed, his
virtue, which might have been animated by a little emulation, and
turned to the advantage of society, droops and dies away in obscurity
and indigence. It is for this reason that the agreeable arts must in
time everywhere be preferred to the useful; and this truth has been
but too much confirmed since the revival of the arts and sciences. We
have physicists, geometricians, chemists, astronomers, poets,
musicians, and painters in plenty; but we have no longer a citizen
among us; or if there be found a few scattered over our abandoned
countryside, they are left to perish there unnoticed and neglected.
Such is the condition to which we are reduced, and such are our
feelings towards those who give us our daily bread, and our children
milk.

I confess, however, that the evil is not so great as it might have
become. The eternal providence, in placing salutary simples beside
noxious plants, and making poisonous animals contain their own
antidote, has taught the sovereigns of the earth, who are its
ministers, to imitate its wisdom. It is by following this example that
the truly great monarch, to whose glory every age will add new lustre,
drew from the very bosom of the arts and sciences, the very fountains
of a thousand lapses from rectitude, those famous societies, which,
while they are depositaries of the dangerous trust of human knowledge,
are yet the sacred guardians of morals, by the attention \page{150}
they pay to their maintenance among themselves in all their purity,
and by the demands which they make on every member whom they admit.

These wise institutions, confirmed by his august successor and
imitated by all the kings of Europe, will serve at least to restrain
men of letters, who, all aspiring to the honour of being admitted into
these Academies, will keep watch over themselves, and endeavour to
make themselves worthy of such honour by useful performances and
irreproachable morals. Those Academies also, which, in proposing
prizes for literary merit, make choice of such subjects as are
calculated to arouse the love of virtue in the hearts of citizens,
prove that it prevails in themselves, and must give men the rare and
real pleasure of finding learned societies devoting themselves to the
enlightenment of mankind, not only by agreeable exercises of the
intellect, but also by useful instructions.

An objection which may be made is, in fact, only an additional proof
of my argument. So much precaution proves but too evidently the need
for it. We never seek remedies for evils that do not exist. Why,
indeed, must these bear all the marks of ordinary remedies, on account
of their inefficacy? The numerous establishments in favour of the
learned are only adapted to make men mistake the objects of the
sciences, and turn men's attention to the cultivation of them. One
would be inclined to think, from the precautions everywhere taken,
that we are overstocked with husbandmen, and are afraid of a shortage
of philosophers. I will not venture here to enter into a comparison
between agriculture and philosophy, as they would not bear it. I shall
only ask What is philosophy? What is contained in the writings of the
most celebrated philosophers? What are the lessons of these friends of
wisdom. To hear them, should we not take them for so many mountebanks,
exhibiting themselves in public, and crying out, \textit{Here, Here,
come to me, I am the only true doctor?} One of them teaches that there
is no such thing as matter, but that everything exists only in
representation. Another declares that there is no other substance than
matter, and no other God than the world itself. A third tells you that
there are no such things as virtue and vice, and that moral good and
evil \page{151} are chimeras; while a fourth informs you that men are
only beasts of prey, and may conscientiously devour one another. Why,
my great philosophers, do you not reserve these wise and profitable
lessons for your friends and children? You would soon reap the benefit
of them, nor should we be under any apprehension of our own becoming
your disciples.

Such are the wonderful men, whom their contemporaries held in the
highest esteem during their lives, and to whom immortality has been
attributed since their decease. Such are the wise maxims we have
received from them, and which are transmitted, from age to age, to our
descendants. Paganism, though given over to all the extravagances of
human reason, has left nothing to compare with the shameful monuments
which have been prepared by the art of printing, during the reign of
the gospel. The impious writings of Leucippus and Diagoras perished
with their authors. The world, in their days, was ignorant of the art
of immortalising the errors and extravagancies of the human mind. But
thanks to the art of printing\footnote{If we consider the frightful
disorders which printing has already caused in Europe, and judge of
the future by the progress of its evils from day to day, it is easy to
foresee that sovereigns will hereafter take as much pains to banish
this dreadful art from their dominions, as they ever took to encourage
it. The Sultan Achmet, yielding to the importunities of certain
pretenders to taste, consented to have a press erected at
Constantinople; but it was hardly set to work before they were obliged
to destroy it, and throw the plant into a well.

It is related that the Caliph Omar, being asked what should be done
with the library at Alexandria, answered in these words. ``If the
books in the library contain anything contrary to the Alcoran, they
are evil and ought to be burnt; if they contain only what the Alcoran
teaches, they are superfluous.'' This reasoning has been cited by our
men of letters as the height of absurdity; but if Gregory the Great
had been in the place of Omar, and the Gospel in the place of the
Alcoran, the library would still have been burnt, and it would have
been perhaps the finest action of his life.} and the use we make of it,
the pernicious reflections of Hobbes and Spinoza will last for ever.
Go, famous writings, of which the ignorance and rusticity of our
forefathers would have been incapable. Go to our descendants, along
with those still more pernicious works which reek of the corrupted
manners of the present age! Let them together convey to posterity a
faithful history of the progress and \page{152} advantages of our arts
and sciences. If they are read, they will leave not a doubt about the
question we are now discussing, and unless mankind should then be
still more foolish than we, they will lift up their hands to Heaven
and exclaim in bitterness of heart: ``Almighty God! thou who holdest
in Thy hand the minds of men, deliver us from the fatal arts and
sciences of our forefathers; give us back ignorance, innocence and
poverty, which alone can make us happy and are precious in Thy
sight.''

But if the progress of the arts and sciences has added nothing to our
real happiness; if it has corrupted our morals, and if that corruption
has vitiated our taste, what are we to think of the herd of text-book
authors, who have removed those impediments which nature purposely
laid in the way to the Temple of the Muses, in order to guard its
approach and try the powers of those who might be tempted to seek
knowledge? What are we to think of those compilers who have
indiscreetly broken open the door of the sciences, and introduced into
their sanctuary a populace unworthy to approach it, when it was
greatly to be wished that all who should be found incapable of making
a considerable progress in the career of learning should have been
repulsed at the entrance, and thereby cast upon those arts which are
useful to society. A man who will be all his life a bad versifier, or
a third-rate geometrician, might have made nevertheless an excellent
clothier. Those whom nature intended for her disciples have not needed
masters. Bacon, Descartes and Newton, those teachers of mankind, had
themselves no teachers. What guide indeed could have taken them so far
as their sublime genius directed them? Ordinary masters would only
have cramped their intelligence, by confining it within the narrow
limits of their own capacity. It was from the obstacles they met with
at first, that they learned to exert themselves, and bestirred
themselves to traverse the vast field which they covered. If it be
proper to allow some men to apply themselves to the study of the arts
and sciences, it is only those who feel themselves able to walk alone
in their footsteps and to outstrip them. It belongs only to these few
to raise monuments to the glory of the human understanding. But if we
are desirous that nothing should be above their genius, nothing should
be \page{153} beyond their hopes. This is the only encouragement they
require. The soul insensibly adapts itself to the objects on which it
is employed, and thus it is that great occasions produce great men.
The greatest orator in the world was Consul of Rome, and perhaps the
greatest of philosophers Lord Chancellor of England. Can it be
conceived that, if the former had only been a professor at some
University, and the latter a pensioner of some Academy, their works
would not have suffered from their situation. Let not princes disdain
to admit into their councils those who are most capable of giving them
good advice. Let them renounce the old prejudice, which was invented
by the pride of the great, that the art of governing mankind is more
difficult than that of instructing them; as if it was easier to induce
men to do good voluntarily, than to compel them to it by force. Let
the learned of the first rank find an honourable refuge in their
courts; let them there enjoy the only recompense worthy of them, that
of promoting by their influence the happiness of the peoples they have
enlightened by their wisdom. It is by this means only that we are
likely to see what virtue, science and authority can do, when animated
by the noblest emulation, and working unanimously for the happiness of
mankind.

But so long as power alone is on one side, and knowledge and
understanding alone on the other, the learned will seldom make great
objects their study, princes will still more rarely do great actions,
and the peoples will continue to be, as they are, mean, corrupt and
miserable.

As for us, ordinary men, on whom Heaven has not been pleased to bestow
such great talents; as we are not destined to reap such glory, let us
remain in our obscurity. Let us not covet a reputation we should never
attain, and which, in the present state of things, would never make up
to us for the trouble it would have cost us, even if we were fully
qualified to obtain it. Why should we build our happiness on the
opinions of others, when we can find it in our own hearts? Let us
leave to others the task of instructing mankind in their duty, and
confine ourselves to the discharge of our own. We have no occasion for
greater knowledge than this.

Virtue! sublime science of simple minds, are such \page{154} industry
and preparation \llb needed if we are to know you? Are not your
principles graven on every heart? Need we do more, to learn your laws,
than examine ourselves, and listen to the voice of conscience, when
the passions are silent?

This is the true philosophy, with which we must learn to be content,
without envying the fame of those celebrated men, whose names are
immortal in the republic of letters. Let us, instead of envying them,
endeavour to make, between them and us, that honourable distinction
which was formerly seen to exist between two great peoples, that the
one knew how to speak, and the other how to act, aright.

