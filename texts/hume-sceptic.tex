
\author{David Hume}
\authdate{1711--1776}
\textdate{1742}
\chapter{The Sceptic}
\source{hume1889.1.18}

\page{213}I have long entertained a suspicion, with regard to the
decisions of philosophers upon all subjects, and found in myself a
greater inclination to dispute, than assent to their conclusions.
There is one mistake, to which they seem liable, almost without
exception; they confine too much their principles, and make no account
of that vast variety, which nature has so much affected in all her
operations. When a philosopher has once laid hold of a favourite
principle, which perhaps accounts for many natural effects, he extends
the \page{214} same principle over the whole creation, and reduces to
it every ph{\ae}nomenon, though by the most violent and absurd
reasoning. Our own mind being narrow and contracted, we cannot extend
our conception to the variety and extent of nature; but imagine, that
she is as much bounded in her operations, as we are in our
speculation.

But if ever this infirmity of philosophers is to be suspected on any
occasion, it is in their reasonings concerning human life, and the
methods of attaining happiness. In that case, they are led astray, not
only by the narrowness of their understandings, but by that also of
their passions. Almost every one has a predominant inclination, to
which his other desires and affections submit, and which governs him,
though, perhaps, with some intervals, through the whole course of his
life. It is difficult for him to apprehend, that any thing, which
appears totally indifferent to him, can ever give enjoyment to any
person, or can possess charms, which altogether escape his
observation. His own pursuits are always, in his account, the most
engaging: The objects of his passion, the most valuable: And the road,
which he pursues, the only one that leads to happiness.

But would these prejudiced reasoners reflect a moment, there are many
obvious instances and arguments, sufficient to undeceive them, and
make them enlarge their maxims and principles. Do they not see the
vast variety of inclinations and pursuits among our species; where
each man seems fully satisfied with his own course of life, and would
esteem it the greatest unhappiness to be confined to that of his
neighbour? Do they not feel in themselves, that what pleases at one
time, displeases at another, by the change of inclination; and that it
is not in their power, by their utmost efforts, to recall that taste
or appetite, which formerly bestowed charms on what now appears
indifferent or disagreeable? What is the meaning therefore of those
general preferences of the town or country life, of a life of action
or one of pleasure, of retirement or society; when besides the
different inclinations of different men, every one's experience may
convince him, that each of these kinds of life is agreeable in its
turn, and that their variety or their judicious mixture chiefly
contributes to the rendering all of them agreeable.

But shall this business be allowed to go altogether at adventures? And
must a man consult only his humour and \page{215} inclination, in
order to determine his course of life, without employing his reason to
inform him what road is preferable, and leads most surely to
happiness? Is there no difference then between one man's conduct and
another?

I answer, there is a great difference. One man, following his
inclination, in chusing his course of life, may employ much surer
means for succeeding than another, who is led by his inclination into
the same course of life, and pursues the same object. \textit{Are
riches the chief object of your desires?} Acquire skill in your
profession; be diligent in the exercise of it; enlarge the circle of
your friends and acquaintance; avoid pleasure and expence; and never
be generous, but with a view of gaining more than you could save by
frugality. \textit{Would you acquire the public esteem?} Guard equally
against the extremes of arrogance and fawning. Let it appear that you
set a value upon yourself, but without despising others. If you fall
into either of the extremes, you either provoke men's pride by your
insolence, or teach them to despise you by your timorous submission,
and by the mean opinion which you seem to entertain of yourself.

These, you say, are the maxims of common prudence, and discretion;
what every parent inculcates on his child, and what every man of sense
pursues in the course of life, which he has chosen.---What is it then
you desire more? Do you come to a philosopher as to a \textit{cunning
man}, to learn something by magic or witchcraft, beyond what can be
known by common prudence and discretion?---Yes; we come to a
philosopher to be instructed, how we shall chuse our ends, more than
the means for attaining these ends: We want to know what desire we
shall gratify, what passion we shall comply with, what appetite we
shall indulge. As to the rest, we trust to common sense, and the
general maxims of the world for our instruction.

I am sorry then, I have pretended to be a philosopher: For I find your
questions very perplexing; and am in danger, if my answer be too rigid
and severe, of passing for a pedant and scholastic; if it be too easy
and free, of being taken for a preacher of vice and immorality.
However, to satisfy you, I shall deliver my opinion upon the matter,
and shall only desire you to esteem it of as little consequence as I
do myself. By that means you will neither think it worthy of your
ridicule nor your anger.

\page{216} If we can depend upon any principle, which we learn from
philosophy, this, I think, may be considered as certain and undoubted,
that there is nothing, in itself, valuable or despicable, desirable or
hateful, beautiful or deformed; but that these attributes arise from
the particular constitution and fabric of human sentiment and
affection. What seems the most delicious food to one animal, appears
loathsome to another: What affects the feeling of one with delight,
produces uneasiness in another. This is confessedly the case with
regard to all the bodily senses: But if we examine the matter more
accurately, we shall find, that the same observation holds even where
the mind concurs with the body, and mingles its sentiment with the
exterior appetite.

Desire this passionate lover to give you a character of his mistress:
He will tell you, that he is at a loss for words to describe her
charms, and will ask you very seriously if ever you were acquainted
with a goddess or an angel? If you answer that you never were: He will
then say, that it is impossible for you to form a conception of such
divine beauties as those which his charmer possesses; so complete a
shape; such well-proportioned features; so engaging an air; such
sweetness of disposition; such gaiety of humour. You can infer
nothing, however, from all this discourse, but that the poor man is in
love; and that the general appetite between the sexes, which nature
has infused into all animals, is in him determined to a particular
object by some qualities, which give him pleasure. The same divine
creature, not only to a different animal, but also to a different man,
appears a mere mortal being, and is beheld with the utmost
indifference.

Nature has given all animals a like prejudice in favour of their
offspring. As soon as the helpless infant sees the light, though in
every other eye it appears a despicable and a miserable creature, it
is regarded by its fond parent with the utmost affection, and is
preferred to every other object, however perfect and accomplished. The
passion alone, arising from the original structure and formation of
human nature, bestows a value on the most insignificant object.

We may push the same observation further, and may conclude, that, even
when the mind operates alone, and feeling the sentiment of blame or
approbation, pronounces one object deformed and odious, another
beautiful and amiable; I say, that, even in this case, those qualities
are not really in the \page{217} objects, but belong entirely to the
sentiment of that mind which blames or praises. I grant, that it will
be more difficult to make this proposition evident, and as it were,
palpable, to negligent thinkers; because nature is more uniform in the
sentiments of the mind than in most feelings of the body, and produces
a nearer resemblance in the inward than in the outward part of human
kind. There is something approaching to principles in mental taste;
and critics can reason and dispute more plausibly than cooks or
perfumers. We may observe, however, that this uniformity among human
kind, hinders not, but that there is a considerable diversity in the
sentiments of beauty and worth, and that education, custom, prejudice,
caprice, and humour, frequently vary our taste of this kind. You will
never convince a man, who is not accustomed to \textsc{Italian} music,
and has not an ear to follow its intricacies, that a \textsc{Scotch}
tune is not preferable. You have not even any single argument, beyond
your own taste, which you can employ in your behalf: And to your
antagonist, his particular taste will always appear a more convincing
argument to the contrary. If you be wise, each of you will allow, that
the other may be in the right; and having many other instances of this
diversity of taste, you will both confess, that beauty and worth are
merely of a relative nature, and consist in an agreeable sentiment,
produced by an object in a particular mind, according to the peculiar
structure and constitution of that mind.

By this diversity of sentiment, observable in human kind, nature has,
perhaps, intended to make us sensible of her authority, and let us see
what surprizing changes she could produce on the passions and desires
of mankind, merely by the change of their inward fabric, without any
alteration on the objects. The vulgar may even be convinced by this
argument: But men, accustomed to thinking, may draw a more convincing,
at least a more general argument, from the very nature of the subject.

In the operation of reasoning, the mind does nothing but run over its
objects, as they are supposed to stand in reality, without adding any
thing to them, or diminishing any thing from them. If I examine the
\textsc{Ptolomaic} and \textsc{Copernican} systems, I endeavour only,
by my enquiries, to know the real situation of the planets; that is in
other words, I endeavour to give them, in my conception, the same
relations, that they \page{218} bear towards each other in the
heavens. To this operation of the mind, therefore, there seems to be
always a real, though often an unknown standard, in the nature of
things; nor is truth or falsehood variable by the various
apprehensions of mankind. Though all human race should for ever
conclude, that the sun moves, and the earth remains at rest, the sun
stirs not an inch from his place for all these reasonings; and such
conclusions are eternally false and erroneous.

But the case is not the same with the qualities of \textit{beautiful
and deformed, desirable and odious}, as with truth and falsehood. In
the former case, the mind is not content with merely surveying its
objects, as they stand in themselves: It also feels a sentiment of
delight or uneasiness, approbation or blame, consequent to that
survey; and this sentiment determines it to affix the epithet
\textit{beautiful or deformed, desirable or odious}. Now, it is
evident, that this sentiment must depend upon the particular fabric or
structure of the mind, which enables such particular forms to operate
in such a particular manner, and produces a sympathy or conformity
between the mind and its objects. Vary the structure of the mind or
inward organs, the sentiment no longer follows, though the form
remains the same. The sentiment being different from the object, and
arising from its operation upon the organs of the mind, an alteration
upon the latter must vary the effect, nor can the same object,
presented to a mind totally different, produce the same sentiment.

This conclusion every one is apt to draw of himself, without much
philosophy, where the sentiment is evidently distinguishable from the
object. Who is not sensible, that power, and glory, and vengeance, are
not desirable of themselves, but derive all their value from the
structure of human passions, which begets a desire towards such
particular pursuits? But with regard to beauty, either natural or
moral, the case is commonly supposed to be different. The agreeable
quality is thought to lie in the object, not in the sentiment; and
that merely because the sentiment is not so turbulent and violent as
to distinguish itself, in an evident manner, from the perception of
the object.

But a little reflection suffices to distinguish them. A man may know
exactly all the circles and ellipses of the \textsc{Copernican}
system, and all the irregular spirals of the \textsc{Ptolomaic},
without perceiving that the former is more beautiful than \page{219}
the latter. \textsc{Euclid} has fully explained every quality of the
circle, but has not, in any proposition, said a word of its beauty.
The reason is evident. Beauty is not a quality of the circle. It lies
not in any part of the line \textit{whose} parts are all equally
distant from a common center. It is only the effect, which that figure
produces upon a mind, whose particular fabric or structure renders it
susceptible of such sentiments. In vain would you look for it in the
circle, or seek it, either by your senses, or by mathematical
reasonings, in all the properties of that figure.

The mathematician, who took no other pleasure in reading
\textsc{Virgil}, but that of examining \textsc{Eneas's} voyage by the
map, might perfectly understand the meaning of every Latin word,
employed by that divine author; and consequently, might have a
distinct idea of the whole narration. He would even have a more
distinct idea of it, than they could attain who had not studied so
exactly the geography of the poem. He knew, therefore, every thing in
the poem: But he was ignorant of its beauty; because the beauty,
properly speaking, lies not in the poem, but in the sentiment or taste
of the reader. And where a man has no such delicacy of temper, as to
make him feel this sentiment, he must be ignorant of the beauty,
though possessed of the science and understanding of an
angel.\footnote{Were I not afraid of appearing too philosophical, I
should remind my reader of that famous doctrine, supposed to be
fully proved in modern times, ``That tastes and colours, and all other
sensible qualities, lie not in the bodies, but merely in the senses.''
The case is the same with beauty and deformity, virtue and vice.
This doctrine, however, takes off no more from the reality of the
latter qualities, than from that of the former; nor need it give any
umbrage either to critics or moralists. Though colours were allowed to
lie only in the eye, would dyers or painters ever be less regarded
or esteemed? There is a sufficient uniformity in the senses and
feelings of mankind, to make all these qualities the objects of art
and reasoning, and to have the greatest influence on life and manners.
And as it is certain, that the discovery above-mentioned in natural
philosophy, makes no alteration on action and conduct; why should a
like discovery in moral philosophy make any alteration?}

The inference upon the whole is, that it is not from the value or
worth of the object, which any person pursues, that we can determine
his enjoyment, but merely from the passion with which he pursues it,
and the success which he meets with in his pursuit. Objects have
absolutely no worth or value in themselves. They derive their worth
merely from the passion. If that be strong, and steady, and
successful, the person is happy. It cannot reasonably be doubted, but
\page{220} a little miss, dressed in a new gown for a dancing-school
ball, receives as compleat enjoyment as the greatest orator, who
triumphs in the splendour of his eloquence, while he governs the
passions and resolutions of a numerous assembly.

All the difference, therefore, between one man and another, with
regard to life, consists either in the \textit{passion}, or in the
\textit{enjoyment:} And these differences are sufficient to produce
the wide extremes of happiness and misery.

To be happy, the \textit{passion} must neither be too violent nor too
remiss. In the first case, the mind is in a perpetual hurry and
tumult; in the second, it sinks into a disagreeable indolence and
lethargy.

To be happy, the passion must be benign and social; not rough or
fierce. The affections of the latter kind are not near so agreeable to
the feeling, as those of the former. Who will compare rancour and
animosity, envy and revenge, to friendship, benignity, clemency, and
gratitude?

To be happy, the passion must be chearful and gay, not gloomy and
melancholy. A propensity to hope and joy is real riches: One to fear
and sorrow, real poverty.

Some passions or inclinations, in the \textit{enjoyment} of their
object, are not so steady or constant as others, nor convey such
durable pleasure and satisfaction. \textit{Philosophical devotion},
for instance, like the enthusiasm of a poet, is the transitory effect
of high spirits, great leisure, a fine genius, and a habit of study
and contemplation: But notwithstanding all these circumstances, an
abstract, invisible object, like that which \textit{natural} religion
alone presents to us, cannot long actuate the mind, or be of any
moment in life. To render the passion of continuance, we must find
some method of affecting the senses and imagination, and must embrace
some \textit{historical}, as well as \textit{philosophical} account of
the divinity. Popular superstitions and observances are even found to
be of use in this particular.

Though the tempers of men be very different, yet we may safely
pronounce in general, that a life of pleasure cannot support itself so
long as one of business, but is much more subject to satiety and
disgust. The amusements, which are the most durable, have all a
mixture of application and attention in them; such as gaming and
hunting. And in general, business and action fill up all the great
vacancies in human life.

\page{221}But where the temper is the best disposed for any
\textit{enjoyment}, the object is often wanting: And in this respect,
the passions, which pursue external objects, contribute not so much to
happiness, as those which rest in ourselves; since we are neither so
certain of attaining such objects, nor so secure in possessing them. A
passion for learning is preferable, with regard to happiness, to one
for riches.

Some men are possessed of great strength of mind; and even when they
pursue \textit{external} objects, are not much affected by a
disappointment, but renew their application and industry with the
greatest chearfulness. Nothing contributes more to happiness than such
a turn of mind.

According to this short and imperfect sketch of human life, the
happiest disposition of mind is the \textit{virtuous;} or, in other
words, that which leads to action and employment, renders us sensible
to the social passions, steels the heart against the assaults of
fortune, reduces the affections to a just moderation, makes our own
thoughts an entertainment to us, and inclines us rather to the
pleasures of society and conversation, than to those of the senses.
This, in the mean time, must be obvious to the most careless reasoner,
that all dispositions of mind are not alike favourable to happiness,
and that one passion or humour may be extremely desirable, while
another is equally disagreeable. And indeed, all the difference
between the conditions of life depends upon the mind; nor is there any
one situation of affairs, in itself, preferable to another. Good and
ill, both natural and moral, are entirely relative to human sentiment
and affection. No man would ever be unhappy, could he alter his
feelings. \textsc{Proteus}-like, he would elude all attacks, by the
continual alterations of his shape and form.

But of this resource nature has, in a great measure, deprived us. The
fabric and constitution of our mind no more depends on our choice,
than that of our body. The generality of men have not even the
smallest notion, that any alteration in this respect can ever be
desirable. As a stream necessarily follows the several inclinations of
the ground, on which it runs; so are the ignorant and thoughtless part
of mankind actuated by their natural propensities. Such are
effectually excluded from all pretensions to philosophy, and the
\textit{medicine of the mind}, so much boasted. But even upon the wise
and thoughtful, nature has a prodigious influence; nor is it always
\page{222} in a man's power, by the utmost art and industry, to
correct his temper, and attain that virtuous character, to which he
aspires. The empire of philosophy extends over a few; and with regard
to these too, her authority is very weak and limited. Men may well be
sensible of the value of virtue, and may desire to attain it; but it
is not always certain, that they will be successful in their wishes.

Whoever considers, without prejudice, the course of human actions,
will find, that mankind are almost entirely guided by constitution and
temper, and that general maxims have little influence, but so far as
they affect our taste or sentiment. If a man have a lively sense of
honour and virtue, with moderate passions, his conduct will always be
conformable to the rules of morality; or if he depart from them, his
return will be easy and expeditious. On the other hand, where one is
born of so perverse a frame of mind, of so callous and insensible a
disposition, as to have no relish for virtue and humanity, no sympathy
with his fellow-creatures, no desire of esteem and applause; such a
one must be allowed entirely incurable, nor is there any remedy in
philosophy. He reaps no satisfaction but from low and sensual objects,
or from the indulgence of malignant passions: He feels no remorse to
controul his vicious inclinations: He has not even that sense or
taste, which is requisite to make him desire a better character: For
my part, I know not how I should address myself to such a one, or by
what arguments I should endeavour to reform him. Should I tell him of
the inward satisfaction which results from laudable and humane
actions, the delicate pleasure of disinterested love and friendship,
the lasting enjoyments of a good name and an established character, he
might still reply, that these were, perhaps, pleasures to such as were
susceptible of them; but that, for his part, he finds himself of a
quite different turn and disposition. I must repeat it; my philosophy
affords no remedy in such a case, nor could I do any thing but lament
this person's unhappy condition. But then I ask, If any other
philosophy can afford a remedy; or if it be possible, by any system,
to render all mankind virtuous, however perverse may be their natural
frame of mind? Experience will soon convince us of the contrary; and I
will venture to affirm, that, perhaps, the chief benefit, which
results from \page{223} philosophy, arises in an indirect
manner,\footnote{[The remainder of this sentence does not occur in
Editions C and D.]} and proceeds more from its secret, insensible
influence, than from its immediate application.

It is certain, that a serious attention to the sciences and liberal
arts softens and humanizes the temper, and cherishes those fine
emotions, in which true virtue and honour consists. It rarely, very
rarely happens, that a man of taste and learning is not, at least, an
honest man, whatever frailties may attend him. The bent of his mind to
speculative studies must mortify in him the passions of interest and
ambition, and must, at the same time, give him a greater sensibility
of all the decencies and duties of life. He feels more fully a moral
distinction in characters and manners; nor is his sense of this kind
diminished, but, on the contrary, it is much encreased, by
speculation.

Besides such insensible changes upon the temper and disposition, it is
highly probable, that others may be produced by study and application.
The prodigious effects of education may convince us, that the mind is
not altogether stubborn and inflexible, but will admit of many
alterations from its original make and structure. Let a man propose to
himself the model of a character, which he approves: Let him be well
acquainted with those particulars, in which his own character deviates
from this model: Let him keep a constant watch over himself, and bend
his mind, by a continual effort, from the vices, towards the virtues;
and I doubt not but, in time, he will find, in his temper, an
alteration for the better.

Habit is another powerful means of reforming the mind, and implanting
in it good dispositions and inclinations. A man, who continues in a
course of sobriety and temperance, will hate riot and disorder: If he
engage in business or study, indolence will seem a punishment to him:
If he constrain himself to practise beneficence and affability, he
will soon abhor all instances of pride and violence. Where one is
thoroughly convinced that the virtuous course of life is preferable;
if he have but resolution enough, for some time, to impose a violence
on himself; his reformation needs not be despaired of. The misfortune
is, that this conviction and this resolution never can have place,
unless a man be, before-hand, tolerably virtuous.

\page{224}Here then is the chief triumph of art and philosophy: It
insensibly refines the temper, and it points out to us those
dispositions which we should endeavour to attain, by a constant
\textit{bent} of mind, and by repeated \textit{habit}. Beyond this I
cannot acknowledge it to have great influence; and I must entertain
doubts concerning all those exhortations and consolations, which are
in such vogue among speculative reasoners.

We have already observed, that no objects are, in themselves,
desirable or odious, valuable or despicable; but that objects acquire
these qualities from the particular character and constitution of the
mind, which surveys them. To diminish therefore, or augment any
person's value for an object, to excite or moderate his passions,
there are no direct arguments or reasons, which can be employed with
any force or influence. The catching of flies, like \textsc{Domitian},
if it give more pleasure, is preferable to the hunting of wild beasts,
like \textsc{William Rufus}, or conquering of kingdoms, like
\textsc{Alexander}.

But though the value of every object can be determined only by the
sentiment or passion of every individual, we may observe, that the
passion, in pronouncing its verdict, considers not the object simply,
as it is in itself, but surveys it with all the circumstances, which
attend it. A man transported with joy, on account of his possessing a
diamond, confines not his view to the glistering stone before him: He
also considers its rarity, and thence chiefly arises his pleasure and
exultation. Here therefore a philosopher may step in, and suggest
particular views, and considerations, and circumstances, which
otherwise would have escaped us; and, by that means, he may either
moderate or excite any particular passion.

It may seem unreasonable absolutely to deny the authority of
philosophy in this respect: But it must be confessed, that there lies
this strong presumption against it, that, if these views be natural
and obvious, they would have occurred of themselves, without the
assistance of philosophy; if they be not natural, they never can have
any influence on the affections. \textit{These} are of a very delicate
nature, and cannot be forced or constrained by the utmost art or
industry. A consideration, which we seek for on purpose, which we
enter into with difficulty, which we cannot retain without \page{225}
care and attention, will never produce those genuine and durable
movements of passion, which are the result of nature, and the
constitution of the mind. A man may as well pretend to cure himself of
love, by viewing his mistress through the \textit{artificial} medium
of a microscope or prospect, and beholding there the coarseness of her
skin, and monstrous disproportion of her features, as hope to excite
or moderate any passion by the \textit{artificial} arguments of a
\textsc{Seneca} or an \textsc{Epictetus}. The remembrance of the
natural aspect and situation of the object, will, in both cases, still
recur upon him. The reflections of philosophy are too subtile and
distant to take place in common life, or eradicate any affection. The
air is too fine to breathe in, where it is above the winds and clouds
of the atmosphere.

Another defect of those refined reflections, which philosophy suggests
to us, is, that commonly they cannot diminish or extinguish our
vicious passions, without diminishing or extinguishing such as are
virtuous, and rendering the mind totally indifferent and unactive.
They are, for the most part, general, and are applicable to all our
affections. In vain do we hope to direct their influence only to one
side. If by incessant study and meditation we have rendered them
intimate and present to us, they will operate throughout, and spread
an universal insensibility over the mind. When we destroy the nerves,
we extinguish the sense of pleasure, together with that of pain, in
the human body.

It will be easy, by one glance of the eye, to find one or other of
these defects in most of those philosophical reflections, so much
celebrated both in ancient and modern times. \textit{Let not the
injuries or violence of men}, say the
philosophers,\footnote{\textsc{Plut}. \textit{de ira cohibenda}.}
\textit{ever discompose you by anger or hatred. Would you be angry at
the ape for its malice, or the tyger for its ferocity?} This
reflection leads us into a bad opinion of human nature, and must
extinguish the social affections. It tends also to prevent all remorse
for a man's own crimes; when he considers, that vice is as natural to
mankind, as the particular instincts to brute-creatures.

\textit{All ills arise from the order of the universe, which is
absolutely perfect. Would you wish to disturb so divine an order for
the sake of your own particular interest?} What if the ills I suffer
arise from malice or oppression? \textit{But the vices and \page{226}
imperfections of men are also comprehended in the order of the
universe:}

\begin{verse}
\textit{If plagues and earthquakes break not heav'n's design,\\
Why then a \textsc{Borgia} or a \textsc{Catiline}?}
\end{verse}

\noindent Let this be allowed; and my own vices will also be a part of
the same order.

\footnote{[This paragraph does not occur in Editions C and D.]}To one
who said, that none were happy, who were not above opinion, a
\textsc{Spartan} replied, \textit{then none are happy but knaves and
robbers}.\footnote{\textsc{Plut}. \textit{Lacon. Apophtheg}.}

\textit{Man is born to be miserable; and is he surprized at any
particular misfortune? And can he give way to sorrow and lamentation
upon account of any disaster?} Yes: He very reasonably laments, that
he should be born to be miserable. Your consolation presents a hundred
ills for one, of which you pretend to ease him.

\textit{You should always have before your eyes death, disease,
poverty, blindness, exile, calumny, and infamy, as ills which are
incident to human nature. If any one of these ills falls to your lot,
you will bear it the better, when you have reckoned upon it}. I
answer, if we confine ourselves to a general and distant reflection on
the ills of human life, \textit{that} can have no effect to prepare us
for them. If by close and intense meditation we render them present
and intimate to us, \textit{that} is the true secret for poisoning all
our pleasures, and rendering us perpetually miserable.

\textit{Your sorrow is fruitless, and will not change the course of
destiny}. Very true: And for that very reason I am sorry.

\textit{Cicero's} consolation for deafness is somewhat curious.
\textit{How many languages are there}, says he, \textit{which you do
not understand? The} \textsc{Punic}, \textsc{Spanish},
\textsc{Gallic}, \textsc{\AE gyptian}, \textit{\&c. With regard to all
these, you are as if you were deaf, yet you are indifferent about the
matter. Is it then so great a misfortune to be deaf to one language
more?}\footnote{\textsc{Tusc}. \textit{Quest}. lib. v. 40.}

I like better the repartee of \textsc{Antipater} the
\textsc{Cyreniac}, when some women were condoling with him for his
blindness: \textit{What!} says he, \textit{Do you think there are no
pleasures in the dark?}

\textit{Nothing can be more destructive}, says \textsc{Fontenelle},
\textit{to ambition, and the passion for conquest, than the true
system of \page{227} astronomy. What a poor thing is even the whole
globe in comparison of the infinite extent of nature?} This
consideration is evidently too distant ever to have any effect. Or, if
it had any, would it not destroy patriotism as well as ambition? The
same gallant author adds with some reason, that the bright eyes of the
ladies are the only objects, which lose nothing of their lustre or
value from the most extensive views of astronomy, but stand proof
against every system. Would philosophers advise us to limit our
affection to them?

\footnote{[The two following paragraphs do not occur in Editions C and
D.]}\textit{Exile}, says \textsc{Plutarch} to a friend in banishment,
\textit{is no evil: Mathematicians tell us, that the whole earth is
but a point, compared to the heavens. To change one's country then is
little more than to remove from one street to another. Man is not a
plant, rooted to a certain spot of earth: All soils and all climates
are alike suited to him}.\footnote{\textit{De exilio}.} These topics
are admirable, could they fall only into the hands of banished
persons. But what if they come also to the knowledge of those who are
employed in public affairs, and destroy all their attachment to their
native country? Or will they operate like the quack's medicine, which
is equally good for a diabetes and a dropsy?

It is certain, were a superior being thrust into a human body, that
the whole of life would to him appear so mean, contemptible, and
puerile, that he never could be induced to take part in any thing, and
would scarcely give attention to what passes around him. To engage him
to such a condescension as to play even the part of a \textsc{Philip}
with zeal and alacrity, would be much more difficult, than to
constrain the same \textsc{Philip}, after having been a king and a
conqueror during fifty years, to mend old shoes with proper care and
attention; the occupation which \textsc{Lucian} assigns him in the
infernal regions. Now all the same topics of disdain towards human
affairs, which could operate on this supposed being, occur also to a
philosopher; but being, in some measure, disproportioned to human
capacity, and not being fortified by the experience of any thing
better, they make not a full impression on him. He sees, but he feels
not sufficiently their truth; and is always a sublime philosopher,
when he needs not; that is, as long as nothing disturbs him, or rouzes
his affections. While others play, he wonders at their
keen-\page{228}ness and ardour; but he no sooner puts in his own
stake, than he is commonly transported with the same passions, that he
had so much condemned, while he remained a simple spectator.

There are two considerations chiefly, to be met with in books of
philosophy, from which any important effect is to be expected, and
that because these considerations are drawn from common life, and
occur upon the most superficial view of human affairs. When we reflect
on the shortness and uncertainty of life, how despicable seem all our
pursuits of happiness? And even, if we would extend our concern beyond
our own life, how frivolous appear our most enlarged and most generous
projects; when we consider the incessant changes and revolutions of
human affairs, by which laws and learning, books and governments are
hurried away by time, as by a rapid stream, and are lost in the
immense ocean of matter? Such a reflection certainly tends to mortify
all our passions: But does it not thereby counterwork the artifice of
nature, who has happily deceived us into an opinion, that human life
is of some importance? And may not such a reflection be employed with
success by voluptuous reasoners, in order to lead us, from the paths
of action and virtue, into the flowery fields of indolence and
pleasure?

% 'ATH\-ENS' necessary for hyphenation

We are informed by \textsc{Thucydides}, that, during the famous plague
of \textsc{Athens}, when death seemed present to every one, a
dissolute mirth and gaiety prevailed among the people, who exhorted
one another to make the most of life as long as it endured.
\footnote{[This sentence does not occur in Editions C and D.]}The same
observation is made by \textsc{Boccace} with regard to the plague of
\textsc{Florence}. A like principle makes soldiers, during war, be
more addicted to riot and expence, than any other race of men.
\footnote{[In place of this sentence Editions C and D read as follows:
And 'tis observable, in this Kingdom, that long Peace, by producing
Security, has much alter'd them in this Particular, and has quite
remov'd our Officers from the generous Character of their
Profession.]}Present pleasure is always of importance; and whatever
diminishes the importance of all other objects must bestow on it an
additional influence and value.

The \textit{second} philosophical consideration, which may often have
an influence on the affections, is derived from a comparison of our
own condition with the condition of others. This comparison we are
continually making, even in common \page{229} life; but the misfortune
is, that we are rather apt to compare our situation with that of our
superiors, than with that of our inferiors. A philosopher corrects
this natural infirmity, by turning his view to the other side, in
order to render himself easy in the situation, to which fortune has
confined him. There are few people, who are not susceptible of some
consolation from this reflection, though, to a very good-natured man,
the view of human miseries should rather produce sorrow than comfort,
and add, to his lamentations for his own misfortunes, a deep
compassion for those of others. Such is the imperfection, even of the
best of these philosophical topics of consolation.\footnote{The
Sceptic, perhaps, carries the matter too far, when he limits all
philosophical topics and reflections to these two. There seem to be
others, whose truth is undeniable, and whose natural tendency is to
tranquillize and soften all the passions. Philosophy greedily seizes
these, studies them, weighs them, commits them to the memory, and
familiarizes them to the mind: And their influence on tempers, which
are thoughtful, gentle, and moderate, may be considerable. But what is
their influence, you will say, if the temper be antecedently disposed
after the same manner as that to which they pretend to form it? They
may, at least, fortify that temper, and furnish it with views, by
which it may entertain and nourish itself. Here are a few examples of
such philosophical reflections.

\begin{enumerate}

\item Is it not certain, that every condition has concealed ills? Then
why envy any body?

\item Every one has known ills; and there is a compensation
throughout. Why not be contented with the present?

\item Custom deadens the sense both of the good and the ill, and
levels every thing.

\item Health and humour all. The rest of little consequence, except
these be affected.

\item How many other good things have I? Then why be vexed for one
ill?

\item How many are happy in the condition of which I complain? How
many envy me?

\item Every good must be paid for: Fortune by labour, favour by
flattery. Would I keep the price, yet have the commodity?

\item Expect not too great happiness in life. Human nature admits it
not.

\item Propose not a happiness too complicated. But does that depend on
me? Yes: The first choice does. Life is like a game: One may choose
the game: And passion, by degrees, seizes the proper object.

\item Anticipate by your hopes and fancy future consolation, which
time infallibly brings to every affliction.

\item I desire to be rich. Why? That I may possess many fine objects;
houses, gardens, equipage, \&c. How many fine objects does nature
offer to every one without expence? If enjoyed, sufficient. If not:
See the effect of custom or of temper, which would soon take off the
relish of the riches.

\item I desire fame. Let this occur: If I act well, I shall have the
esteem of all my acquaintance. And what is all the rest to me?

\end{enumerate}

These reflections are so obvious, that it is a wonder they occur not
to every man: So convincing, that it is a wonder they persuade not
every man. But perhaps they do occur to and persuade most men; when
they consider human life, by a general and calm survey: But where any
real, affecting incident happens; when passion is awakened, fancy
agitated, example draws, and counsel urges; the philosopher is lost in
the man, and he seeks in vain for that persuasion which before seemed
so firm and unshaken. What remedy for this inconvenience? Assist
yourself by a frequent perusal of the entertaining moralists: Have
recourse to the learning of \textsc{Plutarch}, the imagination of
\textsc{Lucian}, the eloquence of \textsc{Cicero}, the wit of
\textsc{Seneca}, the gaiety of \textsc{Montaigne}, the sublimity of
\textsc{Shaftesbury}. Moral precepts, so couched, strike deep, and
fortify the mind against the illusions of passion. But trust not
altogether to external aid: By habit and study acquire that
philosophical temper which both gives force to reflection, and by
rendering a great part of your happiness independent, takes off the
edge from all disorderly passions, and tranquillizes the mind. Despise
not these helps; but confide not too much in them neither; unless
nature has been favourable in the temper, with which she has endowed
you.}

I shall conclude this subject with observing, that, though virtue be
undoubtedly the best choice, when it is attainable; yet such is the
disorder and confusion of human affairs, that \page{230} no perfect or
regular distribution of happiness and misery is ever, in this life, to
be expected. Not only the goods of fortune, and the endowments of
the body (both of which are important), not only these advantages, I
say, are unequally divided between the virtuous and vicious, but even
the mind itself partakes, in some degree, of this disorder, and the
most worthy character, by the very constitution of the passions,
enjoys not always the highest felicity.

It is observable, that, though every bodily pain proceeds from some
disorder in the part or organ, yet the pain is not always proportioned
to the disorder; but is greater or less, according to the greater or
less sensibility of the part, upon which the noxious humours exert
their influence. A \textit{tooth-ach} produces more violent
convulsions of pain than a \textit{phthisis} or a \textit{dropsy}. In
like manner, with regard to the {\oe}conomy of the mind, we may
observe, that all vice is indeed pernicious; yet the disturbance or
pain is not measured out by nature with exact proportion to the degree
of vice, nor is the man of highest virtue, even abstracting from
external accidents, always the most happy. A gloomy and melancholy
disposition is certainly, \textit{to our sentiments}, a vice or
imperfection; but as it may be accompanied with great sense of honour
and great integrity, it may be found in very worthy characters; though
it is sufficient alone to imbitter life, and render the person
affected with it completely miserable. On the other hand, a selfish
villain may possess a spring and alacrity of temper, a certain
\footnote{[Gaiet\'e de C{\oe}ur: Edition C.]}\textit{gaiety of heart},
which is indeed a good quality, but which is rewarded much beyond its
merit, and when attended with good fortune, will compensate for the
uneasiness and remorse arising from all the other vices.

I shall add, as an observation to the same purpose, that, if a man be
liable to a vice or imperfection, it may often happen, that a good
quality, which he possesses along with it, \page{231} will render him
more miserable, than if he were completely vicious. A person of such
imbecility of temper as to be easily broken by affliction, is more
unhappy for being endowed with a generous and friendly disposition,
which gives him a lively concern for others, and exposes him the more
to fortune and accidents. A sense of shame, in an imperfect character,
is certainly a virtue; but produces great uneasiness and remorse, from
which the abandoned villain is entirely free. A very amorous
complexion, with a heart incapable of friendship, is happier than the
same excess in love, with a generosity of temper, which transports a
man beyond himself, and renders him a total slave to the object of his
passion.

In a word, human life is more governed by fortune than by reason; is
to be regarded more as a dull pastime than as a serious occupation;
and is more influenced by particular humour, than by general
principles. Shall we engage ourselves in it with passion and anxiety?
It is not worthy of so much concern. Shall we be indifferent about
what happens? We lose all the pleasure of the game by our phlegm and
carelessness. While we are reasoning concerning life, life is gone;
and death, though \textit{perhaps} they receive him differently, yet
treats alike the fool and the philosopher. To reduce life to exact
rule and method, is commonly a painful, oft a fruitless occupation:
And is it not also a proof, that we overvalue the prize for which we
contend? Even to reason so carefully concerning it, and to fix with
accuracy its just idea, would be overvaluing it, were it not that, to
some tempers, this occupation is one of the most amusing, in which
life could possibly be employed.

