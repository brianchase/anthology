
\author{Augustine of Hippo}
\authdate{354--430}
%\textdate{ca. 427}
\textdate{begun ca. 395/6; completed 427}
\addon{Book 2, Chapters 31 through 34}
\chapter[On Christian Doctrine, bk 2.31--2.34]{On Christian
Doctrine}
\source{augustine1887b}

\page{550}\section{Chap. 31. Uses Of Dialectics. Of Fallacies.}

48. There remain those branches of knowledge which pertain not to the
bodily senses, but to the intellect, among which the science of
reasoning and that of number are the chief. The science of reasoning
is of very great service in searching into and unravelling all sorts
of questions that come up in Scripture, only in the use of it we must
guard against the love of wrangling, and the childish vanity of
entrapping an adversary. For there are many of what are called
\textit{sophisms}, inferences in reasoning that are false, and yet so
close an imitation of the true, as to deceive not only dull people,
but clever men too, when they are not on their guard. For example, one
man lays before another with whom he is talking, the proposition,
``What I am, you are not.'' The other assents, for the proposition is
in part true, the one man being cunning and the other simple. Then the
first speaker adds: ``I am a man;'' and when the other has given his
assent to this also, the first draws his conclusion: ``Then you are
not a man.'' Now at this sort of ensnaring arguments, Scripture, as I
judge, expresses detestation in that place where it is said, ``There
is one that showeth wisdom in words, and is
hated;''\footnote{\textit{Qui sophistice loquitur, odibilis est}.
Ecclus. xxxvii. 20.} although, indeed, a style of speech which is not
intended to entrap, but \page{551} only aims at verbal ornamentation
more than is consistent with seriousness of purpose, is also called
sophistical.

49. There are also valid processes of reasoning which lead to false
conclusions, by following out to its logical consequences the error of
the man with whom one is arguing; and these conclusions are sometimes
drawn by a good and learned man, with the object of making the person
from whose error these consequences result, feel ashamed of them, and
of thus leading him to give up his error, when he finds that if he
wishes to retain his old opinion, he must of necessity also hold other
opinions which he condemns. For example, the apostle did not draw true
conclusions when he said, ``Then is Christ not risen,'' and again,
``Then is our preaching vain, and your faith is also
vain;''\footnote{1 Cor. xv. 13, 14.} and further on drew other
inferences which are all utterly false; for Christ has risen, the
preaching of those who declared this fact was not in vain, nor was
their faith in vain who had believed it. But all these false
inferences followed legitimately from the opinion of those who said
that there is no resurrection of the dead. These inferences, then,
being repudiated as false, it follows that since they would be true if
the dead rise not, there will be a resurrection of the dead. As, then,
valid conclusions may be drawn not only from true but from false
propositions, the laws of valid reasoning may easily be learnt in the
schools, outside the pale of the Church. But the truth of propositions
must be inquired into in the sacred books of the Church.

\section{Chap. 32. Valid Logical Sequence Is Not Devised But Only
Observed By Man.}

50. And yet the validity of logical sequences is not a thing devised
by men, but is observed and noted by them that they may be able to
learn and teach it; for it exists eternally in the reason of things,
and has its origin with God. For as the man who narrates the order of
events does not himself create that order; and as he who describes the
situations of places, or the natures of animals, or roots, or
minerals, does not describe arrangements of man; and as he who points
out the stars and their movements does not point out anything that he
himself or any other man has ordained;---in the same way, he who says,
``When the consequent is false, the antecedent must also be false,''
says what is most true; but he does not himself make it so, he only
points out that it is so. And it is upon this rule that the reasoning
I have quoted from the Apostle Paul proceeds. For the antecedent is,
``There is no resurrection of the dead,''---the position taken up by
those whose error the apostle wished to overthrow. Next, from this
antecedent, the assertion, viz., that there is no resurrection of the
dead, the necessary consequence is, ``Then Christ is not risen.'' But
this consequence is false, for Christ has risen; therefore the
antecedent is also false. But the antecedent is, that there is no
resurrection of the dead. We conclude, therefore, that there is a
resurrection of the dead. Now all this is briefly expressed thus: If
there is no resurrection of the dead, then is Christ not risen; but
Christ is risen, therefore there is a resurrection of the dead. This
rule, then, that when the consequent is removed, the antecedent must
also be removed, is not made by man, but only pointed out by him. And
this rule has reference to the validity of the reasoning, not to the
truth of the statements.

\section{Chap. 33. False Inferences May Be Drawn From Valid
Reasonings, And Vice Versa.}

51. In this passage, however, where the argument is about the
resurrection, both the law of the inference is valid, and the
conclusion arrived at is true. But in the case of false conclusions,
too, there is a validity of inference in some such way as the
following. Let us suppose some man to have admitted: If a snail is an
animal, it has a voice. This being admitted, then, when it has been
proved that the snail has no voice, it follows (since when the
consequent is proved false, the antecedent is also false) that the
snail is not an animal. Now this conclusion is false, but it is a true
and valid inference from the false admission. Thus, the truth of a
statement stands on its own merits; the validity of an inference
depends on the statement or the admission of the man with whom one is
arguing. And thus, as I said above, a false inference may be drawn by
a valid process of reasoning, in order that he whose error we wish to
correct may be sorry that he has admitted the antecedent, when he sees
that its logical consequences are utterly untenable. And hence it is
easy to understand that as the inferences may be valid where the
opinions are false, so the inferences may be unsound where the
opinions are true. For example, suppose that a man propounds the
statement, ``If this man is just, he is good,'' and we admit its
truth. Then he adds, ``But he is not just;'' and when we admit this
too, he draws the conclusion, ``Therefore he is not good.'' Now
although every one of these \page{552} statements may be true, still
the principle of the inference is unsound. For it is not true that, as
when the consequent is proved false the antecedent is also false, so
when the antecedent is proved false the consequent is false. For the
statement is true, ``If he is an orator, he is a man.'' But if we add,
``He is not an orator,'' the consequence does not follow, ``He is not
a man.''

\section{Chap. 34. It Is One Thing To Know The Laws Of Inference,
Another To Know The Truth Of Opinions.}

52. Therefore it is one thing to know the laws of inference, and
another to know the truth of opinions. In the former case we learn
what is consequent, what is inconsequent, and what is incompatible. An
example of a consequent is, ``If he is an orator, he is a man;'' of an
inconsequent, ``If he is a man, he is an orator;'' of an incompatible,
``If he is a man, he is a quadruped.'' In these instances we judge of
the connection. In regard to the truth of opinions, however, we must
consider propositions as they stand by themselves, and not in their
connection with one another; but when propositions that we are not
sure about are joined by a valid inference to propositions that are
true and certain, they themselves, too, necessarily become certain.
Now some, when they have ascertained the validity of the inference,
plume themselves as if this involved also the truth of the
propositions. Many, again, who hold the true opinions have an
unfounded contempt for themselves, because they are ignorant of the
laws of inference; whereas the man who knows that there is a
resurrection of the dead is assuredly better than the man who only
knows that it follows that if there is no resurrection of the dead,
then is Christ not risen.

