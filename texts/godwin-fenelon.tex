
\author{William Godwin}
\authdate{1756--1836}
\textdate{1793}
%\chapter[William Godwin -- F\'enelon or the Chambermaid?]{F\'enelon or
%the Chambermaid?}
%\chapter[William Godwin -- F\'enelon or the Chambermaid?]{F\'enelon or
%the Chambermaid?\\\smaller An Enquiry Concerning Political Justice and
%Its Influence on General Virtue and Happiness, excerpt}
\chapter[William Godwin -- F\'enelon or the Chambermaid?]{F\'enelon or
the Chambermaid?\\\smaller An Enquiry Concerning Political Justice and
Its Influence on General Virtue and Happiness\\\smaller Book 2,
Chapter 2, excerpt}

%\nfootnote{\fullcite[bk. 2, chap. 2, pp. 81--84]{godwin1793.1}}
\nfootnote{\fullcite{godwin1793.1}}

\page{81}In a loose and general view I and my neighbour are both of us
men; and of consequence entitled to equal attention. But in reality it
is probable that one of us is a being of more worth and importance
than the other. A man is of more worth than a beast; because, being
possessed of higher faculties, he is capable of a more refined and
genuine happiness. In the same manner \page{82} the illustrious
archbishop of Cambray was of more worth than his chambermaid, and
there are few of us that would hesitate to pronounce, if his palace
were in flames, and the life of only one of them could be preserved,
which of the two ought to be preferred.

But there is another ground of preference, beside the private
consideration of one of them being further removed from the state of a
mere animal. We are not connected with one or two percipient beings,
but with a society, a nation, and in some sense with the whole family
of mankind. Of consequence that life ought to be preferred which will
be most conducive to the general good. In saving the life of Fenelon,
suppose at the moment when he was conceiving the project of his
immortal \textit{Telemachus}, I should be promoting the benefit of
thousands, who have been cured by the perusal of it of some error,
vice and consequent unhappiness. Nay, my benefit would extend further
than this, for every individual thus cured has become a better member
of society, and has contributed in his turn to the happiness, the
information and improvement of others.

Suppose I had been myself the chambermaid, I ought to have chosen to
die, rather than Fenelon should have died. The life of Fenelon was
really preferable to that of the chambermaid. But understanding is the
faculty that perceives the truth of this and similar propositions; and
justice is the principle that \page{83} regulates my conduct
accordingly. It would have been just in the chambermaid to have
preferred the archbishop to herself. To have done otherwise would have
been a breach of justice.

Supposing the chambermaid had been my wife, my mother, or my
benefactor. This would not alter the truth of the proposition. The
life of Fenelon would still be more valuable than that of the
chambermaid; and justice, pure, unadulterated justice, would still
have preferred that which was most valuable. Justice would have taught
me to save the life of Fenelon at the expense of the other. What magic
is there in the pronoun ``my,'' to overturn the decisions of
everlasting truth? My wife or my mother may be a fool or a prostitute,
malicious, lying or dishonest. If they be, of what consequence is it
that they are mine?

``But my mother endured for me the pains of child bearing, and
nourished me in the helplessness of infancy.'' When she first
subjected herself to the necessity of these cares, she was probably
influenced by no particular motives of benevolence to her future
offspring. Every voluntary benefit however entitles the bestower to
some kindness and retribution. But why so? Because a voluntary benefit
is an evidence of benevolent intention, that is, of virtue. It is the
disposition of the mind, not the external action, that entitles to
respect. But the merit of this disposition is equal, whether the
benefit was conferred upon me or upon another. I and another man
cannot both be right in preferring \page{84} our own individual
benefactor, for my no man can be at the same time both better and
worse than his neighbour. My benefactor ought to be esteemed, not
because he bestowed a benefit upon me, but because he bestowed it upon
a human being. His desert will be in exact proportion to the degree,
in which that human being was worthy of the distinction conferred.
Thus every view of the subject brings us back to the consideration of
my neighbour's moral worth and his importance to the general weal, as
the only standard to determine the treatment to which he is
entitled. Gratitude therefore, a principle which has so often been the
theme of the moralist and the poet, is no part either of justice or
virtue. By gratitude I understand a sentiment, which would lead me to
prefer one man to another, from some other consideration than that of
his superior usefulness or worth: that is, which would make something
true to me (for example this preferableness), which cannot be true to
another man, and is not true in itself.

