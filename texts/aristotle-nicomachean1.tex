
\author{Aristotle}
\authdate{384--322 \BCE}
\textdate{ca. late fourth century \BCE}
\addon{Book 1}
\chapter[Nicomachean Ethics, bk. 1]{Nicomachean Ethics}
\source{aristotle1906}

\page{1}1. Every art and every kind of inquiry, and likewise every act
and purpose, seems to aim at some good: and so it has been well said
that the good is that at which everything aims.

But a difference is observable among these aims or ends. What is aimed
at is sometimes the exercise of a faculty, sometimes a certain result
beyond that exercise. And where there is an end beyond the act, there
the result is better than the exercise of the faculty.

Now since there are many kinds of actions and many arts and sciences,
it follows that there are many ends also; \textit{e.g.} health is the
end of medicine, ships of shipbuilding, victory of the art of war, and
wealth of economy.

But when several of these are subordinated to \page{2} some one art or
science,---as the making of bridles and other trappings to the art of
horsemanship, and this in turn, along with all else that the soldier
does, to the art of war, and so on,\footnote{Reading \grk{τὸν αὐτὸν
δέ}.}---then the end of the master-art is always more desired than the
ends of the subordinate arts, since these are pursued for its sake.
And this is equally true whether the end in view be the mere exercise
of a faculty or something beyond that, as in the above instances.

2. If then in what we do there be some end which we wish for on its
own account, choosing all the others as means to this, but not every
end without exception as a means to something else (for so we should
go on \textit{ad infinitum}, and desire would be left void and
objectless),---this evidently will be the good or the best of all
things. And surely from a practical point of view it much concerns us
to know this good; for then, like archers shooting at a definite mark,
we shall be more likely to attain what we want.

If this be so, we must try to indicate roughly what it is, and first
of all to which of the arts or sciences it belongs.

It would seem to belong to the supreme art or science, that one which
most of all deserves the name of master-art or master-science.

Now Politics\footnote{To Aristotle Politics is a much wider term than
to us; it covers the whole field of human life, since man is
essentially social (7, 6); it has to determine (1) what is the
good?---the question of this treatise (\S9)---and (2) what can law do
to promote this good?---the question of the sequel, which is specially
called ``The Politics;'' \textit{cf.} X. 9.} seems to answer to this
description. \page{3} For it prescribes which of the sciences a state
needs, and which each man shall study, and up to what point; and to it
we see subordinated even the highest arts, such as economy, rhetoric,
and the art of war.

Since then it makes use of the other practical sciences, and since it
further ordains what men are to do and from what to refrain, its end
must include the ends of the others, and must be the proper good of
man.

For though this good is the same for the individual and the state, yet
the good of the state seems a grander and more perfect thing both to
attain and to secure; and glad as one would be to do this service for
a single individual, to do it for a people and for a number of states
is nobler and more divine.

This then is the aim of the present inquiry, which is a sort of
political inquiry.\footnote{\textit{i.e.} covers a part of the ground
only: see preceding note.}

3. We must be content if we can attain to so much precision in our
statement as the subject before us admits of; for the same degree of
accuracy is no more to be expected in all kinds of reasoning than in
all kinds of handicraft.

Now the things that are noble and just (with which Politics deals) are
so various and so uncertain, that some think these are merely
conventional and not natural distinctions.

There is a similar uncertainty also about what is good, because good
things often do people harm: men have before now been ruined by
wealth, and have lost their lives through courage.

Our subject, then, and our data being of this \page{4} nature, we must
be content if we can indicate the truth roughly and in outline, and
if, in dealing with matters that are not amenable to immutable laws,
and reasoning from premises that are but probable, we can arrive at
probable conclusions.\footnote{The expression \grk{τὰ ὡς ἐπὶ τὸ πολύ}
covers both (1) what is generally thought not universally true, and
(2) what is probable though not certain.}

The reader, on his part, should take each of my statements in the same
spirit; for it is the mark of an educated man to require, in each kind
of inquiry, just so much exactness as the subject admits of: it is
equally absurd to accept probable reasoning from a mathematician, and
to demand scientific proof from an orator.

But each man can form a judgment about what he knows, and is called
``a good judge'' of that---of any special matter when he has received
a special education therein, ``a good judge'' (without any qualifying
epithet) when he has received a universal education. And hence a young
man is not qualified to be a student of Politics; for he lacks
experience of the affairs of life, which form the data and the
subject-matter of Politics.

Further, since he is apt to be swayed by his feelings, he will derive
no benefit from a study whose aim is not speculative but practical.

But in this respect young in character counts the same as young in
years; for the young man's disqualification is not a matter of time,
but is due to the fact that feeling rules his life and directs all his
desires. Men of this character turn the knowledge \page{5} they get to
no account in practice, as we see with those we call incontinent; but
those who direct their desires and actions by reason will gain much
profit from the knowledge of these matters.

So much then by way of preface as to the student, and the spirit in
which he must accept what we say, and the object which we propose to
ourselves.

4. Since---to resume---all knowledge and all purpose aims at some
good, what is this which we say is the aim of Politics; or, in other
words, what is the highest of all realizable goods?

As to its name, I suppose nearly all men are agreed; for the masses
and the men of culture alike declare that it is happiness, and hold
that to ``live well'' or to ``do well'' is the same as to be
``happy.''

But they differ as to what this happiness is, and the masses do not
give the same account of it as the philosophers.

The former take it to be something palpable and plain, as pleasure or
wealth or fame; one man holds it to be this, and another that, and
often the same man is of different minds at different times,---after
sickness it is health, and in poverty it is wealth; while when they
are impressed with the consciousness of their ignorance, they admire
most those who say grand things that are above their comprehension.

Some philosophers, on the other hand, have thought that, beside these
several good things, there is an ``absolute'' good which is the cause
of their goodness.

As it would hardly be worth while to review all the opinions that have
been held, we will confine ourselves to those which are most popular,
or which seem to have some foundation in reason.

\page{6}But we must not omit to notice the distinction that is drawn
between the method of proceeding from your starting-points or
principles, and the method of working up to them. Plato used with
fitness to raise this question, and to ask whether the right way is
from or to your starting-points, as in the race-course you may run
from the judges to the boundary, or \textit{vice vers\^a}.

Well, we must start from what is known.

But ``what is known'' may mean two things: ``what is known to us,''
which is one thing, or ``what is known'' simply, which is another.

I think it is safe to say that \textit{we} must start from what is
known to \textit{us}.

And on this account nothing but a good moral training can qualify a
man to study what is noble and just---in a word, to study questions of
Politics. For the undemonstrated fact is here the starting-point, and
if this undemonstrated fact be sufficiently evident to a man, he will
not require a ``reason why.'' Now the man who has had a good moral
training either has already arrived at starting-points or principles
of action, or will easily accept them when pointed out. But he who
neither has them nor will accept them may hear what Hesiod
says\footnote{``Works and Days,'' 291--295.}---

\begin{verse}
\poke{``}The best is he who of himself doth know;\\
Good too is he who listens to the wise;\\
But he who neither knows himself nor heeds\\
The words of others, is a useless man.''
\end{verse}

5. Let us now take up the discussion at the point from which we
digressed.

\page{7}It seems that men not unreasonably take their notions of the
good or happiness from the lives actually led, and that the masses who
are the least refined suppose it to be pleasure, which is the reason
why they aim at nothing higher than the life of enjoyment.

For the most conspicuous kinds of life are three: this life of
enjoyment, the life of the statesman, and, thirdly, the contemplative
life.

The mass of men show themselves utterly slavish in their preference
for the life of brute beasts, but their views receive consideration
because many of those in high places have the tastes of Sardanapalus.

Men of refinement with a practical turn prefer honour; for I suppose
we may say that honour is the aim of the statesman's life.

But this seems too superficial to be the good we are seeking: for it
appears to depend upon those who give rather than upon those who
receive it; while we have a presentiment that the good is something
that is peculiarly a man's own and can scarce be taken away from him.

Moreover, these men seem to pursue honour in order that they may be
assured of their own excellence,---at least, they wish to be honoured by
men of sense, and by those who know them, and on the ground of their
virtue or excellence. It is plain, then, that in their view, at any
rate, virtue or excellence is better than honour; and perhaps we
should take this to be the end of the statesman's life, rather than
honour.

But virtue or excellence also appears too incomplete to be what we
want; for it seems that a man \page{8} might have virtue and yet be
asleep or be inactive all his life, and, moreover, might meet with the
greatest disasters and misfortunes; and no one would maintain that
such a man is happy, except for argument's sake. But we will not dwell
on these matters now, for they are sufficiently discussed in the
popular treatises.

The third kind of life is the life of contemplation: we will treat of
it further on.\footnote{\textit{Cf.} VI. 7, 12, and X. 7, 8.}

As for the money-making life, it is something quite contrary to
nature; and wealth evidently is not the good of which we are in
search, for it is merely useful as a means to something else. So we
might rather take pleasure and virtue or excellence to be ends than
wealth; for they are chosen on their own account. But it seems that
not even they are the end, though much breath has been wasted in
attempts to show that they are.

6. Dismissing these views, then, we have now to consider the
``universal good,'' and to state the difficulties which it presents;
though such an inquiry is not a pleasant task in view of our
friendship for the authors of the doctrine of ideas. But we venture to
think that this is the right course, and that in the interests of
truth we ought to sacrifice even what is nearest to us, especially as
we call ourselves philosophers. Both are dear to us, but it is a
sacred duty to give the preference to truth.

In the first place, the authors of this theory themselves did not
assert a common idea in the case of things of which one is prior to
the other; and for this \page{9} reason they did not hold one common
idea of numbers. Now the predicate good is applied to substances and
also to qualities and relations. But that which has independent
existence, what we call ``substance,'' is logically prior to that
which is relative; for the latter is an offshoot as it were, or [in
logical language] an accident of a thing or substance. So [by their
own showing] there cannot be one common idea of these goods.

Secondly, the term good is used in as many different ways as the term
``is'' or ``being:'' we apply the term to substances or independent
existences, as God, reason; to qualities, as the virtues; to quantity,
as the moderate or due amount; to relatives, as the useful; to time,
as opportunity; to place, as habitation, and so on. It is evident,
therefore, that the word good cannot stand for one and the same notion
in all these various applications; for if it did, the term could not
be applied in all the categories, but in one only.

Thirdly, if the notion were one, since there is but one science of all
the things that come under one idea, there would be but one science of
all goods; but as it is, there are many sciences even of the goods
that come under one category; as, for instance, the science which
deals with opportunity in war is strategy, but in disease is medicine;
and the science of the due amount in the matter of food is medicine,
but in the matter of exercise is the science of gymnastic.

Fourthly, one might ask what they mean by the ``absolute:'' in
``absolute man'' and ``man'' the word ``man'' has one and the same
sense; for in respect of manhood there will be no difference between
them; \page{10} and if so, neither will there be any difference in
respect of goodness between ``absolute good'' and ``good.''

Fifthly, they do not make the good any more good by making it eternal;
a white thing that lasts a long while is no whiter than what lasts but
a day.

There seems to be more plausibility in the doctrine of the
Pythagoreans, who [in their table of opposites] place the one on the
same side with the good things [instead of reducing all goods to
unity]; and even Speusippus\footnote{Plato's nephew and successor.}
seems to follow them in this.

However, these points may be reserved for another occasion; but
objection may be taken to what I have said on the ground that the
Platonists do not speak in this way of all goods indiscriminately, but
hold that those that are pursued and welcomed on their own account are
called good by reference to one common form or type, while those
things that tend to produce or preserve these goods, or to prevent
their opposites, are called good only as means to these, and in a
different sense.

It is evident that there will thus be two classes of goods: one good
in themselves, the other good as means to the former. Let us separate
then from the things that are merely useful those that are good in
themselves, and inquire if they are called good by reference to one
common idea or type.

Now what kind of things would one call ``good in themselves''?

Surely those things that we pursue even apart from their consequences,
such as wisdom and sight \page{11} and certain pleasures and certain
honours; for although we sometimes pursue these things as means, no
one could refuse to rank them among the things that are good in
themselves.

If these be excluded, nothing is good in itself except the idea; and
then the type or form will be meaningless.\footnote{For there is no
meaning in a form which is a form of nothing, in a universal which has
no particulars under it.}

If however, these are ranked among the things that are good in
themselves, then it must be shown that the goodness of all of them can
be defined in the same terms, as white has the same meaning when
applied to snow and to white lead.

But, in fact, we have to give a separate and different account of the
goodness of honour and wisdom and pleasure.

Good, then, is not a term that is applied to all these things alike in
the same sense or with reference to one common idea or form.

But how then do these things come to be called good? for they do not
appear to have received the same name by chance merely. Perhaps it is
because they all proceed from one source, or all conduce to one end;
or perhaps it is rather in virtue of some analogy, just as we call the
reason the eye of the soul because it bears the same relation to the
soul that the eye does to the body, and so on.

But we may dismiss these questions at present; for to discuss them in
detail belongs more properly to another branch of philosophy.

And for the same reason we may dismiss the \page{12} further
consideration of the idea; for even granting that this term good,
which is applied to all these different things, has one and the same
meaning throughout, or that there is an absolute good apart from
these particulars, it is evident that this good will not be anything
that man can realize or attain: but it is a good of this kind that we
are now seeking.

It might, perhaps, be thought that it would nevertheless be well to
make ourselves acquainted with this universal good, with a view to the
goods that are attainable and realizable. With this for a pattern, it
may be said, we shall more readily discern our own good, and
discerning achieve it.

There certainly is some plausibility in this argument, but it seems to
be at variance with the existing sciences; for though they are all
aiming at some good and striving to make up their deficiencies, they
neglect to inquire about this universal good. And yet it is scarce
likely that the professors of the several arts and sciences should not
know, nor even look for, what would help them so much.

And indeed I am at a loss to know how the weaver or the carpenter
would be furthered in his art by a knowledge of this absolute good, or
how a man would be rendered more able to heal the sick or to command
an army by contemplation of the pure form or idea. For it seems to me
that the physician does not even seek for health in this abstract way,
but seeks for the health of man, or rather of some particular man, for
it is individuals that he has to heal.

7. Leaving these matters, then, let us return once \page{13} more to
the question, what this good can be of which we are in search.

It seems to be different in different kinds of action and in different
arts,---one thing in medicine and another in war, and so on. What then
is the good in each of these cases? Surely that for the sake of which
all else is done. And that in medicine is health, in war is victory,
in building is a house,---a different thing in each different case,
but always, in whatever we do and in whatever we choose, the end. For
it is always for the sake of the end that all else is done.

If then there be one end of all that man does, this end will be the
realizable good,---or these ends, if there be more than one.

By this generalization our argument is brought to the same point as
before.\footnote{2, 1. See Stewart.} This point we must try to explain
more clearly.

We see that there are many ends. But some of these are chosen only as
means, as wealth, flutes, and the whole class of instruments. And so
it is plain that not all ends are final.

But the best of all things must, we conceive, be something final.

If then there be only one final end, this will be what we are
seeking,---or if there be more than one, then the most final of them.

Now that which is pursued as an end in itself is more final than that
which is pursued as means to something else, and that which is never
chosen as means than that which is chosen both as an end in itself and
as means, and that is strictly final which \page{14} is always chosen
as an end in itself and never as means.

Happiness seems more than anything else to answer to this description:
for we always choose it for itself, and never for the sake of
something else; while honour and pleasure and reason, and all virtue
or excellence, we choose partly indeed for themselves (for, apart from
any result, we should choose each of them), but partly also for the
sake of happiness, supposing that they will help to make us happy. But
no one chooses happiness for the sake of these things, or as a means
to anything else at all.

We seem to be led to the same conclusion when we start from the notion
of self-sufficiency.

The final good is thought to be self-sufficing [or all-sufficing]. In
applying this term we do not regard a man as an individual leading a
solitary life, but we also take account of parents, children, wife,
and, in short, friends and fellow-citizens generally, since man is
naturally a social being. Some limit must indeed be set to this; for
if you go on to parents and descendants and friends of friends, you
will never come to a stop. But this we will consider further on: for
the present we will take self-sufficing to mean what by itself makes
life desirable and in want of nothing. And happiness is believed to
answer to this description.

And further, happiness is believed to be the most desirable thing in
the world, and that not merely as one among other good things: if it
were merely one among other good things [so that other things could be
added to it], it is plain that the addition of the least \page{15} of
other goods must make it more desirable; for the addition becomes a
surplus of good, and of two goods the greater is always more
desirable.

Thus it seems that happiness is something final and self-sufficing,
and is the end of all that man does.

But perhaps the reader thinks that though no one will dispute the
statement that happiness is the best thing in the world, yet a still
more precise definition of it is needed.

This will best be gained, I think, by asking, What is the function of
man? For as the goodness and the excellence of a piper or a sculptor,
or the practiser of any art, and generally of those who have any
function or business to do, lies in that function, so man's good would
seem to lie in his function, if he has one.

But can we suppose that, while a carpenter and a cobbler has a
function and a business of his own, man has no business and no
function assigned him by nature? Nay, surely as his several members,
eye and hand and foot, plainly have each his own function, so we must
suppose that man also has some function over and above all these.

What then is it?

Life evidently he has in common even with the plants, but we want that
which is peculiar to him. We must exclude, therefore, the life of mere
nutrition and growth.

Next to this comes the life of sense; but this too he plainly shares
with horses and cattle and all kinds of animals.

There remains then the life whereby he acts---the \page{16} life of
his rational nature,\footnote{\grk{πρακτική τις τοῦ λόγον ἔχοντος}.
Aristotle frequently uses the terms \grk{πρᾶξις}, \grk{πρακτός},
\grk{πρακτικός} in this wide sense, covering all that man does,
\textit{i.e.} all that part of man's life that is within the control
of his will, or that is consciously directed to an end, including
therefore speculation as well as action.} with its two sides or
divisions, one rational as obeying reason, the other rational as
having and exercising reason.

But as this expression is ambiguous,\footnote{For it might mean either
the mere possession of the vital faculties, or their exercise.} we
must be understood to mean thereby the life that consists in the
exercise of the faculties; for this seems to be more properly entitled
to the name.

The function of man, then, is exercise of his vital faculties [or
soul] on one side in obedience to reason, and on the other side with
reason.

But what is called the function of a man of any profession and the
function of a man who is good in that profession are generically the
same, \textit{e.g.} of a harper and of a good harper; and this holds
in all cases without exception, only that in the case of the latter
his superior excellence at his work is added; for we say a harper's
function is to harp, and a good harper's to harp well.

(Man's function then being, as we say, a kind of life---that is to
say, exercise of his faculties and action of various kinds with
reason---the good man's function is to do this well and beautifully
[or nobly]. But the function of anything is done well when it is done
in accordance with the proper excellence of that thing.)\footnote{This
paragraph seems to be a repetition (I would rather say a re-writing)
of the previous paragraph. See note on VII. 3, 2.}

\page{17}If this be so the result is that the good of man is exercise
of his faculties in accordance with excellence or virtue, or, if there
be more than one, in accordance with the best and most complete
virtue.\footnote{This ``best and most complete excellence or virtue''
is the trained faculty for philosophic speculation, and the
contemplative life is man's highest happiness. \textit{Cf.} X. 7, 1.}

But there must also be a full term of years for this
exercise;\footnote{\textit{Cf.} 9, 11.} for one swallow or one fine
day does not make a spring, nor does one day or any small space of
time make a blessed or happy man.

This, then, may be taken as a rough outline of the good; for this, I
think, is the proper method,---first to sketch the outline, and then
to fill in the details. But it would seem that, the outline once
fairly drawn, any one can carry on the work and fit in the several
items which time reveals to us or helps us to find. And this indeed is
the way in which the arts and sciences have grown; for it requires no
extraordinary genius to fill up the gaps.

We must bear in mind, however, what was said above, and not demand the
same degree of accuracy in all branches of study, but in each case so
much as the subject-matter admits of and as is proper to that kind of
inquiry. The carpenter and the geometer both look for the right angle,
but in different ways: the former only wants such an approximation to
it as his work requires, but the latter wants to know what constitutes
a right angle, or what is its special quality; his aim is to find out
the truth. And so in other cases we must follow the same course, lest
we spend more \page{18} time on what is immaterial than on the real
business in hand.

Nor must we in all cases alike demand the reason why; sometimes it is
enough if the undemonstrated fact be fairly pointed out, as in the
case of the starting-points or principles of a science. Undemonstrated
facts always form the first step or starting-point of a science; and
these starting-points or principles are arrived at some in one way,
some in another---some by induction, others by perception, others
again by some kind of training. But in each case we must try to
apprehend them in the proper way, and do our best to define them
clearly; for they have great influence upon the subsequent course of
an inquiry. A good start is more than half the race, I think, and our
starting-point or principle, once found, clears up a number of our
difficulties.

8. We must not be satisfied, then, with examining this starting-point
or principle of ours as a conclusion from our data, but must also view
it in its relation to current opinions on the subject; for all
experience harmonizes with a true principle, but a false one is soon
found to be incompatible with the facts.

Now, good things have been divided into three classes, external goods
on the one hand, and on the other goods of the soul and goods of the
body; and the goods of the soul are commonly said to be goods in the
fullest sense, and more good than any other.

But ``actions and exercises of the vital faculties or soul'' may be
said to be ``of the soul.'' So our account is confirmed by this
opinion, which is both of long \page{19} standing and approved by all
who busy themselves with philosophy.

But, indeed, we secure the support of this opinion by the mere
statement that certain actions and exercises are the end; for this
implies that it is to be ranked among the goods of the soul, and not
among external goods.

Our account, again, is in harmony with the common saying that the
happy man lives well and does well; for we may say that happiness,
according to us, is a living well and doing well.

And, indeed, all the characteristics that men expect to find in
happiness seem to belong to happiness as we define it.

Some hold it to be virtue or excellence, some prudence, others a kind
of wisdom; others, again, hold it to be all or some of these, with the
addition of pleasure, either as an ingredient or as a necessary
accompaniment; and some even include external prosperity in their
account of it.

Now, some of these views have the support of many voices and of old
authority; others have few voices, but those of weight; but it is
probable that neither the one side nor the other is entirely wrong,
but that in some one point at least, if not in most, they are both
right.

First, then, the view that happiness is excellence or a kind of
excellence harmonizes with our account; for ``exercise of faculties in
accordance with excellence'' belongs to excellence.

But I think we may say that it makes no small difference whether the
good be conceived as the mere \page{20} possession of something, or as
its use---as a mere habit or trained faculty, or as the exercise of
that faculty. For the habit or faculty may be present, and yet issue
in no good result, as when a man is asleep, or in any other way
hindered from his function; but with its exercise this is not
possible, for it must show itself in acts and in good acts. And as at
the Olympic games it is not the fairest and strongest who receive the
crown, but those who contend (for among these are the victors), so in
life, too, the winners are those who not only have all the
excellences, but manifest these in deed.

And, further, the life of these men is in itself pleasant. For
pleasure is an affection of the soul, and each man takes pleasure in
that which he is said to love,---he who loves horses in horses, he who
loves sight-seeing in sight-seeing, and in the same way he who loves
justice in acts of justice, and generally the lover of excellence or
virtue in virtuous acts or the manifestation of excellence.

And while with most men there is a perpetual conflict between the
several things in which they find pleasure, since these are not
naturally pleasant, those who love what is noble take pleasure in that
which is naturally pleasant. For the manifestations of excellence are
naturally pleasant, so that they are both pleasant to them and
pleasant in themselves.

Their life, then, does not need pleasure to be added to it as an
appendage, but contains pleasure in itself.

Indeed, in addition to what we have said, a man is not good at all
unless he takes pleasure in noble deeds. No one would call a man just
who did not \page{21} take pleasure in doing justice, nor generous who
took no pleasure in acts of generosity, and so on.

If this be so, the manifestations of excellence will be pleasant in
themselves. But they are also both good and noble, and that in the
highest degree---at least, if the good man's judgment about them is
right, for this is his judgment.

Happiness, then, is at once the best and noblest and pleasantest thing
in the world, and these are not separated, as the Delian inscription
would have them to be:---

\begin{verse}
\poke{``}What is most just is noblest, health is best,\\
Pleasantest is to get your heart's desire.''
\end{verse}

\noindent For all these characteristics are united in the best
exercises of our faculties; and these, or some one of them that is
better than all the others, we identify with happiness.

But nevertheless happiness plainly requires external goods too, as we
said; for it is impossible, or at least not easy, to act nobly without
some furniture of fortune. There are many things that can only be done
through instruments, so to speak, such as friends and wealth and
political influence: and there are some things whose absence takes the
bloom off our happiness, as good birth, the blessing of children,
personal beauty; for a man is not very likely to be happy if he is
very ugly in person, or of low birth, or alone in the world, or
childless, and perhaps still less if he has worthless children or
friends, or has lost good ones that he had.

As we said, then, happiness seems to stand in need of this kind of
prosperity; and so some identify it \page{22} with good fortune, just
as others identify it with excellence.

9. This has led people to ask whether happiness is attained by
learning, or the formation of habits, or any other kind of training,
or comes by some divine dispensation or even by chance.

Well, if the Gods do give gifts to men, happiness is likely to be
among the number, more likely, indeed, than anything else, in
proportion as it is better than all other human things.

This belongs more properly to another branch of inquiry; but we may
say that even if it is not heavensent, but comes as a consequence of
virtue or some kind of learning or training, still it seems to be one
of the most divine things in the world; for the prize and aim of
virtue would appear to be better than anything else and something
divine and blessed.

Again, if it is thus acquired it will be widely accessible; for it
will then be in the power of all except those who have lost the
capacity for excellence to acquire it by study and diligence.

And if it be better that men should attain happiness in this way
rather than by chance, it is reasonable to suppose that it is so,
since in the sphere of nature all things are arranged in the best
possible way, and likewise in the sphere of art, and of each mode of
causation, and most of all in the sphere of the noblest mode of
causation. And indeed it would be too absurd to leave what is noblest
and fairest to the dispensation of chance.

But our definition itself clears up the
difficulty;\footnote{\textit{Cf. supra.} 7.21} \page{23} for happiness
was defined as a certain kind of exercise of the vital faculties in
accordance with excellence or virtue. And of the remaining goods
[other than happiness itself], some must be present as necessary
conditions, while others are aids and useful instruments to happiness.
And this agrees with what we said at starting. We then laid down that
the end of the art political is the best of all ends; but the chief
business of that art is to make the citizens of a certain
character---that is, good and apt to do what is noble. It is not
without reason, then, that we do not call an ox, or a horse, or any
brute happy; for none of them is able to share in this kind of
activity.

For the same reason also a child is not happy; he is as yet, because
of his age, unable to do such things. If we ever call a child happy,
it is because we hope he will do them. For, as we said, happiness
requires not only perfect excellence or virtue, but also a full term
of years for its exercise. For our circumstances are liable to many
changes and to all sorts of chances, and it is possible that he who is
now most prosperous will in his old age meet with great disasters, as
is told of Priam in the tales of Troy; and a man who is thus used by
fortune and comes to a miserable end cannot be called happy.

10. Are we, then, to call no man happy as long as he lives, but to
wait for the end, as Solon said?

And, supposing we have to allow this, do we mean that he actually is
happy after he is dead? Surely that is absurd, especially for us who
say that happiness is a kind of activity or life.

\page{24}But if we do not call the dead man happy, and if Solon meant
not this, but that only then could we safely apply the term to a man,
as being now beyond the reach of evil and calamity, then here too we
find some ground for objection. For it is thought that both good and
evil may in some sort befall a dead man (just as they may befall a
living man, although he is unconscious of them), \textit{e.g.} honours
rendered to him, or the reverse of these, and again the prosperity or
the misfortune of his children and all his descendants.

But this, too, has its difficulties; for after a man has lived happily
to a good old age, and ended as he lived, it is possible that many
changes may befall him in the persons of his descendants, and that
some of them may turn out good and meet with the good fortune they
deserve, and others the reverse. It is evident too that the degree in
which the descendants are related to their ancestors may vary to any
extent. And it would be a strange thing if the dead man were to change
with these changes and become happy and miserable by turns. But it
would also be strange to suppose that the dead are not affected at
all, even for a limited time, by the fortunes of their posterity.

But let us return to our former question; for its solution will,
perhaps, clear up this other difficulty.

The saying of Solon may mean that we ought to look for the end and
then call a man happy, not because he now is, but because he once was
happy.

But surely it is strange that when he is happy we should refuse to say
what is true of him, because we do not like to apply the term to
living men in view \page{25} of the changes to which they are liable,
and because we hold happiness to be something that endures and is
little liable to change, while the fortunes of one and the same man
often undergo many revolutions: for, it is argued, it is plain that,
if we follow the changes of fortune, we shall call the same man happy
and miserable many times over, making the happy man ``a sort of
chameleon and one who rests on no sound foundation.''

We reply that it cannot be right thus to follow fortune. For it is not
in this that our weal or woe lies; but, as we said, though good
fortune is needed to complete man's life, yet it is the excellent
employment of his powers that constitutes his happiness, as the
reverse of this constitutes his misery.

But the discussion of this difficulty leads to a further confirmation
of our account. For nothing human is so constant as the excellent
exercise of our faculties. The sciences themselves seem to be less
abiding. And the highest of these exercises\footnote{The ``highest
exercise of our faculties'' is, of course, philosophic contemplation,
as above, I. 7, 15; \textit{cf.} X. 7, 1.} are the most abiding,
because the happy are occupied with them most of all and most
continuously (for this seems to be the reason why we do not forget how
to do them\footnote{We may forget scientific truths that we have
known more easily than we lose the habit of scientific thinking or of
virtuous action; \textit{cf.} X. 7, 2; VI. 5, 8.}).

The happy man, then, as we define him, will have this required
property of permanence, and all through life will preserve his
character; for he will be occupied continually, or with the least
possible interruption, in \page{26} excellent deeds and excellent
speculations; and, whatever his fortune be, he will take it in the
noblest fashion, and bear himself always and in all things suitably,
since he is truly good and ``foursquare without a flaw.''

But the dispensations of fortune are many, some great, some small. The
small ones, whether good or evil, plainly are of no weight in the
scale; but the great ones, when numerous, will make life happier if
they be good; for they help to give a grace to life themselves, and
their use is noble and good; but, if they be evil, will enfeeble and
spoil happiness; for they bring pain, and often impede the exercise of
our faculties.

But nevertheless true worth shines out even here, in the calm
en\-dur\-ance of many great misfortunes, not through insensibility,
but through nobility and greatness of soul. And if it is what a man
does that determines the character of his life, as we said, then no
happy man will become miserable; for he will never do what is hateful
and base. For we hold that the man who is truly good and wise will
bear with dignity whatever fortune sends, and will always make the
best of his circumstances, as a good general will turn the forces at
his command to the best account, and a good shoemaker will make the
best shoe that can be made out of a given piece of leather, and so on
with all other crafts.

If this be so, the happy man will never become miserable, though he
will not be truly happy if he meets with the fate of Priam.

But yet he is not unstable and lightly changed: he \page{27} will not
be moved from his happiness easily, nor by any ordinary misfortunes,
but only by many heavy ones; and after such, he will not recover his
happiness again in a short time, but if at all, only in a considerable
period, which has a certain completeness, and in which he attains to
great and noble things.

We shall meet all objections, then, if we say that a happy man is
``one who exercises his faculties in accordance with perfect
excellence, being duly furnished with external goods, not for any
chance time, but for a full term of years:'' to which perhaps we
should add, ``and who shall continue to live so, and shall die as he
lived,'' since the future is veiled to us, but happiness we take to be
the end and in all ways perfectly final or complete.

If this be so, we may say that those living men are blessed or
perfectly happy who both have and shall continue to have these
characteristics, but happy as men only.

11. Passing now from this question to that of the fortunes of
descendants and of friends generally, the doctrine that they do not
affect the departed at all seems too cold and too much opposed to
popular opinion. But as the things that happen to them are many and
differ in all sorts of ways, and some come home to them more and some
less, so that to discuss them all separately would be a long, indeed
an endless task, it will perhaps be enough to speak of them in general
terms and in outline merely.

Now, as of the misfortunes that happen to a man's self, some have a
certain weight and influence on his life, while others are of less
moment, so is it also with \page{28} what happens to any of his
friends. And, again, it always makes much more difference whether
those who are affected by an occurrence are alive or dead than it does
whether a terrible crime in a tragedy be enacted on the stage or
merely supposed to have already taken place. We must therefore take
these differences into account, and still more, perhaps, the fact that
it is a doubtful question whether the dead are at all accessible to
good and ill. For it appears that even if anything that happens,
whether good or evil, does come home to them, yet it is something
unsubstantial and slight to them if not in itself; or if not that, yet
at any rate its influence is not of that magnitude or nature that it
can make happy those who are not, or take away their happiness from
those that are.

It seems then---to conclude---that the prosperity, and likewise the
adversity, of friends does affect the dead, but not in such a way or
to such an extent as to make the happy unhappy, or to do anything of
the kind.

12. These points being settled, we may now inquire whether happiness
is to be ranked among the goods that we praise, or rather among those
that we revere; for it is plainly not a mere potentiality, but an
actual good.

What we praise seems always to be praised as being of a certain
quality and having a certain relation to something. For instance, we
praise the just and the courageous man, and generally the good man,
and excellence or virtue, because of what they do or produce; and we
praise also the strong or the swift-\page{29}footed man, and so on,
because he has a certain gift or faculty in relation to some good and
admirable thing.

This is evident if we consider the praises bestowed on the Gods. The
Gods are thereby made ridiculous by being made relative to man; and
this happens because, as we said, a thing can only be praised in
relation to something else.

If, then, praise be proper to such things as we mentioned, it is
evident that to the best things is due, not praise, but something
greater and better, as our usage shows; for the Gods we call blessed
and happy, and ``blessed'' is the term we apply to the most godlike
men.

And so with good things: no one praises happiness as he praises
justice, but calls it blessed, as something better and more divine.

On these grounds Eudoxus is thought to have based a strong argument
for the claims of pleasure to the first prize: for he maintained that
the fact that it is not praised, though it is a good thing, shows that
it is higher than the goods we praise, as God and the good are higher;
for these are the standards by reference to which we judge all other
things,---giving praise to excellence or virtue, since it makes us apt
to do what is noble, and passing encomiums on the results of virtue,
whether these be bodily or psychical.

But to refine on these points belongs more properly to those who have
made a study of the subject of encomiums; for us it is plain from what
has been said that happiness is one of the goods which we revere and
count as final.

\page{30}And this further seems to follow from the fact that it is a
starting-point or principle: for everything we do is always done for
its sake; but the principle and cause of all good we hold to be
something divine and worthy of reverence.

13. Since happiness is an exercise of the vital faculties in
accordance with perfect virtue or excellence, we will now inquire
about virtue or excellence; for this will probably help us in our
inquiry about happiness.

And indeed the true statesman seems to be especially concerned with
virtue, for he wishes to make the citizens good and obedient to the
laws. Of this we have an example in the Cretan and the Laced\ae monian
lawgivers, and any others who have resembled them. But if the inquiry
belongs to Politics or the science of the state, it is plain that it
will be in accordance with our original purpose to pursue it.

The virtue or excellence that we are to consider is, of course, the
excellence of man; for it is the good of man and the happiness of man
that we started to seek. And by the excellence of man I mean
excellence not of body, but of soul; for happiness we take to be an
activity of the soul.

If this be so, then it is evident that the statesman must have some
knowledge of the soul, just as the man who is to heal the eye or the
whole body must have some knowledge of them, and that the more in
proportion as the science of the state is higher and better than
medicine. But all educated physicians take much pains to know about
the body.

As statesmen [or students of Politics], then, we \page{31} must
inquire into the nature of the soul, but in so doing we must keep our
special purpose in view and go only so far as that requires; for to go
into minuter detail would be too laborious for the present
undertaking.

Now, there are certain doctrines about the soul which are stated
elsewhere with sufficient precision, and these we will adopt.

Two parts of the soul are distinguished, an irrational and a rational
part.

Whether these are separated as are the parts of the body or any
divisible thing, or whether they are only distinguishable in thought
but in fact inseparable, like concave and convex in the circumference
of a circle, makes no difference for our present purpose.

Of the irrational part, again, one division seems to be common to all
things that live, and to be possessed by plants---I mean that which
causes nutrition and growth; for we must assume that all things that
take nourishment have a faculty of this kind, even when they are
embryos, and have the same faculty when they are full grown; at least,
this is more reasonable than to suppose that they then have a
different one.

The excellence of this faculty, then, is plainly one that man shares
with other beings, and not specifically human.

And this is confirmed by the fact that in sleep this part of the soul,
or this faculty, is thought to be most active, while the good and the
bad man are undistinguishable when they are asleep (whence the saying
that for half their lives there is no difference between the happy and
the miserable; which \page{32} indeed is what we should expect; for
sleep is the cessation of the soul from those functions in respect of
which it is called good or bad), except that they are to some slight
extent roused by what goes on in their bodies, with the result that
the dreams of the good man are better than those of ordinary people.

However, we need not pursue this further, and may dismiss the
nutritive principle, since it has no place in the excellence of man.

But there seems to be another vital principle that is irrational, and
yet in some way partakes of reason. In the case of the continent and
of the incontinent man alike we praise the reason or the rational
part, for it exhorts them rightly and urges them to do what is best;
but there is plainly present in them another principle besides the
rational one, which fights and struggles against the reason. For just
as a paralyzed limb, when you will to move it to the right, moves on
the contrary to the left, so is it with the soul; the incontinent
man's impulses run counter to his reason. Only whereas we see the
refractory member in the case of the body, we do not see it in the
case of the soul. But we must nevertheless, I think, hold that in the
soul too there is something beside the reason, which opposes and runs
counter to it (though in what sense it is distinct from the reason
does not matter here).

It seems, however, to partake of reason also, as we said: at least, in
the continent man it submits to the reason; while in the temperate and
courageous man we may say it is still more obedient; for in him it is
altogether in harmony with the reason.

The irrational part, then, it appears, is twofold. \page{33} There is
the vegetative faculty, which has no share of reason; and the faculty
of appetite or of desire in general, which in a manner partakes of
reason or is rational as listening to reason and submitting to its
sway,---rational in the sense in which we speak of rational obedience
to father or friends, not in the sense in which we speak of rational
apprehension of mathematical truths. But all advice and all rebuke
and exhortation testify that the irrational part is in some way
amenable to reason.

If then we like to say that this part, too, has a share of reason, the
rational part also will have two divisions: one rational in the strict
sense as possessing reason in itself, the other rational as listening
to reason as a man listens to his father.

Now, on this division of the faculties is based the division of
excellence; for we speak of intellectual excellences and of moral
excellences; wisdom and understanding and prudence we call
intellectual, liberality and temperance we call moral virtues or
excellences. When we are speaking of a man's moral character we do not
say that he is wise or intelligent, but that he is gentle or
temperate. But we praise the wise man, too, for his habit of mind or
trained faculty; and a habit or trained faculty that is praiseworthy
is what we call an excellence or virtue.

