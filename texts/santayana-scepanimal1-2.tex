\author{George Santayana}
\authdate{1863--1952}
\textdate{1923}
\addon{Chapters 1 and 2}
\chapter[Scepticism and Animal Faith, chaps. 1--2]{Scepticism and Animal Faith}
\source{santayana1923}

\page{1}\section*{Chapter I. There Is No First Principle of Criticism}

A philosopher is compelled to follow the maxim of epic poets and to
plunge \textit{in medias res}. The origin of things, if things have an
origin, cannot be revealed to me, if revealed at all, until I have
travelled very far from it, and many revolutions of the sun must
precede my first dawn. The light as it appears hides the candle.
Perhaps there is no source of things at all, no simpler form from
which they are evolved, but only an endless succession of different
complexities. In that case nothing would be lost by joining the
procession wherever one happens to come upon it, and following it as
long as one's legs hold out. Every one might still observe a typical
bit of it; he would not have understood anything better if he had seen
more things; he would only have had more to explain. The very notion
of understanding or explaining anything would then be absurd; yet this
notion is drawn from a current presumption or experience to the effect
that in some directions at least things do grow out of simpler things:
bread can be baked, and dough and fire and an oven are conjoined in
baking it. Such an episode is enough to establish the notion of
origins and explanations, without at all implying that the dough and
the hot oven are themselves primary facts. A philosopher may
accordingly perfectly well undertake to find \textit{episodes of
evolution} in the world: parents \page{2} with children, storms with
shipwrecks, passions with tragedies. If he begins in the middle he
will still begin at the beginning of something, and perhaps as much at
the beginning of things as he could possibly begin.

On the other hand, this whole supposition may be wrong. Things may
have had some simpler origin, or may contain simpler elements. In that
case it will be incumbent on the philosopher to prove this fact; that
is, to find in the complex present objects evidence of their
composition out of simples. But in this proof also he would be
beginning in the middle; and he would reach origins or elements only
at the end of his analysis.

The case is similar with respect to first principles of discourse.
They can never be discovered, if discovered at all, until they have
been long taken for granted, and employed in the very investigation
which reveals them. The more cogent a logic is, the fewer and simpler
its first principles will turn out to have been; but in discovering
them, and deducing the rest from them, they must first be employed
unawares, if they are the principles lending cogency to actual
discourse; so that the mind must trust current presumptions no less in
discovering that they are logical---that is, justified by more general
unquestioned presumptions---than in discovering that they are
arbitrary and merely instinctive.

It is true that, quite apart from living discourse, a set of axioms
and postulates, as simple as we like, may be posited in the air, and
deductions drawn from them \textit{ad libitum}; but such pure logic is
otiose, unless we find or assume that discourse or nature actually
follows it; and it is not by deduction from first principles,
arbitrarily chosen, that human reasoning actually proceeds, but by
loose habits of mental evocation which such principles at best may
exhibit afterwards in an idealised form. Moreover, if we could strip
our \page{3} thought for the arena of a perfect logic, we should be
performing, perhaps, a remarkable dialectical feat; but this feat
would be a mere addition to the complexities of nature, and no
simplification. This motley world, besides its other antics, would
then contain logicians and their sports. If by chance, on turning to
the flowing facts, we found by analysis that they obeyed that ideal
logic, we should again be beginning with things as we find them in the
gross, and not with first principles.

It may be observed in passing that no logic to which empire over
nature or over human discourse has ever been ascribed has been a
cogent logic; it has been, in proportion to its exemplification in
existence, a mere description, psychological or historical, of an
actual procedure; whereas pure logic, when at last, quite recently, it
was clearly conceived, turned out instantly to have no necessary
application to anything, and to be merely a parabolic excursion into
the realm of essence.

In the tangle of human beliefs, as conventionally expressed in talk
and in literature, it is easy to distinguish a compulsory factor
called facts or things from a more optional and argumentative factor
called suggestion or interpretation; not that what we call facts are
at all indubitable, or composed of immediate data, but that in the
direction of fact we come much sooner to a stand, and feel that we are
safe from criticism. To reduce conventional beliefs to the facts they
rest on---however questionable those facts themselves may be in other
ways---is to clear our intellectual conscience of voluntary or
avoidable delusion. If what we call a fact still deceives us, we feel
we are not to blame; we should not call it a fact, did we see any way
of eluding the recognition of it. To reduce conventional belief to the
recognition of matters of fact is empirical criticism of knowledge.

\page{4}The more drastic this criticism is, and the more revolutionary
the view to which it reduces me, the clearer will be the contrast
between what I find I know and what I thought I knew. But if these
plain facts were all I had to go on, how did I reach those strange
conclusions? What principles of interpretation, what tendencies to
feign, what habits of inference were at work in me? For if nothing in
the facts justified my beliefs, something in me must have suggested
them. To disentangle and formulate these subjective principles of
interpretation is transcendental criticism of knowledge.

Transcendental criticism in the hands of Kant and his followers was a
sceptical instrument used by persons who were not sceptics. They
accordingly imported into their argument many uncritical assumptions,
such as that these tendencies to feign must be the same in everybody,
that the notions of nature, history, or mind which they led people to
adopt were the right or standard notions on these subjects, and that
it was glorious, rather than ignominious or sophistical, to build on
these principles an encyclop\ae dia of false sciences and to call it
knowledge. A true sceptic will begin by throwing over all those
academic conventions as so much confessed fiction; and he will ask
rather if, when all that these arbitrary tendencies to feign import
into experience has been removed, any factual element remains at all.
The only critical function of transcendentalism is to drive empiricism
home, and challenge it to produce any knowledge of fact whatsoever.
And empirical criticism will not be able to do so. Just as inattention
leads ordinary people to assume as part of the given facts all that
their unconscious transcendental logic has added to them, so
inattention, at a deeper level, leads the empiricist to assume an
existence in his radical facts which does not belong to them. In
\page{5} standing helpless and resigned before them he is, for all his
assurance, obeying his illusion rather than their evidence. Thus
transcendental criticism, used by a thorough sceptic, may compel
empirical criticism to show its hand. It had mistaken its cards, and
was bluffing without knowing it.

\page{6}\section*{Chapter II. Dogma and Doubt}

Custom does not breed understanding, but takes its place, teaching
people to make their way contentedly through the world without knowing
what the world is, nor what they think of it, nor what they are. When
their attention is attracted to some remarkable thing, say to the
rainbow, this thing is not analysed nor examined from various points
of view, but all the casual resources of the fancy are called forth in
conceiving it, and this total reaction of the mind precipitates a
dogma; the rainbow is taken for an omen of fair weather, or for a
trace left in the sky by the passage of some beautiful and elusive
goddess. Such a dogma, far from being an interpenetration or
identification of thought with the truth of the object, is a fresh and
additional object in itself. The original passive perception remains
unchanged; the thing remains unfathomed; and as its diffuse influence
has by chance bred one dogma to-day, it may breed a different dogma
to-morrow. We have therefore, as we progress in our acquaintance with
the world, an always greater confusion. Besides the original fantastic
inadequacy of our perceptions, we have now rival clarifications of
them, and a new uncertainty as to whether these dogmas are relevant to
the original object, or are themselves really clear, or if so, which
of them is true.

\page{7}A prosperous dogmatism is indeed not impossible. We may have
such determinate minds that the suggestions of experience always issue
there in the same dogmas; and these orthodox dogmas, perpetually
revived by the stimulus of things, may become our dominant or even our
sole apprehension of them. We shall really have moved to another level
of mental discourse; we shall be living on ideas. In the gardens of
Seville I once heard, coming through the tangle of palms and orange
trees, the treble voice of a pupil in the theological seminary, crying
to his playmate: ``You booby! of course angels have a more perfect
nature than men.'' With his black and red cassock that child had put
on dialectic; he was playing the game of dogma and dreaming in words,
and was insensible to the scent of violets that filled the air. How
long would that last? Hardly, I suspect, until the next spring; and
the troubled awakening which puberty would presently bring to that
little dogmatist, sooner or later overtakes all elder dogmatists in
the press of the world. The more perfect the dogmatism, the more
insecure. A great high topsail that can never be reefed nor furled is
the first carried away by the gale.

To me the opinions of mankind, taken without any contrary prejudice
(since I have no rival opinions to propose) but simply contrasted with
the course of nature, seem surprising fictions; and the marvel is how
they can be maintained. What strange religions, what ferocious
moralities, what slavish fashions, what sham interests! I can explain
it all only by saying to myself that intelligence is naturally
forthright; it forges ahead; it piles fiction on fiction; and the fact
that the dogmatic structure, for the time being, stands and grows,
passes for a proof of its rightness. Right indeed it is in one sense,
as vegetation is right; it is vital; it has plasticity and warmth, and
a certain \page{8} indirect correspondence with its soil and climate.
Many obviously fabulous dogmas, like those of religion, might for ever
dominate the most active minds, except for one circumstance. In the
jungle one tree strangles another, and luxuriance itself is murderous.
So is luxuriance in the human mind. What kills spontaneous fictions,
what recalls the impassioned fancy from its improvisation, is the
angry voice of some contrary fancy. Nature, silently making fools of
us all our lives, never would bring us to our senses; but the maddest
assertions of the mind may do so, when they challenge one another.
Criticism arises out of the conflict of dogmas.

May I escape this predicament and criticise without a dogmatic
criterion? Hardly; for though the criticism may be expressed
hypothetically, as for instance in saying that if any child knew his
own father he would be a wise child, yet the point on which doubt is
thrown is a point of fact, and that there are fathers and children is
assumed dogmatically. If not, however obscure the essential relation
between fathers and children might be ideally, no one could be wise or
foolish in assigning it in any particular instance, since no such
terms would exist in nature at all. Scepticism is a suspicion of error
about facts, and to suspect error about facts is to share the
enterprise of knowledge, in which facts are presupposed and error is
possible. The sceptic thinks himself shrewd, and often is so; his
intellect, like the intellect he criticises, may have some inkling of
the true hang and connection of things; he may have pierced to a truth
of nature behind current illusions. Since his criticism may thus be
true and his doubt well grounded, they are certainly assertions; and
if he is sincerely a sceptic, they are assertions which he is ready to
maintain stoutly. Scepticism is accordingly a form of belief. Dogma
cannot be abandoned; it can only \page{9} be revised in view of some
more elementary dogma which it has not yet occurred to the sceptic to
doubt; and he may be right in every point of his criticism, except in
fancying that his criticism is radical and that he is altogether a
sceptic.

This vital compulsion to posit and to believe something, even in the
act of doubting, would nevertheless be ignominious, if the beliefs
which life and intelligence forced upon me were always false. I should
then be obliged to honour the sceptic for his heroic though hopeless
effort to eschew belief, and I should despise the dogmatist for his
willing subservience to illusion. The sequel will show, I trust, that
this is not the case; that intelligence is by nature veridical, and
that its ambition to reach the truth is sane and capable of
satisfaction, even if each of its efforts actually fails. To convince
me of this fact, however, I must first justify my faith in many
subsidiary beliefs concerning animal economy and the human mind and
the world they flourish in.

That scepticism should intervene in philosophy at all is an accident
of human history, due to much unhappy experience of perplexity and
error. If all had gone well, assertions would be made spontaneously in
dogmatic innocence, and the very notion of a \textit{right} to make
them would seem as gratuitous as in fact it is; because all the realms
of being lie open to a spirit plastic enough to conceive them, and
those that have ears to hear, may hear. Nevertheless, in the confused
state of human speculation this embarrassment obtrudes itself
automatically, and a philosopher to-day would be ridiculous and
negligible who had not strained his dogmas through the utmost rigours
of scepticism, and who did not approach every opinion, whatever his
own ultimate faith, with the courtesy and smile of the sceptic.

The brute necessity of believing something so long \page{10} as life
lasts does not justify any belief in particular; nor does it assure me
that not to live would not, for this very reason, be far safer and
saner. To be dead and have no opinions would certainly not be to
discover the truth; but if all opinions are necessarily false, it
would at least be not to sin against intellectual honour. Let me then
push scepticism as far as I logically can, and endeavour to clear my
mind of illusion, even at the price of intellectual suicide.

