
\lchapsep{8ex}
\author{Karl Marx}
\authdate{1818--1883}
\textdate{1859}
\addon{Preface, excerpt}
\chapter[A Contribution to the Critique of Political Economy,
excerpt]{A Contribution to the Critique of Political Economy}
\source{marx1904}
\restorechapsep

\page{11}In the social production which men carry on they enter into
definite relations that are indispensable and independent of their
will; these relations of production correspond to a definite stage of
development of their material powers of production. The sum total of
these relations of production constitutes the economic structure of
society---the real foundation, on which rise legal and political
superstructures and to which correspond definite forms of social
consciousness. The mode of production in material life determines the
general character of the social, political and spiritual processes of
life. It is not the consciousness of men that determines their
existence, but, on the contrary, their social existence determines
\page{12} their consciousness. At a certain stage of their
development, the material forces of production in society come in
conflict with the existing relations of production, or---what is but a
legal expression for the same thing---with the property relations
within which they had been at work before. From forms of development
of the forces of production these relations turn into their fetters.
Then comes the period of social revolution. With the change of the
economic foundation the entire immense superstructure is more or less
rapidly transformed. In considering such transformations the
distinction should always be made between the material transformation
of the economic conditions of production which can be determined with
the precision of natural science, and the legal, political, religious,
aesthetic or philosophic---in short ideological forms in which men
become conscious of this conflict and fight it out. Just as our
opinion of an individual is not based on what he thinks of himself, so
can we not judge of such a period of transformation by its own
consciousness; on the contrary, this consciousness must rather be
explained from the contradictions of material life, from the existing
conflict between the social forces of production and the relations of
production. No social order ever disappears before all the productive
forces, for which there is room in it, have been developed; and new
higher relations of production never appear before the material
conditions of their existence have matured in the womb of the old
society. Therefore, mankind always takes up only such problems as it
can solve; since, looking at the matter more closely, we will always
\page{13} find that the problem itself arises only when the material
conditions necessary for its solution already exist or are at least in
the process of formation. In broad outlines we can designate the
Asiatic, the ancient, the feudal, and the modern bourgeois methods of
production as so many epochs in the progress of the economic formation
of society. The bourgeois relations of production are the last
antagonistic form of the social process of production---antagonistic
not in the sense of individual antagonism, but of one arising from
conditions surrounding the life of individuals in society; at the same
time the productive forces developing in the womb of bourgeois society
create the material conditions for the solution of that antagonism.
This social formation constitutes, therefore, the closing chapter of
the prehistoric stage of human society.

