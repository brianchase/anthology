
\author{Jane Addams}
\authdate{1860--1935}
\textdate{1902}
\addon{Chapter 1}
\chapter[Democracy and Social Ethics, chap. 1]{Democracy and Social Ethics}
\source{addams1902.1}

%\page{1}\section*{Introduction}

\page{1}It is well to remind ourselves, from time to time, that
``Ethics'' is but another word for ``righteousness,'' that for which
many men and women of every generation have hungered and thirsted, and
without which life becomes meaningless.

Certain forms of personal righteousness have become to a majority of
the community almost automatic. It is as easy for most of us to keep
from stealing our dinners as it is to digest them, and there is quite
as much voluntary morality involved in one process as in the other. To
steal would be for us to fall sadly below the standard of habit and
expectation which makes virtue easy. In the same way we have been
carefully reared to a sense \page{2} of family obligation, to be
kindly and considerate to the members of our own households, and to
feel responsible for their well-being. As the rules of conduct have
become established in regard to our self-development and our families,
so they have been in regard to limited circles of friends. If the
fulfilment of these claims were all that a righteous life required,
the hunger and thirst would be stilled for many good men and women,
and the clew of right living would lie easily in their hands.

But we all know that each generation has its own test, the
contemporaneous and current standard by which alone it can adequately
judge of its own moral achievements, and that it may not legitimately
use a previous and less vigorous test. The advanced test must indeed
include that which has already been attained; but if it includes no
more, we shall fail to go forward, thinking complacently that we have
``arrived'' when in reality we have not yet started.

To attain individual morality in an age \page{3} demanding social
morality, to pride one's self on the results of personal effort when
the time demands social adjustment, is utterly to fail to apprehend
the situation.

It is perhaps significant that a German critic has of late reminded us
that the one test which the most authoritative and dramatic portrayal
of the Day of Judgment offers, is the social test. The stern questions
are not in regard to personal and family relations, but did ye visit
the poor, the criminal, the sick, and did ye feed the hungry?

All about us are men and women who have become unhappy in regard to
their attitude toward the social order itself; toward the dreary round
of uninteresting work, the pleasures narrowed down to those of
appetite, the declining consciousness of brain power, and the lack of
mental food which characterizes the lot of the large proportion of
their fellow-citizens. These men and women have caught a moral
challenge raised by the exigencies of contemporaneous life; some are
bewildered, others who are denied \page{4} the relief which sturdy
action brings are even seeking an escape, but all are increasingly
anxious concerning their actual relations to the basic organization of
society.

The test which they would apply to their conduct is a social test.
They fail to be content with the fulfilment of their family and
personal obligations, and find themselves striving to respond to a new
demand involving a social obligation; they have become conscious of
another requirement, and the contribution they would make is toward a
code of social ethics. The conception of life which they hold has not
yet expressed itself in social changes or legal enactment, but rather
in a mental attitude of maladjustment, and in a sense of divergence
between their consciences and their conduct. They desire both a
clearer definition of the code of morality adapted to present day
demands and a part in its fulfilment, both a creed and a practice of
social morality. In the perplexity of this intricate situation at
least one thing is becoming clear: if the latter \page{5} day moral
ideal is in reality that of a social morality, it is inevitable that
those who desire it must be brought in contact with the moral
experiences of the many in order to procure an adequate social motive.

These men and women have realized this and have disclosed the fact in
their eagerness for a wider acquaintance with and participation in the
life about them. They believe that experience gives the easy and
trustworthy impulse toward right action in the broad as well as in the
narrow relations. We may indeed imagine many of them saying: ``Cast
our experiences in a larger mould if our lives are to be animated by
the larger social aims. We have met the obligations of our family
life, not because we had made resolutions to that end, but
spontaneously, because of a common fund of memories and affections,
from which the obligation naturally develops, and we see no other way
in which to prepare ourselves for the larger social duties.'' Such a
demand is reasonable, for by our daily experience we have \page{6}
discovered that we cannot mechanically hold up a moral standard, then
jump at it in rare moments of exhilaration when we have the strength
for it, but that even as the ideal itself must be a rational
development of life, so the strength to attain it must be secured from
interest in life itself. We slowly learn that life consists of
processes as well as results, and that failure may come quite as
easily from ignoring the adequacy of one's method as from selfish or
ignoble aims. We are thus brought to a conception of Democracy not
merely as a sentiment which desires the well-being of all men, nor yet
as a creed which believes in the essential dignity and equality of all
men, but as that which affords a rule of living as well as a test of
faith.

We are learning that a standard of social ethics is not attained by
travelling a sequestered byway, but by mixing on the thronged and
common road where all must turn out for one another, and at least see
the size of one another's burdens. To follow the path \page{7} of
social morality results perforce in the temper if not the practice of
the democratic spirit, for it implies that diversified human
experience and resultant sympathy which are the foundation and
guarantee of Democracy.

There are many indications that this conception of Democracy is
growing among us. We have come to have an enormous interest in human
life as such, accompanied by confidence in its essential soundness. We
do not believe that genuine experience can lead us astray any more
than scientific data can.

We realize, too, that social perspective and sanity of judgment come
only from contact with social experience; that such contact is the
surest corrective of opinions concerning the social order, and
concerning efforts, however humble, for its improvement. Indeed, it is
a consciousness of the illuminating and dynamic value of this wider
and more thorough human experience which explains in no small degree
that new curiosity regard-\page{8}ing human life which has more of a
moral basis than an intellectual one.

The newspapers, in a frank reflection of popular demand, exhibit an
omniverous curiosity equally insistent upon the trivial and the
important. They are perhaps the most obvious manifestations of that
desire to know, that ``What is this?'' and ``Why do you do that?'' of
the child. The first dawn of the social consciousness takes this form,
as the dawning intelligence of the child takes the form of constant
question and insatiate curiosity.

Literature, too, portrays an equally absorbing though better adjusted
desire to know all kinds of life. The popular books are the novels,
dealing with life under all possible conditions, and they are widely
read not only because they are entertaining, but also because they in
a measure satisfy an unformulated belief that to see farther, to know
all sorts of men, in an indefinite way, is a preparation for better
social adjustment---for the remedying of social ills.

\page{9}Doubtless one under the conviction of sin in regard to social
ills finds a vague consolation in reading about the lives of the poor,
and derives a sense of complicity in doing good. He likes to feel that
he knows about social wrongs even if he does not remedy them, and in a
very genuine sense there is a foundation for this belief.

Partly through this wide reading of human life, we find in ourselves a
new affinity for all men, which probably never existed in the world
before. Evil itself does not shock us as it once did, and we count
only that man merciful in whom we recognize an understanding of the
criminal. We have learned as common knowledge that much of the
insensibility and hardness of the world is due to the lack of
imagination which prevents a realization of the experiences of other
people. Already there is a conviction that we are under a moral
obligation in choosing our experiences, since the result of those
experiences must ulti-\page{10}mately determine our understanding of
life. We know instinctively that if we grow contemptuous of our
fellows, and consciously limit our intercourse to certain kinds of
people whom we have previously decided to respect, we not only
tremendously circumscribe our range of life, but limit the scope of
our ethics.

We can recall among the selfish people of our acquaintance at least
one common characteristic,—the conviction that they are different from
other men and women, that they need peculiar consideration because
they are more sensitive or more refined. Such people ``refuse to be
bound by any relation save the personally luxurious ones of love and
admiration, or the identity of political opinion, or religious
creed.'' We have learned to recognize them as selfish, although we
blame them not for the will which chooses to be selfish, but for a
narrowness of interest which deliberately selects its experience
within a limited sphere, and we say that they illustrate the danger of
\page{11} concentrating the mind on narrow and unprogressive issues.

We know, at last, that we can only discover truth by a rational and
democratic interest in life, and to give truth complete social
expression is the endeavor upon which we are entering. Thus the
identification with the common lot which is the essential idea of
Democracy becomes the source and expression of social ethics. It is as
though we thirsted to drink at the great wells of human experience,
because we knew that a daintier or less potent draught would not carry
us to the end of the journey, going forward as we must in the heat and
jostle of the crowd.

The six following chapters are studies of various types and groups who
are being impelled by the newer conception of Democracy to an
acceptance of social obligations involving in each instance a new line
of conduct. No attempt is made to reach a conclusion, nor to offer
advice beyond the assumption that the cure for the ills of \page{12}
Democracy is more Democracy, but the quite unlooked-for result of the
studies would seem to indicate that while the strain and perplexity of
the situation is felt most keenly by the educated and self-conscious
members of the community, the tentative and actual attempts at
adjustment are largely coming through those who are simpler and less
analytical.

