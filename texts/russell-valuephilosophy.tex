
\author{Bertrand Russell}
\authdate{1872--1970}
\textdate{1912}
\chapter{The Value of Philosophy}
\source{russell1912.15}

\page{237}\noindent Having now come to the end of our brief and very
incomplete review of the problems of philosophy, it will be well to
consider, in conclusion, what is the value of philosophy and why it
ought to be studied. It is the more necessary to consider this
question, in view of the fact that many men, under the influence of
science or of practical affairs, are inclined to doubt whether
philosophy is anything better than innocent but useless trifling,
hair-splitting distinctions, and controversies on matters concerning
which knowledge is impossible.

This view of philosophy appears to result, partly from a wrong
conception of the ends of life, partly from a wrong conception of the
kind of goods which philosophy strives to \page{238} achieve. Physical
science, through the medium of inventions, is useful to innumerable
people who are wholly ignorant of it; thus the study of physical
science is to be recommended, not only, or primarily, because of the
effect on the student, but rather because of the effect on mankind in
general. This utility does not belong to philosophy. If the study of
philosophy has any value at all for others than students of
philosophy, it must be only indirectly, through its effects upon the
lives of those who study it. It is in these effects, therefore, if
anywhere, that the value of philosophy must be primarily sought.

But further, if we are not to fail in our endeavour to determine the
value of philosophy, we must first free our minds from the prejudices
of what are wrongly called ``practical'' men. The ``practical'' man,
as this word is often used, is one who recognises only material needs,
who realises that men must have food for the body, but is oblivious of
the necessity of providing food for the mind. If all men were well
off, if poverty and disease had been reduced to their lowest possible
point, there \page{239} would still remain much to be done to produce
a valuable society; and even in the existing world the goods of the
mind are at least as important as the goods of the body. It is
exclusively among the goods of the mind that the value of philosophy
is to be found; and only those who are not indifferent to these goods
can be persuaded that the study of philosophy is not a waste of time.

% Without '\linebreak', an overfull hbox warning:

Philosophy, like all other studies, aims primarily at knowledge. The
\linebreak[4] knowledge it aims at is the kind of knowledge which
gives unity and system to the body of the sciences, and the kind which
results from a critical examination of the grounds of our convictions,
prejudices, and beliefs. But it cannot be maintained that philosophy
has had any very great measure of success in its attempts to provide
definite answers to its questions. If you ask a mathematician, a
mineralogist, a historian, or any other man of learning, what definite
body of truths has been ascertained by his science, his answer will
last as long as you are willing to listen. But if you put the same
question to a philosopher, he will, if he is \page{240} candid, have
to confess that his study has not achieved positive results such as
have been achieved by other sciences. It is true that this is partly
accounted for by the fact that, as soon as definite knowledge
concerning any subject becomes possible, this subject ceases to be
called philosophy, and becomes a separate science. The whole study of
the heavens, which now belongs to astronomy, was once included in
philosophy; Newton's great work was called ``the mathematical
principles of natural philosophy.'' Similarly, the study of the human
mind, which was, until very lately, a part of philosophy, has now been
separated from philosophy and has become the science of psychology.
Thus, to a great extent, the uncertainty of philosophy is more
apparent than real: those questions which are already capable of
definite answers are placed in the sciences, while those only to
which, at present, no definite answer can be given, remain to form the
residue which is called philosophy.

This is, however, only a part of the truth concerning the uncertainty
of philosophy. There are many ques\-tions---and among them \page{241}
those that are of the profoundest interest to our spiritual
life---which, so far as we can see, must remain insoluble to the human
intellect unless its powers become of quite a different order from
what they are now. Has the universe any unity of plan or purpose, or
is it a fortuitous concourse of atoms? Is consciousness a permanent
part of the universe, giving hope of indefinite growth in wisdom, or
is it a transitory accident on a small planet on which life must
ultimately become impossible? Are good and evil of importance to the
universe or only to man? Such questions are asked by philosophy, and
variously answered by various philosophers. But it would seem that,
whether answers be otherwise discoverable or not, the answers
suggested by philosophy are none of them demonstrably true. Yet,
however slight may be the hope of discovering an answer, it is part of
the business of philosophy to continue the consideration of such
questions, to make us aware of their importance, to examine all the
approaches to them, and to keep alive that speculative interest in the
universe which is \page{242} apt to be killed by confining ourselves
to definitely ascertainable knowledge.

Many philosophers, it is true, have held that philosophy could
establish the truth of certain answers to such fundamental questions.
They have supposed that what is of most importance in religious
beliefs could be proved by strict demonstration to be true. In order
to judge of such attempts, it is necessary to take a survey of human
knowledge, and to form an opinion as to its methods and its
limitations. On such a subject it would be unwise to pronounce
dogmatically; but if the investigations of our previous chapters have
not led us astray, we shall be compelled to renounce the hope of
finding philosophical proofs of religious beliefs. We cannot,
therefore, include as part of the value of philosophy any definite set
of answers to such questions. Hence, once more, the value of
philosophy must not depend upon any supposed body of definitely
ascertainable knowledge to be acquired by those who study it.

The value of philosophy is, in fact, to be sought largely in its very
uncertainty. The \page{243} man who has no tincture of philosophy goes
through life imprisoned in the prejudices derived from common sense,
from the habitual beliefs of his age or his nation, and from
convictions which have grown up in his mind without the co-operation
or consent of his deliberate reason. To such a man the world tends to
become definite, finite, obvious; common objects rouse no questions,
and unfamiliar possibilities are contemptuously rejected. As soon as
we begin to philosophise, on the contrary, we find, as we saw in our
opening chapters, that even the most everyday things lead to problems
to which only very incomplete answers can be given. Philosophy, though
unable to tell us with certainty what is the true answer to the doubts
which it raises, is able to suggest many possibilities which enlarge
our thoughts and free them from the tyranny of custom. Thus, while
diminishing our feeling of certainty as to what things are, it greatly
increases our knowledge as to what they may be; it removes the
somewhat arrogant dogmatism of those who have never travelled into the
region of liberating \page{244} doubt, and it keeps alive our sense of
wonder by showing familiar things in an unfamiliar aspect.

Apart from its utility in showing unsuspected possibilities,
philosophy has a val\-ue---perhaps its chief val\-ue---through the
greatness of the objects which it contemplates, and the freedom from
narrow and personal aims resulting from this contemplation. The life
of the instinctive man is shut up within the circle of his private
interests: family and friends may be included, but the outer world is
not regarded except as it may help or hinder what comes within the
circle of instinctive wishes. In such a life there is something
feverish and confined, in comparison with which the philosophic life
is calm and free. The private world of instinctive interests is a
small one, set in the midst of a great and powerful world which must,
sooner or later, lay our private world in ruins. Unless we can so
enlarge our interests as to include the whole outer world, we remain
like a garrison in a beleagured fortress, knowing that the enemy
prevents escape and that ultimate surrender \page{245} is inevitable.
In such a life there is no peace, but a constant strife between the
insistence of desire and the powerlessness of will. In one way or
another, if our life is to be great and free, we must escape this
prison and this strife.

One way of escape is by philosophic contemplation. Philosophic
contemplation does not, in its widest survey, divide the universe into
two hostile camps---friends and foes, helpful and hostile, good and
bad---it views the whole impartially. Philosophic contemplation, when
it is unalloyed, does not aim at proving that the rest of the universe
is akin to man. All acquisition of knowledge is an enlargement of the
Self, but this enlargement is best attained when it is not directly
sought. It is obtained when the desire for knowledge is alone
operative, by a study which does not wish in advance that its objects
should have this or that character, but adapts the Self to the
characters which it finds in its objects. This enlargement of Self is
not obtained when, taking the Self as it is, we try to show that the
world is so similar \page{246} to this Self that knowledge of it is
possible without any admission of what seems alien. The desire to
prove this is a form of self-assertion and, like all self-assertion,
it is an obstacle to the growth of Self which it desires, and of which
the Self knows that it is capable. Self-assertion, in philosophic
speculation as elsewhere, views the world as a means to its own ends;
thus it makes the world of less account than Self, and the Self sets
bounds to the greatness of its goods. In contemplation, on the
contrary, we start from the not-Self, and through its greatness the
boundaries of Self are enlarged; through the infinity of the universe
the mind which contemplates it achieves some share in infinity.

For this reason greatness of soul is not fostered by those
philosophies which assimilate the universe to Man. Knowledge is a form
of union of Self and not-Self; like all union, it is impaired by
dominion, and therefore by any attempt to force the universe into
conformity with what we find in ourselves. There is a widespread
philosophical tendency towards the view which tells \page{247} us that
man is the measure of all things, that truth is man-made, that space
and time and the world of universals are properties of the mind, and
that, if there be anything not created by the mind, it is unknowable
and of no account for us. This view, if our previous discussions were
correct, is untrue; but in addition to being untrue, it has the effect
of robbing philosophic contemplation of all that gives it value, since
it fetters contemplation to Self. What it calls knowledge is not a
union with the not-Self, but a set of prejudices, habits, and desires,
making an impenetrable veil between us and the world beyond. The man
who finds pleasure in such a theory of knowledge is like the man who
never leaves the domestic circle for fear his word might not be law.

The true philosophic contemplation, on the contrary, finds its
satisfaction in every enlargement of the not-Self, in everything that
magnifies the objects contemplated, and thereby the subject
contemplating. Everything, in contemplation, that is personal or
private, everything that depends upon habit, \page{248} self-interest,
or desire, distorts the object, and hence impairs the union which the
intellect seeks. By thus making a barrier between subject and object,
such personal and private things become a prison to the intellect. The
free intellect will see as God might see, without a \textit{here} and
\textit{now}, without hopes and fears, without the trammels of
customary beliefs and traditional prejudices, calmly, dispassionately,
in the sole and exclusive desire of knowl\-edge---knowledge as
impersonal, as purely contemplative, as it is possible for man to
attain. Hence also the free intellect will value more the abstract and
universal knowledge into which the accidents of private history do not
enter, than the knowledge brought by the senses, and dependent, as
such knowledge must be, upon an exclusive and personal point of view
and a body whose sense-organs distort as much as they reveal.

The mind which has become accustomed to the freedom and impartiality
of philosophic contemplation will preserve something of the same
freedom and impartiality in the world of action and emotion. It will
view its purposes \page{249} and desires as parts of the whole, with
the absence of insistence that results from seeing them as
infinitesimal fragments in a world of which all the rest is unaffected
by any one man's deeds. The impartiality which, in contemplation, is
the unalloyed desire for truth, is the very same quality of mind
which, in action, is justice, and in emotion is that universal love
which can be given to all, and not only to those who are judged useful
or admirable. Thus contemplation enlarges not only the objects of our
thoughts, but also the objects of our actions and our affections: it
makes us citizens of the universe, not only of one walled city at war
with all the rest. In this citizenship of the universe consists man's
true freedom, and his liberation from the thraldom of narrow hopes and
fears.

Thus, to sum up our discussion of the value of philosophy: Philosophy
is to be studied, not for the sake of any definite answers to its
questions, since no definite answers can, as a rule, be known to be
true, but rather for the sake of the questions themselves; because
these questions enlarge our conception of what \page{250} is possible,
enrich our intellectual imagination, and diminish the dogmatic
assurance which closes the mind against speculation; but above all
because, through the greatness of the universe which philosophy
contemplates, the mind also is rendered great, and becomes capable of
that union with the universe which constitutes its highest good.

