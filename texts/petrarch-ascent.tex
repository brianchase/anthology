
\author{Francesco Petrarca (Petrarch)}
\authdate{1304--1374}
\textdate{1336}
\chapter{The Ascent of Mont Ventoux}
\source{cassirer1948.4}

\page{36}\begin{quote} Letter to Francesco Dionigi de'Roberti of Borgo
San Sepolcro, professor of theology in Paris. Malauc\`{e}ne, April 26,
1336. (\textit{Fam}., IV, 1, in \textit{Le Familiari}, ed. V. Rossi,
I, 153--61; \textit{Opera} [Basel, 1581], pp. 624--27.) \end{quote}

\begin{center} To Dionigi da Borgo San Sepolcro, of the Order of Saint
Augustine, Professor of Theology, about his own troubles. \end{center}

Today I ascended the highest mountain in this region, which, not
without cause, they call the Windy Peak.\footnote{The name of the
mountain appears as ``Ventosus'' in Latin documents as early as the
tenth century, though originally it had nothing to do with the strong
winds blowing about that isolated peak. Its Proven\c{c}al form
``Ventour'' proves that it is related to the name of a deity worshiped
by the pre-Roman (Ligurian) population of the Rhone Basin, a god
believed to dwell on high mountains (cf. C. Jullian, \textit{Histoire
de la Gaule}, VI, 329; P. Julian, ``Glose sur l'\'{e}tymologie du mot
Ventoux,'' in \textit{Le P\'{e}l\'{e}rinage du Mt. Ventoux}
[Carpentras, 1937], pp. 337 ff.).} Nothing but the desire to see its
conspicuous height was the reason for this undertaking. For many years
I have been intending to make this expedition. You know that since my
early childhood, as fate tossed around human affairs, I have been
tossed around in these parts, and this mountain, visible far and wide
from everywhere, is always in your view. So I was at last seized by
the impulse to accomplish what I had always wanted to do. It happened
while I was reading Roman history again in Livy that I hit upon the
passage where Philip, the king of Macedon---the Philip who waged war
against the Roman people---``ascends Mount Haemus in Thessaly, since
he believed the rumor that you can see two seas from its top: the
Adriatic and the Black Sea.''\footnote{In his \textit{History of Rome}
(xl. 21. 2--22. 7) Livy tells that King Philip V of Macedonia went up
to the top of Mount Haemus, one of the highest summits of the Great
Balkans (\textit{ca.} 7,800 ft.), when he wanted to reconnoiter the
field of future operations before the Third Macedonian War, which he
was planning to fight against the Romans (181 \textsc{B.C.}. Since
Petrarca knew the exact location of this mountain from Pliny's
\textit{Natural History} (iv. 1. 3 and xi. 18. 41), it must have been
a slip of his pen that made him substitute ``Thessaly'' for
``Thrace.''} Whether he was right or wrong I cannot make out
be-\page{37}cause the mountain is far from our region, and the
disagreement among authors renders the matter uncertain. I do not
intend to consult all of them: the cosmographer Pomponius Mela does
not hesitate to report the fact as true;\footnote{Mela
\textit{Cosmographia} ii. 2. 17.} Livy supposes the rumor to be false.
I would not leave it long in doubt if that mountain were as easy to
explore as the one here. At any rate, I had better let it go, in order
to come back to the mountain I mentioned at first. It seemed to me
that a young man who holds no public office\footnote{Cf. Cicero
\textit{De imperio Cn. Pompei} 21. 61, where he praises the courage of
Pompey, who took over the command of the Roman armies in 77
\textsc{B.C.} though he was then but an ``adulescentulus privatus.''}
might be excused for doing what an old king is not blamed for.

I now began to think over whom to choose as a companion. It will sound
strange to you that hardly a single one of all my friends seemed to me
suitable in every respect, so rare a thing is absolute congeniality in
every attitude and habit even among dear friends. One was too
sluggish, the other too vivacious; one too slow, the other too quick;
this one too gloomy of temper, that one too gay. One was duller, the
other brighter than I should have liked. This man's taciturnity, that
man's flippancy; the heavy weight and obesity of the next, the
thinness and weakness of still another were reasons to deter me. The
cool lack of curiosity of one, like another's too eager interest,
dissuaded me from choosing either. All such qualities, however
difficult they are to bear, can be borne at home: loving friendship is
able to endure everything; it refuses no burden. But on a journey they
become intolerable. Thus my delicate mind, craving honest
entertainment, looked about carefully, weighing every detail with no
offense to friendship. Tacitly it rejected whatever it could foresee
would become troublesome on the projected excursion. \page{38} What do
you think I did? At last I applied for help at home and revealed my
plan to my only brother, who is younger than I and whom you know well
enough. He could hear of nothing he would have liked better and was
happy to fill the place of friend as well as brother.

We left home on the appointed day and arrived at Malauc\`{e}ne at
night. This is a place at the northern foot of the mountain.

We spent a day there and began our ascent this morning, each of us
accompanied by a single servant. From the start we encountered a good
deal of trouble, for the mountain is a steep and almost inaccessible
pile of rocky material. However, what the Poet says is appropriate:
``Ruthless striving overcomes everything.''\footnote{Vergil
\textit{Georgica} i. 145--46; Macrobius, \textit{Saturnalia} v. 6.}

The day was long, the air was mild; this and vigorous minds, strong
and supple bodies, and all the other conditions assisted us on our
way. The only obstacle was the nature of the spot. We found an aged
shepherd in the folds of the mountain who tried with many words to
dissuade us from the ascent. He said he had been up to the highest
summit in just such youthful fervor fifty years ago and had brought
home nothing but regret and pains, and his body as well as his clothes
torn by rocks and thorny underbrush. Never before and never since had
the people there heard of any man who dared a similar feat. While he
was shouting these words at us, our desire increased just because of
his warnings; for young people's minds do not give credence to
advisers. When the old man saw that he was exerting himself in vain he
went with us a little way forward through the rocks and pointed with
his finger to a steep path. He gave us much good advice and repeated
it again and again at our backs when we were already at quite a
distance. We left with him whatever of our clothes and other
belongings might encumber us, intent only on the ascent, and began to
climb with merry alacrity. However, as almost always happens, the
daring attempt was soon followed by quick fatigue.

\page{39}Not far from our start we stopped at a rock. From there we
went on again, proceeding at a slower pace, to be sure. I in
particular made my way up with considerably more modest steps. My
brother endeavored to reach the summit by the very ridge of the
mountain on a short cut; I, being so much more of a weakling, was
bending down toward the valley. When he called me back and showed me
the better way, I answered that I hoped to find an easier access on
the other side and was not afraid of a longer route on which I might
proceed more smoothly. With such an excuse I tried to palliate my
laziness, and, when the others had already reached the higher zones, I
was still wandering through the valleys, where no more comfortable
access was revealed, while the way became longer and longer and the
vain fatigue grew heavier and heavier. At last I felt utterly
disgusted, began to regret my perplexing error, and decided to attempt
the heights with a wholehearted effort. Weary and exhausted, I reached
my brother, who had been waiting for me and was refreshed by a good
long rest. For a while we went on together at the same pace. However,
hardly had we left that rock behind us when I forgot the detour I had
made just a short while before and was once more drawing down the
lower regions. Again I wandered through the valleys, looking for the
longer and easier path and stumbling only into longer difficulties.
Thus I indeed put off the disagreeable strain of climbing. But nature
is not overcome by man's devices; a corporeal thing cannot reach the
heights by descending. What shall I say? My brother laughed at me; I
was indignant; this happened to me three times and more within a few
hours. So often was I frustrated in my hopes that at last I sat down
in a valley. There I leaped in my winged thoughts from things
corporeal to what is incorporeal and addressed myself in words like
these:

``What you have so often experienced today while climbing this
mountain happens to you, you must know, and to many others who are
making their way toward the blessed life. This \page{40} is not easily
understood by us men, because the motions of the body lie open while
those of the mind are invisible and hidden. The life we call blessed
is located on a high peak. `A narrow way,'\footnote{Matt. 7:14 (Sermon
on the Mount).} they say, leads up to it. Many hilltops intervene, and
we must proceed `from virtue to virtue' with exalted steps.\footnote{A
typical metaphor familiar to ecclesiastical writers; cf. e.g., Anselm
of Canterbury, \textit{Letters} i.43 (Migne, \textit{Patrologia
Latina}, CLVIII, 1113, etc.), where it is used as a friendly wish in
salutations.} On the highest summit is set the end of all, the goal
toward which our pilgrimage is directed. Every man wants to arrive
there. However, as Naso says: `Wanting is not enough; long and you
attain it.'\footnote{Ovid, \textit{Ex Ponto} iii. 1. 35.} You
certainly do not merely want; you have a longing, unless you are
deceiving yourself in this respect as in so many others. What is it,
then, that keeps you back? Evidently nothing but the smoother way that
leads through the meanest earthly pleasures and looks easier at first
sight. However, having strayed far in error, you must either ascend to
the summit of the blessed life under the heavy burden of hard
striving, ill deferred, or lie prostrate in your slothfulness in the
valleys of your sins. If `darkness and the shadow of
death'\footnote{Ps. 106 (107): 10; Job 34:22.} find you there---I
shudder while I pronounce these ominous words---you must pass the
eternal night in incessant torments.''

You cannot imagine how much comfort this thought brought my mind and
body for what lay still ahead of me. Would that I might achieve with
my mind the journey for which I am longing day and night as I achieved
with the feet of my body my journey today after overcoming all
obstacles. And I wonder whether it ought not to be much easier to
accomplish what can be done by means of the agile and immortal mind
without any local motion ``in the twinkling of the trembling
eye''\footnote{1 Cor. 15:52; Augustine, \textit{Confessions} vii. 1.1
(cf. Shakespeare, \textit{Merchant of Venice}, Act II, scene 2, line
183).} than what is to be performed in the succession of time by the
service \page{41} of the frail body that is doomed to die and under
the heavy load of the limbs.

There is a summit, higher than all the others. The people in the woods
up there call it ``Sonny,''\footnote{Though Petrarca was familiar with
the idiom of southern France, he misinterpreted the Proven\c{c}al
world \textit{fiholo}. There is still today a spring just below the
summit of Mont Ventoux called ``Font-filiole'' and a ravine near by by
name of ``combe filiole,'' the word meaning a water conduit or a
rivulet, but the summit can have received the name only secondarily
(P. de Champeville, ``L'Itin\'{e}raire du po\`{e}te F.P.,'' in
\textit{L'Ascension du Mt. Ventoux} [Carpentras, 1937], p. 41).} I do
not know why. However, I suspect they use the word in a sense opposite
to its meaning, as is done sometimes in other cases too. For it really
looks like the father of all the surrounding mountains. On its top is
a small level stretch. There at last we rested from our fatigue.

And now, my dear father, since you have heard what sorrows arose in my
breast during my climb, listen also to what remains to be told.
Devote, I beseech you, one of your hours to reading what I did during
one of my days. At first I stood there almost benumbed, overwhelmed by
a gale such as I had never felt before and by the unusually open and
wide view. I looked around me: clouds were gathering below my feet,
and Athos and Olympus grew less incredible, since I saw on a mountain
of lesser fame what I had heard and read about them. From there I
turned my eyes in the direction of Italy, for which my mind is so
fervently yearning. The Alps were frozen stiff and covered with
snow---those mountains through which that ferocious enemy of the Roman
name once passed, blasting his way through the rocks with vinegar if
we may believe tradition.\footnote{Hannibal is said to have made his
troops burn down the trees on rocks obstructing their way and pour
vinegar on the ashes to pulverize the burned material when he crossed
the Alps in 218 \textsc{B.C.} (Livy, \textit{History of Rome} xxi. 37;
cf. Pliny, \textit{Nat. Hist.} xxiii. 57). Later authors referred to
this incident as an example of Hannibal's ingenuity in overcoming
seemingly unsurmountable obstacles (Juvenal \textit{Satire} 10, 153).}
They looked as if they were quite near me, though they are far, far
away. I was longing, I must confess, for Italian air, \page{42} which
appeared rather to my mind than my eyes. An incredibly strong desire
seized me to see my friend\footnote{Petrarch is referring to Giacomo
Colonna, bishop of Lombez, who had gone to Rome in the summer of 1333;
cf. \textit{Fam}., I, 5 (4), and I, 6 (5).} and my native land again.
At the same time I rebuked the weakness of a mind not yet grown to
manhood, manifest in both these desires, although in both cases an
excuse would not lack support from famous champions.

Then another thought took possession of my mind, leading it from the
contemplation of space to that of time, and I said to myself: ``This
day marks the completion of the tenth year since you gave up the
studies of your boyhood and left Bologna. O immortal God, O immutable
Wisdom! How many and how great were the changes you have had to
undergo in your moral habits since then.'' I will not speak of what is
still left undone, for I am not yet in port that I might think in
security of the storms I have had to endure. The time will perhaps
come when I can review all this in the order in which it happened,
using as a prologue that passage of your favorite Augustine: ``Let me
remember my past mean acts and the carnal corruption of my soul, not
that I love them, but that I may love Thee, my
God.''\footnote{\textit{Confessions} ii. 1. 1.}

Many dubious and troublesome things are still in store for me. What I
used to love, I love no longer. But I lie: I love it still, but less
passionately. Again have I lied: I love it, but more timidly, more
sadly. Now at last I have told the truth; for thus it is: I love, but
what I should love not to love, what I should wish to hate.
Nevertheless I love it, but against my will, under compulsion and in
sorrow and mourning. To my own misfortune I experience in myself now
the meaning of that most famous line: ``Hate I shall, if I can; if I
can't, I shall love though not willing.''\footnote{Ovid,
\textit{Amores} iii. 11. 35.} The third year has not yet elapsed since
that perverted and malicious will, which had totally seized me and
\page{43} reigned in the court of my heart without an opponent, began
to encounter a rebel offering resistance. A stubborn and still
undecided battle has been long raging on the field of my thoughts for
the supremacy of one of the two men within me.\footnote{Two rival
wills are struggling in Petrarch's breast, the old not releasing him
from his amorous servitude and blocking his spiritual progress, the
other urging him forward on the way to perfection (cf. Augustine
\textit{Confessions} viii. 5. 10; x. 22--23, and Petrara's Sonnet 52
(68).}

Thus I revolved in my thoughts the history of the last decade. Then I
dismissed my sorrow at the past and asked myself: ``Suppose you
succeed in protracting this rapidly fleeing life for another decade,
and come as much nearer to virtue, in proportion to the span of time,
as you have been freed from your former obstinacy during these last
two years as a result of the struggle of the new and the old
wills---would you then not be able---perhaps not with certainty but
with reasonable hope at least---to meet death in your fortieth year
with equal mind and cease to care for that remnant of life which
descends into old age?''

These and like considerations rose in my breast again and again, dear
father. I was glad of the progress I had made, but I wept over my
imperfection and was grieved by the fickleness of all that men do. In
this manner I seemed to have somehow forgotten the place I had come to
and why, until I was warned to throw off such sorrows, for which
another place would be more appropriate. I had better look around and
see what I had intended to see in coming here. The time to leave was
approaching, they said. The sun was already setting, and the shadow of
the mountain was growing longer and longer. Like a man aroused from
sleep, I turned back and looked toward the west. The boundary wall
between France and Spain, the ridge of the Pyrenees, is not visible
from there, though there is no obstacle of which I knew, and nothing
but the weakness of the mortal eye is the cause. However, one could
see most distinctly the mountains of the province of Lyons to the
right and, to the \page{44} left, the sea near Marseilles as well as
the waves that break against Aigues Mortes, although it takes several
days to travel to this city. The Rhone River was directly under our
eyes.

I admired every detail, now relishing earthly enjoyment, now lifting
up my mind to higher spheres after the example of my body, and I
thought it fit to look into the volume of Augustine's Confessions
which I owe to your loving kindness and preserve carefully, keeping it
always in my hands, in remembrance of the author as well as the
donor.\footnote{The small-sized manuscript codex of Augustine
\textit{Confessions}, a present from Dionigi, accompanied Petrarch
wherever he went until the last year of his life, when he could no
longer read its minute script and gave the book to Luigi Marsili\ldots
as a token of his friendship.} It is a little book of smallest size
but full of infinite sweetness. I opened it with the intention of
reading whatever might occur to me first: nothing, indeed, but pious
and devout sentences could come to hand. I happened to hit upon the
tenth book of the work. My brother stood beside me, intently expecting
to hear something from Augustine on my mouth. I ask God to be my
witness and my brother who was with me: Where I fixed my eyes first,
it was written: ``And men go to admire the high mountains, the vast
floods of the sea, the huge streams of the rivers, the circumference
of the ocean and the revolutions of the stars---and desert
themselves.''\footnote{Augustine, \textit{Confessions} x. 8. 15.} I
was stunned, I confess. I bade my brother, who wanted to hear more,
not to molest me, and closed the book, angry with myself that I still
admired earthly things. Long since I ought to have learned, even from
pagan philosophers, that ``nothing is admirable besides the mind;
compared to its greatness nothing is great.''\footnote{Seneca,
\textit{Epistle} 8. 5.}

I was completely satisfied with what I had seen of the mountain and
turned my inner eye toward myself. From this hour nobody heard me say
a word until we arrived at the bottom. These words occupied me
sufficiently. I could not imagine that this had happened to me by
chance: I was convinced that what-\page{45}ever I had read there was
said to me and to nobody else. I remembered that Augustine once
suspected the same regarding himself, when, while he was reading the
Apostolic Epistles, the first passage that occurred to him was, as he
himself relates: ``Not in banqueting and drunkenness, not in
chambering and wantonness, not in strife and envying; but put ye on
the Lord Jesus Christ, and make no provision for the flesh to fulfil
your lusts.''\footnote{The Romans 13:13--14, quoted by Augustine
\textit{Confessions} viii. 12. 29.} The same had happened before to
Anthony: he heard the Gospel where it is written: ``If thou wilt be
perfect, go and sell that thou hast, and give to the poor, and come
and follow me, and thou shalt have treasure in
heaven.''\footnote{Matt. 19:21, quoted by Athanasius in his
\textit{Life of St. Anthony} (Latin version by Euagrius), chap. 2, and
from there by Augustine \textit{Confessions} viii. 12. 29.} As his
biographer Athanasius says, he applied the Lord's command to himself,
just as if the Scripture had been recited for his sake. And as
Anthony, having heard this, sought nothing else, and as Angustine,
having read the other passage, proceeded no further, the end of all my
reading was the few words I have already set down. Silently I thought
over how greatly mortal men lack counsel who, neglecting the noblest
part of themselves in empty parading, look without for what can be
found within. I admired the nobility of the mind, had it not
voluntarily degenerated and strayed from the primordial state of its
origin, converting into disgrace what God had given to be its honor.

How often, do you think, did I turn back and look up to the summit of
the mountain today while I was walking down? It seemed to me hardly
higher than a cubit compared to the height of human contemplation,
were the latter not plunged into the filth of earthly sordidness. This
too occurred to me at every step: ``If you do not regret undergoing so
much sweat and hard labor to lift the body a bit nearer to heaven,
ought any cross or jail or torture to frighten the mind that is trying
to come nearer to God and set its feet upon the swollen summit of
insolence \page{46} and upon the fate of mortal men?'' And this too:
``How few will ever succeed in not diverging from this path because of
fear of hardship or desire for smooth comfort?\footnote{Cf. Matt.
7:13--15.} Too fortunate would be any man who accomplished such a
feat---were there ever such anywhere. This would be him of whom I
should judge the Poet was thinking when he wrote:

\begin{quote} Happy the man who succeeded in baring the causes of
things\\ And who trod underfoot all fear, inexorable Fate and\\ Greedy
Acheron's uproar\ldots\footnote{Virgil, \textit{Georgica} ii.
490--92.} \end{quote}

\noindent How intensely ought we to exert our strength to get under
foot not a higher spot of earth but the passions which are puffed up
by earthly instincts.''

Such emotions were rousing a storm in my breast as, without perceiving
the roughness of the path, I returned late at night to the little
rustic inn from which I had set out before dawn. The moon was shining
all night long and offered her friendly service to the wanderers.
While the servants were busy preparing our meal I withdrew quite alone
into a remote part of the house to write this letter to you in all
haste and on the spur of the moment. I was afraid the intention to
write might evaporate, since the rapid change of scene was likely to
cause a change of mood if I deferred it.

And, thus, most loving father, gather from this letter how eager I am
to leave nothing whatever in my heart hidden from your eyes. Not only
do I lay my whole life open to you with the utmost care but every
single thought of mine. Pray for these thoughts, I beseech you, that
they may at last find stability. So long have they been idling about
and, finding no firm stand, been uselessly driven through so many
matters. May they now turn at last to the One, the Good, the True, the
stably Abiding.

Farewell

On the twenty-sixth day of April, at Malauc\`{e}ne.

