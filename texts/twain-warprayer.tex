
%\author{Mark Twain [Samuel Clemens]}
\author{Mark Twain}
\authdate{1835--1910}
\textdate{1904/5}
\chapter{The War Prayer}
\source{twain1923.34}

\page{394}It was a time of great and exalting excitement. The country
was up in arms, the war was on, in every breast burned the holy fire
of patriotism; the drums were beating, the bands playing, the toy
pistols popping, the bunched firecrackers hissing and spluttering; on
every hand and far down the receding and fading spread of roofs and
balconies a fluttering wilderness of flags flashed in the sun; daily
the young volunteers marched down the wide avenue gay and fine in
their new uniforms, the proud fathers and mothers and sisters and
sweethearts cheering them with voices choked with happy emotion as
they swung by; nightly the packed mass meetings listened, panting, to
patriot oratory which stirred the deepest deeps of their hearts, and
which they interrupted at briefest intervals with cyclones of
applause, the tears running down their cheeks the while; in the
churches the pastors preached devotion to flag and country, and
invoked the God of Battles beseeching His aid in our good cause in
outpourings of fervid eloquence which moved every listener. It was
indeed a glad and gracious time, and the half dozen rash spirits that
ventured to disapprove of the war and cast a doubt upon its
righteousness straightway got such a stern and angry \page{395}
warning that for their personal safety's sake they quickly shrank out
of sight and offended no more in that way.

% NOTE: text has 'them home from the war'

Sunday morning came---next day the battalions would leave for the
front; the church was filled; the volunteers were there, their young
faces alight with martial dreams---vi\-sions of the stern advance, the
gathering momentum, the rushing charge, the flashing sabers, the
flight of the foe, the tumult, the enveloping smoke, the fierce
pursuit, the surrender!---then home from the war, bronzed heroes,
welcomed, adored, submerged in golden seas of glory! With the
volunteers sat their dear ones, proud, happy, and envied by the
neighbors and friends who had no sons and brothers to send forth to
the field of honor, there to win for the flag, or, failing, die the
noblest of noble deaths. The service proceeded; a war chapter from the
Old Testament was read; the first prayer was said; it was followed by
an organ burst that shook the building, and with one impulse the house
rose, with glowing eyes and beating hearts, and poured out that
tremendous invocation---

\begin{verse}
``God the all-terrible! Thou who ordainest,\\
Thunder thy clarion and lightning thy sword!''
\end{verse}

\noindent Then came the ``long'' prayer. None could remember the like
of it for passionate pleading and moving and beautiful language. The
burden of its supplication was, that an ever-merciful and benignant
Father of us all would watch over our noble young soldiers, and aid,
comfort, and encourage them in their patriotic work; bless them,
shield them in the day of battle \page{396} and the hour of peril,
bear them in His mighty hand, make them strong and confident,
invincible in the bloody onset; help them to crush the foe, grant to
them and to their flag and country imperishable honor and glory---

An aged stranger entered and moved with slow and noiseless step up the
main aisle, his eyes fixed upon the minister, his long body clothed in
a robe that reached to his feet, his head bare, his white hair
descending in a frothy cataract to his shoulders, his seamy face
unnaturally pale, pale even to ghastliness. With all eyes following
him and wondering, he made his silent way; without pausing, he
ascended to the preacher's side and stood there, waiting. With shut
lids the preacher, unconscious of his presence, continued with his
moving prayer, and at last finished it with the words, uttered in
fervent appeal, ``Bless our arms, grant us the victory, O Lord our
God, Father and Protector of our land and flag!''

The stranger touched his arm, motioned him to step a\-side---which the
startled minister did---and took his place. During some moments he
surveyed the spellbound audience with solemn eyes, in which burned an
uncanny light; then in a deep voice he said:

``I come from the Throne---bear\-ing a message from Almighty God!''
The words smote the house with a shock; if the stranger perceived it
he gave no attention. ``He has heard the prayer of His servant your
shepherd, and will grant it if such shall be your desire after I, His
messenger, shall have explained to you its im\-port---that is to say,
its full import. \page{397} For it is like unto many of the prayers of
men, in that it asks for more than he who utters it is aware
of---except he pause and think.

``God's servant and yours has prayed his prayer. Has he paused and
taken thought? Is it one prayer? No, it is two---one uttered, the
other not. Both have reached the ear of Him Who heareth all
supplications, the spoken and the unspoken. Ponder this---keep it in
mind. If you would beseech a blessing upon yourself, beware! lest
without intent you invoke a curse upon a neighbor at the same time. If
you pray for the blessing of rain upon your crop which needs it, by
that act you are possibly praying for a curse upon some neighbor's
crop which may not need rain and can be injured by it.

``You have heard your servant's prayer---the uttered part of it. I am
commissioned of God to put into words the other part of it---that part
which the pas\-tor---and also you in your hearts---fervently prayed
silently. And ignorantly and unthinkingly? God grant that it was so!
You heard these words: `Grant us the victory, O Lord our God!' That is
sufficient. The \textit{whole} of the uttered prayer is compact into
those pregnant words. Elaborations were not necessary. When you have
prayed for victory you have prayed for many unmentioned results which
follow vic\-to\-ry---\textit{must} follow it, cannot help but follow
it. Upon the listening spirit of God the Father fell also the unspoken
part of the prayer. He commandeth me to put it into words. Listen!

``O Lord our Father, our young patriots, idols of \page{398} our
hearts, go forth to bat\-tle---be Thou near them! With them---in
spir\-it---we also go forth from the sweet peace of our beloved
firesides to smite the foe. O Lord our God, help us to tear their
soldiers to bloody shreds with our shells; help us to cover their
smiling fields with the pale forms of their patriot dead; help us to
drown the thunder of the guns with the shrieks of their wounded,
writhing in pain; help us to lay waste their humble homes with a
hurricane of fire; help us to wring the hearts of their unoffending
widows with unavailing grief; help us to turn them out roofless with
little children to wander unfriended the wastes of their desolated
land in rags and hunger and thirst, sports of the sun flames of summer
and the icy winds of winter, broken in spirit, worn with travail,
imploring Thee for the refuge of the grave and denied it---for our
sakes who adore Thee, Lord, blast their hopes, blight their lives,
protract their bitter pilgrimage, make heavy their steps, water their
way with their tears, stain the white snow with the blood of their
wounded feet! We ask it, in the spirit of love, of Him Who is the
Source of Love, and Who is the ever-faithful refuge and friend of all
that are sore beset and seek His aid with humble and contrite hearts.
Amen.''

(\textit{After a pause}.) ``Ye have prayed it; if ye still desire it,
speak! The messenger of the Most High waits!''

It was believed afterward that the man was a lunatic, because there
was no sense in what he said.

