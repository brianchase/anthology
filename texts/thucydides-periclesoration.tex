
\author{Thucydides}
\authdate{ca. 460 -- ca. 395 {\smaller BCE}}
\textdate{ca. late 5th, early 4th century {\smaller BCE}}
\chapter[Thucydides -- Pericles' Funeral Oration]{Pericles' Funeral
Oration\\\smaller History of the Peloponnesian War, bk. 2.34--2.46}

\nfootnote{\fullcite*[bk. 2.34--2.46]{thucydides1900}}

\page{125}During the same winter, in accordance with an old national
custom, the funeral of those who first fell in this war was celebrated
by the Athenians at the public charge. The ceremony is as follows:
Three days before the celebration they erect a tent in which the bones
of the dead are laid out, and every one brings to his own dead any
offering which he pleases. At the time of the funeral the bones are
placed in chests of cypress wood, which are conveyed on hearses; there
is one chest for each tribe. They also carry a single empty litter
decked with a pall for all whose bodies are missing, and cannot be
recovered after the battle. The procession is accompanied by any one
who chooses, whether citizen or stranger, and the female relatives of
the deceased are present at the place of interment and make
lamentation. The public sepulchre is situated in the most beautiful
spot outside the walls; there they always bury those who fall in war;
only after the battle of Marathon the dead, in recognition of their
pre-eminent valour, were interred on the field. When the remains have
been laid in the earth, some man of known ability and high reputation,
chosen by the city, delivers a suitable oration over them; after
which the people depart. Such is the manner of interment; and
\page{126} the ceremony was repeated from time to time throughout the
war. Over those who were the first buried Pericles was chosen to
speak. At the fitting moment he advanced from the sepulchre to a lofty
stage, which had been erected in order that he might be heard as far
as possible by the multitude, and spoke as follows:---

\begin{center}\small(FUNERAL SPEECH.)\end{center}

`Most of those who have spoken here before me have commended the
lawgiver who added this oration to our other funeral customs; it
seemed to them a worthy thing that such an honour should be given at
their burial to the dead who have fallen on the field of battle. But I
should have preferred that, when men's deeds have been brave, they
should be honoured in deed only, and with such an honour as this
public funeral, which you are now witnessing. Then the reputation of
many would not have been imperiled on the eloquence or want of
eloquence of one, and their virtues believed or not as he spoke well
or ill. For it is difficult to say neither too little nor too much;
and even moderation is apt not to give the impression of truthfulness.
The friend of the dead who knows the facts is likely to think that the
words of the speaker fall short of his knowledge and of his wishes;
another who is not so well informed, when he hears of anything which
surpasses his own powers, will be envious and will suspect
exaggeration. Mankind are tolerant of the praises of others so long as
each hearer thinks that he can do as well or nearly as well himself,
but, when the speaker rises above him, jealousy is aroused and he
begins to be incredulous. However, since our ancestors have set the
seal of their approval upon the practice, I must obey, and to the
utmost of my power shall endeavour to satisfy the wishes and beliefs
of all who hear me.

\page{127}`I will speak first of our ancestors, for it is right and
seemly that now, when we are lamenting the dead, a tribute should be
paid to their memory. There has never been a time when they did not
inhabit this land, which by their valour they will have handed down
from generation to generation, and we have received from them a free
state. But if they were worthy of praise, still more were our fathers,
who added to their inheritance, and after many a struggle transmitted
to us their sons this great empire. And we ourselves assembled here
to-day, who are still most of us in the vigour of life, have carried
the work of improvement further, and have richly endowed our city with
all things, so that she is sufficient for herself both in peace and
war. Of the military exploits by which our various possessions were
acquired, or of the energy with which we or our fathers drove back the
tide of war, Hellenic or Barbarian, I will not speak; for the tale
would be long and is familiar to you. But before I praise the dead, I
should like to point out by what principles of action we
rose\footnote{Reading \grk{ἤλθομεν}.} to power, and under what
institutions and through what manner of life our empire became great.
For I conceive that such thoughts are not unsuited to the occasion,
and that this numerous assembly of citizens and strangers may
profitably listen to them.

`Our form of government does not enter into rivalry with the
institutions of others. We do not copy our neighbours, but are an
example to them. It is true that we are called a democracy, for the
administration is in the hands of the many and not of the few. But
while there exists equal justice to all and alike in their private
disputes, the claim of excellence is also recognised; and when a
\page{128} citizen is in any way distinguished, he is preferred to the
public service, not as a matter of privilege, but as the reward of
merit. Neither is poverty a bar, but a man may benefit his country
whatever be the obscurity of his condition. There is no exclusiveness
in our public life, and in our private intercourse we are not
suspicious of one another, nor angry with our neighbour if he does
what he likes; we do not put on sour looks at him which, though
harmless, are not pleasant. While we are thus unconstrained in our
private business, a spirit of reverence pervades our public acts; we
are prevented from doing wrong by respect for the authorities and for
the laws, having an especial regard to those which are ordained for
the protection of the injured as well as those unwritten laws which
bring upon the transgressor of them the reprobation of the general
sentiment.

`And we have not forgotten to provide for our weary spirits many
relaxations from toil; we have regular games and sacrifices throughout
the year; our homes are beautiful and elegant; and the delight which
we daily feel in all these things helps to banish melancholy. Because
of the greatness of our city the fruits of the whole earth flow in
upon us; so that we enjoy the goods of other countries as freely as of
our own.

`Then, again, our military training is in many respects superior to
that of our adversaries. Our city is thrown open to the world, and we
never expel a foreigner and prevent him from seeing or learning
anything of which the secret if revealed to an enemy might profit him.
We rely not upon management or trickery, but upon our own hearts and
hands. And in the matter of education, whereas they from early youth
are always undergoing laborious exercises which are to make them
brave, we live at ease, and yet are equally \page{129} ready to face
the perils which they face\footnote{Or, `perils such as our strength
can bear;' or `perils which are enough to daunt us.'}. And here is the
proof. The Lacedaemonians come into Attica not by themselves, but with
their whole confederacy following; we go alone into a neighbour's
country; and although our opponents are fighting for their homes and
we on a foreign soil, we have seldom any difficulty in overcoming
them. Our enemies have never yet felt our united strength; the care of
a navy divides our attention, and on land we are obliged to send our
own citizens everywhere. But they, if they meet and defeat a part of
our army, are as proud as if they had routed us all, and when defeated
they pretend to have been vanquished by us all.

`If then we prefer to meet danger with a light heart but without
laborious training, and with a courage which is gained by habit and
not enforced by law, are we not greatly the gainers? Since we do not
anticipate the pain, although, when the hour comes, we can be as brave
as those who never allow themselves to rest; and thus too our city is
equally admirable in peace and in war. For we are lovers of the
beautiful, yet simple in our tastes, and we cultivate the mind without
loss of manliness. Wealth we employ, not for talk and ostentation, but
when there is a real use for it. To avow poverty with us is no
disgrace; the true disgrace is in doing nothing to avoid it. An
Athenian citizen does not neglect the state because he takes care of
his own household; and even those of us who are engaged in business
have a very fair idea of politics. We alone regard a man who takes no
interest in public affairs, not as a harmless, but as a useless
character; and if few of us are originators, we are all sound judges
of a policy. The great impediment to action \page{130} is, in our
opinion, not discussion, but the want of that knowledge which is
gained by discussion preparatory to action. For we have a peculiar
power of thinking before we act and of acting too, whereas other men
are courageous from ignorance but hesitate upon reflection. And they
are surely to be esteemed the bravest spirits who, having the clearest
sense both of the pains and pleasures of life, do not on that account
shrink from danger. In doing good, again, we are unlike others; we
make our friends by conferring, not by receiving favours. Now he who
confers a favour is the firmer friend, because he would fain by
kindness keep alive the memory of an obligation; but the recipient is
colder in his feelings, because he knows that in requiting another's
generosity he will not be winning gratitude but only paying a debt. We
alone do good to our neighbours not upon a calculation of interest,
but in the confidence of freedom and in a frank and fearless spirit.
To sum up: I say that Athens is the school of Hellas, and that the
individual Athenian in his own person seems to have the power of
adapting himself to the most varied forms of action with the utmost
versatility and grace. This is no passing and idle word, but truth and
fact; and the assertion is verified by the position to which these
qualities have raised the state. For in the hour of trial Athens alone
among her contemporaries is superior to the report of her. No enemy
who comes against her is indignant at the reverses which he sustains
at the hands of such a city; no subject complains that his masters are
unworthy of him. And we shall assuredly not be without witnesses;
there are mighty monuments of our power which will make us the wonder
of this and of succeeding ages; we shall not need the praises of Homer
or of any other panegyrist whose poetry may please for the
moment\footnote{Cp. i. 10 med., and 21.}, \page{131} although his
representation of the facts will not bear the light of day. For we
have compelled every land and every sea to open a path for our valour,
and have everywhere planted eternal memorials of our friendship and of
our enmity. Such is the city for whose sake these men nobly fought and
died; they could not bear the thought that she might be taken from
them; and every one of us who survive should gladly toil on her
behalf.

`I have dwelt upon the greatness of Athens because I want to show you
that we are contending for a higher prize than those who enjoy none of
these privileges, and to establish by manifest proof the merit of
these men whom I am now commemorating. Their loftiest praise has been
already spoken. For in magnifying the city I have magnified them, and
men like them whose virtues made her glorious. And of how few Hellenes
can it be said as of them, that their deeds when weighed in the
balance have been found equal to their fame! Methinks that a death
such as theirs has been the true measure of a man's worth; it may be
the first revelation of his virtues, but is at any rate their final
seal. For even those who come short in other ways may justly plead the
valour with which they have fought for their country; they have
blotted out the evil with the good, and have benefited the state more
by their public services than they have injured her by their private
actions. None of these men were enervated by wealth or hesitated to
resign the pleasures of life; none of them put off the evil day in the
hope, natural to poverty, that a man, though poor, may one day become
rich. But, deeming that the punishment of their enemies was sweeter
than any of these things, and that they could fall in no nobler cause,
they determined at the hazard of their lives to be honorably avenged,
and to leave the rest. They resigned to hope their unknown chance of
happiness; but in the face of death they resolved to rely upon
them-\page{132}selves alone. And when the moment came they were minded
to resist and suffer, rather than to fly and save their lives; they
ran away from the word of dishonour, but on the battle-field their
feet stood fast, and in an instant, at the height of their fortune,
they passed away from the scene, not of their fear, but of their
glory\footnote{Or, taking \grk{τύχης} with \grk{καιροῦ}: `while for a
moment they were in the hands of fortune, at the height, not of terror
but of glory, they passed away.'}.

`Such was the end of these men; they were worthy of Athens, and the
living need not desire to have a more heroic spirit, although they may
pray for a less fatal issue. The value of such a spirit is not to be
expressed in words. Any one can discourse to you for ever about the
advantages of a brave defence, which you know already. But instead of
listening to him I would have you day by day fix your eyes upon the
greatness of Athens, until you become filled with the love of her; and
when you are impressed by the spectacle of her glory, reflect that
this empire has been acquired by men who knew their duty and had the
courage to do it, who in the hour of conflict had the fear of
dishonour always present to them, and who, if ever they failed in an
enterprise, would not allow their virtues to be lost to their country,
but freely gave their lives to her as the fairest offering which they
could present at her feast. The sacrifice which they collectively made
was individually repaid to them; for they received again each one for
himself a praise which grows not old, and the noblest of all
sepulchres---I speak not of that in which their remains are laid, but
of that in which their glory survives, and is proclaimed always and on
every \page{133} fitting occasion both in word and deed. For the whole
earth is the sepulchre of famous men; not only are they commemorated
by columns and inscriptions in their own country, but in foreign lands
there dwells also an unwritten memorial of them, graven not on stone
but in the hearts of men. Make them your examples, and, esteeming
courage to be freedom and freedom to be happiness, do not weigh too
nicely the perils of war. The unfortunate who has no hope of a change
for the better has less reason to throw away his life than the
prosperous who, if he survive, is always liable to a change for the
worse, and to whom any accidental fall makes the most serious
difference. To a man of spirit, cowardice and disaster coming together
are far more bitter than death striking him unperceived at a time when
he is full of courage and animated by the general hope.

`Wherefore I do not now commiserate the parents of the dead who stand
here; I would rather comfort them. You know that your life has been
passed amid manifold vicissitudes; and that they may be deemed
fortunate who have gained most honour, whether an honourable death
like theirs, or an honorable sorrow like yours, and whose days have
been so ordered that the term of their happiness is likewise the term
of their life. I know how hard it is to make you feel this, when the
good fortune of others will too often remind you of the gladness which
once lightened your hearts. And sorrow is felt at the want of those
blessings, not which a man never knew, but which were a part of his
life before they were taken from him. Some of you are of an age at
which they may hope to have other children, and they ought to bear
their sorrow better; not only will the children who may hereafter be
born make them forget their own lost ones, but the city \page{134}
will be doubly a gainer. She will not be left desolate, and she will
be safer. For a man's counsel cannot have equal weight or worth, when
he alone has no children to risk in the general danger. To those of
you who have passed their prime, I say: ``Congratulate yourselves that
you have been happy during the greater part of your days; remember
that your life of sorrow will not last long, and be comforted by the
glory of those who are gone. For the love of honour alone is ever
young, and not riches, as some say, but honour is the delight of men
when they are old and useless.''

`To you who are the sons and brothers of the departed, I see that the
struggle to emulate them will be an arduous one. For all men praise
the dead, and, however preeminent your virtue may be, hardly will you
be thought, I do not say to equal, but even to approach them. The
living have their rivals and detractors, but when a man is out of the
way, the honour and good-will which he receives is unalloyed. And, if
I am to speak of womanly virtues to those of you who will henceforth
be widows, let me sum them up in one short admonition: To a woman not
to show more weakness than is natural to her sex is a great glory, and
not to be talked about for good or for evil among men.

`I have paid the required tribute, in obedience to the law, making use
of such fitting words as I had. The tribute of deeds has been paid in
part; for the dead have been honourably interred, and it remains only
that their children should be maintained at the public charge until
they are grown up: this is the solid prize with which, as with a
garland, Athens crowns her sons living and dead, after a struggle like
theirs. For where the rewards of virtue are greatest, there the
noblest citizens \page{135} are enlisted in the service of the state.
And now, when you have duly lamented, every one his own dead, you may
depart.'

