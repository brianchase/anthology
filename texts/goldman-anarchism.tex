
\author{Emma Goldman}
\authdate{1869--1940}
\textdate{1910}
\chapter{Anarchism: What It Really Stands For}
\source{goldman1917.1}

% The book in which this essay appears was first published in 1910. As
% far as I can tell, the essay was first published then. It does not
% appear in Mother Earth.

\page{53}\poemtitle{Anarchy.}
\begin{verse}
Ever reviled, accursed, ne'er understood,\\
\vin  Thou art the grisly terror of our age.\\
``Wreck of all order,'' cry the multitude,\\
\vin ``Art thou, and war and murder's endless rage.''\\
O, let them cry. To them that ne'er have striven\\
\vin The truth that lies behind a word to find,\\
To them the word's right meaning was not given.\\
\vin They shall continue blind among the blind.\\
But thou, O word, so clear, so strong, so pure,\\
\vin Thou sayest all which I for goal have taken.\\
I give thee to the future! Thine secure\\
\vin When each at least unto himself shall waken.\\
Comes it in sunshine? In the tempest's thrill?\\
\vin I cannot tell---but it the earth shall see!\\
I am an Anarchist! Wherefore I will\\
\vin Not rule, and also ruled I will not be!
\end{verse}
\attrib{John Henry Mackay.}

\noindent The history of human growth and development is at the same
time the history of the terrible struggle of every new idea heralding
the approach of a brighter dawn. In its tenacious hold on tradition,
the Old \page{54} has never hesitated to make use of the foulest and
cruelest means to stay the advent of the New, in whatever form or
period the latter may have asserted itself. Nor need we retrace our
steps into the distant past to realize the enormity of opposition,
difficulties, and hardships placed in the path of every progressive
idea. The rack, the thumbscrew, and the knout are still with us; so
are the convict's garb and the social wrath, all conspiring against
the spirit that is serenely marching on.

Anarchism could not hope to escape the fate of all other ideas of
innovation. Indeed, as the most revolutionary and uncompromising
innovator, Anarchism must needs meet with the combined ignorance and
venom of the world it aims to reconstruct.

To deal even remotely with all that is being said and done against
Anarchism would necessitate the writing of a whole volume. I shall
therefore meet only two of the principal objections. In so doing, I
shall attempt to elucidate what Anarchism really stands for.

The strange phenomenon of the opposition to Anarchism is that it
brings to light the relation between so-called intelligence and
ignorance. And yet this is not so very strange when we consider the
relativity of all things. The ignorant mass has in its favor that it
makes no pretense of knowledge or tolerance. Acting, as it always
does, by mere impulse, its reasons are like those of a child. ``Why?''
``Because.'' Yet the opposition of the uneducated to Anarchism
deserves the same consideration as that of the intelligent man.

\page{55}What, then, are the objections? First, Anarchism is
impractical, though a beautiful ideal. Second, Anarchism stands for
violence and destruction, hence it must be repudiated as vile and
dangerous. Both the intelligent man and the ignorant mass judge not
from a thorough knowledge of the subject, but either from hearsay or
false interpretation.

A practical scheme, says Oscar Wilde, is either one already in
existence, or a scheme that could be carried out under the existing
conditions; but it is exactly the existing conditions that one objects
to, and any scheme that could accept these conditions is wrong and
foolish. The true criterion of the practical, therefore, is not
whether the latter can keep intact the wrong or foolish; rather is it
whether the scheme has vitality enough to leave the stagnant waters of
the old, and build, as well as sustain, new life. In the light of this
conception, Anarchism is indeed practical. More than any other idea,
it is helping to do away with the wrong and foolish; more than any
other idea, it is building and sustaining new life.

The emotions of the ignorant man are continuously kept at a pitch by
the most blood-curdling stories about Anarchism. Not a thing too
outrageous to be employed against this philosophy and its exponents.
Therefore Anarchism represents to the unthinking what the proverbial
bad man does to the child,---a black monster bent on swallowing
everything; in short, destruction and violence.

Destruction and violence! How is the ordinary man to know that the
most violent element in society \page{56} is ignorance; that its power
of destruction is the very thing Anarchism is combating? Nor is he
aware that Anarchism, whose roots, as it were, are part of nature's
forces, destroys, not healthful tissue, but parasitic growths that
feed on the life's essence of society. It is merely clearing the soil
from weeds and sagebrush, that it may eventually bear healthy fruit.

Someone has said that it requires less mental effort to condemn than
to think. The widespread mental indolence, so prevalent in society,
proves this to be only too true. Rather than to go to the bottom of
any given idea, to examine into its origin and meaning, most people
will either condemn it altogether, or rely on some superficial or
prejudicial definition of non-essentials.

Anarchism urges man to think, to investigate, to analyze every
proposition; but that the brain capacity of the average reader be not
taxed too much, I also shall begin with a definition, and then
elaborate on the latter.

\begin{description}

\item[\textsc{Anarchism}]--- The philosophy of a new social order
based on liberty unrestricted by man-made law; the theory that all
forms of government rest on violence, and are therefore wrong and
harmful, as well as unnecessary.

\end{description}

The new social order rests, of course, on the materialistic basis of
life; but while all Anarchists agree that the main evil today is an
economic one, they maintain that the solution of that evil can be
brought about only through the consideration of \textit{every phase}
of life,---in\-di\-vid\-u\-al, as well as the collective; the
internal, as well as the external phases.

\page{57}A thorough perusal of the history of human development will
disclose two elements in bitter conflict with each other; elements
that are only now beginning to be understood, not as foreign to each
other, but as closely related and truly harmonious, if only placed in
proper environment: the individual and social instincts. The
individual and society have waged a relentless and bloody battle for
ages, each striving for supremacy, because each was blind to the value
and importance of the other. The individual and social
instincts,---the one a most potent factor for individual endeavor, for
growth, aspiration, self-realization; the other an equally potent
factor for mutual helpfulness and social well-being.

The explanation of the storm raging within the individual, and between
him and his surroundings, is not far to seek. The primitive man,
unable to understand his being, much less the unity of all life, felt
himself absolutely dependent on blind, hidden forces ever ready to
mock and taunt him. Out of that attitude grew the religious concepts
of man as a mere speck of dust dependent on superior powers on high,
who can only be appeased by complete surrender. All the early sagas
rest on that idea, which continues to be the \textit{Leitmotiv} of the
biblical tales dealing with the relation of man to God, to the State,
to society. Again and again the same motif, \textit{man is nothing,
the powers are everything}. Thus Jehovah would only endure man on
condition of complete surrender. Man can have all the glories of the
earth, but he must not become conscious of himself. The State,
society, and moral laws all sing the same re-\page{58}frain: Man can
have all the glories of the earth, but he must not become conscious of
himself.

Anarchism is the only philosophy which brings to man the consciousness
of himself; which maintains that God, the State, and society are
non-existent, that their promises are null and void, since they can be
fulfilled only through man's subordination. Anarchism is therefore the
teacher of the unity of life; not merely in nature, but in man. There
is no conflict between the individual and the social instincts, any
more than there is between the heart and the lungs: the one the
receptacle of a precious life essence, the other the repository of the
element that keeps the essence pure and strong. The individual is the
heart of society, conserving the essence of social life; society is
the lungs which are distributing the element to keep the life
essence---that is, the in\-di\-vid\-u\-al---pure and strong.

``The one thing of value in the world,'' says Emerson, ``is the active
soul; this every man contains within him. The soul active sees
absolute truth and utters truth and creates.'' In other words, the
individual instinct is the thing of value in the world. It is the true
soul that sees and creates the truth alive, out of which is to come a
still greater truth, the re-born social soul.

Anarchism is the great liberator of man from the phantoms that have
held him captive; it is the arbiter and pacifier of the two forces for
individual and social harmony. To accomplish that unity, Anarchism has
declared war on the pernicious influences which have so far prevented
the harmonious \page{59} blending of individual and social instincts,
the individual and society.

Religion, the dominion of the human mind; Property, the dominion of
human needs; and Government, the dominion of human conduct, represent
the stronghold of man's enslavement and all the horrors it entails.
Religion! How it dominates man's mind, how it humiliates and degrades
his soul. God is everything, man is nothing, says religion. But out of
that nothing God has created a kingdom so des\-potic, so tyrannical,
so cruel, so terribly exacting that naught but gloom and tears and
blood have ruled the world since gods began. Anarchism rouses man to
rebellion against this black monster. Break your mental fetters, says
Anarchism to man, for not until you think and judge for yourself will
you get rid of the dominion of darkness, the greatest obstacle to all
progress.

Property, the dominion of man's needs, the denial of the right to
satisfy his needs. Time was when property claimed a divine right, when
it came to man with the same refrain, even as religion, ``Sacrifice!
Abnegate! Submit!'' The spirit of Anarchism has lifted man from his
prostrate position. He now stands erect, with his face toward the
light. He has learned to see the insatiable, devouring, devastating
nature of property, and he is preparing to strike the monster dead.

``Property is robbery,'' said the great French Anarchist Proudhon.
Yes, but without risk and danger to the robber. Monopolizing the
accumulated efforts of man, property has robbed him of his
birth-\page{60}right, and has turned him loose a pauper and an
outcast. Property has not even the time-worn excuse that man does not
create enough to satisfy all needs. The A B C student of economics
knows that the productivity of labor within the last few decades far
exceeds normal demand. But what are normal demands to an abnormal
institution? The only demand that property recognizes is its own
gluttonous appetite for greater wealth, because wealth means power;
the power to subdue, to crush, to exploit, the power to enslave, to
outrage, to degrade. America is particularly boastful of her great
power, her enormous national wealth. Poor America, of what avail is
all her wealth, if the individuals comprising the nation are
wretchedly poor? If they live in squalor, in filth, in crime, with
hope and joy gone, a homeless, soilless army of human prey.

It is generally conceded that unless the returns of any business
venture exceed the cost, bankruptcy is inevitable. But those engaged
in the business of producing wealth have not yet learned even this
simple lesson. Every year the cost of production in human life is
growing larger (50,000 killed, 100,000 wounded in America last year);
the returns to the masses, who help to create wealth, are ever getting
smaller. Yet America continues to be blind to the inevitable
bankruptcy of our business of production. Nor is this the only crime
of the latter. Still more fatal is the crime of turning the producer
into a mere particle of a machine, with less will and decision than
his master of steel and iron. Man is being robbed not merely of the
products of his labor, but \page{61} of the power of free initiative,
of originality, and the interest in, or desire for, the things he is
making.

Real wealth consists in things of utility and beauty, in things that
help to create strong, beautiful bodies and surroundings inspiring to
live in. But if man is doomed to wind cotton around a spool, or dig
coal, or build roads for thirty years of his life, there can be no
talk of wealth. What he gives to the world is only gray and hideous
things, reflecting a dull and hideous existence,---too weak to live,
too cowardly to die. Strange to say, there are people who extol this
deadening method of centralized production as the proudest achievement
of our age. They fail utterly to realize that if we are to continue in
machine subserviency, our slavery is more complete than was our
bondage to the King. They do not want to know that centralization is
not only the death-knell of liberty, but also of health and beauty, of
art and science, all these being impossible in a clock-like,
mechanical atmosphere.

Anarchism cannot but repudiate such a method of production: its goal
is the freest possible expression of all the latent powers of the
individual. Oscar Wilde defines a perfect personality as ``one who
develops under perfect conditions, who is not wounded, maimed, or in
danger.'' A perfect personality, then, is only possible in a state of
society where man is free to choose the mode of work, the conditions
of work, and the freedom to work. One to whom the making of a table,
the building of a house, or the tilling of the soil, is what the
painting is to the artist and the discovery to the scientist,---the
\page{62} result of inspiration, of intense longing, and deep interest
in work as a creative force. That being the ideal of Anarchism, its
economic arrangements must consist of voluntary productive and
distributive associations, gradually developing into free communism,
as the best means of producing with the least waste of human energy.
Anarchism, however, also recognizes the right of the individual, or
numbers of individuals, to arrange at all times for other forms of
work, in harmony with their tastes and desires.

Such free display of human energy being possible only under complete
individual and social freedom, Anarchism directs its forces against
the third and greatest foe of all social equality; namely, the State,
organized authority, or statutory law,---the dominion of human
conduct.

Just as religion has fettered the human mind, and as property, or the
monopoly of things, has subdued and stifled man's needs, so has the
State enslaved his spirit, dictating every phase of conduct. ``All
government in essence,'' says Emerson, ``is tyranny.'' It matters not
whether it is government by divine right or majority rule. In every
instance its aim is the absolute subordination of the individual.

Referring to the American government, the greatest American Anarchist,
David Thoreau, said: ``Government, what is it but a tradition, though
a recent one, endeavoring to transmit itself unimpaired to posterity,
but each instance losing its integrity; it has not the vitality and
force of a single living man. Law never made man a whit more just; and
by \page{63} means of their respect for it, even the well disposed are
daily made agents of injustice.''

Indeed, the keynote of government is injustice. With the arrogance and
self-sufficiency of the King who could do no wrong, governments
ordain, judge, condemn, and punish the most insignificant offenses,
while maintaining themselves by the greatest of all offenses, the
annihilation of individual liberty. Thus Ouida is right when she
maintains that ``the State only aims at instilling those qualities in
its public by which its demands are obeyed, and its exchequer is
filled. Its highest attainment is the reduction of mankind to
clockwork. In its atmosphere all those finer and more delicate
liberties, which require treatment and spacious expansion, inevitably
dry up and perish. The State requires a taxpaying machine in which
there is no hitch, an exchequer in which there is never a deficit, and
a public, monotonous, obedient, colorless, spiritless, moving humbly
like a flock of sheep along a straight high road between two walls.''

Yet even a flock of sheep would resist the chicanery of the State, if
it were not for the corruptive, tyrannical, and oppressive methods it
employs to serve its purposes. Therefore Bakunin repudiates the State
as synonymous with the surrender of the liberty of the individual or
small minorities,---the destruction of social relationship, the
curtailment, or complete denial even, of life itself, for its own
aggrandizement. The State is the altar of political freedom and, like
the religious altar, it is maintained for the purpose of human
sacrifice.

In fact, there is hardly a modern thinker who \page{64} does not agree
that government, organized authority, or the State, is necessary
\textit{only} to maintain or protect property and monopoly. It has
proven efficient in that function only.

Even George Bernard Shaw, who hopes for the miraculous from the State
under Fabianism, nevertheless admits that ``it is at present a huge
machine for robbing and slave-driving of the poor by brute force.''
This being the case, it is hard to see why the clever prefacer wishes
to uphold the State after poverty shall have ceased to exist.

Unfortunately there are still a number of people who continue in the
fatal belief that government rests on natural laws, that it maintains
social order and harmony, that it diminishes crime, and that it
prevents the lazy man from fleecing his fellows. I shall therefore
examine these contentions.

A natural law is that factor in man which asserts itself freely and
spontaneously without any external force, in harmony with the
requirements of nature. For instance, the demand for nutrition, for
sex gratification, for light, air, and exercise, is a natural law. But
its expression needs not the machinery of government, needs not the
club, the gun, the handcuff, or the prison. To obey such laws, if we
may call it obedience, requires only spontaneity and free opportunity.
That governments do not maintain themselves through such harmonious
factors is proven by the terrible array of violence, force, and
coercion all governments use in order to live. Thus Blackstone is
right when he says, ``Human laws are invalid, because they are
contrary to the laws of nature.''

\page{65}Unless it be the order of Warsaw after the slaughter of
thousands of people, it is difficult to ascribe to governments any
capacity for order or social harmony. Order derived through submission
and maintained by terror is not much of a safe guaranty; yet that is
the only ``order'' that governments have ever maintained. True social
harmony grows naturally out of solidarity of interests. In a society
where those who always work never have anything, while those who never
work enjoy everything, solidarity of interests is non-existent; hence
social harmony is but a myth. The only way organized authority meets
this grave situation is by extending still greater privileges to those
who have already monopolized the earth, and by still further enslaving
the disinherited masses. Thus the entire arsenal of government---laws,
police, soldiers, the courts, legislatures, prisons,---is strenuously
engaged in ``harmonizing'' the most antagonistic elements in society.

The most absurd apology for authority and law is that they serve to
diminish crime. Aside from the fact that the State is itself the
greatest criminal, breaking every written and natural law, stealing in
the form of taxes, killing in the form of war and capital punishment,
it has come to an absolute standstill in coping with crime. It has
failed utterly to destroy or even minimize the horrible scourge of its
own creation.

Crime is naught but misdirected energy. So long as every institution
of today, economic, political, social, and moral, conspires to
misdirect human energy into wrong channels; so long as most people
\page{66} are out of place doing the things they hate to do, living a
life they loathe to live, crime will be inevitable, and all the laws
on the statutes can only increase, but never do away with, crime. What
does society, as it exists today, know of the process of despair, the
poverty, the horrors, the fearful struggle the human soul must pass on
its way to crime and degradation. Who that knows this terrible process
can fail to see the truth in these words of Peter Kropotkin:

``Those who will hold the balance between the benefits thus attributed
to law and punishment and the degrading effect of the latter on
humanity; those who will estimate the torrent of depravity poured
abroad in human society by the informer, favored by the Judge even,
and paid for in clinking cash by governments, under the pretext of
aiding to unmask crime; those who will go within prison walls and
there see what human beings become when deprived of liberty, when
subjected to the care of brutal keepers, to coarse, cruel words, to a
thousand stinging, piercing humiliations, will agree with us that the
entire apparatus of prison and punishment is an abomination which
ought to be brought to an end.''

The deterrent influence of law on the lazy man is too absurd to merit
consideration. If society were only relieved of the waste and expense
of keeping a lazy class, and the equally great expense of the
paraphernalia of protection this lazy class requires, the social
tables would contain an abundance for all, including even the
occasional lazy \page{67} individual. Besides, it is well to consider
that laziness results either from special privileges, or physical and
mental abnormalities. Our present insane system of production fosters
both, and the most astounding phenomenon is that people should want to
work at all now. Anarchism aims to strip labor of its deadening,
dulling aspect, of its gloom and compulsion. It aims to make work an
instrument of joy, of strength, of color, of real harmony, so that the
poorest sort of a man should find in work both recreation and hope.

To achieve such an arrangement of life, government, with its unjust,
arbitrary, repressive measures, must be done away with. At best it has
but imposed one single mode of life upon all, without regard to
individual and social variations and needs. In destroying government
and statutory laws, Anarchism proposes to rescue the self-respect and
independence of the individual from all restraint and invasion by
authority. Only in freedom can man grow to his full stature. Only in
freedom will he learn to think and move, and give the very best in
him. Only in freedom will he realize the true force of the social
bonds which knit men together, and which are the true foundation of a
normal social life.

But what about human nature? Can it be changed? And if not, will it
endure under Anarchism?

Poor human nature, what horrible crimes have been committed in thy
name! Every fool, from king to policeman, from the flatheaded
par-\page{68}son to the visionless dabbler in science, presumes to
speak authoritatively of human nature. The greater the mental
charlatan, the more definite his insistence on the wickedness and
weaknesses of human nature. Yet, how can any one speak of it today,
with every soul in a prison, with every heart fettered, wounded, and
maimed?

John Burroughs has stated that experimental study of animals in
captivity is absolutely useless. Their character, their habits, their
appetites undergo a complete transformation when torn from their soil
in field and forest. With human nature caged in a narrow space,
whipped daily into submission, how can we speak of its potentialities?

Freedom, expansion, opportunity, and, above all, peace and repose,
alone can teach us the real dominant factors of human nature and all
its wonderful possibilities.

Anarchism, then, really stands for the liberation of the human mind
from the dominion of religion; the liberation of the human body from
the dominion of property; liberation from the shackles and restraint
of government. Anarchism stands for a social order based on the free
grouping of individuals for the purpose of producing real social
wealth; an order that will guarantee to every human being free access
to the earth and full enjoyment of the necessities of life, according
to individual desires, tastes, and inclinations.

This is not a wild fancy or an aberration of the mind. It is the
conclusion arrived at by hosts of intellectual men and women the world
over; a con-\page{69}clusion resulting from the close and studious
observation of the tendencies of modern society: individual liberty
and economic equality, the twin forces for the birth of what is fine
and true in man.

% NOTE: period after 'methods' is correct

As to methods. Anarchism is not, as some may suppose, a theory of the
future to be realized through divine inspiration. It is a living force
in the affairs of our life, constantly creating new conditions. The
methods of Anarchism therefore do not comprise an iron-clad program to
be carried out under all circumstances. Methods must grow out of the
economic needs of each place and clime, and of the intellectual and
temperamental requirements of the individual. The serene, calm
character of a Tolstoy will wish different methods for social
reconstruction than the intense, overflowing personality of a Michael
Bakunin or a Peter Kropotkin. Equally so it must be apparent that the
economic and political needs of Russia will dictate more drastic
measures than would England or America. Anarchism does not stand for
military drill and uniformity; it does, however, stand for the spirit
of revolt, in whatever form, against everything that hinders human
growth. All Anarchists agree in that, as they also agree in their
opposition to the political machinery as a means of bringing about the
great social change.

``All voting,'' says Thoreau, ``is a sort of gaming, like checkers, or
backgammon, a playing with right and wrong; its obligation never
exceeds that of expediency. Even voting for the right thing is doing
nothing for it. A wise man will not leave \page{70} the right to the
mercy of chance, nor wish it to prevail through the power of the
majority.'' A close examination of the machinery of politics and its
achievements will bear out the logic of Thoreau.

What does the history of parliamentarism show? Nothing but failure and
defeat, not even a single reform to ameliorate the economic and social
stress of the people. Laws have been passed and enactments made for
the improvement and protection of labor. Thus it was proven only last
year that Illinois, with the most rigid laws for mine protection, had
the greatest mine disasters. In States where child labor laws prevail,
child exploitation is at its highest, and though with us the workers
enjoy full political opportunities, capitalism has reached the most
brazen zenith.

Even were the workers able to have their own representatives, for
which our good Socialist politicians are clamoring, what chances are
there for their honesty and good faith? One has but to bear in mind
the process of politics to realize that its path of good intentions is
full of pitfalls: wire-pulling, intriguing, flattering, lying,
cheating; in fact, chicanery of every description, whereby the
political aspirant can achieve success. Added to that is a complete
demoralization of character and conviction, until nothing is left that
would make one hope for anything from such a human derelict. Time and
time again the people were foolish enough to trust, believe, and
support with their last farthing aspiring politicians, only to find
themselves betrayed and cheated.

\page{71}It may be claimed that men of integrity would not become
corrupt in the political grinding mill. Perhaps not; but such men
would be absolutely helpless to exert the slightest influence in
behalf of labor, as indeed has been shown in numerous instances. The
State is the economic master of its servants. Good men, if such there
be, would either remain true to their political faith and lose their
economic support, or they would cling to their economic master and be
utterly unable to do the slightest good. The political arena leaves
one no alternative, one must either be a dunce or a rogue.

The political superstition is still holding sway over the hearts and
minds of the masses, but the true lovers of liberty will have no more
to do with it. Instead, they believe with Stirner that man has as much
liberty as he is willing to take. Anarchism therefore stands for
direct action, the open defiance of, and resistance to, all laws and
restrictions, economic, social, and moral. But defiance and resistance
are illegal. Therein lies the salvation of man. Everything illegal
necessitates integrity, self-reliance, and courage. In short, it calls
for free, independent spirits, for ``men who are men, and who have a
bone in their backs which you cannot pass your hand through.''

Universal suffrage itself owes its existence to direct action. If not
for the spirit of rebellion, of the defiance on the part of the
American revolutionary fathers, their posterity would still wear the
King's coat. If not for the direct action of a John \page{72} Brown
and his comrades, America would still trade in the flesh of the black
man. True, the trade in white flesh is still going on; but that, too,
will have to be abolished by direct action. Trade-unionism, the
economic arena of the modern gladiator, owes its existence to direct
action. It is but recently that law and government have attempted to
crush the trade-union movement, and condemned the exponents of man's
right to organize to prison as conspirators. Had they sought to assert
their cause through begging, pleading, and compromise, trade-unionism
would today be a negligible quantity. In France, in Spain, in Italy,
in Russia, nay even in England (witness the growing rebellion of
English labor unions), direct, revolutionary, economic action has
become so strong a force in the battle for industrial liberty as to
make the world realize the tremendous importance of labor's power. The
General Strike, the supreme expression of the economic consciousness
of the workers, was ridiculed in America but a short time ago. Today
every great strike, in order to win, must realize the importance of
the solidaric general protest.

Direct action, having proven effective along economic lines, is
equally potent in the environment of the individual. There a hundred
forces encroach upon his being, and only persistent resistance to them
will finally set him free. Direct action against the authority in the
shop, direct action against the authority of the law, direct action
against the invasive, meddlesome authority of our moral code, is the
logical, consistent method of Anarchism.

\page{73}Will it not lead to a revolution? Indeed, it will. No real
social change has ever come about without a revolution. People are
either not familiar with their history, or they have not yet learned
that revolution is but thought carried into action.

Anarchism, the great leaven of thought, is today permeating every
phase of human endeavor. Science, art, literature, the drama, the
effort for economic betterment, in fact every individual and social
opposition to the existing disorder of things, is illumined by the
spiritual light of Anarchism. It is the philosophy of the sovereignty
of the individual. It is the theory of social harmony. It is the
great, surging, living truth that is reconstructing the world, and
that will usher in the Dawn.

