
\author{William James}
\authdate{1842--1910}
\textdate{1907}
\chapter{What Pragmatism Means}
\source{james1908.2}

\page{43}\noindent Some years ago, being with a camping party in the
mountains, I returned from a solitary ramble to find everyone engaged
in a ferocious metaphysical dispute. The \textit{corpus} of the
dispute was a squir\-rel---a live squirrel supposed to be clinging to
one side of a tree-trunk; while over against the tree's opposite side
a human being was imagined to stand. This human witness tries to get
sight of the squirrel by moving rapidly round the tree, but no matter
how fast he goes, the squirrel moves as fast in the opposite
direction, and always keeps the tree between himself and the man, so
that never a glimpse of him is caught. The resultant metaphysical
problem now is this: \textit{Does the man go round the squirrel or
not?} He goes round the tree, sure enough, and the squirrel is on the
tree; but does he go round the squirrel? In the unlimited leisure of
the wilderness, discussion had been worn threadbare. Everyone had
taken sides, and was obstinate; and the numbers on \page{44} both
sides were even. Each side, when I appeared therefore appealed to me
to make it a majority. Mindful of the scholastic adage that whenever
you meet a contradiction you must make a distinction, I immediately
sought and found one, as follows: ``Which party is right,'' I said,
``depends on what you \textit{practically mean} by `going round' the
squirrel. If you mean passing from the north of him to the east, then
to the south, then to the west, and then to the north of him again,
obviously the man does go round him, for he occupies these successive
positions. But if on the contrary you mean being first in front of
him, then on the right of him, then behind him, then on his left, and
finally in front again, it is quite as obvious that the man fails to
go round him, for by the compensating movements the squirrel makes, he
keeps his belly turned towards the man all the time, and his back
turned away. Make the distinction, and there is no occasion for any
farther dispute. You are both right and both wrong according as you
conceive the verb `to go round' in one practical fashion or the
other.''

\page{45}Although one or two of the hotter disputants called my speech
a shuffling evasion, saying they wanted no quibbling or scholastic
hair-splitting, but meant just plain honest English `round,' the
majority seemed to think that the distinction had assuaged the
dispute.

I tell this trivial anecdote because it is a peculiarly simple example
of what I wish now to speak of as \textit{the pragmatic method}. The
pragmatic method is primarily a method of settling metaphysical
disputes that otherwise might be interminable. Is the world one or
many?---fated or free?---material or spiritual?---here are notions
either of which may or may not hold good of the world; and disputes
over such notions are unending. The pragmatic method in such cases is
to try to interpret each notion by tracing its respective practical
consequences. What difference would it practically make to anyone if
this notion rather than that notion were true? If no practical
difference whatever can be traced, then the alternatives mean
practically the same thing, and all dispute is idle. Whenever a
dispute is serious, we ought to be \page{46} able to show some
practical difference that must follow from one side or the other's
being right.

A glance at the history of the idea will show you still better what
pragmatism means. The term is derived from the same Greek word
\grk{πράγμα}, meaning action, from which our words `practice' and
`practical' come. It was first introduced into philosophy by Mr.
Charles Peirce in 1878. In an article entitled `How to Make Our Ideas
Clear,' in the `Popular Science Monthly' for January of that
year\footnote{Translated in the \textit{Revue Philosophique} for
January, 1879 (vol. vii).} Mr. Peirce, after pointing out that our
beliefs are really rules for action, said that, to develop a thought's
meaning, we need only determine what conduct it is fitted to produce:
that conduct is for us its sole significance. And the tangible fact at
the root of all our thought-distinctions, however subtle, is that
there is no one of them so fine as to consist in anything but a
possible difference of practice. To attain perfect clearness in our
thoughts of an object, then, we need only consider what conceivable
\page{47} effects of a practical kind the object may in\-volve---what
sensations we are to expect from it, and what reactions we must
prepare. Our conception of these effects, whether immediate or remote,
is then for us the whole of our conception of the object, so far as
that conception has positive significance at all.

This is the principle of Peirce, the principle of pragmatism. It lay
entirely unnoticed by an yone for twenty years, until I, in an address
before Professor Howison's philosophical union at the university of
California, brought it forward again and made a special application of
it to religion. By that date (1898) the times seemed ripe for its
reception. The word `pragmatism' spread, and at present it fairly
spots the pages of the philosophic journals. On all hands we find the
`pragmatic movement' spoken of, sometimes with respect, sometimes with
contumely, seldom with clear understanding. It is evident that the
term applies itself conveniently to a number of tendencies that
hitherto have lacked a collective name, and that it has `come to
stay.'

\page{48}To take in the importance of Peirce's principle, one must get
accustomed to applying it to concrete cases. I found a few years ago
that Ostwald, the illustrious Leipzig chemist, had been making
perfectly distinct use of the principle of pragmatism in his lectures
on the philosophy of science, tho he had not called it by that name.

``All realities influence our practice,'' he wrote me, ``and that
influence is their meaning for us. I am accustomed to put questions to
my classes in this way: In what respects would the world be different
if this alternative or that were true? If I can find nothing that
would become different, then the alternative has no sense.''

That is, the rival views mean practically the same thing, and meaning,
other than practical, there is for us none. Ostwald in a published
lecture gives this example of what he means. Chemists have long
wrangled over the inner constitution of certain bodies called
`tautomerous.' Their properties seemed equally consistent with the
notion that an instable hydrogen \page{49} atom oscillates inside of
them, or that they are instable mixtures of two bodies. Controversy
raged, but never was decided. ``It would never have begun,'' says
Ostwald, ``if the combatants had asked themselves what particular
experimental fact could have been made different by one or the other
view being correct. For it would then have appeared that no difference
of fact could possibly ensue; and the quarrel was as unreal as if,
theorizing in primitive times about the raising of dough by yeast, one
party should have invoked a `brownie,' while another insisted on an
`elf' as the true cause of the phenomenon.''\footnote{`Theorie und
Praxis,' \textit{Zeitsch. des Oesterreichischen Ingenieur u.
Architecten-Vereines}, 1905, Nr. 4 u. 6. I find a still more radical
pragmatism than Ostwald's in an address by Professor W. S. Franklin:
``I think that the sickliest notion of physics, even if a student gets
it, is that it is `the science of masses, molecules and the ether.'
And I think that the healthiest notion, even if a student does not
wholly get it, is that physics is the science of the ways of taking
hold of bodies and pushing them!'' (\textit{Science}, January 2,
1903.)}

It is astonishing to see how many philosophical disputes collapse into
insignificance the moment you subject them to this simple test of
tracing a concrete consequence. There can \textit{be} \page{50} no
difference anywhere that doesn't \textit{make} a difference
else\-where---no difference in abstract truth that doesn't express
itself in a difference in concrete fact and in conduct consequent upon
that fact, imposed on somebody, somehow, somewhere, and somewhen. The
whole function of philosophy ought to be to find out what definite
difference it will make to you and me, at definite instants of our
life, if this world-formula or that world-formula be the true one.

There is absolutely nothing new in the pragmatic method. Socrates was
an adept at it. Aristotle used it methodically. Locke, Berkeley, and
Hume made momentous contributions to truth by its means. Shadworth
Hodgson keeps insisting that realities are only what they are
`known as.' But these forerunners of pragmatism used it in fragments:
they were preluders only. Not until in our time has it generalized
itself, become conscious of a universal mission, pretended to a
conquering destiny. I believe in that destiny, and I hope I may end by
inspiring you with my belief.

\page{51}Pragmatism represents a perfectly familiar attitude in
philosophy, the empiricist attitude, but it represents it, as it seems
to me, both in a more radical and in a less objectionable form than it
has ever yet assumed. A pragmatist turns his back resolutely and
once for all upon a lot of inveterate habits dear to professional
philosophers. He turns away from abstraction and insufficiency, from
verbal solutions, from bad \textit{a priori} reasons, from fixed
principles, closed systems, and pretended absolutes and origins. He
turns towards concreteness and adequacy, towards facts, towards
action, and towards power. That means the empiricist temper regnant
and the rationalist temper sincerely given up. It means the open air
and possibilities of nature, as against dogma, artificiality, and the
pretence of finality in truth.

At the same time it does not stand for any special results. It is a
method only. But the general triumph of that method would mean an
enormous change in what I called in my last lecture the `temperament'
of philosophy. \page{52} Teachers of the ultra-rationalistic type
would be frozen out, much as the courtier type is frozen out in
republics, as the ultramontane type of priest is frozen out in
protestant lands. Science and metaphysics would come much nearer
together, would in fact work absolutely hand in hand.

Metaphysics has usually followed a very primitive kind of quest. You
know how men have always hankered after unlawful magic, and you know
what a great part in magic \textit{words} have always played. If you
have his name, or the formula of incantation that binds him, you can
control the spirit, genie, afrite, or whatever the power may be.
Solomon knew the names of all the spirits, and having their names,
he held them subject to his will. So the universe has always appeared
to the natural mind as a kind of enigma, of which the key must be
sought in the shape of some illuminating or power-bringing word or
name. That word names the universe's \textit{principle}, and to
possess it is after a fashion to possess the universe itself. `God,'
`Matter,' `Reason,' `the Absolute,' `Energy,' are so \page{53} many
solving names. You can rest when you have them. You are at the end of
your metaphysical quest.

But if you follow the pragmatic method, you cannot look on any such
word as closing your quest. You must bring out of each word its
practical cash-value, set it at work within the stream of your
experience. It appears less as a solution, then, than as a program for
more work, and more particularly as an indication of the ways in which
existing realities may be \textit{changed}.

\textit{Theories thus become instruments, not answers to enigmas, in
which we can rest}. We don't lie back upon them, we move forward, and,
on occasion, make nature over again by their aid. Pragmatism
unstiffens all our theories, limbers them up and sets each one at
work. Being nothing essentially new, it harmonizes with many ancient
philosophic tendencies. It agrees with nominalism for instance, in
always appealing to particulars; with utilitarianism in emphasizing
practical aspects; with positivism in its disdain for verbal \page{54}
solutions, useless questions and metaphysical abstractions.

All these, you see, are \textit{anti-intellectualist} tendencies.
Against rationalism as a pretension and a method pragmatism is fully
armed and militant. But, at the outset, at least, it stands for no
particular results. It has no dogmas, and no doctrines save its
method. As the young Italian pragmatist Papini has well said, it lies
in the midst of our theories, like a corridor in a hotel. Innumerable
chambers open out of it. In one you may find a man writing an
atheistic volume; in the next some one on his knees praying for faith
and strength; in a third a chemist investigating a body's properties.
In a fourth a system of idealistic metaphysics is being excogitated;
in a fifth the impossibility of metaphysics is being shown. But they
all own the corridor, and all must pass through it if they want a
practicable way of getting into or out of their respective rooms.

No particular results then, so far, but only an attitude of
orientation, is what the pragmatic method means. \textit{The attitude
of looking away \page{55} from first things, principles, `categories,'
supposed necessities; and of looking towards last things, fruits,
consequences, facts}.

So much for the pragmatic method! You may say that I have been
praising it rather than explaining it to you, but I shall presently
explain it abundantly enough by showing how it works on some familiar
problems. Meanwhile the word pragmatism has come to be used in a still
wider sense, as meaning also a certain \textit{theory of truth}. I
mean to give a whole lecture to the statement of that theory, after
first paving the way, so I can be very brief now. But brevity is hard
to follow, so I ask for your redoubled attention for a quarter of an
hour. If much remains obscure, I hope to make it clearer in the later
lectures.

One of the most successfully cultivated branches of philosophy in our
time is what is called inductive logic, the study of the conditions
under which our sciences have evolved. Writers on this subject have
begun to show a singular unanimity as to what the laws of nature and
elements of fact mean, when formu-\page{56}lated by mathematicians,
physicists and chemists. When the first mathematical, logical, and
natural uniformities, the first \textit{laws}, were discovered, men
were so carried away by the clearness, beauty and simplification that
resulted, that they believed themselves to have deciphered
authentically the eternal thoughts of the Almighty. His mind also
thundered and reverberated in syllogisms. He also thought in conic
sections, squares and roots and ratios, and geometrized like Euclid.
He made Kepler's laws for the planets to follow; he made velocity
increase proportionally to the time in falling bodies; he made the law
of the sines for light to obey when refracted; he established the
classes, orders, families and genera of plants and animals, and fixed
the distances between them. He thought the archetypes of all things,
and devised their variations; and when we rediscover any one of these
his wondrous institutions, we seize his mind in its very literal
intention.

But as the sciences have developed farther, the notion has gained
ground that most, per-\page{57}haps all, of our laws are only
approximations. The laws themselves, moreover, have grown so numerous
that there is no counting them; and so many rival formulations are
proposed in all the branches of science that investigators have
become accustomed to the notion that no theory is absolutely a
transcript of reality, but that any one of them may from some point of
view be useful. Their great use is to summarize old facts and to lead
to new ones. They are only a man-made language, a conceptual
shorthand, as some one calls them, in which we write our reports of
nature; and languages, as is well known, tolerate much choice of
expression and many dialects.

Thus human arbitrariness has driven divine necessity from scientific
logic. If I mention the names of Sigwart, Mach, Ostwald, Pearson,
Milhaud, Poin\-ca\-r\'e, Duhem, Ruyssen, those of you who are students
will easily identify the tendency I speak of, and will think of
additional names.

Riding now on the front of this wave of scientific logic Messrs.
Schiller and Dewey appear \page{58} with their pragmatistic account of
what truth everywhere signifies. Everywhere, these teachers say,
`truth' in our ideas and beliefs means the same thing that it means in
science. It means, they say, nothing but this, \textit{that ideas
(which themselves are but parts of our experience) become true just in
so far as they help us to get into satisfactory relation with other
parts of our experience}, to summarize them and get about among them
by conceptual short-cuts instead of following the interminable
succession of particular phenomena. Any idea upon which we can ride,
so to speak; any idea that will carry us prosperously from any one
part of our experience to any other part, linking things
satisfactorily, working securely, simplifying, saving labor; is true
for just so much, true in so far forth, true \textit{instrumentally}.
This is the `instrumental' view of truth taught so successfully at
Chicago, the view that truth in our ideas means their power to `work,'
promulgated so brilliantly at Oxford.

Messrs. Dewey, Schiller and their allies, in reaching this general
conception of all truth, \page{59} have only followed the example of
geologists, biologists and philologists. In the establishment of these
other sciences, the successful stroke was always to take some simple
process actually observable in op\-er\-a\-tion---as denudation by
weather, say, or variation from parental type, or change of dialect
by incorporation of new words and pro\-nun\-ci\-a\-tions---and then to
generalize it, making it apply to all times, and produce great results
by summating its effects through the ages.

The observable process which Schiller and Dewey particularly singled
out for generalization is the familiar one by which any individual
settles into \textit{new opinions}. The process here is always the
same. The individual has a stock of old opinions already, but he meets
a new experience that puts them to a strain. Somebody contradicts
them; or in a reflective moment he discovers that they contradict
each other; or he hears of facts with which they are incompatible; or
desires arise in him which they cease to satisfy. The result is an
inward trouble to which his mind till then had been a stranger,
\page{60} and from which he seeks to escape by modifying his previous
mass of opinions. He saves as much of it as he can, for in this matter
of belief we are all extreme conservatives. So he tries to change
first this opinion, and then that (for they resist change very
variously), until at last some new idea comes up which he can graft
upon the ancient stock with a minimum of disturbance of the latter,
some idea that mediates between the stock and the new experience and
runs them into one another most felicitously and expediently.

This new idea is then adopted as the true one. It preserves the older
stock of truths with a minimum of modification, stretching them just
enough to make them admit the novelty, but conceiving that in ways as
familiar as the case leaves possible. An \textit{outr\'ee}
explanation, violating all our preconceptions, would never pass for a
true account of a novelty. We should scratch round industriously till
we found something less excentric. The most violent revolutions in an
individual's beliefs leave most of his old order standing. Time and
space, cause and \page{61} effect, nature and history, and one's own
biography remain untouched. New truth is always a go-between, a
smoother-over of transitions. It marries old opinion to new fact so as
ever to show a minimum of jolt, a maximum of continuity. We hold a
theory true just in proportion to its success in solving this `problem
of maxima and minima.' But success in solving this problem is
eminently a matter of approximation. We say this theory solves it on
the whole more satisfactorily than that theory; but that means more
satisfactorily to ourselves, and individuals will emphasize their
points of satisfaction differently. To a certain degree, therefore,
everything here is plastic.

The point I now urge you to observe particularly is the part played by
the older truths. Failure to take account of it is the source of much
of the unjust criticism levelled against pragmatism. Their influence
is absolutely controlling. Loyalty to them is the first
prin\-ci\-ple---in most cases it is the only principle; for by far the
most usual way of handling phenomena so novel that they would make for
a serious re-\page{62}arrangement of our preconceptions is to ignore
them altogether, or to abuse those who bear witness for them.

You doubtless wish examples of this process of truth's growth, and the
only trouble is their superabundance. The simplest case of new truth
is of course the mere numerical addition of new kinds of facts, or of
new single facts of old kinds, to our ex\-pe\-ri\-ence---an addition
that involves no alteration in the old beliefs. Day follows day, and
its contents are simply added. The new contents themselves are not
true, they simply \textit{come} and \textit{are}. Truth is
\textit{what we say about} them, and when we say that they have come,
truth is satisfied by the plain additive formula.

But often the day's contents oblige a rearrangement. If I should now
utter piercing shrieks and act like a maniac on this platform, it
would make many of you revise your ideas as to the probable worth of
my philosophy. `Radium' came the other day as part of the day's
content, and seemed for a moment to contradict our ideas of the whole
order of nature, that \page{63} order having come to be identified
with what is called the conservation of energy. The mere sight of
radium paying heat away indefinitely out of its own pocket seemed to
violate that conservation. What to think? If the radiations from it
were nothing but an escape of unsuspected `potential' energy,
pre-existent inside of the atoms, the principle of conservation would
be saved. The discovery of `helium' as the radiation's outcome, opened
a way to this belief. So Ramsay's view is generally held to be true,
because, although it extends our old ideas of energy, it causes a
minimum of alteration in their nature.

I need not multiply instances. A new opinion counts as `true' just in
proportion as it gratifies the individual's desire to assimilate the
novel in his experience to his beliefs in stock. It must both lean on
old truth and grasp new fact; and its success (as I said a moment ago)
in doing this, is a matter for the individual's appreciation. When old
truth grows, then, by new truth's addition, it is for subjective
reasons. We are in the process and obey the reasons. That \page{64}
new idea is truest which performs most felicitously its function of
satisfying our double urgency. It makes itself true, gets itself
classed as true, by the way it works; grafting itself then upon the
ancient body of truth, which thus grows much as a tree grows by the
activity of a new layer of cambium.

Now Dewey and Schiller proceed to generalize this observation and to
apply it to the most ancient parts of truth. They also once were
plastic. They also were called true for human reasons. They also
mediated between still earlier truths and what in those days were
novel observations. Purely objective truth, truth in whose
establishment the function of giving human satisfaction in marrying
previous parts of experience with newer parts played no r\^ole
whatever, is nowhere to be found. The reasons why we call things true
is the reason why they \textit{are} true, for `to be true'
\textit{means} only to perform this marriage-function.

The trail of the human serpent is thus over everything. Truth
independent; truth that we \textit{find} merely; truth no longer
malleable to hu-\page{65}man need; truth incorrigible, in a word; such
truth exists indeed su\-per\-a\-bun\-dant\-ly---or is supposed to
exist by rationalistically minded thinkers; but then it means only the
dead heart of the living tree, and its being there means only that
truth also has its paleontology and its `prescription,' and may grow
stiff with years of veteran service and petrified in men's regard by
sheer antiquity. But how plastic even the oldest truths nevertheless
really are has been vividly shown in our day by the transformation of
logical and mathematical ideas, a transformation which seems even to
be invading physics. The ancient formulas are reinterpreted as special
expressions of much wider principles, principles that our ancestors
never got a glimpse of in their present shape and formulation.

Mr. Schiller still gives to all this view of truth the name of
`Humanism,' but, for this doctrine too, the name of pragmatism seems
fairly to be in the ascendant, so I will treat it under the name of
pragmatism in these lectures.

Such then would be the scope of prag\-ma\-tism---first, a method; and
second, a genetic theory \page{66} of what is meant by truth. And
these two things must be our future topics.

What I have said of the theory of truth will, I am sure, have appeared
obscure and unsatisfactory to most of you by reason of us brevity. I
shall make amends for that hereafter. In a lecture on `common sense' I
shall try to show what I mean by truths grown petrified by antiquity.
In another lecture I shall expatiate on the idea that our thoughts
become true in proportion as they successfully exert their go-between
function. In a third I shall show how hard it is to discriminate
subjective from objective factors in Truth's development. You may not
follow me wholly in these lectures; and if you do, you may not wholly
agree with me. But you will, I know, regard me at least as serious,
and treat my effort with respectful consideration.

You will probably be surprised to learn, then, that Messrs. Schiller's
and Dewey's theories have suffered a hailstorm of contempt and
ridicule. All rationalism has risen against them. In influential
quarters Mr. Schiller, in partic-\page{67}ular, has been treated like
an impudent schoolboy who deserves a spanking. I should not mention
this, but for the fact that it throws so much sidelight upon that
rationalistic temper to which I have opposed the temper of pragmatism.
Pragmatism is uncomfortable away from facts. Rationalism is
comfortable only in the presence of abstractions. This pragmatist talk
about truths in the plural, about their utility and satisfactoriness,
about the success with which they `work,' etc., suggests to the
typical intellectualist mind a sort of coarse lame second-rate
makeshift article of truth. Such truths are not real truth. Such tests
are merely subjective. As against this, objective truth must be
something non-utilitarian, haughty, refined, remote, august, exalted.
It must be an absolute correspondence of our thoughts with an equally
absolute reality. It must be what we \textit{ought} to think
unconditionally. The conditioned ways in which we \textit{do} think
are so much irrelevance and matter for psychology. Down with
psychology, up with logic, in all this question!

\page{68}See the exquisite contrast of the types of mind! The
pragmatist clings to facts and concreteness, observes truth at its
work in particular cases, and generalizes. Truth, for him, becomes a
class-name for all sorts of definite working-values in experience. For
the rationalist it remains a pure abstraction, to the bare name of
which we must defer. When the pragmatist undertakes to show in detail
just \textit{why} we must defer, the rationalist is unable to
recognize the concretes from which his own abstraction is taken. He
accuses us of \textit{denying} truth; whereas we have only sought to
trace exactly why people follow it and always ought to follow it. Your
typical ultra-abstractionist fairly shudders at concreteness: other
things equal, he positively prefers the pale and spectral. If the two
universes were offered, he would always choose the skinny outline
rather than the rich thicket of reality. It is so much purer, clearer,
nobler.

I hope that as these lectures go on, the concreteness and closeness to
facts of the pragmatism which they advocate may be what approves
\page{69} itself to you as its most satisfactory peculiarity. It only
follows here the example of the sister-sciences, interpreting the
unobserved by the observed. It brings old and new harmoniously
together. It converts the absolutely empty notion of a static
relation of `correspondence' (what that may mean we must ask later)
between our minds and reality, into that of a rich and active commerce
(that anyone may follow in detail and understand) between particular
thoughts of ours, and the great universe of other experiences in
which they play their parts and have their uses.

But enough of this at present? The justification of what I say must be
postponed. I wish now to add a word in further explanation of the
claim I made at our last meeting, that pragmatism may be a happy
harmonizer of empiricist ways of thinking with the more religious
demands of human beings.

\vspace{1\baselineskip}

Men who are strongly of the fact-loving temperament, you may remember
me to have said, are liable to be kept at a distance by the small
\page{70} sympathy with facts which that philosophy from the
present-day fashion of idealism offers them. It is far too
intellectualistic. Old fashioned theism was bad enough, with its
notion of God as an exalted monarch, made up of a lot of
unintelligible or preposterous `attributes'; but, so long as it held
strongly by the argument from design, it kept some touch with concrete
realities. Since, however, darwinism has once for all displaced design
from the minds of the `scientific,' theism has lost that foothold; and
some kind of an immanent or pantheistic deity working \textit{in}
things rather than above them is, if any, the kind recommended to our
contemporary imagination. Aspirants to a philosophic religion turn, as
a rule, more hopefully nowadays towards idealistic pantheism than
towards the older dualistic theism, in spite of the fact that the
latter still counts able defenders.

But, as I said in my first lecture, the brand of pantheism offered is
hard for them to assimilate if they are lovers of facts, or
empirically minded. It is the absolutistic brand, spurning the dust
and reared upon pure logic. It keeps no con-\page{71}nexion whatever
with concreteness. Affirming the Absolute Mind, which is its
substitute for God, to be the rational presupposition of all
particulars of fact, whatever they may be, it remains supremely
indifferent to what the particular facts in our world actually are. Be
they what they may, the Absolute will father them. Like the sick lion
in Esop's fable, all footprints lead into his den, but \textit{nulla
vestigia retrorsum}. You cannot redescend into the world of
particulars by the Absolute's aid, or deduce any necessary
consequences of detail important for your life from your idea of his
nature. He gives you indeed the assurance that all is well with
\textit{Him}, and for his eternal way of thinking; but thereupon he
leaves you to be finitely saved by your own temporal devices.

Far be it from me to deny the majesty of this conception, or its
capacity to yield religious comfort to a most respectable class of
minds. But from the human point of view, no one can pretend that it
doesn't suffer from the faults of remoteness and abstractness. It is
eminently a product of what I have ventured to call the \page{72}
rationalistic temper. It disdains empiricism's needs. It substitutes a
pallid outline for the real world's richness. It is dapper, it is
noble in the bad sense, in the sense in which to be noble is to be
inapt for humble service. In this real world of sweat and dirt, it
seems to me that when a view of things is `noble,' that ought to count
as a presumption against its truth, and as a philosophic
disqualification. The prince of darkness may be a gentleman, as we are
told he is, but whatever the God of earth and heaven is, he can surely
be no gentleman. His menial services are needed in the dust of our
human trials, even more than his dignity is needed in the empyrean.

Now pragmatism, devoted tho she be to facts, has no such materialistic
bias as ordinary empiricism labors under. Moreover, she has no
objection whatever to the realizing of abstractions, so long as you
get about among particulars with their aid and they actually carry you
somewhere. Interested in no conclusions but those which our minds and
our experiences work out together, she has no \textit{a priori}
preju-\page{73}dices against theology. \textit{If theological ideas
prove to have a value for concret life, they will be true, for
pragmatism, in the sense of being good for so much. For how much more
they are true, will depend entirely on their relations to the other
truths that also have to be acknowledged}.

What I said just now about the Absolute, of transcendental idealism,
is a case in point. First, I called it majestic and said it yielded
religious comfort to a class of minds, and then I accused it of
remoteness and sterility. But so far as it affords such comfort, it
surely is not sterile; it has that amount of value; it performs a
concrete function. As a good pragmatist, I myself ought to call the
Absolute true `in so far forth,' then; and I unhesitatingly now do
so.

But what does \textit{true in so far forth} mean in this case? To
answer, we need only apply the pragmatic method. What do believers in
the Absolute mean by saying that their belief affords them comfort?
They mean that since, in the Absolute finite evil is `overruled'
already, we may, therefore, whenever we wish, treat the \page{74}
temporal as if it were potentially the eternal, be sure that we can
trust its outcome, and, without sin, dismiss our fear and drop the
worry of our finite responsibility. In short, they mean that we have a
right ever and anon to take a moral holiday, to let the world wag in
its own way, feeling that its issues are in better hands than ours and
are none of our business.

The universe is a system of which the individual members may relax
their anxieties occasionally, in which the don't-care mood is also
right for men, and moral holidays in or\-der,---that, if I mistake
not, is part, at least, of what the Absolute is `known-as,' that is
the great difference in our particular experiences which his being
true makes for us, that is part of his cash-value when he is
pragmatically interpreted. Farther than that the ordinary lay-reader
in philosophy who thinks favorably of absolute idealism does not
venture to sharpen his conceptions. He can use the Absolute for so
much, and so much is very precious. He is pained at hearing you speak
incredulously of the Absolute, therefore, and disregards your
criticisms \page{75} because they deal with aspects of the
conception that he fails to follow.

If the Absolute means this, and means no more than this, who can
possibly deny the truth of it? To deny it would be to insist that men
should never relax, and that holidays are never in order.

I am well aware how odd it must seem to some of you to hear me say
that an idea is `true' so long as to believe it is profitable to our
lives. That it is \textit{good}, for as much as it profits, you will
gladly admit. If what we do by its aid is good, you will allow the
idea itself to be good in so far forth, for we are the better for
possessing it. But is it not a strange misuse of the word `truth,'
you will say, to call ideas also `true' for this reason?

To answer this difficulty fully is impossible at this stage of my
account. You touch here upon the very central point of Messrs.
Schiller's, Dewey's and my own doctrine of truth, which I cannot
discuss with detail until my sixth lecture. Let me now say only this,
that truth is \textit{one species of good}, and not, as is usually
sup-\page{76}posed, a category distinct from good, and co-ordinate
with it. \textit{The true is the name of whatever proves itself to be
good in the way of belief, and good, too, for definite, assignable
reasons}. Surely you must admit this, that if there were \textit{no}
good for life in true ideas, or if the knowledge of them were
positively disadvantageous and false ideas the only useful ones, then
the current notion that truth is divine and precious, and its
pursuit a duty, could never have grown up or become a dogma. In a
world like that, our duty would be to \textit{shun} truth, rather. But
in this world, just as certain foods are not only agreeable to our
taste, but good for our teeth, our stomach, and our tissues; so
certain ideas are not only agreeable to think about, or agreeable as
supporting other ideas that we are fond of, but they are also helpful
in life's practical struggles. If there be any life that it is
really better we should lead, and if there be any idea which, if
believed in, would help us to lead that life, then it would be really
\textit{better for us} to believe in that idea, \textit{unless,
indeed, belief in it incidentally clashed with other greater vital
benefits}.

\page{77}`What would be better for us to believe'! This sounds very
like a definition of truth. It comes very near to saying `what we
\textit{ought} to believe': and in \textit{that} definition none of
you would find any oddity. Ought we ever not to believe what it is
\textit{better for us} to believe? And can we then keep the notion of
what is better for us, and what is true for us, permanently apart?

Pragmatism says no, and I fully agree with her. Probably you also
agree, so far as the abstract statement goes, but with a suspicion
that if we practically did believe everything that made for good in
our own personal lives, we should be found indulging all kinds of
fancies about this world's affairs, and all kinds of sentimental
superstitions about a world hereafter. Your suspicion here is
undoubtedly well founded, and it is evident that something happens
when you pass from the abstract to the concrete that complicates the
situation.

I said just now that what is better for us to believe is true
\textit{unless the belief incidentally clashes with some other vital
benefit}. Now in \page{78} real life what vital benefits is any
particular belief of ours most liable to clash with? What indeed
except the vital benefits yielded by \textit{other beliefs} when these
prove incompatible with the first ones? In other words, the greatest
enemy of any one of our truths may be the rest of our truths. Truths
have once for all this desperate instinct of self-preservation and of
desire to extinguish whatever contradicts them. My belief in the
Absolute, based on the good it does me, must run the gauntlet of all
my other beliefs. Grant that it may be true in giving me a moral
holiday. Nevertheless, as I conceive it,---and let me speak now
confidentially, as it were, and merely in my own private person,---it
clashes with other truths of mine whose benefits I hate to give up on
its account. It happens to be associated with a kind of logic of
which I am the enemy, I find that it entangles me in metaphysical
paradoxes that are inacceptable, etc., etc. But as I have enough
trouble in life already without adding the trouble of carrying these
intellectual inconsistencies, I personally just give up the \page{79}
Absolute. I just \textit{take} my moral holidays; or else as a
professional philosopher, I try to justify them by some other
principle.

If I could restrict my notion of the Absolute to its bare
holiday-giving value, it wouldn't clash with my other truths. But we
cannot easily thus restrict our hypotheses. They carry supernumerary
features, and these it is that clash so. My disbelief in the Absolute
means then disbelief in those other supernumerary features, for I
fully believe in the legitimacy of taking moral holidays.

You see by this what I meant when I called pragmatism a mediator and
reconciler and said, borrowing the word from Papini, that she
`unstiffens' our theories. She has in fact no prejudices whatever, no
obstructive dogmas, no rigid canons of what shall count as proof. She
is completely genial. She will entertain any hypothesis, she will
consider any evidence. It follows that in the religious field she is
at a great advantage both over positivistic empiricism, with its
anti-theological bias, and over religious rationalism, with its
exclusive interest \page{80} in the remote, the noble, the simple, and
the abstract in the way of conception.

In short, she widens the field of search for God. Rationalism sticks
to logic and the empyrean. Empiricism sticks to the external senses.
Pragmatism is willing to take anything, to follow either logic or the
senses and to count the humblest and most personal experiences. She
will count mystical experiences if they have practical consequences.
She will take a God who lives in the very dirt of private fact---if
that should seem a likely place to find him.

Her only test of probable truth is what works best in the way of
leading us, what fits every part of life best and combines with the
collectivity of experience's demands, nothing being omitted. If
theological ideas should do this, if the notion of God, in particular,
should prove to do it, how could pragmatism possibly deny God's
existence? She could see no meaning in treating as `not true' a notion
that was pragmatically so successful. What other kind of truth could
there be, for her, than all this agreement with concrete reality?

\page{81}In my last lecture I shall return again to the relations of
pragmatism with religion. But you see already how democratic she is.
Her manners are as various and flexible, her resources as rich and
endless, and her conclusions as friendly as those of mother nature.

