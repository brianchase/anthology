
\author{John Locke}
\authdate{1632--1704}
\textdate{1689}
\addon[Book 2, Chapter 27, excerpt]{An Essay Concerning Human Understanding}
%\chapter[An Essay Concerning Human Understanding, excerpt]{An Essay
%Concerning Human Understanding}
\chapter[Of Identity and Diversity, excerpt]{Of Identity and Diversity}
%\source[bk. 2, chap. 27.9--27.29]{locke1894.1}
%\source[bk. 2, chap. 27, secs. 9--29]{locke1894.1}
\source{locke1894.1}

\page{445}9. An animal is a living organized body; and consequently
the same animal, as we have observed, is the same continued
\textit{life} communicated to different particles of matter, as they
happen successively to be united to that organized living body\ldots

\page{448}This being premised, to find wherein personal
identity consists, we must consider what \textit{person} stands
for;---which, I think, is a thinking intelligent being, that has
reason and reflection, and can consider itself as itself, the same
thinking thing, in different times and places; which it does only by
that consciousness which is inseparable from thinking, and, \page{449}
as it seems to me, essential to it: it being impossible for any one
to perceive without \textit{perceiving} that he does perceive. When we
see, hear, smell, taste, feel, meditate, or will anything, we know
that we do so. Thus it is always as to our present sensations and
perceptions: and by this every one is to himself that which he calls
\textit{self}:---it not being considered, in this case, whether the
same self be continued in the same or divers substances. For, since
consciousness always accompanies thinking, and it is that which
makes every one to be what he calls self, and thereby distinguishes
himself from all other thinking things, in this alone consists
personal identity, i.e. the sameness of a rational being: and as far
as this consciousness can be extended backwards to any past action or
thought, so far reaches the identity of that person; it is the same
self now it was then; and it is by the same self with this present one
that now reflects on it, that that action was done.

10. But it is further inquired, whether it be the same identical
substance. This few would think they had reason to doubt of, if these
perceptions, with their consciousness, always remained present in the
mind, whereby the same thinking thing would be always consciously
present, and, as would be thought, evidently the same to itself. But
that which seems to make the difficulty is this, that this
consciousness being interrupted always by forgetfulness, there being
no moment of our lives wherein we have the whole train of all our past
actions before our eyes in one view, but even the best memories losing
the sight of one part whilst they are viewing another; and we
sometimes, and that the greatest part of our lives, not reflecting on
our past selves, being intent on our present thoughts, and in sound
sleep having no thoughts at all, or at least none with that
consciousness which remarks our waking thoughts,---I say, in all these
cases, our consciousness being interrupted, and we losing the sight of
our past selves, doubts are raised whether we are the same thinking
thing, i.e. the same \textit{substance} or no. Which, however
reasonable or unreasonable, concerns not \textit{personal} identity at
all. The question being what makes the same person; and not whether it
be the same identical substance, which always thinks in the same
person, which, in this case, matters not at all: different substances,
by the same consciousness (where they do partake in it) being united
into one person, as well as different bodies by the same life are
united into one animal, whose identity is preserved in that change of
substances by the unity of one continued life. For, it being the
\page{451} same consciousness that makes a man be himself to himself,
personal identity depends on that only, whether it be annexed solely
to one individual substance, or can be continued in a succession of
several substances. For as far as any intelligent being \textit{can}
repeat the idea of any past action with the same consciousness it had
of it at first, and with the same consciousness it has of any present
action; so far it is the same personal self. For it is by the
consciousness it has of its present thoughts and actions, that it is
\textit{self to itself} now, and so will be the same self, as far as
the same consciousness can extend to actions past or to come; and
would be by distance of time, or change of substance, no more two
persons, than a man be two men by wearing other clothes to-day than he
did yesterday, with a long or a short sleep between: the same
consciousness uniting those distant \page{452} actions into the same
person, whatever substances contributed to their production.

11. That this is so, we have some kind of evidence in our very bodies,
all whose particles, whilst vitally united to this same thinking
conscious self, so that \textit{we feel} when they are touched, and
are affected by, and conscious of good or harm that happens to them,
are a part of ourselves; i.e. of our thinking conscious self. Thus,
the limbs of his body are to every one a part of himself; he
sympathizes and is concerned for them. Cut off a hand, and thereby
separate it from that consciousness he had of its heat, cold, and
other affections, and it is then no longer a part of that which is
himself, any more than the remotest part of matter. Thus, we see the
\textit{substance} whereof personal self consisted at one time may be
varied at another, without the change of personal identity; there
being no question about the same person, though the limbs which but
now were a part of it, be cut off.

12. But the question is, Whether if the same substance which thinks be
changed, it can be the same person; or, remaining the same, it can be
different persons?

And to this I answer: First, This can be no question at all to those
who place thought in a purely material animal constitution, void of an
immaterial substance. For, whether \page{453} their supposition be
true or no, it is plain they conceive personal identity preserved in
something else than identity of substance; as animal identity is
preserved in identity of life, and not of substance. And therefore
those who place thinking in an immaterial substance only, before they
can come to deal with these men, must show why personal identity
cannot be preserved in the change of immaterial substances, or
variety of particular immaterial substances, as well as animal
identity is preserved in the change of material substances, or variety
of particular bodies: unless they will say, it is one immaterial
spirit that makes the same life in brutes, as it is one immaterial
spirit that makes the same person in men; which the Cartesians at
least will not admit, for fear of making brutes thinking things too.

13. But next, as to the first part of the question, Whether, if the
same thinking substance (supposing immaterial substances only to
think) be changed, it can be the same person? I answer, that cannot be
resolved but by those who know there can what kind of substances they
are that do think; and whether the consciousness of past actions can
be transferred from one thinking substance to another. I grant were
the same consciousness the same individual action it could not: but it
being a present representation of a past action, why it may not be
possible, that that may be represented to the mind to have been which
really never was, will remain to be shown. And therefore how far the
consciousness of past actions is annexed to any individual agent, so
that another \page{454} cannot possibly have it, will be hard for us
to determine, till we know what kind of action it is that cannot be
done without a reflex act of perception accompanying it, and how
performed by thinking substances, who cannot think without being
conscious of it. But that which we call the same consciousness, not
being the same individual act, why one intellectual substance may not
have represented to it, as done by itself, what \textit{it} never did,
and was perhaps done by some other a\-gent---why, I say, such a
representation may not possibly be without reality of matter of fact,
as well as several representations in dreams are, which yet whilst
dreaming we take for true---will be difficult to conclude from the
nature of things. And that it never is so, will by us, till we have
clearer views of the nature of thinking substances, be best resolved
into the goodness of God; who, as far as the happiness or misery of
any of his sensible creatures is concerned in it, will not, by a fatal
error of theirs, transfer from one to another that consciousness which
draws reward or punishment with it. How far this may be an argument
against those who would place thinking in a system of fleeting animal
spirits, I leave to be considered. But yet, to return to the question
before us, it must be allowed, that, if the same consciousness (which,
as has been shown, is quite a different thing from the same numerical
figure or motion in body) can be transferred from one thinking
substance to another, it will be possible that two thinking substances
may make but one person. For the same consciousness being preserved,
whether in the same or different substances, the personal identity is
preserved.

\page{455}14. As to the second part of the question, Whether the same
immaterial substance remaining, there may be two distinct persons;
which question seems to me to be built on this,---Whether the same
immaterial being, being conscious of the action of its past duration,
may be wholly stripped of all the consciousness of its past existence,
and lose it beyond the power of ever retrieving it again: and so as it
were beginning a new account from a new period, have a consciousness
that \textit{cannot} reach beyond this new state. All those who hold
pre-existence are evidently of this mind; since they allow the soul to
have no remaining consciousness of what it did in that pre-existent
state, either wholly separate from body, or informing any other body;
and if they should not, it is plain experience would be against them.
So that personal identity, reaching no further than consciousness
reaches, a pre-existent spirit not having continued so many ages in a
state of silence, must needs make different persons. Suppose a
Christian Platonist or a Pythagorean should, upon God's having ended
all his works of creation the seventh day, think his soul hath existed
ever since; and should imagine it has revolved in several human
bodies; as I once met with one, who was persuaded his had been the
\textit{soul} of Socrates (how reasonably I will not dispute; this I
know, that in the post he filled, which was no inconsiderable one, he
passed for a very rational man, and the press has shown that he wanted
not parts or learning;)---would any one say, that he, being not
conscious of any of Socrates's actions or thoughts, could be the same
\textit{person} with Socrates? Let any one reflect upon himself, and
conclude that he has in himself an immaterial spirit, which is that
which thinks in him, and, in the constant \page{456} change of his
body keeps him the same: and is that which he calls \textit{himself}:
let him also suppose it to be the same soul that was in Nestor or
Thersites, at the siege of Troy, (for souls being, as far as we know
anything of them, in their nature indifferent to any parcel of matter,
the supposition has no apparent absurdity in it,) which it may have
been, as well as it is now the soul of any other man: but he now
having no consciousness of any of the actions either of Nestor or
Thersites, does or can he conceive himself the same person with either
of them? Can he be concerned in either of their actions? attribute
them to himself, or think them his own, more than the actions of any
other men that ever existed? So that this consciousness, not reaching
to any of the actions of either of those men, he is no more one
\textit{self} with either of them than if the soul or immaterial
spirit that now informs him had been created, and began to exist, when
it began to inform his present body; though it were never so true,
that the same \textit{spirit} that informed Nestor's or Thersites'
body were numerically the same that now informs his. For this would no
more make him the same person with Nestor, than if some of the
particles of matter that were once a part of Nestor were now a part of
this man; the same immaterial substance, without the same
consciousness, no more making the same person, by being united to any
body, than the same particle of matter, without consciousness, united
to any body, makes the same person. But let him once find himself
conscious of any of the actions of Nestor, he then finds himself the
same person with Nestor.

15. And thus may we be able, without any difficulty, to conceive the
same person at the resurrection, though in a \page{457} body not
exactly in make or parts the same which he had here,---the same
consciousness going along with the soul that inhabits it. But yet the
soul alone, in the change of bodies, would scarce to any one but to
him that makes the soul the man, be enough to make the same man. For
should the soul of a prince, carrying with it the consciousness of the
prince's past life, enter and inform the body of a cobbler, as soon as
deserted by his own soul, every one sees he would be the same
\textit{person} with the prince, accountable only for the prince's
actions: but who would say it was the same \textit{man}? The body too
goes to the making the man, and would, I guess, to everybody determine
the man in this case, wherein the soul, with all its princely thoughts
about it, would not make another man: but he would be the same cobbler
to every one besides himself. I know that, in the ordinary way of
speaking, the same person, and the same man, stand for one and the
same thing. And indeed every one will always have a liberty to speak
as he pleases, and to apply what articulate sounds to what ideas he
thinks fit, and change them as often as he pleases. But yet, when we
will inquire what makes the same \textit{spirit}, \textit{man}, or
\textit{person}, we must fix the ideas of spirit, man, or person in
our minds; and having resolved with ourselves what we mean by them, it
will not be hard to determine, in either of them, or the like, when it
is the same, and when not.

\page{458}16. But though the same immaterial substance or soul does
not alone, wherever it be, and in whatsoever state, make the same
\textit{man}; yet it is plain, consciousness, as far as ever it can be
ex\-tend\-ed---should it be to ages past---u\-nites existences and
actions very remote in time into the same \textit{person}, as well as
it does the existences and actions of the immediately preceding
moment: so that whatever has the consciousness of present and past
actions, is the same person to whom they both belong. Had I the same
consciousness that I saw the ark and Noah's flood, as that I saw an
overflowing of the Thames last winter, or as that I write now, I could
no more doubt that I who write this now, that saw the Thames
overflowed last winter, and that viewed the flood at the general
deluge, was the same \textit{self},---place that self in what
\textit{substance} you please---than that I who write this am the same
\textit{myself} now whilst I write (whether I consist of all the same
substance, material or immaterial, or no) that I was yesterday. For as
to this point of being the same self, it matters not whether this
present self be made up of the same or other sub\-stanc\-es---I being
as much concerned, and as justly accountable for any action that was
done a thousand years since, appropriated to me now by this
self-consciousness, as I am for what I did the last moment.

17. \textit{Self} is that conscious thinking thing,---whatever
sub-\page{459}stance made up of, (whether spiritual or material,
simple or compounded, it matters not)---which is sensible or conscious
of pleasure and pain, capable of happiness or misery, and so is
concerned for itself, as far as that consciousness extends. Thus every
one finds that, whilst comprehended under that consciousness, the
little finger is as much a part of himself as what is most so. Upon
separation of this little finger, should this consciousness go along
with the little finger, and leave the rest of the body, it is evident
the little finger would be the person, the same person; and self then
would have nothing to do with the rest of the body. As in this case it
is the consciousness that goes along with the substance, when one part
is separate from another, which makes the same person, and constitutes
this inseparable self: so it is in reference to substances remote in
time. That with which the consciousness of this present thinking thing
\textit{can} join itself, makes the same person, and is one self with
it, and with nothing else; and so attributes to itself, and owns all
the actions of that thing, as its own, as far as that consciousness
reaches, and no further; as every one who reflects will perceive.

18. In this personal identity is founded all the right and justice of
reward and punishment; happiness and misery being that for which every
one is concerned for \textit{himself}, and not mattering what becomes
of any \textit{substance}, not joined to, or affected with that
consciousness. For, as it is evident in the instance I gave but now,
if the consciousness went along with the little finger when it was cut
off, that would be the \page{460} same self which was concerned for
the whole body yesterday, as making part of itself, whose actions then
it cannot but admit as its own now. Though, if the same body should
still live, and immediately from the separation of the little finger
have its own peculiar consciousness, whereof the little finger knew
nothing, it would not at all be concerned for it, as a part of itself,
or could own any of its actions, or have any of them imputed to him.

19. This may show us wherein personal identity consists: not in the
identity of substance, but, as I have said, in the identity of
consciousness, wherein if Socrates and the present mayor of
Queinborough agree, they are the same person: if the same Socrates
waking and sleeping do not partake of the same consciousness, Socrates
waking and sleeping is not the same person. And to punish Socrates
waking for what sleeping Socrates thought, and waking Socrates was
never conscious of, would be no more of right, than to punish one twin
for what his brother-twin did, whereof he knew nothing, because their
outsides were so like, that they could not be distinguished; for such
twins have been seen.

20. But yet possibly it will still be objected,---Sup\-pose I wholly
lose the memory of some parts of my life, beyond a possibility of
retrieving them, so that perhaps I shall never be conscious of them
again; yet am I not the same person that did those actions, had those
thoughts that I once was conscious of, though I have now forgot them?
To which I answer, that we must here take notice what the word
\textit{I} is applied to; which, in this case, is the \textit{man}
only. And the same man being presumed to be the same person, I is
easily here supposed to stand also for the same person. But if it
\page{461} be possible for the same man to have distinct
incommunicable consciousness at different times, it is past doubt the
same man would at different times make different persons; which, we
see, is the sense of mankind in the solemnest declaration of their
opinions, human laws not punishing the mad man for the sober man's
actions, nor the sober man for what the mad man did,---there\-by
making them two persons: which is somewhat explained by our way of
speaking in English when we say such an one is `not himself' or is
`beside himself'; in which phrases it is insinuated, as if those who
now, or at least first used them, thought that self was changed; the
selfsame person was no longer in that man.

21. But yet it is hard to conceive that Socrates, the same individual
man, should be two persons. To help us a little in this, we must
consider what is meant by Socrates, or the same individual
\textit{man}.

First, it must be either the same individual, immaterial, thinking
substance; in short, the same numerical soul, and nothing else.

Secondly, or the same animal, without any regard to an immaterial
soul.

Thirdly, or the same immaterial spirit united to the same animal.

Now, take which of these suppositions you please, it is impossible to
make personal identity to consist in anything but consciousness; or
reach any further than that does.

For, by the first of them, it must be allowed possible that a man born
of different women, and in distant times, may be the same man. A way
of speaking which, whoever admits, must allow it possible for the same
man to be two distinct persons, as any two that have lived in
different ages without the knowledge of one another's thoughts.

By the second and third, Socrates, in this life and after it, cannot
be the same man any way, but by the same consciousness; and so making
human identity to consist in the same \page{462} thing wherein we
place personal identity, there will be no difficulty to allow the same
man to be the same person. But then they who place human identity in
consciousness only and not in something else, must consider how they
will make the infant Socrates the same man with Socrates after the
resurrection. But whatsoever to some men makes a man, and consequently
the same individual man, wherein perhaps few are agreed, personal
identity can by us be placed in nothing but consciousness, (which is
that alone which makes what we call \textit{self},) without involving
us in great absurdities.

% Perhaps stop here

22. But is not a man drunk and sober the same person? why else is he
punished for the fact he commits when \page{463} drunk, though he be
never afterwards conscious of it? Just as much the same person as a
man that walks, and does other things in his sleep, is the same
person, and is answerable for any mischief he shall do in it. Human
laws punish both, with a justice suitable to \textit{their} way of
knowledge;---because, in these cases, they cannot distinguish
certainly what is real, what counterfeit: and so the ignorance in
drunkenness or sleep is not admitted as a plea. [For, though
punishment be annexed to personality, and personality to
consciousness, and the drunkard perhaps be not conscious of what he
did, yet human judicatures justly punish him; because the fact is
proved against him, but want of consciousness cannot be proved for
him.] But in the Great Day, wherein the secrets of all hearts shall be
laid open, it may be reasonable \page{464} to think, no one shall be
made to answer for what he knows nothing of; but shall receive his
doom, his conscience accusing or excusing him.

23. Nothing but consciousness can unite remote existences into the
same person: the identity of substance will not do it; for whatever
substance there is, however framed, without consciousness there is no
person: and a carcass may be a person, as well as any sort of
substance be so, without consciousness.

% NOTE: the source has an em dash in 'night-man' and no dash after
% 'day'

Could we suppose two distinct incommunicable consciousnesses acting
the same body, the one constantly by day, the other by night; and, on
the other side, the same consciousness, acting by intervals, two
distinct bodies: I ask, in the first case, whether the day- and the
night-man would not be two as distinct persons as Socrates and Plato?
And whether, in the second case, there would not be one person in two
distinct bodies, as much as one man is the same in two distinct
clothings? Nor is it at all material to say, that this same, and
this distinct consciousness, in the cases above mentioned, is owing to
the same and distinct immaterial substances, bringing it with them to
those bodies; which, whether true or no, alters not the case: since
it is evident the personal identity would equally be determined by the
consciousness, whether that consciousness were annexed to some
individual immaterial substance or no. For, granting that the thinking
substance in man must be necessarily supposed immaterial, it is
evident that immaterial thinking thing may sometimes part with its
past consciousness, and be restored to it again: as appears in the
forgetfulness men often have of their past actions; and the mind
many times recovers the memory of a past consciousness, which it had
lost for twenty years together. Make these intervals of memory and
forgetfulness to take their turns regularly by day and night, and you
have two persons with the same immaterial spirit, \page{465} as much
as in the former instance two persons with the same body. So that self
is not determined by identity or diversity of substance, which it
cannot be sure of, but only by identity of consciousness.

24. Indeed it may conceive the substance whereof it is now made up to
have existed formerly, united in the same conscious being: but,
consciousness removed, that substance is no more itself, or makes no
more a part of it, than any other other substance; as is evident in
the instance we have already given of a limb cut off, of whose heat,
or cold, or other affections, having no longer any consciousness, it
is no more of a man's self than any other matter of the universe. In
like manner it will be in reference to any immaterial substance, which
is void of that consciousness whereby I am myself to myself: [if there
be any part of its existence which] I cannot upon recollection join
with that present consciousness whereby I am now myself, it is, in
that part of its existence, no more \textit{myself} than any other
immaterial being. For, whatsoever any substance has thought or done,
which I cannot recollect, and by my consciousness make my own thought
and action, it will no more belong to me, whether a part of me thought
or did it, than if it had been thought or done by any other immaterial
being anywhere existing.

25. I agree, the more probable opinion is, that this consciousness is
annexed to, and the affection of, one individual immaterial substance.

But let men, according to their diverse hypotheses, resolve of that as
they please. This every intelligent being, sensible of happiness or
misery, must grant---that there is something that \page{466} is
\textit{himself}, that he is concerned for, and would have happy; that
this self has existed in a continued duration more than one instant,
and therefore it is possible may exist, as it has done, months and
years to come, without any certain bounds to be set to its duration;
and may be the same self, by the same consciousness continued on for
the future. And thus, by this consciousness he finds himself to be the
same self which did such and such an action some years since, by which
he comes to be happy or miserable now. In all which account of self,
the same numerical \textit{substance} is not considered as making the
same self; but the same continued \textit{consciousness}, in which
several substances may have been united, and again separated from it,
which, whilst they continued in a vital union with that wherein this
consciousness then resided, made a part of that same self. Thus any
part of our bodies, vitally united to that which is conscious in us,
makes a part of ourselves: but upon separation from the vital union by
which that consciousness is communicated, that which a moment since
was part of ourselves, is now no more so than a part of another man's
self is a part of me: and it is not impossible but in a little time
may become a real part of another person. And so we have the same
numerical substance become a part of two different persons; and the
same person preserved under the change of various substances. Could we
suppose any spirit wholly stripped of all its memory or consciousness
of past actions, as we find our minds always are of a great part of
ours, and sometimes of them all; the union or separation of such a
spiritual substance would make no variation of personal identity, any
more than that of any particle of matter does. Any substance vitally
united to the present thinking being is a part of that very same self
which now is; anything united to it by a consciousness of former
actions, makes also a part of the same self, which is the same both
then and now.

26. \textit{Person}, as I take it, is the name for this self. Wherever
a man finds what he calls himself, there, I think, another \page{467}
may say is the same person. It is a forensic term, appropriating
actions and their merit; and so belongs only to intelligent agents,
capable of a law, and happiness, and misery. This personality extends
itself beyond present existence to what is past, only by
consciousness,---where\-by it becomes concerned and accountable; owns
and imputes to itself past actions, just upon the same ground and for
the same reason as it does the present. All which is founded in a
concern for happiness, the unavoidable concomitant of consciousness;
that which is conscious of pleasure and pain, desiring that that self
that is conscious should be happy. And therefore whatever past actions
it cannot reconcile or \textit{appropriate} to that present self by
consciousness, it can be no more concerned in than if they had never
been done: and to receive pleasure or pain, i.e. reward or punishment,
on the account of any such action, is all one as to be made happy or
miserable \page{468} in its first being, without any demerit at all.
For, supposing a \textit{man} punished now for what he had done in
another life, whereof he could be made to have no consciousness at
all, what difference is there between that punishment and being
\textit{created} miserable? And therefore, conformable to this, the
apostle tells us, that, at the great day, when every one shall
`receive according to his doings, the secrets of all hearts shall be
laid open.' The sentence shall be justified by the consciousness all
persons shall have, that \textit{they themselves}, in what bodies
soever they appear, or what substances soever that consciousness
adheres to, are the \textit{same} that committed those actions, and
deserve that punishment for them.

27. I am apt enough to think I have, in treating of this subject, made
some suppositions that will look strange to are some readers, and
possibly they are so in themselves. But yet, I think they are such as
are pardonable, in this ignorance we are in of the nature of that
thinking thing that is in us, and which we look on as
\textit{ourselves}. Did we know what it \page{469} was; or how it was
tied to a certain system of fleeting animal spirits; or whether it
could or could not perform its operations of thinking and memory out
of a body organized as ours is; and whether it has pleased God that no
one such spirit shall ever be united to any but one such body, upon
the right constitution of whose organs its memory should depend; we
might see the absurdity of some of those suppositions I have made. But
taking, as we ordinarily now do (in the dark concerning these
matters,) the soul of a man for an immaterial substance, independent
from matter, and indifferent alike to it all; there can, from the
nature of things, be no absurdity at all to suppose that the same
\textit{soul} may at different times be united to different
\textit{bodies}, and with them make up for that time one \textit{man}:
as well as we suppose a part of a sheep's body yesterday should be a
part of a man's body to-morrow, and in that union make a vital part of
Melib\oe us himself, as well as it did of his ram.

28. To conclude: Whatever substance begins to exist, it must, during
its existence, necessarily be the same: whatever compositions of
substances begin to exist, during the union of those substances, the
concrete must be the same: whatsoever mode begins to exist, during its
existence it is the same: and so if the composition be of distinct
substances and different modes, the same rule holds. Whereby it will
appear, that the difficulty or obscurity that has been about this
matter rather rises from the names ill-used, than from any obscurity
in things themselves. For whatever makes the specific idea to which
the name is applied, if that idea be steadily kept to, the distinction
of anything into the same and divers will easily be conceived, and
there can arise no doubt about it.

29. For, supposing a rational spirit be the idea of a \textit{man},
\page{470} it is easy to know what is the same man, viz. the same
spir\-it---wheth\-er separate or in a bo\-dy---will be the
\textit{same man}. Supposing a rational spirit vitally united to a
body of a certain conformation of parts to make a man; whilst that
rational spirit, with that vital conformation of parts, though
continued in a fleeting successive body, remains, it will be the
\textit{same man}. But if to any one the idea of a man be but the
vital union of parts in a certain shape; as long as that vital union
and shape remain in a concrete, no otherwise the same but by a
continued succession of fleeting particles, it will be the
\textit{same man}. For, whatever be the composition whereof the
complex idea is made, whenever existence makes it one particular thing
under any denomination, \textit{the same existence continued}
preserves it the \textit{same} individual under the same denomination.

