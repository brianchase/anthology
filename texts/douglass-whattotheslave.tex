
\author{Frederick Douglass}
\authdate{ca. 1818--1895}
\textdate{1852}
\chapter{What to the Slave is the Fourth of July?}
\source{douglass1857.30}

\page{441}\begin{center}\textsc{Extract from an Oration, at Rochester,
July 5, 1852.}\end{center}

\noindent Fellow-citizens---Pardon me, allow me to ask, why am I
called upon to speak here to-day? What have I, or those I represent,
to do with your national independence? Are the great principles of
political freedom and of natural justice, embodied in that Declaration
of Independence, extended to us? and am I, therefore, called upon to
bring our humble offering to the national altar, and to confess the
benefits, and express devout gratitude for the blessings, resulting
from your independence to us?

Would to God, both for your sakes and ours, that an affirmative answer
could be truthfully returned to these questions! Then would my task be
light, and my burden easy and delightful. For who is there so cold
that a nation's sympathy could not warm him? Who so obdurate and dead
to the claims of gratitude, that would not thankfully acknowledge such
priceless benefits? Who so stolid and selfish, that would not give his
voice to swell the hallelujahs of a nation's jubilee, when the chains
of servitude had been torn from his limbs? I am not that man. In a
case like that, the dumb might eloquently speak, and the ``lame man
leap as an hart.''

But, such is not the state of the case. I say it with a sad sense of
the disparity between us. I am not included within the pale of this
glorious anniversary! Your high independence only reveals the
immeasurable distance between us. The blessings in which you this
day rejoice, are not enjoyed in common. The rich inheritance of
justice, liberty, prosperity and independence, bequeathed by your
fathers, is shared by you, not by me. The sunlight that brought life
and healing to you, has brought stripes and death to me. This Fourth
of July is \textit{yours}, not \textit{mine}. \textit{You} may
rejoice, \textit{I} must mourn. To drag a man in fetters into the
grand illuminated temple of liberty, and call upon him to join you in
joyous anthems, were inhuman mockery and sacrilegious irony. Do you
mean, cit-\page{442}izens, to mock me, by asking me to speak to-day?
If so, there is a parallel to your conduct. And let me warn you that
it is dangerous to copy the example of a nation whose crimes, towering
up to heaven, were thrown down by the breath of the Almighty, burying
that nation in irrecoverable ruin! I can to-day take up the plaintive
lament of a peeled and woe-smitten people!

``By the rivers of Babylon, there we sat down. Yea! we wept when we
remembered Zion. We hanged our harps upon the willows in the midst
thereof. For there, they that carried us away captive, required of us
a song; and they who wasted us required of us mirth, saying, Sing us
one of the songs of Zion. How can we sing the Lord's song in a strange
land? If I forget thee, O Jerusalem, let my right hand forget her
cunning. If I do not remember thee, let my tongue cleave to the roof
of my mouth.''

Fellow-citizens, above your national, tumultuous joy, I hear the
mournful wail of millions, whose chains, heavy and grievous yesterday,
are to-day rendered more intolerable by the jubilant shouts that reach
them. If I do forget, if I do not faithfully remember those bleeding
children of sorrow this day, ``may my right hand forget her cunning,
and may my tongue cleave to the roof of my mouth!'' To forget them, to
pass lightly over their wrongs, and to chime in with the popular
theme, would be treason most scandalous and shocking, and would make
me a reproach before God and the world. My subject, then
fellow-citizens, is \textsc{American Slavery}. I shall see this day
and its popular characteristics from the slave's point of view.
Standing, there, identified with the American bondman, making his
wrongs mine, I do not hesitate to declare, with all my soul, that the
character and conduct of this nation never looked blacker to me than
on this Fourth of July. Whether we turn to the declarations of the
past, or to the professions of the present, the conduct of the nation
seems equally hideous and revolting. America is false to the past,
false to the present, and solemnly binds herself to be false to the
future. Standing with God and the crushed and bleeding slave on this
occasion, I will, in the name of humanity which is outraged, in the
name of liberty which is fettered, in the name of the constitution and
the bible, which are disregarded and trampled upon, dare to call in
question and to denounce, with all the emphasis I can command,
everything that serves to perpetuate slav\-er\-y---the great sin and
shame of America! ``I will not equivocate; I will not excuse;'' I will
use the severest language I can \page{443} command; and yet not one
word shall escape me that any man, whose judgment is not blinded by
prejudice, or who is not at heart a slaveholder, shall not confess to
be right and just.

But I fancy I hear some one of my audience say, it is just in this
circumstance that you and your brother abolitionists fail to make a
favorable impression on the public mind. Would you argue more, and
denounce less, would you persuade more and rebuke less, your cause
would be much more likely to succeed. But, I submit, where all is
plain there is nothing to be argued. What point in the anti-slavery
creed would you have me argue? On what branch of the subject do the
people of this country need light? Must I undertake to prove that the
slave is a man? That point is conceded already. Nobody doubts it. The
slaveholders themselves acknowledge it in the enactment of laws for
their government. They acknowledge it when they punish disobedience on
the part of the slave. There are seventy-two crimes in the state of
Virginia, which, if committed by a black man, (no matter how ignorant
he be), subject him to the punishment of death; while only two of
these same crimes will subject a white man to the like punishment.
What is this but the acknowledgment that the slave is a moral,
intellectual, and responsible being. The manhood of the slave is
conceded. It is admitted in the fact that southern statute books are
covered with enactments forbidding, under severe fines and penalties,
the teaching of the slave to read or to write. When you can point to
any such laws, in reference to the beasts of the field, then I may
consent to argue the manhood of the slave. When the dogs in your
streets, when the fowls of the air, when the cattle on your hills,
when the fish of the sea, and the reptiles that crawl, shall be unable
to distinguish the slave from a brute, then will I argue with you that
the slave is a man!

% NOTE: text has 'christian's', corrected below

For the present, it is enough to affirm the equal manhood of the negro
race. Is it not astonishing that, while we are plowing, planting, and
reaping, using all kinds of mechanical tools, erecting houses,
constructing bridges, building ships, working in metals of brass,
iron, copper, silver, and gold; that, while we are reading, writing,
and ciphering, acting as clerks, merchants, and secretaries, having
among us lawyers, doctors, ministers, poets, authors, editors,
orators, and teachers; that, while we are engaged in all manner of
enterprises common to other men---digging gold in California,
capturing the whale in the Pacific, feeding sheep and cattle on the
hill-\page{444}side, living, moving, acting, thinking, planning,
living in families as husbands, wives, and children, and, above all,
confessing and worshiping the Christian's God, and looking hopefully
for life and immortality beyond the grave,---we are called upon to
prove that we are men!

Would you have me argue that man is entitled to liberty? that he is
the rightful owner of his own body? You have already declared it. Must
I argue the wrongfulness of slavery? Is that a question for
republicans? Is it to be settled by the rules of logic and
argumentation, as a matter beset with great difficulty, involving a
doubtful application of the principle of justice, hard to be
understood? How should I look to-day in the presence of Americans,
dividing and subdividing a discourse, to show that men have a natural
right to freedom, speaking of it relatively and positively, negatively
and affirmatively? To do so, would be to make myself ridiculous, and
to offer an insult to your understanding. There is not a man beneath
the canopy of heaven that does not know that slavery is wrong
\textit{for him}.

What! am I to argue that it is wrong to make men brutes, to rob them
of their liberty, to work them without wages, to keep them ignorant of
their relations to their fellow-men, to beat them with sticks, to flay
their flesh with the lash, to load their limbs with irons, to hunt
them with dogs, to sell them at auction, to sunder their families, to
knock out their teeth, to burn their flesh, to starve them into
obedience and submission to their masters? Must I argue that a system
thus marked with blood and stained with pollution, is wrong? No; I
will not. I have better employments for my time and strength than such
arguments would imply.

What, then, remains to be argued? Is it that slavery is not divine;
that God did not establish it; that our doctors of divinity are
mistaken? There is blasphemy in the thought. That which is inhuman
cannot be divine. Who can reason on such a proposition? They that can,
may; I cannot. The time for such argument is past.

At a time like this, scorching irony, not convincing argument, is
needed. Oh! had I the ability, and could I reach the nation's ear, I
would to-day pour out a fiery stream of biting ridicule, blasting
reproach, withering sarcasm, and stern rebuke. For it is not light
that is needed, but fire; it is not the gentle shower, but thunder. We
need the storm, the whirlwind, and the earthquake. \page{445} The
feeling of the nation must be quickened; the conscience of the nation
must be roused; the propriety of the nation must be startled; the
hypocrisy of the nation must be exposed; and its crimes against God
and man must be proclaimed and denounced.

What to the American slave is your Fourth of July? I answer, a day
that reveals to him, more than all other days in the year, the gross
injustice and cruelty to which he is the constant victim. To him, your
celebration is a sham; your boasted liberty, an unholy license; your
national greatness, swelling vanity; your sounds of rejoicing are
empty and heartless; your denunciations of tyrants, brass-fronted
impudence; your shouts of liberty and equality, hollow mockery; your
prayers and hymns, your sermons and thanksgivings, with all your
religious parade and solemnity, are to him mere bombast, fraud,
deception, impiety, and hy\-poc\-ri\-sy---a thin veil to cover up
crimes which would disgrace a nation of savages. There is not a nation
on the earth guilty of practices more shocking and bloody, than are
the people of these United States, at this very hour.

Go where you may, search where you will, roam through all the
monarchies and despotisms of the old world, travel through South
America, search out every abuse, and when you have found the last, lay
your facts by the side of the every-day practices of this nation, and
you will say with me, that, for revolting barbarity and shameless
hypocrisy, America reigns without a rival.

