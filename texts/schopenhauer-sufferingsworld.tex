
\author{Arthur Schopenhauer}
\authdate{1788--1860}
\textdate{1851}
\chapter{On the Sufferings of the World}
\source{schopenhauer1913.1}

\page{11}Unless \textit{suffering} is the direct and immediate object
of life, our existence must entirely fail of its aim. It is absurd to
look upon the enormous amount of pain that abounds everywhere in the
world, and originates in needs and necessities inseparable from life
itself, as serving no purpose at all and the result of mere chance.
Each separate misfortune, as it comes, seems, no doubt, to be
something exceptional; but misfortune in general is the rule.

I know of no greater absurdity than that propounded by most systems of
philosophy in declaring evil to be negative in its character. Evil is
just what is positive; it makes its own existence felt. Leibnitz is
particularly concerned to defend this absurdity; and he seeks to
strengthen his position by using a palpable and paltry
sophism.\footnote{\textit{Translator's Note}, cf. \textit{Th\`eod},
\S153.---Leibnitz argued that evil is a negative
qual\-i\-ty---\textit{i.e}., the absence of good; and that its active
and seemingly positive character is an incidental and not an essential
part of its nature. Cold, he said, is only the absence of the power of
heat, and the active power of expansion in freezing water is an
incidental and not an essential part of the nature of cold. The fact
is that the power of expansion in freezing water is really an increase
of repulsion amongst its molecules; and Schopenhauer is quite right in
calling the whole argument a sophism.} It is the good which is
negative; in other words, happiness and satisfaction always imply some
desire fulfilled, some state of pain brought to an end.

This explains the fact that we generally find \page{12} pleasure to be
not nearly so pleasant as we expected, and pain very much more
painful.

The pleasure in this world, it has been said, outweighs the pain; or,
at any rate, there is an even balance between the two. If the reader
wishes to see shortly whether this statement is true, let him compare
the respective feelings of two animals, one of which is engaged in
eating the other.

The best consolation in misfortune or affliction of any kind will be
the thought of other people who are in a still worse plight than
yourself; and this is a form of consolation open to every one. But
what an awful fate this means for mankind as a whole!

We are like lambs in a field, disporting themselves under the eye of
the butcher, who chooses out first one and then another for his prey.
So it is that in our good days we are all unconscious of the evil Fate
may have presently in store for us---sickness, poverty, mutilation,
loss of sight or reason.

No little part of the torment of existence lies in this, that Time is
continually pressing upon us, never letting us take breath, but always
coming after us like a taskmaster with a whip. If at any moment Time
stays his hand, it is only when we are delivered over to the misery of
boredom.

But misfortune has its uses; for, as our bodily frame would burst
asunder if the pressure of the atmosphere was removed, so, if the
lives of men were relieved of all need, hardship and adversity; if
everything they took in hand were successful, they would be so swollen
with arrogance that, though they might not burst, they would present
the spectacle of \page{13} unbridled folly---nay, they would go mad.
And I may say, further, that a certain amount of care or pain or
trouble is necessary for every man at all times. A ship without
ballast is unstable and will not go straight.

Certain it is that \textit{work, worry, labor} and \textit{trouble},
form the lot of almost all men their whole life long. But if all
wishes were fulfilled as soon as they arose, how would men occupy
their lives? what would they do with their time? If the world were a
paradise of luxury and ease, a land flowing with milk and honey, where
every Jack obtained his Jill at once and without any difficulty, men
would either die of boredom or hang themselves; or there would be
wars, massacres, and murders; so that in the end mankind would inflict
more suffering on itself than it has now to accept at the hands of
Nature.

In early youth, as we contemplate our coming life, we are like
children in a theatre before the curtain is raised, sitting there in
high spirits and eagerly waiting for the play to begin. It is a
blessing that we do not know what is really going to happen. Could we
foresee it, there are times when children might seem like innocent
prisoners, condemned, not to death, but to life, and as yet all
unconscious of what their sentence means. Nevertheless every man
desires to reach old age; in other words, a state of life of which it
may be said: ``It is bad to-day, and it will be worse to-morrow; and
so on till the worst of all.''

If you try to imagine, as nearly as you can, what an amount of misery,
pain and suffering of every kind \page{14} the sun shines upon in its
course, you will admit that it would be much better if on the earth as
little as on the moon the sun were able to call forth the phenomena of
life; and if, here as there, the surface were still in a crystalline
state.

Again, you may look upon life as an unprofitable episode, disturbing
the blessed calm of non-existence. And, in any case, even though
things have gone with you tolerably well, the longer you live the more
clearly you will feel that, on the whole, life is \textit{a
disappointment, nay, a cheat}.

If two men who were friends in their youth meet again when they are
old, after being separated for a life-time, the chief feeling they
will have at the sight of each other will be one of complete
disappointment at life as a whole; because their thoughts will be
carried back to that earlier time when life seemed so fair as it lay
spread out before them in the rosy light of dawn, promised so
much---and then performed so little. This feeling will so completely
predominate over every other that they will not even consider it
necessary to give it words; but on either side it will be silently
assumed, and form the ground-work of all they have to talk about.

He who lives to see two or three generations is like a man who sits
some time in the conjurer's booth at a fair, and witnesses the
performance twice or thrice in succession. The tricks were meant to be
seen only once; and when they are no longer a novelty and cease to
deceive their effect is gone.

While no man is much to be envied for his lot, there are countless
numbers whose fate is to be deplored.

\page{15}Life is a task to be done. It is a fine thing to say
\textit{defunctus est}; it means that the man has done his task.

If children were brought into the world by an act of pure reason
alone, would the human race continue to exist? Would not a man rather
have so much sympathy with the coming generation as to spare it the
burden of existence? or at any rate not take it upon himself to impose
that burden upon it in cold blood.

I shall be told, I suppose, that my philosophy is
comfortless---because I speak the truth; and people prefer to be
assured that everything the Lord has made is good. Go to the priests,
then, and leave philosophers in peace! At any rate, do not ask us to
accommodate our doctrines to the lessons you have been taught. That is
what those rascals of sham philosophers will do for you. Ask them for
any doctrine you please, and you will get it. Your University
professors are bound to preach optimism; and it is an easy and
agreeable task to upset their theories.

I have reminded the reader that every state of welfare, every feeling
of satisfaction, is negative in its character; that is to say, it
consists in freedom from pain, which is the positive element of
existence. It follows, therefore, that the happiness of any given life
is to be measured, not by its joys and pleasures, but by the extent to
which it has been free from suffering---from positive evil. If this is
the true standpoint, the lower animals appear to enjoy a happier
destiny than man. Let us examine the matter a little more closely.

However varied the forms that human happiness \page{16} and misery may
take, leading a man to seek the one and shun the other, the material
basis of it all is bodily pleasure or bodily pain. This basis is very
restricted: it is simply health, food, protection from wet and cold,
the satisfaction of the sexual instinct; or else the absence of these
things. Consequently, as far as real physical pleasure is concerned,
the man is not better off than the brute, except in so far as the
higher possibilities of his nervous system make him more sensitive to
every kind of pleasure, but also, it must be remembered, to every kind
of pain. But then compared with the brute, how much stronger are the
passions aroused in him! what an immeasurable difference there is in
the depth and vehemence of his emotions!---and yet, in the one case,
as in the other, all to produce the same result in the end: namely,
health, food, clothing, and so on.

The chief source of all this passion is that thought for what is
absent and future, which, with man, exercises such a powerful
influence upon all he does. It is this that is the real origin of his
cares, his hopes his fears---emotions which affect him much more
deeply than could ever be the case with those present joys and
sufferings to which the brute is confined. In his powers of
reflection, memory and foresight, man possesses, as it were, a machine
for condensing and storing up his pleasures and his sorrows. But the
brute has nothing of the kind; whenever it is in pain, it is as though
it were suffering for the first time, even though the same thing
should have previously happened to it times out of number. It has no
power of summing up its feelings. Hence its careless and \page{17}
placid temper: how much it is to be envied! But in man reflection
comes in, with all the emotions to which it gives rise; and taking up
the same elements of pleasure and pain which are common to him and the
brute, it develops his susceptibility to happiness and misery to such
a degree that, at one moment the man is brought in an instant to a
state of delight that may even prove fatal, at another to the depths
of despair and suicide.

If we carry our analysis a step farther, we shall find that, in order
to increase his pleasures, man has intentionally added to the number
and pressure of his needs, which in their original state were not much
more difficult to satisfy than those of the brute. Hence luxury in all
its forms: delicate food, the use of tobacco and opium, spirituous
liquors, fine clothes and the thousand and one things than he
considers necessary to his existence.

And above and beyond all this, there is a separate and peculiar source
of pleasure, and consequently of pain, which man has established for
himself, also as the result of using his powers of reflection; and
this occupies him out of all proportion to its value, nay, almost more
than all his other interests put together---I mean ambition and the
feeling of honour and shame; in plain words, what he thinks about the
opinion other people have of him. Taking a thousand forms, often very
strange ones, this becomes the goal of almost all the efforts he makes
that are not rooted in physical pleasure or pain. It is true that
besides the sources of pleasure which he has in common with the brute,
man has the pleasures of the mind as well. These \page{18} admit of
many gradations, from the most innocent trifling or the merest talk up
to the highest intellectual achievements; but there is the
accompanying boredom to be set against them on the side of suffering.
Boredom is a form of suffering unknown to brutes, at any rate in their
natural state; it is only the very cleverest of them who show faint
traces of it when they are domesticated; whereas in the case of man it
has become a downright scourge. The crowd of miserable wretches whose
one aim in life is to fill their purses, but never to put anything
into their heads, offers a singular instance of this torment of
boredom. Their wealth becomes a punishment by delivering them up to
misery of having nothing to do; for, to escape it, they will rush
about in all directions, traveling here, there and everywhere. No
sooner do they arrive in a place than they are anxious to know what
amusements it affords; just as though they were beggars asking where
they could receive a dole! Of a truth, need and boredom are the two
poles of human life. Finally, I may mention that as regards the sexual
relation, a man is committed to a peculiar arrangement which drives
him obstinately to choose one person. This feeling grows, now and
then, into a more or less passionate love,\footnote{I have treated
this subject at length in a special chapter of the second volume of my
chief work.} which is the source of little pleasure and much
suffering.

It is, however, a wonderful thing that the mere addition of thought
should serve to raise such a vast and lofty structure of human
happiness and misery; resting, too, on the same narrow basis of joy
and \page{19} sorrow as man holds in common with the brute, and
exposing him to such violent emotions, to so many storms of passion,
so much convulsion of feeling, that what he has suffered stands
written and may be read in the lines on his face. And yet, when all is
told, he has been struggling ultimately for the very same things as
the brute has attained, and with an incomparably smaller expenditure
of passion and pain.

But all this contributes to increase the measures of suffering in
human life out of all proportion to its pleasures; and the pains of
life are made much worse for man by the fact that death is something
very real to him. The brute flies from death instinctively without
really knowing what it is, and therefore without ever contemplating it
in the way natural to a man, who has this prospect always before his
eyes. So that even if only a few brutes die a natural death, and most
of them live only just long enough to transmit their species, and
then, if not earlier, become the prey of some other animal,---whilst
man, on the other hand, manages to make so-called natural death the
rule, to which, however, there are a good many exceptions,---the
advantage is on the side of the brute, for the reason stated above.
But the fact is that man attains the natural term of years just as
seldom as the brute; because the unnatural way in which he lives, and
the strain of work and emotion, lead to a degeneration of the race;
and so his goal is not often reached.

The brute is much more content with mere existence than man; the plant
is wholly so; and man finds satisfaction in it just in proportion as
he is dull \page{20} and obtuse. Accordingly, the life of the brute
carries less of sorrow with it, but also less of joy, when compared
with the life of man; and while this may be traced, on the one side,
to freedom from the torment of \textit{care} and \textit{anxiety}, it
is also due to the fact that \textit{hope}, in any real sense, is
unknown to the brute. It is thus deprived of any share in that which
gives us the most and best of our joys and pleasures, the mental
anticipation of a happy future, and the inspiriting play of phantasy,
both of which we owe to our power of imagination. If the brute is free
from care, it is also, in this sense, without hope; in either case
because its consciousness is limited to the present moment, to what it
can actually see before it. The brute is an embodiment of present
impulses, and hence what elements of fear and hope exist in its
nature---and they do not go very far---arise only in relation to
objects that lie before it and within reach of those impulses: whereas
a man's range of vision embraces the whole of his life, and extends
far into the past and future.

Following upon this, there is one respect in which brutes show real
wisdom when compared with us---I mean their quiet, placid enjoyment of
the present moment. The tranquillity of mind which this seems to give
them often puts us to shame for the many times we allow our thoughts
and our cares to make us restless and discontented. And, in fact,
those pleasures of hope and anticipation which I have been mentioning
are not to be had for nothing. The delight which a man has in hoping
for and looking forward to some special satisfaction is a part of the
\page{21} real pleasure attaching to it enjoyed in advance. This is
afterwards deducted; for the more we look forward to anything the less
satisfaction we find in it when it comes. But the brute's enjoyment is
not anticipated and therefore suffers no deduction; so that the actual
pleasure of the moment comes to it whole and unimpaired. In the same
way, too, evil presses upon the brute only with its own intrinsic
weight; whereas with us the fear of its coming often makes its burden
ten times more grievous.

It is just this characteristic way in which the brute gives itself up
entirely to the present moment that contributes so much to the delight
we take in our domestic pets. They are the present moment personified,
and in some respects they make us feel the value of every hour that is
free from trouble and annoyance, which we, with our thoughts and
preoccupations, mostly disregard. But man, that selfish and heartless
creature, misuses this quality of the brute to be more content than we
are with mere existence, and often works it to such an extent that he
allows the brute absolutely nothing more than mere, bare life. The
bird which was made so that it might rove over half of the world, he
shuts up into the space of a cubic foot, there to die a slow death in
longing and crying for freedom; for in a cage it does not sing for the
pleasure of it. And when I see how man misuses the dog, his best
friend; how he ties up this intelligent animal with a chain, I feel
the deepest sympathy with the brute and burning indignation against
its master.

We shall see later that by taking a very high standpoint it is
possible to justify the sufferings of \page{22} mankind. But this
justification cannot apply to animals, whose sufferings, while in a
great measure brought about by men, are often considerable even apart
from their agency.\footnote{Cf. \textit{Welt als Wille und
Vorstellung}, vol. ii. p. 404.} And so we are forced to ask, Why and
for what purpose does all this torment and agony exist? There is
nothing here to give the will pause; it is not free to deny itself
and so obtain redemption. There is only one consideration that may
serve to explain the sufferings of animals. It is this: that the will
to live, which underlies the whole world of phenomena, must in their
case satisfy its cravings by feeding upon itself. This it does by
forming a gradation of phenomena, every one of which exists at the
expense of another. I have shown, however, that the capacity for
suffering is less in animals than in man. Any further explanation that
may be given of their fate will be in the nature of hypothesis if not
actually mythical in its character; and I may leave the reader to
speculate upon the matter for himself.

\vspace{1\baselineskip}

\textit{Brahma} is said to have produced the world by a kind of fall
or mistake; and in order to atone for his folly he is bound to remain
in it himself until he works out his redemption. As an account of the
origin of things, that is admirable! According to the doctrines of
\textit{Buddhism}, the world came into being as the result of some
inexplicable disturbance in the heavenly calm of Nirvana, that blessed
state obtained by expiation, which had endured so long a time---the
change taking place by a kind of fatality. This \page{23} explanation
must be understood as having at bottom some moral bearing; although it
is illustrated by an exactly parallel theory in the domain of physical
science, which places the origin of the sun in a primitive streak of
mist, formed one knows not how. Subsequently, by a series of moral
errors, the world became gradually worse and worse---true of the
physical orders as well---until it assumed the dismal aspect it wears
to-day. Excellent! The \textit{Greeks} looked upon the world and the
gods as the work of an inscrutable necessity. A passable explanation:
we may be content with it until we can get a better. Again,
\textit{Ormuzd} and \textit{Ahriman} are rival powers, continually at
war. That is not bad. But that a God like Jehovah should have created
this world of misery and woe, out of pure caprice, and because he
enjoyed doing it, and should then have clapped his hands in praise of
his own work, and declared everything to be very good---that will not
do at all! In its explanation of the origin of the world, Judaism is
inferior to any other form of religious doctrine professed by a
civilised nation; and it is quite in keeping with this that it is the
only one which presents no trace whatever of any belief in the
immortality of the soul.\footnote{See \textit{Parerga}, vol. i. pp.
139 \textit{et seq}.}

Even though Leibnitz' contention, that this is the best of all
possible worlds, were correct, that would not justify God in having
created it. For he is the Creator not of the world only, but of
possibility itself; and, therefore, he ought to have so ordered
possibility as that it would admit of something better.

\page{24}There are two things which make it impossible to believe that
this world is the successful work of an all-wise, all-good, and, at
the same time, all-powerful Being; firstly, the misery which abounds
in it everywhere; and secondly, the obvious imperfection of its
highest product, man, who is a burlesque of what he should be. These
things cannot be reconciled with any such belief. On the contrary,
they are just the facts which support what I have been saying; they
are our authority for viewing the world as the outcome of our own
misdeeds, and therefore, as something that had better not have been.
Whilst, under the former hypothesis, they amount to a bitter
accusation against the Creator, and supply material for sarcasm; under
the latter they form an indictment against our own nature, our own
will, and teach us a lesson of humility. They lead us to see that,
like the children of a libertine, we come into the world with the
burden of sin upon us; and that it is only through having continually
to atone for this sin that our existence is so miserable, and that its
end is death.

There is nothing more certain than the general truth that it is the
grievous \textit{sin of the world} which has produced the grievous
\textit{suffering of the world}. I am not referring here to the
physical connection between these two things lying in the realm of
experience; my meaning is metaphysical. Accordingly, the sole thing
that reconciles me to the Old Testament is the story of the Fall. In
my eyes, it is the only metaphysical truth in that book, even though
it appears in the form of an allegory. There seems to me no \page{25}
better explanation of our existence than that it is the result of some
false step, some sin of which we are paying the penalty. I cannot
refrain from recommending the thoughtful reader a popular, but at the
same time, profound treatise on this subject by
Claudius\footnote{\textit{Translator's Note}. Matthias Claudius
(1740-1815), a popular poet, and friend of Klopstock, Herder and
Lessing. He edited the \textit{Wandsbecker Bote}, in the fourth part
of which appeared the treatise mentioned above. He generally wrote
under the pseudonym of \textit{Asmus}, and Schopenhauer often refers
to him by this name.} which exhibits the essentially pessimistic
spirit of Christianity. It is entitled: \textit{Cursed is the ground
for thy sake}.

Between the ethics of the Greeks and the ethics of the Hindoos, there
is a glaring contrast. In the one case (with the exception, it must be
confessed, of Plato), the object of ethics is to enable a man to lead
a happy life; in the other, it is to free and redeem him from life
altogether---as is directly stated in the very first words of the
\textit{Sankhya Karika}.

Allied with this is the contrast between the Greek and the Christian
idea of death. It is strikingly presented in a visible form on a fine
antique sarcophagus in the gallery at Florence, which exhibits, in
relief, the whole series of ceremonies attending a wedding in ancient
times, from the formal offer to the evening when Hymen's torch lights
the happy couple home. Compare with that the Christian coffin, draped
in mournful black and surmounted with a crucifix! How much
significance there is in these two ways of finding comfort in death.
They \page{26} are opposed to each other, but each is right. The one
points to the \textit{affirmation} of the will to live, which remains
sure of life for all time, however rapidly its forms may change. The
other, in the symbol of suffering and death, points to the
\textit{denial} of the will to live, to redemption from this world,
the domain of death and devil. And in the question between the
affirmation and the denial of the will to live, Christianity is in the
last resort right.

The contrast which the New Testament presents when compared with the
old, according to the ecclesiastical view of the matter, is just that
existing between my ethical system and the moral philosophy of Europe.
The Old Testament represents man as under the dominion of Law, in
which, however, there is no redemption. The New Testament declares Law
to have failed, frees man from its dominion,\footnote{Cf. Romans vii;
Galatians ii, iii.} and in its stead preaches the kingdom of grace, to
be won by faith, love of neighbor and entire sacrifice of self. This
is the path of redemption from the evil of the world. The spirit of
the New Testament is undoubtedly asceticism, however your protestants
and rationalists may twist it to suit their purpose. Asceticism is the
denial of the will to live; and the transition from the old Testament
to the New, from the dominion of Law to that of Faith, from
justification by works to redemption through the Mediator, from the
domain of sin and death to eternal life in Christ, means, when taken
in its real sense, the transition from the merely moral virtues to the
denial of the will to live. My philosophy shows the
meta-\page{27}physical foundation of justice and the love of mankind,
and points to the goal to which these virtues necessarily lead, if
they are practised in perfection. At the same time it is candid in
confessing that a man must turn his back upon the world, and that the
denial of the will to live is the way of redemption. It is therefore
really at one with the spirit of the New Testament, whilst all other
systems are couched in the spirit of the Old; that is to say,
theoretically as well as practically, their result is Judaism---mere
despotic theism. In this sense, then, my doctrine might be called the
only true Christian philosophy---however paradoxical a statement this
may seem to people who take superficial views instead of penetrating
to the heart of the matter.

If you want a safe compass to guide you through life, and to banish
all doubt as to the right way of looking at it, you cannot do better
than accustom yourself to regard this world as a penitentiary, a sort
of a penal colony, or \grk{ἐργαστήριον}, as the earliest philosophers
called it.\footnote{Cf. Clem. Alex. Strom. L. iii, c. 3, p. 399.}
Amongst the Christian Fathers, Origen, with praiseworthy courage, took
this view,\footnote{Augustine \textit{de civitate Dei}., L. xi. c.
23.} which is further justified by certain objective theories of life.
I refer, not to my own philosophy alone, but to the wisdom of all
ages, as expressed in Brahmanism and Buddhism, and in the sayings of
Greek philosophers like Empedocles and Pythagoras; as also by Cicero,
in his remark that the wise men of old used to teach that we come into
this world to pay the penalty of crime committed in another state of
existence---a \page{28} doctrine which formed part of the initiation
into the mysteries.\footnote{Cf. \textit{Fragmenta de philosophia}.}
And Vanini---whom his contemporaries burned, finding that an easier
task than to confute him---puts the same thing in a very forcible way.
\textit{Man}, he says, \textit{is so full of every kind of misery
that, were it not repugnant to the Christian religion, I should
venture to affirm that if evil spirits exist at all, they have posed
into human form and are now atoning for their
crimes}.\footnote{\textit{De admirandis naturae arcanis}; dial L. p.
35.} And true Christianity---using the word in its right sense---also
regards our existence as the consequence of sin and error.

If you accustom yourself to this view of life you will regulate your
expectations accordingly, and cease to look upon all its disagreeable
incidents, great and small, its sufferings, its worries, its misery,
as anything unusual or irregular; nay, you will find that everything
is as it should be, in a world where each of us pays the penalty of
existence in his own peculiar way. Amongst the evils of a penal colony
is the society of those who form it; and if the reader is worthy of
better company, he will need no words from me to remind him of what he
has to put up with at present. If he has a soul above the common, or
if he is a man of genius, he will occasionally feel like some noble
prisoner of state, condemned to work in the galleys with common
criminals; and he will follow his example and try to isolate himself.

In general, however, it should be said that this view of life will
enable us to contemplate the so-called imperfections of the great
majority of men \page{29} their moral and intellectual deficiencies
and the resulting base type of countenance, without any surprise, to
say nothing of indignation; for we shall never cease to reflect where
we are, and that the men about us are beings conceived and born in
sin, and living to atone for it. That is what Christianity means in
speaking of the sinful nature of man.

\textit{Pardon's the word to all!}\footnote{``Cymbeline,'' Act v. Sc.
5.} Whatever folly men commit, be their shortcomings or their vices
what they may, let us exercise forbearance; remembering that when
these faults appear in others it is our follies and vices that we
behold. They are the shortcomings of humanity, to which we belong;
whose faults, one and all, we share; yes, even those very faults at
which we now wax so indignant, merely because they have not yet
appeared in ourselves. They are faults that do not lie on the surface.
But they exist down there in the depths of our nature; and should
anything call them forth they will come and show themselves, just as
we now see them in others. One man, it is true, may have faults that
are absent in his fellow; and it is undeniable that the sum total of
bad qualities is in some cases very large; for the difference of
individuality between man and man passes all measure.

In fact, the conviction that the world and man is something that had
better not have been is of a kind to fill us with indulgence towards
one another. Nay, from this point of view, we might well consider the
proper form of address to be, not \textit{Monsieur, Sir, mein Herr},
but \textit{my fellow-sufferer, Soc\^i malorum, compagnon de
mis\`eres}! This may perhaps sound strange, \page{30} but it is in
keeping with the facts; it puts others in a right light; and it
reminds us of that which is after all the most necessary thing in
life---the tolerance, patience, regard, and love of neighbor, of which
everyone stands in need, and which, therefore, every man owes to his
fellow.

