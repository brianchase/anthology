
\author{Miguel de Unamuno}
\authdate{1864--1936}
\textdate{1912}
\chapter[Miguel de Unamuno -- Tragic Sense of Life, chap. 3]{Tragic
Sense of Life\\\smaller Chapter 3}

\nfootnote{\fullcite{unamuno1921.3}}

\page{38}Let us pause to consider this immortal yearning for
immortality---even though the gnostics or intellectuals may be able to
say that what follows is not philosophy but rhetoric. Moreover, the
divine Plato, when he discussed the immortality of the soul in his
\textit{Ph\ae do}, said that it was proper to clothe it in legend,
\grk{μυθολογεῖν}.

First of all let us recall once again---and it will not be for the
last time---that saying of Spinoza that every being endeavours to
persist in itself, and that this endeavour is its actual essence, and
implies indefinite time, and that the soul, in fine, sometimes with a
clear and distinct idea, sometimes confusedly, tends to persist in its
being with indefinite duration, and is aware of its persistency
(\textit{Ethic}, Part III., Props. VI.--X.).

It is impossible for us, in effect, to conceive of ourselves as not
existing, and no effort is capable of enabling consciousness to
realize absolute unconsciousness, its own annihilation. Try, reader,
to imagine to yourself, when you are wide awake, the condition of your
soul when you are in a deep sleep; try to fill your consciousness with
the representation of no-consciousness, and you will see the
impossibility of it. The effort to comprehend it causes the most
tormenting dizziness. We cannot conceive ourselves as not existing.

The visible universe, the universe that is created by the instinct of
self-pre\-ser\-va\-tion, becomes all too narrow for me. It is like a
cramped cell, against the bars of which my soul beats its wings in
vain. Its lack of air stifles me. More, more, and always more! I want
to be myself, and yet without ceasing to be myself to be others
\page{39} as well, to merge myself into the totality of things visible
and invisible, to extend myself into the illimitable of space and to
prolong myself into the infinite of time. Not to be all and for ever
is as if not to be---at least, let me be my whole self, and be so for
ever and ever. And to be the whole of myself is to be everybody else.
Either all or nothing!

All or nothing! And what other meaning can the Shakespearean ``To be
or not to be'' have, or that passage in \textit{Coriolanus} where it
is said of Marcius ``He wants nothing of a god but eternity''?
Eternity, eternity!---that is the supreme desire! The thirst of
eternity is what is called love among men, and whosoever loves another
wishes to eternalize himself in him. Nothing is real that is not
eternal.

From the poets of all ages and from the depths of their souls this
tremendous vision of the flowing away of life like water has wrung
bitter cries---from Pindar's ``dream of a shadow,'' \grk{σκιὰσ ὄναρ},
to Calder\'on's ``life is a dream'' and Shakespeare's ``we are such
stuff as dreams are made on,'' this last a yet more tragic sentence
than Calder\'on's, for whereas the Castilian only declares that our
life is a dream, but not that we ourselves are the dreamers of it, the
Englishman makes us ourselves a dream, a dream that dreams.

The vanity of the passing world and love are the two fundamental and
heart-penetrating notes of true poetry. And they are two notes of
which neither can be sounded without causing the other to vibrate. The
feeling of the vanity of the passing world kindles love in us, the
only thing that triumphs over the vain and transitory, the only thing
that fills life again and eternalizes it. In appearance at any rate,
for in reality\ldots And love, above all when it struggles against
destiny, overwhelms us with the feeling of the vanity of this world of
appearances and gives us a glimpse of another world, in which destiny
is overcome and liberty is law.

\page{40}Everything passes! Such is the refrain of those who have
drunk, lips to the spring, of the fountain of life, of those who have
tasted of the fruit of the tree of the knowledge of good and evil.

To be, to be for ever, to be without ending! thirst of being, thirst
of being more! hunger of God! thirst of love eternalizing and eternal!
to be for ever! to be God!

``Ye shall be as gods!'' we are told in Genesis that the serpent said
to the first pair of lovers (Gen. iii. 5). ``If in this life only we
have hope in Christ, we are of all men most miserable,'' wrote the
Apostle (1 Cor. xv. 19); and all religion has sprung historically from
the cult of the dead---that is to say, from the cult of immortality.

The tragic Portuguese Jew of Amsterdam wrote that the free man thinks
of nothing less than of death; but this free man is a dead man, free
from the impulse of life, for want of love, the slave of his liberty.
This thought that I must die and the enigma of what will come after
death is the very palpitation of my consciousness. When I contemplate
the green serenity of the fields or look into the depths of clear eyes
through which shines a fellow-soul, my consciousness dilates, I feel
the diastole of the soul and am bathed in the flood of the life that
flows about me, and I believe in my future; but instantly the voice of
mystery whispers to me, ``Thou shalt cease to be!'' the angel of Death
touches me with his wing, and the systole of the soul floods the
depths of my spirit with the blood of divinity.

Like Pascal, I do not understand those who assert that they care not a
farthing for these things, and this indifference ``in a matter that
touches themselves, their eternity, their all, exasperates me rather
than moves me to compassion, astonishes and shocks me,'' and he who
feels thus ``is for me,'' as for Pascal, whose are the words just
quoted, ``a monster.''

It has been said a thousand times and in a thousand \page{41} books
that ancestor-worship is for the most part the source of primitive
religions, and it may be strictly said that what most distinguishes
man from the other animals is that, in one form or another, he guards
his dead and does not give them over to the neglect of teeming mother
earth; he is an animal that guards its dead. And from what does he
thus guard them? From what does he so futilely protect them? The
wretched consciousness shrinks from its own annihilation, and, just as
an animal spirit, newly severed from the womb of the world, finds
itself confronted with the world and knows itself distinct from it, so
consciousness must needs desire to possess another life than that of
the world itself. And so the earth would run the risk of becoming a
vast cemetery before the dead themselves should die again.

When mud huts or straw shelters, incapable of resisting the inclemency
of the weather, sufficed for the living, tumuli were raised for the
dead, and stone was used for sepulchres before it was used for houses.
It is the strong-builded houses of the dead that have withstood the
ages, not the houses of the living; not the temporary lodgings but the
permanent habitations.

This cult, not of death but of immortality, originates and preserves
religions. In the midst of the delirium of destruction, Robespierre
induced the Convention to declare the existence of the Supreme Being
and ``the consolatory principle of the immortality of the soul,'' the
Incorruptible being dismayed at the idea of having himself one day to
turn to corruption.

A disease? Perhaps; but he who pays no heed to his disease is heedless
of his health, and man is an animal essentially and substantially
diseased. A disease? Perhaps it may be, like life itself to which it
is thrall, and perhaps the only health possible may be death; but this
disease is the fount of all vigorous health. From the depth of this
anguish, from the abyss of the feeling of our mortality, we emerge
into the light of another \page{42} heaven, as from the depth of Hell
Dante emerged to behold the stars once again---

\begin{quote} \textit{e quindi uscimmo a riveder le stelle.}
\end{quote}

Although this meditation upon mortality may soon induce in us a sense
of anguish, it fortifies us in the end. Retire, reader, into yourself
and imagine a slow dissolution of your\-self---the light dimming about
you---all things becoming dumb and soundless, enveloping you in
silence---the objects that you handle crumbling away between your
hands---the ground slipping from under your feet---your very memory
vanishing as if in a swoon---eve\-ry\-thing melting away from you into
nothingness and you yourself also melting a\-way---the very
consciousness of nothingness, merely as the phantom harbourage of a
shadow, not even remaining to you.

I have heard it related of a poor harvester who died in a hospital
bed, that when the priest went to anoint his hands with the oil of
extreme unction, he refused to open his right hand, which clutched a
few dirty coins, not considering that very soon neither his hand nor
he himself would be his own any more. And so we close and clench, not
our hand, but our heart, seeking to clutch the world in it.

A friend confessed to me that, foreseeing while in the full vigour of
physical health the near approach of a violent death, he proposed to
concentrate his life and spend the few days which he calculated still
remained to him in writing a book. Vanity of vanities!

If at the death of the body which sustains me, and which I call mine
to distinguish it from the self that is I, my consciousness returns to
the absolute unconsciousness from which it sprang, and if a like fate
befalls all my brothers in humanity, then is our toil-worn human race
nothing but a fatidical procession of phantoms, going from nothingness
to nothingness, and humanitarianism the most inhuman thing known.

\page{43}And the remedy is not that suggested in the quatrain that
runs---

\begin{verse}
\textit{Cada vez que considero\\
que me tengo de morir,\\
tiendo la capa en el suelo\\
y no me harto de dormir.}\footnote{Each time that I
consider that it is my lot to die, I spread my cloak upon the ground
and am never surfeited with sleeping.}
\end{verse}

No! The remedy is to consider our mortal destiny without flinching, to
fasten our gaze upon the gaze of the Sphinx, for it is thus that the
malevolence of its spell is discharmed.

If we all die utterly, wherefore does everything exist? Wherefore? It
is the Wherefore of the Sphinx; it is the Wherefore that corrodes the
marrow of the soul; it is the begetter of that anguish which gives us
the love of hope.

Among the poetic laments of the unhappy Cowper there are some lines
written under the oppression of delirium, in which, believing himself
to be the mark of the Divine vengeance, he exclaims---

\begin{quote}{Hell might afford my miseries a shelter.}
\end{quote}

\noindent This is the Puritan sentiment, the preoccupation with sin
and predestination; but read the much more terrible words of
S\'enancour, expressive of the Catholic, not the Protestant, despair,
when he makes his Obermann say, ``L'homme est p\'erissable. Il se
peut; mais p\'erissons en r\'esistant, et, si le n\'eant nous est
r\'eserv\'e, ne faisons pas que ce soit une justice.'' And I must
confess, painful though the confession be, that in the days of the
simple faith of my childhood, descriptions of the tortures of hell,
however terrible, never made me tremble, for I always felt that
nothingness was much more terrifying. He who suffers lives, and he who
lives suffering, even though over the portal of his abode is written
``Abandon all hope!'' loves and hopes. It is better to live in pain
\page{44} than to cease to be in peace. The truth is that I could not
believe in this atrocity of Hell, of an eternity of punishment, nor
did I see any more real hell than nothingness and the prospect of it.
And I continue in the belief that if we all believed in our salvation
from nothingness we should all be better.

What is this \textit{joie de vivre} that they talk about nowadays? Our
hunger for God, our thirst of immortality, of survival, will always
stifle in us this pitiful enjoyment of the life that passes and abides
not. It is the frenzied love of life, the love that would have life to
be unending, that most often urges us to long for death. ``If it is
true that I am to die utterly,'' we say to ourselves, ``then once I am
annihilated the world has ended so far as I am con\-cerned---it is
finished. Why, then, should it not end forthwith, so that no new
consciousnesses, doomed to suffer the tormenting illusion of a
transient and apparential existence, may come into being? If, the
illusion of living being shattered, living for the mere sake of living
or for the sake of others who are likewise doomed to die, does not
satisfy the soul, what is the good of living? Our best remedy is
death.'' And thus it is that we chant the praises of the never-ending
rest because of our dread of it, and speak of liberating death.

Leopardi, the poet of sorrow, of annihilation, having lost the
ultimate illusion, that of believing in his immortality---

\begin{verse}
\hspace{1.1em}\textit{Peri l'inganno estremo}\\
\textit{ch'eterno io mi credei,}
\end{verse}

\noindent spoke to his heart of \textit{l'infinita vanit\'a del
tutto}, and perceived how close is the kinship between love and death,
and how ``when love is born deep down in the heart, simultaneously a
languid and weary desire to die is felt in the breast.'' The greater
part of those who seek death at their own hand are moved thereto by
love; it is the supreme longing for life, for more life, the longing
\page{45} to prolong and perpetuate life, that urges them to death,
once they are persuaded of the vanity of this longing.

The problem is tragic and eternal, and the more we seek to escape from
it, the more it thrusts itself upon us. Four-and-twenty centuries ago,
in his dialogue on the immortality of the soul, the serene
Pla\-to---but was he serene?---spoke of the uncertainty of our dream
of being immortal and of the \textit{risk} that the dream might be
vain, and from his own soul there escaped this profound cry---Glorious
is the risk!---\grk{καλὸς γὰρ ὁ κίνδυνος}, glorious is the risk that
we are able to run of our souls never dy\-ing---a sentence that was
the germ of Pascal's famous argument of the wager.

Faced with this risk, I am presented with arguments designed to
eliminate it, arguments demonstrating the absurdity of the belief in
the immortality of the soul; but these arguments fail to make any
impression upon me, for they are reasons and nothing more than
reasons, and it is not with reasons that the heart is appeased. I do
not want to die---no; I neither want to die nor do I want to want to
die; I want to live for ever and ever and ever. I want this ``I'' to
live---this poor ``I'' that I am and that I feel myself to be here and
now, and therefore the problem of the duration of my soul, of my own
soul, tortures me.

I am the centre of my universe, the centre of the universe, and in my
supreme anguish I cry with Michelet, ``Mon moi, ils m'arrachent mon
moi!'' What is a man profited if he shall gain the whole world and
lose his own soul? (Matt. xvi. 26). Egoism, you say? There is nothing
more universal than the individual, for what is the property of each
is the property of all. Each man is worth more than the whole of
humanity, nor will it do to sacrifice each to all save in so far as
all sacrifice themselves to each. That which we call egoism is the
principle of psychic gravity, the necessary postulate. ``Love thy
neighbour as thyself,'' we are told, the presupposi-\page{46}tion
being that each man loves himself; and it is not said ``Love
thyself.'' And, nevertheless, we do not know how to love ourselves.

Put aside the persistence of your own self and ponder what they tell
you. Sacrifice yourself to your children! And sacrifice yourself to
them because they are yours, part and prolongation of yourself, and
they in their turn will sacrifice themselves to their children, and
these children to theirs, and so it will go on without end, a sterile
sacrifice by which nobody profits. I came into the world to create my
self, and what is to become of all our selves? Live for the True, the
Good, the Beautiful! We shall see presently the supreme vanity and the
supreme insincerity of this hypocritical attitude.

``That art thou!'' they tell me with the Upanishads. And I answer:
Yes, I am that, if that is I and all is mine, and mine the totality of
things. As mine I love the All, and I love my neighbour because he
lives in me and is part of my consciousness, because he is like me,
because he is mine.

Oh, to prolong this blissful moment, to sleep, to eternalize oneself
in it! Here and now, in this discreet and diffused light, in this lake
of quietude, the storm of the heart appeased and stilled the echoes of
the world! Insatiable desire now sleeps and does not even dream; use
and wont, blessed use and wont, are the rule of my eternity; my
disillusions have died with my memories, and with my hopes my fears.

And they come seeking to deceive us with a deceit of deceits, telling
us that nothing is lost, that everything is transformed, shifts and
changes, that not the least particle of matter is annihilated, not the
least impulse of energy is lost, and there are some who pretend to
console us with this! Futile consolation! It is not my matter or my
energy that is the cause of my disquiet, for they are not mine if I
myself am not mine---that is, if I am not eternal. No, my longing is
not to be submerged in the \page{47} vast All, in an infinite and
eternal Matter or Energy, or in God; not to be possessed by God, but
to possess Him, to become myself God, yet without ceasing to be I
myself, I who am now speaking to you. Tricks of monism avail us
nothing; we crave the substance and not the shadow of immortality.

Materialism, you say? Materialism? Without doubt; but either our
spirit is likewise some kind of matter or it is nothing. I dread the
idea of having to tear myself away from my flesh; I dread still more
the idea of having to tear myself away from everything sensible and
material, from all substance. Yes, perhaps this merits the name of
materialism; and if I grapple myself to God with all my powers and all
my senses, it is that He may carry me in His arms beyond death,
looking into these eyes of mine with the light of His heaven when the
light of earth is dimming in them for ever. Self-illusion? Talk not to
me of illusion---let me live!

They also call this pride---``stinking pride'' Leopardi called
it---and they ask us who are we, vile earthworms, to pretend to
immortality; in virtue of what? wherefore? by what right? ``In virtue
of what?'' you ask; and I reply, In virtue of what do we now live?
``Wherefore?''---and wherefore do we now exist? ``By what
right?''---and by what right are we? To exist is just as gratuitous as
to go on existing for ever. Do not let us talk of merit or of right or
of the wherefore of our longing, which is an end in itself, or we
shall lose our reason in a vortex of absurdities. I do not claim any
right or merit; it is only a necessity; I need it in order to live.

And you, who are you? you ask me; and I reply with Obermann, ``For the
universe, nothing; for myself, everything!'' Pride? Is it pride to
want to be immortal? Unhappy men that we are! 'Tis a tragic fate,
without a doubt, to have to base the affirmation of immortality upon
the insecure and slippery foundation of the desire for immortality;
but to condemn this \page{48} desire on the ground that we believe it
to have been proved to be unattainable, without undertaking the proof,
is merely supine. I am dreaming\ldots ? Let me dream, if this dream is
my life. Do not awaken me from it. I believe in the immortal origin of
this yearning for immortality, which is the very substance of my soul.
But do I really believe in it\ldots ? And wherefore do you want to be
immortal? you ask me, wherefore? Frankly, I do not understand the
question, for it is to ask the reason of the reason, the end of the
end, the principle of the principle.

But these are things which it is impossible to discuss.

It is related in the book of the Acts of the Apostles how wherever
Paul went the Jews, moved with envy, were stirred up to persecute him.
They stoned him in Iconium and Lystra, cities of Lycaonia, in spite of
the wonders that he worked therein; they scourged him in Philippi of
Macedonia and persecuted his brethren in Thessalonica and Berea. He
arrived at Athens, however, the noble city of the intellectuals, over
which brooded the sublime spirit of Pla\-to---the Plato of the
gloriousness of the risk of immortality; and there Paul disputed with
Epicureans and Stoics. And some said of him, ``What doth this babbler
(\grk{σπερμολόγος}) mean?'' and others, ``He seemeth to be a setter
forth of strange gods'' (Acts xvii. 18), ``and they took him and
brought him unto Areopagus, saying, May we know what this new
doctrine, whereof thou speakest, is? for thou bringest certain strange
things to our ears; we would know, therefore, what these things mean''
(verses 19--20). And then follows that wonderful characterization of
those Athenians of the decadence, those dainty connoisseurs of the
curious, ``for all the Athenians and strangers which were there spent
their time in nothing else, but either to tell or to hear some new
thing'' (verse 21). A wonderful stroke which depicts for us the
condition of mind of those who had learned from the \textit{Odyssey}
that the gods plot and \page{49} achieve the destruction of mortals in
order that their posterity may have something to narrate!

Here Paul stands, then, before the subtle Athenians, before the
\textit{gr\ae uli}, men of culture and tolerance, who are ready to
welcome and examine every doctrine, who neither stone nor scourge nor
imprison any man for professing these or those doc\-trines---here he
stands where liberty of conscience is respected and every opinion is
given an attentive hearing. And he raises his voice in the midst of
the Areopagus and speaks to them as it was fitting to speak to the
cultured citizens of Athens, and all listen to him, agog to hear the
latest novelty. But when he begins to speak to them of the
resurrection of the dead their stock of patience and tolerance comes
to an end, and some mock him, and others say: ``We will hear thee
again of this matter!'' intending not to hear him. And a similar thing
happened to him at C\ae sarea when he came before the Roman pr\ae tor
Felix, likewise a broad-minded and cultured man, who mitigated the
hardships of his imprisonment, and wished to hear and did hear him
discourse of righteousness and of temperance; but when he spoke of the
judgement to come, Felix said, terrified (\grk{ἔμφοβος γενόμενος}):
``Go thy way for this time; when I have a convenient season I will
call for thee'' (Acts xxiv. 22--25). And in his audience before King
Agrippa, when Festus the governor heard him speak of the resurrection
of the dead, he exclaimed: ``Thou art mad, Paul; much learning hath
made thee mad'' (Acts xxvi. 24).

Whatever of truth there may have been in Paul's discourse in the
Areopagus, and even if there were none, it is certain that this
admirable account plainly shows how far Attic tolerance goes and where
the patience of the intellectuals ends. They all listen to you, calmly
and smilingly, and at times they encourage you, saying: ``That's
strange!'' or, ``He has brains!'' or ``That's suggestive,'' or ``How
fine!'' or ``Pity that a thing so \page{50} beautiful should not be
true!'' or ``This makes one think!'' But as soon as you speak to them
of resurrection and life after death, they lose their patience and cut
short your remarks and exclaim, ``Enough of this! We will talk about
this another day!'' And it is about this, my poor Athenians, my
intolerant intellectuals, it is about this that I am going to talk to
you here.

And even if this belief be absurd, why is its exposition less
tolerated than that of others much more absurd? Why this manifest
hostility to such a belief? Is it fear? Is it, perhaps, spite provoked
by inability to share it?

And sensible men, those who do not intend to let themselves be
deceived, keep on dinning into our ears the refrain that it is no use
giving way to folly and kicking against the pricks, for what cannot be
is impossible. The manly attitude, they say, is to resign oneself to
fate; since we are not immortal, do not let us want to be so; let us
submit ourselves to reason without tormenting ourselves about what is
irremediable, and so making life more gloomy and miserable. This
obsession, they add, is a disease. Disease, madness, reason\ldots the
everlasting refrain! Very well then---No! I do not submit to reason,
and I rebel against it, and I persist in creating by the energy of
faith my immortalizing God, and in forcing by my will the stars out of
their courses, for if we had faith as a grain of mustard seed we
should say to that mountain, ``Remove hence,'' and it would remove,
and nothing would be impossible to us (Matt. xvii. 20).

There you have that ``thief of energies,'' as he\footnote{Nietzsche.}
so obtusely called Christ who sought to wed nihilism with the struggle
for existence, and he talks to you about courage. His heart craved the
eternal All while his head convinced him of nothingness, and,
desperate and mad to defend himself from himself, he cursed that which
he \page{51} most loved. Because he could not be Christ, he blasphemed
against Christ. Bursting with his own self, he wished himself unending
and dreamed his theory of eternal recurrence, a sorry counterfeit of
immortality, and, full of pity for himself, he abominated all pity.
And there are some who say that his is the philosophy of strong men!
No, it is not. My health and my strength urge me to perpetuate myself.
His is the doctrine of weaklings who aspire to be strong, but not of
the strong who are strong. Only the feeble resign themselves to final
death and substitute some other desire for the longing for personal
immortality. In the strong the zeal for perpetuity overrides the doubt
of realizing it, and their superabundance of life overflows upon the
other side of death.

Before this terrible mystery of mortality, face to face with the
Sphinx, man adopts different attitudes and seeks in various ways to
console himself for having been born. And now it occurs to him to take
it as a diversion, and he says to himself with Renan that this
universe is a spectacle that God presents to Himself, and that it
behoves us to carry out the intentions of the great Stage-Manager and
contribute to make the spectacle the most brilliant and the most
varied that may be. And they have made a religion of art, a cure for
the metaphysical evil, and invented the meaningless phrase of art for
art's sake.

And it does not suffice them. If the man who tells you that he writes,
paints, sculptures, or sings for his own amusement, gives his work to
the public, he lies; he lies if he puts his name to his writing,
painting, statue, or song. He wishes, at the least, to leave behind a
shadow of his spirit, something that may survive him. If the
\textit{Imitation of Christ} is anonymous, it is because its author
sought the eternity of the soul and did not trouble himself about that
of the name. The man of letters who shall tell you that he despises
fame is a lying rascal. \page{52} Of Dante, the author of those
three-and-thirty vigorous verses (\textit{Purg}. xi. 85--117) on the
vanity of worldly glory, Boccaccio says that he relished honours and
pomps more perhaps than suited with his conspicuous virtue. The
keenest desire of his condemned souls is that they may be remembered
and talked of here on earth, and this is the chief solace that
lightens the darkness of his Inferno. And he himself confessed that
his aim in expounding the concept of Monarchy was not merely that he
might be of service to others, but that he might win for his own glory
the palm of so great prize (\textit{De Monarchia}, lib. i., cap. i.).
What more? Even of that holy man, seemingly the most indifferent to
worldly vanity, the Poor Little One of Assisi, it is related in the
\textit{Legenda Trium Sociorum} that he said: \textit{Adhuc adorabor
per totum mundum!}---You will see how I shall yet be adored by all the
world! (II. \textit{Celano}, i. 1). And even of God Himself the
theologians say that He created the world for the manifestation of His
glory.

When doubts invade us and cloud our faith in the immortality of the
soul, a vigorous and painful impulse is given to the anxiety to
perpetuate our name and fame, to grasp at least a shadow of
immortality. And hence this tremendous struggle to singularize
ourselves, to survive in some way in the memory of others and of
posterity. It is this struggle, a thousand times more terrible than
the struggle for life, that gives its tone, colour, and character to
our society, in which the medieval faith in the immortal soul is
passing away. Each one seeks to affirm himself, if only in appearance.

Once the needs of hunger are sat\-is\-fied---and they are soon
sat\-is\-fied---the vanity, the necessity---for it is a
ne\-ces\-si\-ty---a\-ris\-es of imposing ourselves upon and surviving
in others. Man habitually sacrifices his life to his purse, but he
sacrifices his purse to his vanity. He boasts even of his weaknesses
and his misfortunes, for want of anything better to boast of, and is
like a child who, in order \page{53} to attract attention, struts
about with a bandaged finger. And vanity, what is it but eagerness for
survival?

The vain man is in like case with the av\-a\-ri\-cious---he takes the
means for the end; forgetting the end he pursues the means for its own
sake and goes no further. The seeming to be something, conducive to
being it, ends by forming our objective. We need that others should
believe in our superiority to them in order that we may believe in it
ourselves, and upon their belief base our faith in our own
persistence, or at least in the persistence of our fame. We are more
grateful to him who congratulates us on the skill with which we defend
a cause than we are to him who recognizes the truth or the goodness of
the cause itself. A rabid mania for originality is rife in the modern
intellectual world and characterizes all individual effort. We would
rather err with genius than hit the mark with the crowd. Rousseau has
said in his \textit{\'Emile} (book iv.): ``Even though philosophers
should be in a position to discover the truth, which of them would
take any interest in it? Each one knows well that his system is not
better founded than the others, but he supports it because it is his.
There is not a single one of them who, if he came to know the true and
the false, would not prefer the falsehood that he had found to the
truth discovered by another. Where is the philosopher who would not
willingly deceive mankind for his own glory? Where is he who in the
secret of his heart does not propose to himself any other object than
to distinguish himself? Provided that he lifts himself above the
vulgar, provided that he outshines the brilliance of his competitors,
what does he demand more? The essential thing is to think differently
from others. With believers he is an atheist; with atheists he would
be a believer.'' How much substantial truth there is in these gloomy
confessions of this man of painful sincerity!

This violent struggle for the perpetuation of our name extends
backwards into the past, just as it aspires to \page{54} conquer the
future; we contend with the dead because we, the living, are obscured
beneath their shadow. We are jealous of the geniuses of former times,
whose names, standing out like the landmarks of history, rescue the
ages from oblivion. The heaven of fame is not very large, and the more
there are who enter it the less is the share of each. The great names
of the past rob us of our place in it; the space which they fill in
the popular memory they usurp from us who aspire to occupy it. And so
we rise up in revolt against them, and hence the bitterness with which
all those who seek after fame in the world of letters judge those who
have already attained it and are in enjoyment of it. If additions
continue to be made to the wealth of literature, there will come a day
of sifting, and each one fears lest he be caught in the meshes of the
sieve. In attacking the masters, irreverent youth is only defending
itself; the iconoclast or image-breaker is a Stylite who erects
himself as an image, an \textit{icon}. ``Comparisons are odious,''
says the familiar adage, and the reason is that we wish to be unique.
Do not tell Fernandez that he is one of the most talented Spaniards of
the younger generation, for though he will affect to be gratified by
the eulogy he is really annoyed by it; if, however, you tell him that
he is the most talented man in Spain---well and good! But even that is
not sufficient: one of the worldwide reputations would be more to his
liking, but he is only fully satisfied with being esteemed the first
in all countries and all ages. The more alone, the nearer to that
unsubstantial immortality, the immortality of the name, for great
names diminish one another.

What is the meaning of that irritation which we feel when we believe
that we are robbed of a phrase, or a thought, or an image, which we
believed to be our own, when we are plagiarized? Robbed? Can it indeed
be ours once we have given it to the public? Only because it is ours
we prize it; and we are fonder of the false money \page{55} that
preserves our impress than of the coin of pure gold from which our
effigy and our legend has been effaced. It very commonly happens that
it is when the name of a writer is no longer in men's mouths that he
most influences his public, his mind being then disseminated and
infused in the minds of those who have read him, whereas he was quoted
chiefly when his thoughts and sayings, clashing with those generally
received, needed the guarantee of a name. What was his now belongs to
all, and he lives in all. But for him the garlands have faded, and he
believes himself to have failed. He hears no more either the applause
or the silent tremor of the heart of those who go on reading him. Ask
any sincere artist which he would prefer, whether that his work should
perish and his memory survive, or that his work should survive and his
memory perish, and you will see what he will tell you, if he is really
sincere. When a man does not work merely in order to live and carry
on, he works in order to survive. To work for the work's sake is not
work but play. And play? We will talk about that later on.

A tremendous passion is this longing that our memory may be rescued,
if it is possible, from the oblivion which overtakes others. From it
springs envy, the cause, according to the biblical narrative, of the
crime with which human history opened: the murder of Abel by his
brother Cain. It was not a struggle for bread---it was a struggle to
survive in God, in the divine memory. Envy is a thousand times more
terrible than hunger, for it is spiritual hunger. If what we call the
problem of life, the problem of bread, were once solved, the earth
would be turned into a hell by the emergence in a more violent form of
the struggle for survival.

For the sake of a name man is ready to sacrifice not only life but
hap\-pi\-ness---life as a matter of course. ``Let me die, but let my
fame live!'' exclaimed Rodrigo Arias in \textit{Las Mocedades del Cid}
when he fell mortally wounded \page{56} by Don Ord\'o\~nez de Lara.
``Courage, Girolamo, for you will long be remembered; death is bitter,
but fame eternal!'' cried Girolamo Olgiati, the disciple of Cola
Montano and the murderer, together with his fellow-conspirators
Lampugnani and Visconti, of Galeazzo Sforza, tyrant of Milan. And
there are some who covet even the gallows for the sake of acquiring
fame, even though it be an infamous fame: \textit{avidus mal{\ae}
fam\ae}, as Tacitus says.

And this erostratism, what is it at bottom but the longing for
immortality, if not for substantial and concrete immortality, at any
rate for the shadowy immortality of the name?

And in this there are degrees. If a man despises the applause of the
crowd of to-day, it is because he seeks to survive in renewed
minorities for generations. ``Posterity is an accumulation of
minorities,'' said Gounod. He wishes to prolong himself in time rather
than in space. The crowd soon overthrows its own idols and the statue
lies broken at the foot of the pedestal without anyone heeding it; but
those who win the hearts of the elect will long be the objects of a
fervent worship in some shrine, small and secluded no doubt, but
capable of preserving them from the flood of oblivion. The artist
sacrifices the extensiveness of his fame to its duration; he is
anxious rather to endure for ever in some little corner than to occupy
a brilliant second place in the whole universe; he prefers to be an
atom, eternal and conscious of himself, rather than to be for a brief
moment the consciousness of the whole universe; he sacrifices
infinitude to eternity.

And they keep on wearying our ears with this chorus of Pride! stinking
Pride! Pride, to wish to leave an ineffaceable name? Pride? It is like
calling the thirst for riches a thirst for pleasure. No, it is not so
much the longing for pleasure that drives us poor folk to seek money
as the terror of poverty, just as it was not the \page{57} desire for
glory but the terror of hell that drove men in the Middle Ages to the
cloister with its \textit{acedia}. Neither is this wish to leave a
name pride, but terror of extinction. We aim at being all because in
that we see the only means of escaping from being nothing. We wish to
save our mem\-o\-ry---at any rate, our memory. How long will it last?
At most as long as the human race lasts. And what if we shall save our
memory in God?

Unhappy, I know well, are these confessions; but from the depth of
unhappiness springs new life, and only by draining the lees of
spiritual sorrow can we at last taste the honey that lies at the
bottom of the cup of life. Anguish leads us to consolation.

This thirst for eternal life is appeased by many, especially by the
simple, at the fountain of religious faith; but to drink of this is
not given to all. The institution whose primordial end is to protect
this faith in the personal immortality of the soul is Catholicism; but
Catholicism has sought to rationalize this faith by converting
religion into theology, by offering a philosophy, and a philosophy of
the thirteenth century, as a basis for vital belief. This and its
consequences we will now proceed to examine.

