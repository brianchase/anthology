
\author{Anselm of Canterbury}
\authdate{1033--1109}
\textdate{ca. 1077}
\addon{Chapters 2 through 4}
\chapter[Proslogium, chaps. 2--4]{Proslogium}
\source{anselm1903.1}

\page{7}\section{Chapter II.}

\hangindent=\parindent
\hangafter=1
\noindent{\small Truly there is a God, although the fool hath said in
his heart, There is no God.}

\vspace{1\baselineskip}

\noindent And so, Lord, do thou, who dost give understanding to faith,
give me, so far as thou knowest it to be profitable, to understand
that thou art as we believe; and that thou art that which we believe.
And, indeed, we believe that thou art a being than which nothing
greater can be conceived. Or is there no such nature, since the fool
hath said in his heart, there is no God? (Psalms xiv. 1). But, at any
rate, this very fool, when he hears of this being of which I speak---a
being than which nothing greater can be con\-ceived---understands what
he hears, and what he understands is in his understanding; although he
does not understand it to exist.

For, it is one thing for an object to be in the understanding, and
another to understand that the object exists. When a painter first
conceives of what he will afterwards perform, he has it in his
understanding, but he does not yet understand it to be, because he has
not yet performed it. But after he has made the painting, he both has
it in his understanding, and he understands that it exists, because he
has made it.

\page{8}Hence, even the fool is convinced that something exists in the
understanding, at least, than which nothing greater can be conceived.
For, when he hears of this, he understands it. And whatever is
understood, exists in the understanding. And assuredly that, than
which nothing greater can be conceived, cannot exist in the
understanding alone. For, suppose it exists in the understanding
alone: then it can be conceived to exist in reality; which is greater.

Therefore, if that, than which nothing greater can be conceived,
exists in the understanding alone, the very being, than which nothing
greater can be conceived, is one, than which a greater can be
conceived. But obviously this is impossible. Hence, there is no doubt
that there exists a being, than which nothing greater can be
conceived, and it exists both in the understanding and in reality.

\section{Chapter III.}

\hangindent=\parindent
\hangafter=1
\noindent{\small God cannot be conceived not to ex\-ist.---God is
that, than which nothing greater can be con\-ceived.---That which can
be conceived not to exist is not God.}

\vspace{1\baselineskip}

\noindent And it assuredly exists so truly, that it cannot be
conceived not to exist. For, it is possible to conceive of a being
which cannot be conceived not to exist; and this is greater than one
which can be conceived not to exist. Hence, if that, than which
nothing greater can be conceived, can be conceived not to exist, it is
not that, than which nothing greater can be conceived. But this is an
irreconcilable contradiction. There is, then, so truly a being than
which nothing greater can be conceived to exist, that it cannot even
\page{9} be conceived not to exist; and this being thou art, O Lord,
our God.

So truly, therefore, dost thou exist, O Lord, my God, that thou canst
not be conceived not to exist; and rightly. For, if a mind could
conceive of a being better than thee, the creature would rise above
the Creator; and this is most absurd. And, indeed, whatever else there
is, except thee alone, can be conceived not to exist. To thee alone,
therefore, it belongs to exist more truly than all other beings, and
hence in a higher degree than all others. For, whatever else exists
does not exist so truly, and hence in a less degree it belongs to it
to exist. Why, then, has the fool said in his heart, there is no God
(Psalms xiv. 1), since it is so evident, to a rational mind, that thou
dost exist in the highest degree of all? Why, except that he is dull
and a fool?

\section{Chapter IV.}

\hangindent=\parindent
\hangafter=1
\noindent{\small How the fool has said in his heart what cannot be
con\-ceived.---A thing may be conceived in two ways: (1) when the word
signifying it is conceived; (2) when the thing itself is understood As
far as the word goes, God can be conceived not to exist; in reality he
cannot.}

\vspace{1\baselineskip}

\noindent But how has the fool said in his heart what he could not
conceive; or how is it that he could not conceive what he said in his
heart? since it is the same to say in the heart, and to conceive.

But, if really, nay, since really, he both conceived, because he said
in his heart; and did not say in his heart, because he could not
conceive; there is more than one way in which a thing is said in the
heart or conceived. For, in one sense, an object is conceived,
\page{10} when the word signifying it is conceived; and in another,
when the very entity, which the object is, is understood.

In the former sense, then, God can be conceived not to exist; but in
the latter, not at all. For no one who understands what fire and water
are can conceive fire to be water, in accordance with the nature of
the facts themselves, although this is possible according to the
words. So, then, no one who understands what God is can conceive that
God does not exist; although he says these words in his heart, either
without any or with some foreign, signification. For, God is that than
which a greater cannot be conceived. And he who thoroughly understands
this, assuredly understands that this being so truly exists, that not
even in concept can it be non-existent. Therefore, he who understands
that God so exists, cannot conceive that he does not exist.

I thank thee, gracious Lord, I thank thee; because what I formerly
believed by thy bounty, I now so understand by thine illumination,
that if I were unwilling to believe that thou dost exist, I should not
be able not to understand this to be true.

