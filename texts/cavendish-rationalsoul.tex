
\author{Margaret Cavendish}
\authdate{1623--1673}
\textdate{1666}
\addon{Observations upon Experimental Philosophy\\Section 2.15}
\chapter{Of the Rational Soul of Man}
\source{cavendish1666.2.15}

% 'Of the Rational Soul of Man' is the fifteenth division of 'Further
% Observations upon Experimental Philosophy; Reflecting Withal, upon
% Some Principal Subjects in Contemplative Philosophy', which in turn
% is the second division of 'Observations upon Experimental
% Philosophy'. The text is unclear about how to refer to these
% divisions.

\page{301}Of all the Opinions concerning the Natural Soul, of Man, I
like that best which affirms the Soul to be a self-moving Substance;
but yet I will add a \textit{Material} self-moving Substance; for the
Soul of Man is part of the Soul of Nature, and the Soul of Nature is
Material; I mean onely the Natural, not the Divine Soul of Man, which
I leave to the Church. And this Natural Soul, otherwise called Reason,
is nothing else but corporeal natural self-motion, or a particle of
the purest, most subtil and active part of Matter, which I call
Animate; which Animate Matter is the Life and Soul of Nature, and
consequently of Man, and all other Creatures; for we cannot in
Reason conceive that Man should be the onely Creature that partakes of
this Soul of Nature, and that all the rest of Nature's parts, or most
of them, should be Soul-less, or (which is all one) Irrational,
although they are commonly called, nay, believed to be such. Truly, if
all other Creatures cannot be denied to be Material, they can neither
be accounted Irrational, Insensible, or Inanimate, by reason there is
no part, nay, not the smallest particle in Nature, our reason
\page{302} is able to conceive, which is not composed of Animate
Matter, as well as of Inanimate; of Life and Soul, as well as of Body;
and therefore no particular Creature can claim a Prerogative in this
case before an other; for there is a thorow mixture of Animate and
Inanimate Matter in Nature, and all her Parts. But some may object,
That if there be sense and reason in every part of Nature, it must
be in all parts alike; and then a stone, or any other the like
Creature, may have reason, or a rational soul, as well as Man. To
which, I answer: I do not deny that a Stone has reason, or doth
partake of the Rational Soul of Nature, as well as Man doth, because
it is part of the same Matter Man consists of; but yet it has not
Animal or Human sense and reason, because it is not of Animal kind;
but being a Mineral, it has Mineral sense and reason; for it is to be
observed, that as Animate self-moving Matter moves not one and the
same way in all Creatures, so there can neither be the same way of
knowledg and understanding, which is sense and reason, in all
Creatures alike; but Nature being various, not onely in her parts, but
in her actions, it causes a variety also amongst her Creatures; and
hence come so many kinds, sorts and particulars of Natural Creatures,
quite different from each other; though not in the General and
Universal Principle of Nature, which is self-moving Matter, (for in
this they agree all) yet in their particular interior natures, figures
and proprieties. Thus, although there be Sense \page{303} and Reason,
which is not onely Motion, but a regular and well-ordered self-motion,
apparent in the wonderful and various Productions, Generations,
Transformations, Dissolutions, Compositions, and other actions of
Nature, in all Nature's parts and particles; yet by reason of the
variety of this self-motion, whose ways and modes do differ according
to the nature of each particular figure, no figure or creature can
have the same sense and reason, that is, the same natural motions
which another has; and therefore no Stone can be said to feel pain as
an Animal doth, or be called blind, because it has no Eyes; for this
kind of Sense, as Seeing, Hearing, Tasting, Touching and Smelling, is
proper onely to an Animal figure, and not to a Stone, which is a
Mineral; so that those which frame an argument from the want of animal
sense and sensitive organs, to the defect of all sense and motion; as
for example, that a Stone would withdraw it self from the Carts going
over it, or a piece of Iron from the hammering of a Smith, conclude,
in my opinion, very much against the Artificial Rules of Logick; and
although I understand none of them, yet I question not but I shall
make a better Argument by the Rules of Natural Logick: But that this
difference of sense and reason, is not altogether impossible, or at
least improbable to our understanding, I will explain by another
instance. We see so many several Creatures in their several kinds, to
wit, Elements, Vegetables, Mi-\page{304}nerals and Animals, which are
the chief distinctions of those kinds of Creatures as are subject to
our sensitive perceptions; and in all those, what variety and
difference do we find both in their exterior figures, and in their
interior natures? Truly such as most of both ancient and modern
Philosophers have imagined some of them, \textit{viz}. the Elements,
to be simple Bodies, and the Principles of all other Creatures; nay,
those several Creatures do not onely differ so much from each other in
their general kinds, but there is no less difference perceived in
their particular kinds: For example, Concerning Elements, what
difference is there not between heavy and contracting Earth, and
between light and dilating Air? between flowing Water, and ascending
Fire? So as it would be an endless labour to consider all the
different natures of those Creatures onely that are subject to our
exterior senses. And yet who dares deny that they all consist of
Matter, or are material? Thus we see that Infinite Matter is not like
a piece of Clay, out of which no figure can be made, but it must be
Clayie; for natural Matter has no such narrow bounds, and is not
forced to make all Creatures alike; for, though Gold and Stone are
both material, nay, of the same kind, to wit, Minerals, yet one is not
the other, nor like the other. And if this be true of Matter, why may
not the same be said of self-motion, which is Sense and Reason?
Wherefore, in all probability of truth, there is sense and reason in a
Mineral, as well as in an \page{305} Animal, and in a Vegetable as
well as in an Element, although there is as great a difference between
the manner and way of their sensitive and rational perceptions, as
there is between both their exterior and interior figures and natures.
Nay, there is a difference of Sense and Reason even in the parts of
one and the same Creature, and consequently of sensitive and rational
perception or knowledg; for, as I have declared heretofore more at
large, every sensitive Organ in man hath its peculiar way of knowledg
and perception; for the Eye doth not know what the Ear knows, nor the
Ear what the Nose knows, \textit{\&c}. all which is the cause of a
general ignorance between Nature's parts; and the chief cause of all
this difference, is the variety of self-motion; for, if natural Motion
were in all Creatures alike, all Sense and Reason would be alike too;
and if there were no degrees of Matter, all the figures of Creatures
would be alike, either all hard, or all soft; all dense, or all rare
and fluid, \textit{\&c}. and yet neither this variety of Motion causes
an absence of Motion, or of Sense and Reason, nor the variety of
figures an absence of Matter, but onely a difference between the parts
of Nature, all being, nevertheless, self-moving, sensible and
rational, as well as Material; for wheresoever is natural Matter,
there is also Self-motion, and consequently, Sense and Reason. By this
we may see how easie it is to conceive the actions of Nature, and to
resolve all the \textit{Ph\ae nomena} or appearances upon this ground;
and I can-\page{306}not admire enough, how so many eminent and
learned Philosophers have been, and are still puzled about the natural
rational Soul of Man. Some will have her to be a \textit{Light}; some
an \textit{Entelichy}, or they know not what; some the
\textit{Quintessence of the four Elements}; some composed of
\textit{Earth and Water}; some of \textit{Fire}, some of
\textit{Blood}, some an \textit{hot Complexion}, some an
\textit{heated and dispersed Air}, some an \textit{Immaterial
Spirit}, and some \textit{Nothing}. All which opinions seem the more
strange, the wiser their Authors are accounted; for if they did
proceed from some ignorant persons, it would not be so much taken
notice of; but coming from great Philosophers, who pretend to have
searched the depth of Nature, and disclosed her secrets, it causes
great admiration in any body, and may well serve for an argument to
confirm the variety and difference of sensitive and rational knowledg,
and the ignorance amongst natural parts; for if Creatures of the same
particular kind, as Men, have so many different Perceptions, what may
there be in all Nature? But Infinite Nature is wise, and will not
have, that one part of hers should know more than its particular
nature requires; and she taking delight in variety, orders her works
accordingly.

