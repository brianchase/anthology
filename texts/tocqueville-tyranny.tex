
\author{Alexis de Tocqueville}
\authdate{1805--1859}
\textdate{1835}
\addon{Democracy in America, Chapter 16, Section 2}
\chapter{Tyranny of the Majority}
\source{tocqueville1898.1}

\page{330}\begin{abstract}{h} How the Principle of the Sovereignty of
the People is to be understood---Im\-pos\-si\-bil\-i\-ty of conceiving
a Mixed Government---The Sovereign Power must exist
some\-where---Precautions to be taken to control its
Ac\-tion---These Precautions have not been taken in the United
States---Consequences. \end{abstract}

\noindent I hold it to be an impious and detestable maxim, that,
politically speaking, the people have a right to do anything; and yet
I have asserted that all authority originates in the will of the
majority. Am I, then, in contradiction with myself?

A general law, which bears the name of justice, has been made and
sanctioned, not only by a majority of this or that people, but by a
majority of mankind. The rights of every people are therefore confined
within the limits of what is just. A nation may be considered as a
jury which is empowered to represent society at large, and to apply
justice, which is its law. Ought such a jury, which represents
society, to have more power than the society itself, whose laws it
executes?

When I refuse to obey an unjust law, I do not contest the right of the
majority to command, but I simply appeal from the sovereignty of the
people to the sovereignty of mankind. Some have not feared to assert
that a people can never outstep the boundaries of justice and reason
in those affairs which are peculiarly its own; and that consequently
full power may be given to the majority by which it is represented.
But this is the language of a slave.

A majority taken collectively is only an individual, whose opinions,
and frequently whose interests, are opposed to those of another
individual, who is styled a minority. If it be admitted that a man
possessing absolute power may misuse that power by wronging his
adversaries, why should not a majority be liable to the same
\page{331} reproach? Men do not change their characters by uniting
with one another; nor does their patience in the presence of obstacles
increase with their strength.\footnote{No one will assert that a
people cannot forcibly wrong another people; but parties may be looked
upon as lesser nations within a great one, and they are aliens to each
other: if, therefore, it be admitted that a nation can act
tyrannically towards another nation, it cannot be denied that a party
may do the same towards another party.} For my own part, I cannot
believe it; the power to do everything, which I should refuse to one
of my equals, I will never grant to any number of them.

I do not think that, for the sake of preserving liberty, it is
possible to combine several principles in the same government so as
really to oppose them to one another. The form of government that is
usually termed \textit{mixed} has always appeared to me a mere
chimera. Accurately speaking, there is no such thing as a
\textit{mixed government}, in the sense usually given to that word,
because, in all communities, some one principle of action may be
discovered which preponderates over the others. England, in the last
cen\-tu\-ry,---which has been especially cited as an example of this
sort of gov\-ern\-ment,---was essentially an aristocratic state,
although it comprised some great elements of democracy; for the laws
and customs of the country were such that the aristocracy could not
but preponderate in the long run, and direct public affairs according
to its own will. The error arose from seeing the interests of the
nobles perpetually contending with those of the people, without
considering the issue of the contest, which was really the important
point. When a community actually has a mixed government,---that is to
say, when it is equally divided between adverse principles,---it must
either experience a revolution, or fall into anarchy.

I am therefore of the opinion, that social power superior to all
others must always be placed somewhere; but I think \page{332} that
liberty is endangered when this power finds no obstacle which can
retard its course, and give it time to moderate its own vehemence.

Unlimited power is in itself a bad and dangerous thing. Human beings
are not competent to exercise it with discretion. God alone can be
omnipotent, because his wisdom and his justice are always equal to his
power. There is no power on earth so worthy of honor in itself, or
clothed with rights so sacred, that I would admit its uncontrolled and
all-predominant authority. When I see that the right and the means of
absolute command are conferred on any power whatever, be it called a
people or a king, an aristocracy or a democracy, a monarchy or a
republic, I say there is the germ of tyranny, and I seek to live
elsewhere, under other laws.

In my opinion, the main evil of the present democratic institutions of
the United States does not arise, as is often asserted in Europe, from
their weakness, but from their irresistible strength. I am not so much
alarmed at the excessive liberty which reigns in that country, as at
the inadequate securities which one finds there against tyranny.

When an individual or a party is wronged in the United States, to whom
can he apply for redress? If to public opinion, public opinion
constitutes the majority; if to the legislature, it represents the
majority, and implicitly obeys it; if to the executive power, it is
appointed by the majority, and serves as a passive tool in its hands.
The public force consists of the majority under arms; the jury is the
majority invested with the right of hearing judicial cases; and in
certain States, even the judges are elected by the majority. However
iniquitous or absurd the measure of which you complain, you must
submit to it as well as you can.\footnote{A striking instance of the
excesses that may be occasioned by the despotism of the majority
occurred at Baltimore during the War of 1812. At that time, the war
was very popular in Baltimore. A newspaper which had taken the other
side excited by its opposition the indignation of the inhabitants. The
mob assembled, broke the printing-presses, and attacked the house of
the editors. The militia was called out, but did not obey the call;
and the only means of saving the wretches who were threatened by the
frenzy of the mob, was to throw them into prison as common
malefactors. But even this precaution was ineffectual; the mob
collected again during the night; the magistrates again made a vain
attempt to call out the militia; the prison was forced, one of the
newspaper editors was killed upon the spot, and the others were left
for dead. The guilty parties, when they were brought to trial, were
acquitted by the jury.

I said one day to an inhabitant of Pennsylvania: ``Be so good as to
explain to me how it happens, that in a State founded by Quakers, and
celebrated for its toleration, free Blacks are not allowed to exercise
civil rights. They pay taxes; is it not fair that they should
vote?''

``You insult us,'' replied my informant, ``if you imagine that our
legislators could have committed so gross an act of injustice and
intolerance.''

``Then the Blacks possess the right of voting in this country?''

``Without doubt.''

``How comes it, then, that at the polling-booth, this morning, I did
not perceive a single Negro in the meeting?''

``That is not the fault of the law: the Negroes have an undisputed
right of voting; but they voluntarily abstain from making their
appearance.''

``A very pretty piece of modesty on their part!'' rejoined I.

``Why, the truth is, that they are not disinclined to vote, but they
are afraid of being maltreated; in this country, the law is sometimes
unable to maintain its authority, without the support of the majority.
But in this case, the majority entertains very strong prejudices
against the Blacks, and the magistrates are unable to protect them in
the exercise of their legal rights.''

``Then the majority claims the right not only of making the laws, but
of breaking the laws it has made?''}

\page{333}If, on the other hand, a legislative power could be so
constituted as to represent the majority without necessarily being the
slave of its passions, an executive so as to retain a proper share of
authority, and a judiciary so as to remain independent of the other
two powers, a government \page{334} would be formed which would still
be democratic, while incurring hardly any risk of tyranny.

I do not say that there is a frequent use of tyranny in America at the
present day; but I maintain that there is no sure barrier against it,
and that the causes which mitigate the government there are to be
found in the circumstances and the manners of the country, more than
in its laws.

