
\author{Madame de Sta\"{e}l}
\authdate{1766--1817}
\textdate{1810}
\addon{Germany, Part 4, Chapter 12}
\chapter{Of the Influence of Enthusiasm upon Happiness}
\source{destael1810.4.12}

\page{367}The course of my subject necessarily leads me here to treat
of happiness. I have hitherto studiously avoided the word, because
now, for almost a century, it has been the custom to place it
principally in pleasures so gross, in a way of life so selfish, in
calculations so narrow and confined, that its very image is sullied
and profaned. It, however, may be pronounced with confidence, that of
all the feelings of the human heart, enthusiasm confers the greatest
happiness, that indeed it alone confers real happiness, alone can
enable us to bear the lot of mortality in every situation in which
fortune has the power to place us.

Vainly would we reduce ourselves to sensual enjoyments; the soul
asserts itself on every side. Pride, ambition, self-love, all these
are still from the soul, although in them a poisonous \page{368} and
pestilential blast mixes with its essence. Meanwhile, how wretched is
the existence of that crowd of mortals, who, playing the hypocrite
with themselves almost as much as with others, are continually
employed in repressing the generous emotions which struggle to revive
within their bosoms, as diseases of the imagination, which the open
air should at once dispel! How impoverished is the existence of those
who content themselves with abstaining from doing evil, and treat as
weakness and delusion the source of the most beautiful deeds, and the
most noble conceptions! From mere vanity they imprison themselves in
obstinate mediocrity, which they might easily have opened to the light
of knowledge which everywhere surrounds them; they sentence and
condemn themselves to that monotony of ideas, to that deadness of
feeling, which suffers the days to pass, one after the other, without
deriving from them any advantage, without making in them any progress,
without treasuring up any matter for future recollection. If time in
its course had not cast a change upon their features, what proofs
would they have preserved of its having passed at all? If to grow old
and to die were not the necessary law of our nature, what serious
reflection would ever have arisen in their minds?

Some reasoners there are, who object that enthusiasm produces a
distaste for ordinary life; and that, as we cannot always remain in
the same frame of mind, it is more for our advantage never to indulge
it: and why, then, I would ask them, have they accepted the gift of
youth, why of life itself, since they well knew that they were not to
last forever? Why have they loved (if indeed they ever have loved),
since death at any moment might separate them from the objects of
their affection? Can there be a more wretched economy than of the
faculties of the soul? They were given us to be improved and expanded,
to be carried as near as possible to perfection, even to be prodigally
lavished for a high and noble end.

The more we benumb our feelings and render ourselves insensible, the
nearer, it will be said, we approach to a state of material existence,
and the more we diminish the dominion of \page{369} pain and sorrow
over us. This argument imposes upon many; it consists, in fact, in
recommending us to make an attempt to live with as little of life as
possible. But our own degradation is always accompanied by an
uneasiness of mind for which we cannot account, and which
unremittingly attends upon us in secret. The discontent, the shame,
and the weariness which it causes, are arrayed by vanity in the garb
of impertinence and contempt; but it is very rare that any man can
settle peaceably in this confined and desert sphere of being, which
leaves him without resource in himself when he is abandoned by the
prosperity of the world. Man has a consciousness of the beautiful as
well as of the virtuous; and in the absence of the former he feels a
void, as in a deviation from the latter he finds remorse.

It is a common accusation against enthusiasm, that it is transitory;
man were too much blessed, if he could fix and retain emotions so
beautiful; but it is because they are so easily dissipated and lost,
that we should strive and exert ourselves to preserve them. Poetry and
the fine arts are the means of calling forth in man this happiness of
illustrious origin, which raises the depressed heart; and, instead of
an unquiet satiety of life, gives an habitual feeling of the divine
harmony, in which nature and ourselves claim a part. There is no duty,
there is no pleasure, there is no sentiment, which does not borrow
from enthusiasm I know not what charm, which is still in perfect
unison with the simple beauty of truth.

% NOTE: the source has 'the the greatest happiness'

All men take up arms indeed for the defence of the land which they
inhabit, when circumstances demand this duty of them; but if they are
inspired by the enthusiasm of their country, what warm emotions do
they not feel within them? The sun, which shone upon their birth, the
land of their fathers, \textit{the sea which bathes their
rocks},\footnote{It is easy to perceive, that by this phrase, and by
those which follow, I have been trying to designate England; in fact,
I could not speak of war with enthusiasm, without representing it to
myself as the contest of a free nation for her independence.} their
many recollections of the past, their many hopes for the future, every
thing around \page{370} them presents itself as a summons and
encouragement for battle; and in every pulsation of the heart rises a
thought of affection and of honor. God has given this country to men
who can defend it; to women, who, for its sake, consent to the
dangers of their brothers, their husbands, and their sons. At the
approach of the perils which threaten it, a fever, exempt from
shuddering as from delirium, quickens the blood in the veins. Every
effort, in such a struggle, comes from the deepest source of inward
thought. As yet nothing can be seen in the features of these generous
citizens but tranquillity; there is too much dignity in their emotions
for outward demonstration; but let the signal once be heard, let the
banner of their country wave in the air, and you will see those looks,
before so gentle, and so ready to resume that character at the sight
of misfortune, at once animated by a determination holy and terrible!
They shudder no more, neither at wounds nor at blood; it is no longer
pain, it is no longer death, it is an offering to the God of armies;
no regret, no hesitation, now intrudes itself into the most
desperate resolutions; and when the heart is entirely in its object,
then is the highest enjoyment of existence! As soon as man has, within
his own mind, separated himself from himself, to him life is only an
evil; and if it be true, that of all the feelings enthusiasm confers
the greatest happiness, it is because, more than any other, it unites
all the forces of the soul in the same direction for the same end.

The labors of the understanding are considered by many writers as an
occupation almost merely mechanical, and which fills up their life in
the same manner as any other profession. It is still something that
their choice has fallen upon literature; but have such men even an
idea of the sublime happiness of thought when it is animated by
enthusiasm? Do they know the hope which penetrates the soul, when
there arises in it the confident belief, that by the gift of eloquence
we are about to demonstrate and declare some profound truth, some
truth which will be at once a generous bond of union between us and
every soul that sympathizes with ours?

\page{371}Writers without enthusiasm, know of the career of literature
nothing but the criticisms, the rivalries, the jealousies which attend
upon it, and which necessarily must endanger our peace of mind, if we
allow ourselves to be entangled among the passions of men. Unjust
attacks of this nature may, indeed, sometimes do us injury; but, can
the true, the heartfelt internal enjoyment which belongs to talent, be
affected by them? Even at the moment of the first public appearance of
a work, and before its character is yet decided, how many hours of
happiness has it not already been worth to him who wrote it from his
heart, and as an act and office of his worship! How many tears of
rapture has he not shed in his solitude over those wonders of life,
love, glory, and religion! Has he not, in his transports, enjoyed the
air of heaven like a bird; the waters like a thirsty hunter; the
flowers like a lover, who believes that he is breathing the sweets
which surround his mistress? In the world, we have the feeling of
being oppressed beneath our own faculties, and we often suffer from
the consciousness that we are the only one of our own disposition, in
the midst of so many beings, who exist so easily, and at the expense
of so little intellectual exertion; but the creative talent of
imagination, for some moments at least, satisfies all our wishes and
desires; it opens to us treasures of wealth; it offers to us crowns of
glory; it raises before our eyes the pure and bright image of an ideal
world; and so mighty sometimes is its power, that by it we hear in our
hearts the very voice and accents of one whom we have loved.

Does he who is not endowed with an enthusiastic imagination flatter
himself that he is, in any degree, acquainted with the earth upon
which he lives, or that he has travelled through any of its various
countries? Does his heart beat at the echo of the mountains? or has
the air of the South lulled his senses in its voluptuous softness?
Does he perceive wherein countries differ, the one from the other?
Does he remark the accent, and does he understand the peculiar
character of the idioms of their languages? Does he hear in the
popular song, and see in the national dance, the manners and the
genius of \page{372} the people? Does one single sensation at once
fill his mind with a crowd of recollections?

Is Nature to be felt without enthusiasm? Can common men address to her
the tale of their mean interests and low desires? What have the sea
and the stars to answer to the little vanities with which each
individual is content to fill up each day? But if the soul be really
moved within us, if in the universe it seeks a God, even if it be
still sensible to glory and to love, the clouds of heaven will hold
converse with it, the torrents will listen to its voice, and the
breeze that passes through the grove seems to deign to whisper to us
something of those we love.

There are some who, although devoid of enthusiasm, still believe that
they have a taste and relish for the fine arts; and indeed they do
love the refinement of luxury, and they wish to acquire a knowledge of
music and of painting, that they may be able to converse upon them
with ease and with taste, and even with that confidence which becomes
the man of the world, when the subject turns upon imagination, or upon
Nature; but what are these barren pleasures, when compared with true
enthusiasm? What an emotion runs through the brain when we contemplate
in the Niobe that settled look of calm and terrible despair which
seems to reproach the gods with their jealousy of her maternal
happiness! What consolation does the sight of beauty breathe upon us!
Beauty also is from the soul, and pure and noble is the admiration it
inspires. To feel the grandeur of the Apollo demands in the spectator
a pride which tramples under foot all the serpents of the earth. None
but a Christian can penetrate the countenance of the Virgins of
Raphael, and the St. Jerome of Domenichino. None but a Christian can
recognize the same expression in fascinating beauty, and in the
depressed and grief-worn visage; in the brilliancy of youth, and in
features changed by age and disfigured by suffering,---the same
expression which springs from the soul, and which, like a ray of
celestial light, shoots across the early morning of life, or the
closing darkness of age.

\page{373}Can it be said that there is such an art as that of music
for those who cannot feel enthusiasm? Habit may render harmonious
sounds, as it were, a necessary gratification to them, and they enjoy
them as they do the flavor of fruits, or the ornament of colors; but
has their whole being vibrated and trembled responsively, like a lyre,
if at any time the midnight silence has been suddenly broken by the
song, or by any of those instruments which resemble the human voice?
Have they in that moment felt the mystery of their existence in that
softening emotion which reunites our separate natures, and blends in
the same enjoyment the senses of the soul? Have the beatings of the
heart followed the cadence of the music? Have they learned, under the
influence of these emotions so full of charms, to shed those tears
which have nothing of self in them; those tears which do not ask for
the compassion of others, but which relieve ourselves from the
inquietude which arises from the need of something to admire and to
love?

The taste for public spectacles is universal; for the greater part of
mankind have more imagination than they themselves think, and that
which they consider as the allurement of pleasure, as a remnant of the
weakness of childhood which still hangs about them, is often the
better part of their nature; while they are beholding the scenes of
fictions, they are true, natural, and feeling; whereas in the world,
dissimulation, calculation, and vanity, are the absolute masters of
their words, sentiments, and actions. But do they think that they have
felt all that a really fine tragedy can inspire, who find in the
representation of the strongest affections nothing but a diversion and
amusement? Do they doubt and disbelieve that rapturous agitation which
the passions, purified by poetry, excite within us? Ah! how many and
how great are the pleasures which spring from fictions! The interest
they raise is without either apprehension or remorse; and the
sensibility which they call forth has none of that painful harshness
from which real passions are scarcely ever exempt.

What enchantment does not the language of love borrow from poetry and
the fine arts! How beautiful is it to love at \page{374} once with the
heart and with the mind! thus to vary in a thousand fashions a
sentiment which one word is indeed sufficient to express, but for
which all the words of the world are but poverty and weakness! to
submit entirely to the influence of those masterpieces of the
imagination, which all depend upon love, and to discover in the
wonders of nature and genius new expressions to declare the feelings
of our own heart!

What have they known of love who have not reverenced and admired the
woman whom they loved, in whom the sentiment is not a hymn breathed
from the heart, and who do not perceive in grace and beauty the
heavenly image of the most touching passions? What has she felt of
love who has not seen in the object of her choice an exalted
protector, a powerful and a gentle guide, whose look at once commands
and supplicates, and who receives upon his knees the right of
disposing of her fate? How inexpressible is the delight which serious
reflections, united and blended with warm and lively impressions,
produce! The tenderness of a friend, in whose hands our happiness is
deposited, ought, at the gates of the tomb, in the same manner as in
the beautiful days of our youth, to form our chief blessing; and every
thing most serious and solemn in our existence transforms itself into
emotions of delight, when, as in the fable of the ancients, it is the
office of love to light and to extinguish the torch of life.

If enthusiasm fills the soul with happiness, by a strange and wondrous
charm, it forms also its chief support under misfortune; it leaves
behind it a deep trace and a path of light, which do not allow absence
itself to efface us from the hearts of our friends. It affords also to
ourselves an asylum from the utmost bitterness of sorrow, and is the
only feeling which can give tranquillity without indifference.

Even the most simple affections which every heart believes itself
capable of feeling, even filial and maternal love, cannot be felt in
their full strength, unless enthusiasm be blended with them. How can
we love a son without indulging the flattering hope that he will be
generous and gallant, without wishing him that renown which may, as it
were, multiply his existence, \page{375} and make us hear from every
side the name which our own heart is continually repeating? Why should
we not enjoy with rapture the talents of a son, the beauty of a
daughter? Can there be a more strange ingratitude towards the Deity
than indifference for his gifts? Are they not from heaven, since they
render it a more easy task for us to please him whom we love?

Meanwhile, should some misfortune deprive our child of these
advantages, the same sentiment would then assume another form; it
would increase and exalt within us the feeling of compassion, of
sympathy, the happiness of being necessary to him. Under all
circumstances, enthusiasm either animates or consoles; and even in the
moment when the blow, the most cruel that can be struck, reaches us,
when we lose him to whom we owe our own being, him whom we loved as a
tutelary angel, and who inspired us at once with a fearless respect
and a boundless confidence, still enthusiasm comes to our assistance
and support. It brings together within us some sparks of that soul
which has passed away to heaven; we still live before him, and we
promise ourselves that we will one day transmit to posterity the
history of his life. Never, we feel assured, never will his paternal
hand abandon us entirely in this world; and his image, affectionate
and tender, still inclines towards us, to support us, until we are
called unto him.

And in the end, when the hour of trial comes, when it is for us in our
turn to meet the struggle of death, the increasing weakness of our
faculties, the loss and ruin of our hopes, this life, before so
strong, which now begins to give way within us, the crowd of feelings
and ideas which lived within our bosoms, and which the shades of the
tomb already surround and envelope, our interests, our passions, this
existence itself, which lessens to a shadow, before it vanishes
away---all deeply distress us, and the common man appears, when he
expires, to have less of death to undergo. Blessed be God, however,
for the assistance which he has prepared for us even in that moment;
our utterance shall be imperfect, our eyes shall no longer distinguish
the light, our reflections, before clear and connected, shall
wan-\page{376}der vague and confused; but enthusiasm will not abandon
us, her brilliant wings shall wave over the funeral couch; she will
lift the veil of death; she will recall to our recollection those
moments, when, in the fulness of energy, we felt that the heart was
imperishable; and our last sigh shall be a high and generous thought,
reascending to that heaven from which it had its birth.

``O France! land of glory and of love! if the day should ever come
when enthusiasm shall be extinct upon your soil, when all shall be
governed and disposed upon calculation, and even the contempt of
danger shall be founded only upon the conclusions of reason, in that
day what will avail you the loveliness of your climate, the splendor
of your intellect, the general fertility of your nature? Their
intelligent activity, and an impetuosity directed by prudence and
knowledge, may indeed give your children the empire of the world; but
the only traces you will leave on the face of that world will be like
those of the sandy whirlpool, terrible as the waves, and sterile as
the desert!''\footnote{This last sentence is that which excited in the
French police the greatest indignation against my book. It seems to
me, that Frenchmen at least cannot be displeased with it.}

