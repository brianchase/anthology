
\author{Robert Louis Stevenson}
\authdate{1850--1894}
\textdate{1877}
\chapter[Robert Louis Stevenson -- An Apology for Idlers]{An Apology
for Idlers}

\nfootnote{\fullcite{stevenson1887.3}}

% It was first published in Cornhill Magazine in July 1877. I don't
% know how the essay changed between then and the first edition of the
% book, but it underwent some changes, mainly in punctuation, between
% the book's first and second editions.

\begin{quote} \page{107}``\textsc{Boswell}: We grow weary when idle.

``\textsc{Johnson}: That is, sir, because others being busy, we want
company; but if we were idle, there would be no growing weary; we
should all entertain one another.'' \end{quote}

Just now, when every one is bound, under pain of a decree in absence
convicting them of \textit{l\`{e}se}-respectability, to enter on some
lucrative profession, and labour therein with something not far short
of enthusiasm, a cry from the opposite party who are content when they
have enough, and like to look on and enjoy in the meanwhile, savours
a little bravado and gasconade. And yet this should not be. Idleness
so called, which does not consist in doing nothing, but in doing a
great deal not recognised in the dogmatic formularies of the ruling
class, has \page{108} as good a right to state its position as
industry itself. It is admitted that the presence of people who refuse
to enter in the great handicap race for sixpenny pieces, is at once an
insult and a disenchantment for those who do. A fine fellow (as we see
so many) takes his determination, votes for the sixpences, and in the
emphatic Americanism, ``goes for'' them. And while such an one is
ploughing distressfully up the road, it is not hard to understand his
resentment, when he perceives cool persons in the meadows by the
wayside, lying with a handkerchief over their ears and a glass at
their elbow. Alexander is touched in a very delicate place by the
disregard of Diogenes. Where was the glory of having taken Rome for
these tumultuous barbarians, who poured into the Senate house, and
found the Fathers sitting silent and unmoved by their success? It is a
sore thing to have laboured along and scaled the arduous hilltops, and
when all is done, find humanity indifferent to your achievement. Hence
physicists condemn the unphysical; \page{109} financiers have only a
superficial toleration for those who know little of stocks; literary
persons despise the unlettered; and people of all pursuits combine to
disparage those who have none.

But though this is one difficulty of the subject, it is not the
greatest. You could not be put in prison for speaking against
industry, but you can be sent to Coventry for speaking like a fool.
The greatest difficulty with most subjects is to do them well;
therefore, please to remember this is an apology. It is certain that
much may be judiciously argued in favour of diligence; only there is
something to be said against it, and that is what, on the present
occasion, I have to say. To state one argument is not necessarily to
be deaf to all others, and that a man has written a book of travels in
Montenegro, is no reason why he should never have been to Richmond.

It is surely beyond a doubt that people should be a good deal idle in
youth. For though here and there a Lord Macaulay may \page{110} escape
from school honours with all his wits about him, most boys pay so dear
for their medals that they never afterwards have a shot in their
locker, and begin the world bankrupt. And the same holds true during
all the time a lad is educating himself, or suffering others to
educate him. It must have been a very foolish old gentleman who
addressed Johnson at Oxford in these words: ``Young man, ply your book
diligently now, and acquire a stock of knowledge; for when years come
upon you, you will find that poring upon books will be but an irksome
task.'' The old gentleman seems to have been unaware that many other
things besides reading grow irksome, and not a few become impossible,
by the time a man has to use spectacles and cannot walk without a
stick. Books are good enough in their own way, but they are a mighty
bloodless substitute for life. It seems a pity to sit, like the Lady
of Shalott, peering into a mirror, with your back turned on all the
bustle and glamour of reality. And if a man reads very hard, as
\page{111} the old anecdote reminds us, he will have little time for
thought.

If you look back on your own education, I am sure it will not be full,
vivid, instructive hours of truantry that you regret; you would rather
cancel some lack-lustre periods between sleep and waking in the class.
For my own part, I have attended a good many lectures in my time. I
still remember that the spinning of a top is a case of Kinetic
Stability. I still remember that Emphyteusis is not a disease, nor
Stillicide a crime. But though I would not willingly part with such
scraps of science, I do not set the same store by them as by certain
other odds and ends that I came by in the open street while I was
playing truant. This is not the moment to dilate on that mighty place
of education, which was the favourite school of Dickens and of Balzac,
and turns out yearly many inglorious masters in the Science of the
Aspects of Life. Suffice it to say this: if a lad does not learn in
the streets, it is because he has no faculty of \page{112} learning.
Nor is the truant always in the streets, for if he prefers, he may go
out by the gardened suburbs into the country. He may pitch on some
tuft of lilacs over a burn, and smoke innumerable pipes to the tune of
the water on the stones. A bird will sing in the thicket. And there he
may fall into a vein of kindly thought, and see things in a new
perspective. Why, if this be not education, what is? We may conceive
Mr. Worldly Wiseman accosting such an one, and the conversation that
should thereupon ensue:---

``How now, young fellow, what dost thou here?''

``Truly, sir, I take mine ease.''

``Is not this the hour of the class? and should'st thou not be plying
thy Book with diligence, to the end thou mayest obtain knowledge?''

``Nay, but thus also I follow after Learning, by your leave.''

``Learning, quotha! After what fashion, I pray thee? Is it
mathematics?''\page{113}

``No, to be sure.''

``Is it metaphysics?''

``Nor that.''

``Is it some language?''

``Nay, it is no language.''

``Is it a trade?''

``Nor a trade neither.''

``Why, then, what is't?''

``Indeed, sir, as a time may soon come for me to go upon Pilgrimage, I
am desirous to note what is commonly done by persons in my case, and
where are the ugliest Sloughs and Thickets on the Road; as also, what
manner of Staff is of the best service. Moreover, I lie here, by this
water, to learn by root-of-heart a lesson which my master teaches me
to call Peace, or Contentment.''

Hereupon Mr. Worldly Wiseman was much commoved with passion, and
shaking his cane with a very threatful countenance, broke forth upon
this wise: ``Learning, quotha!'' said he; ``I would have all such
rogues scourged by the Hangman!''

And so he would go his way, ruffling out \page{114} his cravat with a
crackle of starch, like a turkey when it spreads its feathers.

Now this, of Mr. Wiseman's, is the common opinion. A fact is not
called a fact, but a piece of gossip, if it does not fall into one of
your scholastic categories. An inquiry must be in some acknowledged
direction, with a name to go by; or else you are not inquiring at all,
only lounging; and the workhouse is too good for you. It is supposed
that all knowledge is at the bottom of a well, or the far end of a
telescope. Sainte-Beuve, as he grew older, came to regard all
experience as a single great book, in which to study for a few years
ere we go hence; and it seemed all one to him whether you should read
in Chapter xx., which is the differential calculus, or in Chapter
xxxix., which is hearing the band play in the gardens. As a matter of
fact, an intelligent person, looking out of his eyes and hearkening in
his ears, with a smile on his face all the time, will get more true
education than many another in a life of heroic vigils. \page{115}
There is certainly some chill and arid knowledge to be found upon the
summits of formal and laborious science; but it is all round about
you, and for the trouble of looking, that you will acquire the warm
and palpitating facts of life. While others are filling their memory
with a lumber of words, one-half of which they will forget before the
week be out, your truant may learn some really useful art: to play the
fiddle, to know a good cigar, or to speak with ease and opportunity to
all varieties of men. Many who have ``plied their book diligently,''
and know all about some one branch or another of accepted lore, come
out of the study with an ancient and owl-like demeanour, and prove
dry, stockish, and dyspeptic in all the better and brighter parts of
life. Many make a large fortune, who remain underbred and pathetically
stupid to the last. And meantime there goes the idler, who began life
along with them---by your leave, a different picture. He has had time
to take care of his health and his spirits; he \page{116} has been a
great deal in the open air, which is the most salutary of all things
for both body and mind; and if he has never read the great Book in
very recondite places, he has dipped into it and skimmed it over to
excellent purpose. Might not the student afford some Hebrew roots, and
the business man some of his half-crowns, for a share of the idler's
knowledge of life at large, and Art of Living? Nay, and the idler has
another and more important quality than these. I mean his wisdom. He
who has much looked on at the childish satisfaction of other people in
their hobbies, will regard his own with only a very ironical
indulgence. He will not be heard among the dogmatists. He will have a
great and cool allowance for all sorts of people and opinions. If he
finds no out-of-the-way truths, he will identify himself with no very
burning falsehood. His way takes him along a by-road, not much
frequented, but very even and pleasant, which is called Commonplace
Lane, and leads to the Belvedere of Commonsense. \page{117} Thence he
shall command an agreeable, if not very noble prospect; and while
others behold the East and West, the Devil and the Sunrise, he will be
contentedly aware of a sort of morning hour upon all sublunary things,
with an army of shadows running speedily and in many different
directions into the great daylight of Eternity. The shadows and the
generations, the shrill doctors and the plangent wars, go by into
ultimate silence and emptiness; but underneath all this, a man may
see, out of the Belvedere windows, much green and peaceful landscape;
many firelit parlours; good people laughing, drinking, and making love
as they did before the Flood or the French Revolution; and the old
shepherd telling his tale under the hawthorn.

Extreme \textit{busyness}, whether at school or college, kirk or
market, is a symptom of deficient vitality; and a faculty for idleness
implies a catholic appetite and a strong sense of personal identity.
There is a sort of dead-alive, hackneyed people about, who \page{118}
are scarcely conscious of living except in the exercise of some
conventional occupation. Bring these fellows into the country, or set
them aboard ship, and you will see how they pine for their desk or
their study. They have no curiosity; they cannot give themselves over
to random provocations; they do not take pleasure in the exercise of
their faculties for its own sake; and unless Necessity lays about them
with a stick, they will even stand still. It is no good speaking to
such folk: they \textit{cannot} be idle, their nature is not generous
enough; and they pass those hours in a sort of coma, which are not
dedicated to furious moiling in the gold-mill. When they do not
require to go to the office, when they are not hungry and have no mind
to drink, the whole breathing world is a blank to them. If they have
to wait an hour or so for a train, they fall into a stupid trance with
their eyes open. To see them, you would suppose there was nothing to
look at and no one to speak with; you would imagine \page{119} they
were paralysed or alienated; and yet very possibly they are hard
workers in their own way, and have good eyesight for a flaw in a deed
or a turn of the market. They have been to school and college, but all
the time they had their eye on the medal; they have gone about in the
world and mixed with clever people, but all the time they were
thinking of their own affairs. As if a man's soul were not too small
to begin with, they have dwarfed and narrowed theirs by a life of all
work and no play; until here they are at forty, with a listless
attention, a mind vacant of all material of amusement, and not one
thought to rub against another, while they wait for the train. Before
he was breeched, he might have clambered on the boxes; when he was
twenty, he would have stared at the girls; but now the pipe is smoked
out, the snuffbox empty, and my gentleman sits bolt upright upon a
bench, with lamentable eyes. This does not appeal to me as being
Success in Life.

\page{120}But it is not only the person himself who suffers from his
busy habits, but his wife and children, his friends and relations, and
down to the very people he sits with in a railway carriage or an
omnibus. Perpetual devotion to what a man calls his business, is only
to be sustained by perpetual neglect of many other things. And it is
not by any means certain that a man's business is the most important
thing he has to do. To an impartial estimate it will seem clear that
many of the wisest, most virtuous, and most beneficent parts that are
to be played upon the Theatre of Life are filled by gratuitous
performers, and pass, among the world at large, as phases of idleness.
For in that Theatre, not only the walking gentlemen, singing
chambermaids, and diligent fiddlers in the orchestra, but those who
look on and clap their hands from the benches, do really play a part
and fulfil important offices towards the general result. You are no
doubt very dependent on the care of your lawyer and stockbroker, of
the guards and \page{121} signalmen who convey you rapidly from place
to place, and the policemen who walk the streets for your protection;
but is there not a thought of gratitude in your heart for certain
other benefactors who set you smiling when they fall in your way, or
season your dinner with good company? Colonel Newcome helped to lose
his friend's money; Fred Bayham had an ugly trick of borrowing shirts;
and yet they were better people to fall among than Mr. Barnes. And
though Falstaff was neither sober nor very honest, I think I could
name one or two long-faced Barabbases whom the world could better have
done without. Hazlitt mentions that he was more sensible of obligation
to Northcote, who had never done him anything he could call a service,
than to his whole circle of ostentatious friends; for he thought a
good companion emphatically the greatest benefactor. I know there are
people in the world who cannot feel grateful unless the favour has
been done them at the cost of pain and difficulty. But this is a
churlish \page{122} disposition. A man may send you six sheets of
letter-paper covered with the most entertaining gossip, or you may
pass half an hour pleasantly, perhaps profitably, over an article of
his; do you think the service would be greater, if he had made the
manuscript in his heart's blood, like a compact with the devil? Do you
really fancy you should be more beholden to your correspondent, if he
had been damning you all the while for your importunity? Pleasures are
more beneficial than duties because, like the quality of mercy, they
are not strained, and they are twice blest. There must always be two
to a kiss, and there may be a score in a jest; but wherever there is
an element of sacrifice, the favour is conferred with pain, and, among
generous people, received with confusion. There is no duty we so much
underrate as the duty of being happy. By being happy, we sow anonymous
benefits upon the world, which remain unknown even to ourselves, or
when they are disclosed, surprise nobody so much as the benefactor.
\page{123} The other day, a ragged, barefoot boy ran down the street
after a marble, with so jolly an air that he set every one he passed
into a good humour; one of these persons, who had been delivered from
more than usually black thoughts, stopped the little fellow and gave
him some money with this remark: ``You see what sometimes comes of
looking pleased.'' If he had looked pleased before, he had now to look
both pleased and mystified. For my part, I justify this encouragement
of smiling rather than tearful children; I do not wish to pay for
tears anywhere but upon the stage; but I am prepared to deal largely
in the opposite commodity. A happy man or woman is a better thing to
find than a five-pound note. He or she is a radiating focus of
goodwill; and their entrance into a room is as though another candle
had been lighted. We need not care whether they could prove the
forty-seventh proposition; they do a better thing than that, they
practically demonstrate the great Theorem of the Liveableness of Life.
Consequently, if a \page{124} person cannot be happy without remaining
idle, idle he should remain. It is a revolutionary precept; but thanks
to hunger and the workhouse, one not easily to be abused; and within
practical limits, it is one of the most incontestable truths in the
whole Body of Morality. Look at one of your industrious fellows for a
moment, I beseech you. He sows hurry and reaps indigestion; he puts a
vast deal of activity out to interest, and receives a large measure of
nervous derangement in return. Either he absents himself entirely from
all fellowship, and lives a recluse in a garret, with carpet slippers
and a leaden inkpot; or he comes among people swiftly and bitterly, in
a contraction of his whole nervous system, to discharge some temper
before he returns to work. I do not care how much or how well he
works, this fellow is an evil feature in other people's lives. They
would be happier if he were dead. They could easier do without his
services in the Circumlocution Office, than they can tolerate his
fractious spirits. He \page{125} poisons life at the well-head. It is
better to be beggared out of hand by a scapegrace nephew, than daily
hag-ridden by a peevish uncle.

And what, in God's name, is all this pother about? For what cause do
they embitter their own and other people's lives? That a man should
publish three or thirty articles a year, that he should finish or not
finish his great allegorical picture, are questions of little interest
to the world. The ranks of life are full; and although a thousand
fall, there are always some to go into the breach. When they told Joan
of Arc she should be at home minding women's work, she answered there
were plenty to spin and wash. And so, even with your own rare gifts!
When nature is ``so careless of the single life,'' why should we
coddle ourselves into the fancy that our own is of exceptional
importance? Suppose Shakespeare had been knocked on the head some dark
night in Sir Thomas Lucy's preserves, the world would have wagged on
better or worse, the pitcher \page{126} gone to the well, the scythe
to the corn, and the student to his book; and no one been any the
wiser of the loss. There are not many works extant, if you look the
alternative all over, which are worth the price of a pound of tobacco
to a man of limited means. This is a sobering reflection for the
proudest of our earthly vanities. Even a tobacconist may, upon
consideration, find no great cause for personal vainglory in the
phrase; for although tobacco is an admirable sedative, the qualities
necessary for retailing it are neither rare nor precious in
themselves. Alas and alas! you may take it how you will, but the
services of no single individual are indispensable. Atlas was just a
gentleman with a protracted nightmare! And yet you see merchants who
go and labour themselves into a great fortune and thence into the
bankruptcy court; scribblers who keep scribbling at little articles
until their temper is a cross to all who come about them, as though
Pharaoh should set the Israelites to make a pin instead of a pyramid;
and fine \page{127} young men who work themselves into a decline, and
are driven off in a hearse with white plumes upon it. Would you not
suppose these persons had been whispered, by the Master of the
Ceremonies, the promise of some momentous destiny? and that this
lukewarm bullet on which they play their farces was the bull's-eye and
centrepoint of all the universe? And yet it is not so. The ends for
which they give away their priceless youth, for all they know, may be
chimerical or hurtful; the glory and riches they expect may never
come, or may find them indifferent; and they and the world they
inhabit are so inconsiderable that the mind freezes at the thought.

