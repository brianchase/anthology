
\author{James Madison}
\authdate{1751--1836}
\textdate{1787}
\chapter{The Federalist. No. X.}
\source{madison1787.10}

% The 1868 Dawson edition of The Federalist, which this selection
% uses, is not only in the public domain but follows the format of the
% original publications in periodicals.

\page{55}\begin{abstract}{c}{[From the New York Packet, Friday,
November 23, 1787.]}\end{abstract}

\noindent\textsc{To the People of the State of New York}:

Among the numerous advantages promised by a well-constructed Union,
none deserves to be more accurately developed than its tendency to
break and control the violence of faction. The friend of popular
Governments never finds himself so much alarmed for their character
and fate, as when he contemplates their propensity to this dangerous
vice. He will not fail, therefore, to set a due value on any plan
which, without violating the principles to which he is attached,
provides a proper cure for it. The instability, injustice, and
confusion introduced into the public councils, have, in truth, been
the mortal diseases under which popular Governments have everywhere
perished; as they continue to be the favorite and fruitful topics from
which the adversaries to liberty derive their most specious
declamations. The valuable improvements made by the American
Constitutions on the popular models, both ancient and modern, cannot
certainly be too much admired; but it would be an unwarrantable
partiality, to contend that they have as effectually obviated the
danger on this side, as was wished and expected. Complaints are every
where heard from our most considerate and virtuous citizens, equally
the friends of public and private faith, and of public and personal
liberty, that our Governments are too unstable; that the public good
is disregarded in the conflicts of rival parties; and that measures
are too often decided, not according to the rules of justice, and the
rights of the minor party, but by the superior force \page{56} of an
interested and overbearing majority. However anxiously we may wish
that these complaints had no foundation, the evidence of known facts
will not permit us to deny that they are in some degree true. It will
be found, indeed, on a candid review of our situation, that some of
the distresses under which we labor have been erroneously charged on
the operation of our Governments; but it will be found, at the same
time, that other causes will not alone account for many of our
heaviest misfortunes; and, particularly, for that prevailing and
increasing distrust of public engagements, and alarm for private
rights, which are echoed from one end of the continent to the other.
These must be chiefly, if not wholly, effects of the unsteadiness and
injustice, with which a factious spirit has tainted our public
administrations.

By a faction, I understand a number of citizens, whether amounting to
a majority or minority of the whole, who are united and actuated by
some common impulse of passion, or of interest, adverse to the rights
of other citizens, or to the permanent and aggregate interests of the
community.

There are two methods of curing the mischiefs of faction: the one, by
removing its causes; the other, by controlling its effects.

There are again two methods of removing the causes of faction: the
one, by destroying the liberty which is essential to its existence;
the other, by giving to every citizen the same opinions, the same
passions, and the same interests.

It could never be more truly said than of the first remedy, that it
was worse than the disease. Liberty is to faction, what air is to
fire, an aliment without which it instantly expires. But it could not
be less folly to abolish liberty, which is essential to political
life, because it nourishes faction, than it would be to wish the
\page{57} annihilation of air, which is essential to animal life,
because it imparts to fire its destructive agency.

The second expedient is as impracticable, as the first would be
unwise. As long as the reason of man continues fallible, and he is at
liberty to exercise it, different opinions will be formed. As long as
the connection subsists between his reason and his self-love, his
opinions and his passions will have a reciprocal influence on each
other; and the former will be objects to which the latter will attach
themselves. The diversity in the faculties of men, from which the
rights of property originate, is not less an insuperable obstacle to
an uniformity of interests. The protection of these faculties is the
first object of Government. From the protection of different and
unequal faculties of acquiring property, the possession of different
degrees and kinds of property immediately results; and from the
influence of these on the sentiments and views of the respective
proprietors, ensues a division of the society into different interests
and parties.

The latent causes of faction are thus sown in the nature of man; and
we see them everywhere brought into different degrees of activity,
according to the different circumstances of civil society. A zeal for
different opinions concerning religion, concerning Government, and
many other points, as well of speculation as of practice; an
attachment to different leaders ambitiously contending for
pre\"{e}minence and power; or to persons of other descriptions whose
fortunes have been interesting to the human passions, have, in turn,
divided mankind into parties, inflamed them with mutual animosity, and
rendered them much more disposed to vex and oppress each other, than
to co\"{o}perate for their common good. So strong is this propensity
of mankind to fall into mutual animosities, that where no substantial
occasion presents itself, the most frivolous and fanciful
dis-\page{58}tinctions have been sufficient to kindle their unfriendly
passions, and excite their most violent conflicts. But the most common
and durable source of factions has been the various and unequal
distribution of property. Those who hold, and those who are without
property, have ever formed distinct interests in society. Those who
are creditors, and those who are debtors, fall under a like
discrimination. A landed interest, a manufacturing interest, a
mercantile interest, a moneyed interest, with many lesser interests,
grow up of necessity in civilized nations, and divide them into
different classes, actuated by different sentiments and views. The
regulation of these various and interfering interests forms the
principal task of modern Legislation, and involves the spirit of party
and faction in the necessary and ordinary operations of the Government.

No man is allowed to be a judge in his own cause; because his interest
would certainly bias his judgment, and, not improbably, corrupt his
integrity. With equal, nay with greater reason, a body of men are
unfit to be both judges and parties at the same time; yet what are
many of the most important acts of legislation, but so many judicial
determinations, not indeed concerning the rights of single persons,
but concerning the rights of large bodies of citizens? and what are
the different classes of Legislators, but advocates and parties to the
causes which they determine? Is a law proposed concerning private
debts? It is a question to which the creditors are parties on one
side, and the debtors on the other. Justice ought to hold the balance
between them. Yet the parties are, and must be, themselves the judges;
and the most numerous party, or, in other words, the most powerful
faction, must be expected to prevail. Shall domestic manufactures be
encouraged, and in what degree, by restrictions on foreign
manufactures? are questions which would be differently decided by the
\page{59} landed and the manufacturing classes; and probably by
neither, with a sole regard to justice and the public good. The
apportionment of taxes on the various descriptions of property is an
act which seems to require the most exact impartiality; yet there is,
perhaps, no legislative act in which greater opportunity and
temptation are given to a predominant party, to trample on the rules
of justice. Every shilling, with which they overburden the inferior
number, is a shilling saved to their own pockets.

It is in vain to say, that enlightened statesmen will be able to
adjust these clashing interests, and render them all subservient to
the public good. Enlightened statesmen will not always be at the helm:
Nor, in many cases, can such an adjustment be made at all, without
taking into view indirect and remote considerations, which will rarely
prevail over the immediate interest which one party may find in
disregarding the rights of another, or the good of the whole.

The inference to which we are brought is, that the \textit{causes} of
faction cannot be removed; and that relief is only to be sought in the
means of controlling its \textit{effects}.

If a faction consists of less than a majority, relief is supplied by
the republican principle, which enables the majority to defeat its
sinister views by regular vote. It may clog the administration, it may
convulse the society; but it will be unable to execute and mask its
violence under the forms of the Constitution. When a majority is
included in a faction, the form of popular Government, on the other
hand, enables it to sacrifice to its ruling passion or interest both
the public good and the rights of other citizens. To secure the public
good, and private rights, against the danger of such a faction, and at
the same time to preserve the spirit and the form of popular
Government, is then the great object to which \page{60} our inquiries
are directed: Let me add, that it is the great desideratum, by which
alone this form of Government can be rescued from the opprobrium under
which it has so long labored, and be recommended to the esteem and
adoption of mankind.

By what means is this object attainable? Evidently by one of two only.
Either the existence of the same passion or interest in a majority, at
the same time, must be prevented; or the majority, having such
coexistent passion or interest, must be rendered, by their number and
local situation, unable to concert and carry into effect schemes of
oppression. If the impulse and the opportunity be suffered to
coincide, we well know that neither moral nor religious motives can be
relied on as an adequate control. They are not found to be such on the
injustice and violence of individuals, and lose their efficacy in
proportion to the number combined together; that is, in proportion as
their efficacy becomes needful.

From this view of the subject, it may be concluded, that a pure
Democracy, by which I mean a Society consisting of a small number of
citizens, who assemble and administer the Government in person, can
admit of no cure for the mischiefs of faction. A common passion or
interest will, in almost every case, be felt by a majority of the
whole; a communication and concert results from the form of Government
itself; and there is nothing to check the inducements to sacrifice the
weaker party, or an obnoxious individual. Hence it is, that such
Democracies have ever been spectacles of turbulence and contention;
have ever been found incompatible with personal security, or the
rights of property; and have in general been as short in their lives,
as they have been violent in their deaths. Theoretic politicians, who
have patronized this species of Government, have erroneously supposed,
that by reducing mankind to a perfect equality in their political
rights, they would, at the \page{61} same time, be perfectly equalized
and assimilated in their possessions, their opinions, and their
passions.

A Republic, by which I mean a Government in which the scheme of
representation takes place, opens a different prospect, and promises
the cure for which we are seeking. Let us examine the points in which
it varies from pure Democracy, and we shall comprehend both the nature
of the cure, and the efficacy which it must derive from the Union.

The two great points of difference, between a Democracy and a
Republic, are, first, the delegation of the Government, in the latter,
to a small number of citizens elected by the rest: Secondly, the
greater number of citizens, and greater sphere of country, over which
the latter may be extended.

The effect of the first difference is, on the one hand, to refine and
enlarge the public views, by passing them through the medium of a
chosen body of citizens, whose wisdom may best discern the true
interest of their country, and whose patriotism and love of justice
will be least likely to sacrifice it to temporary or partial
considerations. Under such a regulation, it may well happen, that the
public voice, pronounced by the representatives of the People, will be
more consonant to the public good, than if pronounced by the People
themselves, convened for the purpose. On the other hand, the effect
may be inverted. Men of factious tempers, of local prejudices, or of
sinister designs, may by intrigue, by corruption, or by other means,
first obtain the suffrages, and then betray the interests of the
people. The question resulting is, whether small or extensive
Republics are most favorable to the election of proper guardians of
the public weal; and it is clearly decided in favor of the latter by
two obvious considerations.

In the first place, it is to be remarked that however small the
Republic may be, the Representatives must be \page{62} raised to a
certain number, in order to guard against the cabals of a few; and
that however large it may be, they must be limited to a certain
number, in order to guard against the confusion of a multitude. Hence,
the number of Representatives in the two cases not being in proportion
to that of the Constituents, and being proportionally greatest in the
small Republic, it follows, that if the proportion of fit characters
be not less in the large than in the small Republic, the former will
present a greater option, and consequently a greater probability of a
fit choice.

In the next place, as each Representative will be chosen by a greater
number of citizens in the large than in the small Republic, it will be
more difficult for unworthy candidates to practise with success the
vicious arts, by which elections are too often carried; and the
suffrages of the People, being more free, will be more likely to
centre in men who possess the most attractive merit, and the most
diffusive and established characters.

It must be confessed, that in this, as in most other cases, there is a
mean, on both sides of which inconveniences will be found to lie. By
enlarging too much the number of electors, you render the
representative too little acquainted with all their local
circumstances and lesser interests; as by reducing it too much, you
render him unduly attached to these, and too little fit to comprehend
and pursue great and National objects. The F\oe ederal Constitution
forms a happy combination in this respect; the great and aggregate
interests being referred to the National, the local and particular to
the State Legislatures.

The other point of difference is, the greater number of citizens and
extent of territory which may be brought within the compass of
Republican, than of Democratic Government; and it is this circumstance
principally which renders factious combinations less to be dreaded
\page{63} in the former, than in the latter. The smaller the society,
the fewer probably will be the distinct parties and interests
composing it; the fewer the distinct parties and interests, the more
frequently will a majority be found of the same party; and the smaller
the number of individuals composing a majority, and the smaller the
compass within which they are placed, the more easily will they
concert and execute their plans of oppression. Extend the sphere, and
you take in a greater variety of parties and interests; you make it
less probable that a majority of the whole will have a common motive
to invade the rights of other citizens; or if such a common motive
exists, it will be more difficult for all who feel it to discover
their own strength, and to act in unison with each other. Besides
other impediments, it may be remarked, that where there is a
consciousness of unjust or dishonorable purposes, communication is
always checked by distrust, in proportion to the number whose
concurrence is necessary.

Hence, it clearly appears, that the same advantage which a Republic
has over a Democracy, in controlling the effects of faction, is
enjoyed by a large over a small Republic,---is enjoyed by the Union
over the States composing it. Does this advantage consist in the
substitution of Representatives, whose enlightened views and virtuous
sentiments render them superior to local prejudices, and to schemes of
injustice? It will not be denied, that the Representation of the Union
will be most likely to possess these requisite endowments. Does it
consist in the greater security afforded by a greater variety of
parties, against the event of any one party being able to outnumber
and oppress the rest? In an equal degree does the increased variety of
parties, comprised within the Union, increase this security. Does it,
in fine, consist in the greater obstacles opposed to the concert and
accomplishment of the secret wishes \page{64} of an unjust and
interested majority? Here, again, the extent of the Union gives it the
most palpable advantage.

% NOTE: the source has a semicolon after 'source' below, but that is
% clearly supposed to be a period

The influence of factious leaders may kindle a flame within their
particular States, but will be unable to spread a general
conflagration through the other States: A religious sect may
degenerate into a political faction in a part of the Confederacy; but
the variety of sects dispersed over the entire face of it, must secure
the National Councils against any danger from that source. A rage for
paper money, for an abolition of debts, for an equal division of
property, or for any other improper or wicked project, will be less
apt to pervade the whole body of the Union, than a particular member
of it; in the same proportion as such a malady is more likely to taint
a particular county or district, than an entire State.

In the extent and proper structure of the Union, therefore, we behold
a Republican remedy for the diseases most incident to Republican
Government. And according to the degree of pleasure and pride we feel
in being Republicans, ought to be our zeal in cherishing the spirit,
and supporting the character, of F\oe deralists.

\hfill\textsc{Publius}.

