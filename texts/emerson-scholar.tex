
\author{Ralph Waldo Emerson}
\authdate{1803--1882}
\textdate{1837}
\chapter{The American Scholar}
\source{emerson1903.2}

% NOTE: The text below appears on an unnumbered title page.

\begin{center}An Oration delivered before the Phi Beta Kappa
Society,\\at Cambridge, August 31, 1837.\end{center}

\page{81}\noindent\textsc{Mr. President and Gentlemen}:

I greet you on the recommencement of our literary year. Our
anniversary is one of hope, and, perhaps, not enough of labor. We do
not meet for games of strength or skill, for the recitation of
histories, tragedies, and odes, like the ancient Greeks; for
parliaments of love and poesy, like the Troubadours; nor for the
advancement of science, like our contemporaries in the British and
European capitals. Thus far, our holiday has been simply a friendly
sign of the survival of the love of letters amongst a people too busy
to give to letters any more. As such it is precious as the sign of an
indestructible instinct. Perhaps the time is already come when it
ought to be, and will be, something else; when the sluggard intellect
of this continent will look from under its iron lids and fill the
postponed expectation of the world with something better than the
exertions of mechanical skill. Our day of dependence, our long
apprenticeship to the learning of other lands, draws to a close. 
\page{82} The millions that around us are rushing into life, cannot
always be fed on the sere remains of foreign harvests. Events, actions
arise, that must be sung, that will sing themselves. Who can doubt
that poetry will revive and lead in a new age, as the star in the
constellation Harp, which now flames in our zenith, astronomers
announce, shall one day be the pole-star for a thousand years?

In this hope I accept the topic which not only usage but the nature of
our association seem to prescribe to this day,---the \textsc{American
Scholar}. Year by year we come up hither to read one more chapter of
his biography. Let us inquire what light new days and events have
thrown on his character and his hopes.

It is one of those fables which out of an unknown antiquity convey an
unlooked-for wisdom, that the gods, in the beginning, divided Man into
men, that he might be more helpful to himself; just as the hand was
divided into fingers, the better to answer its end.

The old fable covers a doctrine ever new and sublime; that there is
One Man,---pre\-sent to all particular men only partially, or through
one faculty; and that you must take the whole society to find the
whole man. Man is not a farmer, \page{83} or a professor, or an
engineer, but he is all. Man is priest, and scholar, and statesman,
and producer, and soldier. In the \textit{divided} or social state
these functions are parcelled out to individuals, each of whom aims to
do his stint of the joint work, whilst each other performs his. The
fable implies that the individual, to possess himself, must sometimes
return from his own labor to embrace all the other laborers. But,
unfortunately, this original unit, this fountain of power, has been so
distributed to multitudes, has been so minutely subdivided and peddled
out, that it is spilled into drops, and cannot be gathered. The state
of society is one in which the members have suffered amputation from
the trunk, and strut about so many walking monsters,---a good finger,
a neck, a stomach, an elbow, but never a man.

Man is thus metamorphosed into a thing, into many things. The planter,
who is Man sent out into the field to gather food, is seldom cheered
by any idea of the true dignity of his ministry. He sees his bushel
and his cart, and nothing beyond, and sinks into the farmer, instead
of Man on the farm. The tradesman scarcely ever gives an ideal worth
to his work, but is ridden by the routine of his craft, and the soul
is sub-\page{84}ject to dollars. The priest becomes a form; the
attorney a statute-book; the mechanic a machine; the sailor a rope of
the ship.

In this distribution of functions the scholar is the delegated
intellect. In the right state he is \textit{Man Thinking}. In the
degenerate state, when the victim of society, he tends to become a
mere thinker, or still worse, the parrot of other men's thinking.

In this view of him, as Man Thinking, the theory of his office is
contained. Him Nature solicits with all her placid, all her monitory
pictures; him the past instructs; him the future invites. Is not
indeed every man a student, and do not all things exist for the
student's behoof? And, finally, is not the true scholar the only true
master? But the old oracle said, ``All things have two handles: beware
of the wrong one.'' In life, too often, the scholar errs with mankind
and forfeits his privilege. Let us see him in his school, and consider
him in reference to the main influences he receives.

\vspace{1\baselineskip}

I. The first in time and the first in importance of the influences
upon the mind is that of nature. Every day, the sun; and, after
sunset, Night and her stars. Ever the winds blow; ever \page{85} the
grass grows. Every day, men and women, con\-ver\-sing---be\-hold\-ing
and beholden. The scholar is he of all men whom this spectacle most
engages. He must settle its value in his mind. What is nature to him?
There is never a beginning, there is never an end, to the
inexplicable continuity of this web of God, but always circular power
returning into itself. Therein it resembles his own spirit, whose
beginning, whose ending, he never can find,---so entire, so boundless.
Far too as her splendors shine, system on system shooting like rays,
upward, downward, without centre, without cir\-cum\-fer\-ence,---in
the mass and in the particle, Nature hastens to render account of
herself to the mind. Classification begins. To the young mind every
thing is individual, stands by itself. By and by, it finds how to join
two things and see in them one nature; then three, then three
thousand; and so, tyrannized over by its own unifying instinct, it
goes on tying things together, diminishing anomalies, discovering
roots running under ground whereby contrary and remote things cohere
and flower out from one stem. It presently learns that since the dawn
of history there has been a constant accumulation and classifying of
facts. But what is classification but the perceiving that \page{85}
these objects are not chaotic, and are not foreign, but have a law
which is also a law of the human mind? The astronomer discovers that
geometry, a pure abstraction of the human mind, is the measure of
planetary motion. The chemist finds proportions and intelligible
method throughout matter; and science is nothing but the finding of
analogy, identity, in the most remote parts. The ambitious soul sits
down before each refractory fact; one after another reduces all
strange constitutions, all new powers, to their class and their law,
and goes on forever to animate the last fibre of organization, the
outskirts of nature, by insight.

Thus to him, to this schoolboy under the bending dome of day, is
suggested that he and it proceed from one root; one is leaf and one is
flower; relation, sympathy, stirring in every vein. And what is that
root? Is not that the soul of his soul? A thought too bold; a dream
too wild. Yet when this spiritual light shall have revealed the law of
more earthly na\-tures,---when he has learned to worship the soul, and
to see that the natural philosophy that now is, is only the first
gropings of its gigantic hand, he shall look forward to an ever
expanding knowledge as to a becoming creator. He shall see \page{87}
that nature is the opposite of the soul, answering to it part for
part. One is seal and one is print. Its beauty is the beauty of his
own mind. Its laws are the laws of his own mind. Nature then becomes
to him the measure of his attainments. So much of nature as he is
ignorant of, so much of his own mind does he not yet possess. And, in
fine, the ancient precept, ``Know thyself,'' and the modern precept,
``Study nature,'' become at last one maxim.

II. The next great influence into the spirit of the scholar is the
mind of the Past,---in whatever form, whether of literature, of art,
of institutions, that mind is inscribed. Books are the best type of
the influence of the past, and perhaps we shall get at the
truth,---learn the amount of this influence more
con\-ven\-ient\-ly,---by considering their value alone.

The theory of books is noble. The scholar of the first age received
into him the world around; brooded thereon; gave it the new
arrangement of his own mind, and uttered it again. It came into him
life; it went out from him truth. It came to him short-lived actions;
it went out from him immortal thoughts. It came to him business; it
went from him poetry. It was dead fact; now, it is quick thought. It
can \page{88} stand, and it can go. It now endures, it now flies, it
now inspires. Precisely in proportion to the depth of mind from which
it issued, so high does it soar, so long does it sing.

Or, I might say, it depends on how far the process had gone, of
transmuting life into truth. In proportion to the completeness of the
distillation, so will the purity and imperishableness of the product
be. But none is quite perfect. As no air-pump can by any means make a
perfect vacuum, so neither can any artist entirely exclude the
conventional, the local, the perishable from his book, or write a book
of pure thought, that shall be as efficient, in all respects, to a
remote posterity, as to contemporaries, or rather to the second age.
Each age, it is found, must write its own books; or rather, each
generation for the next succeeding. The books of an older period will
not fit this.

Yet hence arises a grave mischief. The sacredness which attaches to
the act of creation, the act of thought, is transferred to the record.
The poet chanting was felt to be a divine man: henceforth the chant is
divine also. The writer was a just and wise spirit: henceforward it is
settled the book is perfect; as love of the hero corrupts into worship
of his statue. Instantly \page{89} the book becomes noxious: the guide
is a tyrant. The sluggish and perverted mind of the multitude, slow to
open to the incursions of Reason, having once so opened, having once
received this book, stands upon it, and makes an outcry if it is
disparaged. Colleges are built on it. Books are written on it by
thinkers, not by Man Thinking; by men of talent, that is, who start
wrong, who set out from accepted dogmas, not from their own sight of
principles. Meek young men grow up in libraries, believing it their
duty to accept the views which Cicero, which Locke, which Bacon, have
given; forgetful that Cicero, Locke, and Bacon were only young men in
libraries when they wrote these books.

Hence, instead of Man Thinking, we have the bookworm. Hence the
book-learn\-ed class, who value books, as such; not as related to
nature and the human constitution, but as making a sort of Third
Estate with the world and the soul. Hence the restorers of readings,
the emendators, the bibliomaniacs of all degrees.

Books are the best of things, well used; abused, among the worst. What
is the right use? What is the one end which all means go to effect?
They are for nothing but to inspire. I had bet-\page{90}ter never see
a book than to be warped by its attraction clean out of my own orbit,
and made a satellite instead of a system. The one thing in the world,
of value, is the active soul. This every man is entitled to; this
every man contains within him, although in almost all men obstructed
and as yet unborn. The soul active sees absolute truth and utters
truth, or creates. In this action it is genius; not the privilege of
here and there a favorite, but the sound estate of every man. In its
essence it is progressive. The book, the college, the school of art,
the institution of any kind, stop with some past utterance of genius.
This is good, say they,---let us hold by this. They pin me down. They
look backward and not forward. But genius looks forward: the eyes of
man are set in his forehead, not in his hindhead: man hopes: genius
creates. Whatever talents may be, if the man create not, the pure
efflux of the Deity is not his;---cin\-ders and smoke there may be,
but not yet flame. There are creative manners, there are creative
actions, and creative words; manners, actions, words, that is,
indicative of no custom or authority, but springing spontaneous from
the mind's own sense of good and fair.

On the other part, instead of being its own \page{91} seer, let it
receive from another mind its truth, though it were in torrents of
light, without periods of solitude, inquest, and self-recovery, and a
fatal disservice is done. Genius is always sufficiently the enemy of
genius by over-influence. The literature of every nation bears me
witness. The English dramatic poets have Shakspearized now for two
hundred years.

Undoubtedly there is a right way of reading, so it be sternly
subordinated. Man Thinking must not be subdued by his instruments.
Books are for the scholar's idle times. When he can read God directly,
the hour is too precious to be wasted in other men's transcripts of
their readings. But when the intervals of darkness come, as come they
must,---when the sun is hid and the stars withdraw their
shin\-ing,---we repair to the lamps which were kindled by their ray,
to guide our steps to the East again, where the dawn is. We hear, that
we may speak. The Arabian proverb says, ``A fig tree, looking on a fig
tree, becometh fruitful.''

It is remarkable, the character of the pleasure we derive from the
best books. They impress us with the conviction that one nature wrote
and the same reads. We read the verses of one of the great English
poets, of Chaucer, of Marvell, \page{92} of Dryden, with the most
modern joy,---with a pleasure, I mean, which is in great part caused
by the abstraction of all \textit{time} from their verses. There is
some awe mixed with the joy of our surprise, when this poet, who lived
in some past world, two or three hundred years ago, says that which
lies close to my own soul, that which I also had well-nigh thought and
said. But for the evidence thence afforded to the philosophical
doctrine of the identity of all minds, we should suppose some
pre\"{e}stablished harmony, some foresight of souls that were to be,
and some preparation of stores for their future wants, like the fact
observed in insects, who lay up food before death for the young grub
they shall never see.

I would not be hurried by any love of system, by any exaggeration of
instincts, to underrate the Book. We all know, that as the human body
can be nourished on any food, though it were boiled grass and the
broth of shoes, so the human mind can be fed by any knowledge. And
great and heroic men have existed who had almost no other information
than by the printed page. I only would say that it needs a strong head
to bear that diet. One must be an inventor to read well. As the
proverb says, ``He that would bring home the wealth of the Indies,
must \page{93} carry out the wealth of the Indies.'' There is then
creative reading as well as creative writing. When the mind is braced
by labor and invention, the page of whatever book we read becomes
luminous with manifold allusion. Every sentence is doubly significant,
and the sense of our author is as broad as the world. We then see,
what is always true, that as the seer's hour of vision is short and
rare among heavy days and months, so is its record, perchance, the
least part of his volume. The discerning will read, in his Plato or
Shakspeare, only that least part,---only the authentic utterances of
the or\-a\-cle;---all the rest he rejects, were it never so many times
Plato's and Shakspeare's.

Of course there is a portion of reading quite indispensable to a wise
man. History and exact science he must learn by laborious reading.
Colleges, in like manner, have their indispensable of\-fice,---to
teach elements. But they can only highly serve us when they aim not to
drill, but to create; when they gather from far every ray of various
genius to their hospitable halls, and by the concentrated fires, set
the hearts of their youth on flame. Thought and knowledge are natures
in which apparatus and pretension avail nothing. Gowns and pecuniary
foundations, \page{94} though of towns of gold, can never countervail
the least sentence or syllable of wit. Forget this, and our American
colleges will recede in their public importance, whilst they grow
richer every year.

III. There goes in the world a notion that the scholar should be a
recluse, a val\-e\-tu\-di\-nar\-i\-an,---as unfit for any handiwork or
public labor as a penknife for an axe. The so-called ``practical men''
sneer at speculative men, as if, because they speculate or
\textit{see}, they could do nothing. I have heard it said that the
cler\-gy,---who are always, more universally than any other class, the
scholars of their day,---are addressed as women; that the rough,
spontaneous conversation of men they do not hear, but only a mincing
and diluted speech. They are often virtually disfranchised; and indeed
there are advocates for their celibacy. As far as this is true of the
studious classes, it is not just and wise. Action is with the
scholar subordinate, but it is essential. Without it he is not yet
man. Without it thought can never ripen into truth. Whilst the world
hangs before the eye as a cloud of beauty, we cannot even see its
beauty. Inaction is cowardice, but there can be no scholar without the
heroic mind. The preamble of thought, the transition \page{95} through
which it passes from the unconscious to the conscious, is action. Only
so much do I know, as I have lived. Instantly we know whose words
are loaded with life, and whose not.

The world,---this shadow of the soul, or \textit{other me},---lies
wide a\-round. Its attractions are the keys which unlock my thoughts
and make me acquainted with myself. I run eagerly into this resounding
tumult. I grasp the hands of those next me, and take my place in the
ring to suffer and to work, taught by an instinct that so shall the
dumb abyss be vocal with speech. I pierce its order; I dissipate its
fear; I dispose of it within the circuit of my expanding life. So much
only of life as I know by experience, so much of the wilderness have I
vanquished and planted, or so far have I extended my being, my
dominion. I do not see how any man can afford, for the sake of his
nerves and his nap, to spare any action in which he can partake. It is
pearls and rubies to his discourse. Drudgery, calamity, exasperation,
want, are instructors in eloquence and wisdom. The true scholar
grudges every opportunity of action past by, as a loss of power. It is
the raw material out of which the intellect moulds her splendid
pro-\page{96}ducts. A strange process too, this by which experience is
converted into thought, as a mulberry leaf is converted into satin.
The manufacture goes forward at all hours.

The actions and events of our childhood and youth are now matters of
calmest observation. They lie like fair pictures in the air. Not so
with our recent ac\-tions,---with the business which we now have in
hand. On this we are quite unable to speculate. Our affections as yet
circulate through it. We no more feel or know it than we feel the
feet, or the hand, or the brain of our body. The new deed is yet a
part of life,---re\-mains for a time immersed in our unconscious life.
In some contemplative hour it detaches itself from the life like a
ripe fruit, to become a thought of the mind. Instantly it is raised,
transfigured; the corruptible has put on incorruption. Henceforth it
is an object of beauty, however base its origin and neighborhood.
Observe too the impossibility of antedating this act. In its grub
state, it cannot fly, it cannot shine, it is a dull grub. But
suddenly, without observation, the selfsame thing unfurls beautiful
wings, and is an angel of wisdom. So is there no fact, no event, in
our private history, which shall not, sooner or later, lose its
\page{97} adhesive, inert form, and astonish us by soaring from our
body into the empyrean. Cradle and infancy, school and playground, the
fear of boys, and dogs, and ferules, the love of little maids and
berries, and many another fact that once filled the whole sky, are
gone already; friend and relative, profession and party, town and
country, nation and world, must also soar and sing.

% In the Centenary Edition, 'flower-pot' is broken by a line break,
% which might lead you to think that Emerson wrote 'flowerpot'. But in
% the Fireside Edition, where it does not appear at the end of a line,
% the word is 'flower-pot'

Of course, he who has put forth his total strength in fit actions has
the richest return of wisdom. I will not shut myself out of this globe
of action, and transplant an oak into a flower-pot, there to hunger
and pine; nor trust the revenue of some single faculty, and exhaust
one vein of thought, much like those Savoyards, who, getting their
livelihood by carving shepherds, shepherdesses, and smoking Dutchmen,
for all Europe, went out one day to the mountain to find stock, and
discovered that they had whittled up the last of their pine trees.
Authors we have, in numbers, who have written out their vein, and who,
moved by a commendable prudence, sail for Greece or Palestine, follow
the trapper into the prairie, or ramble round Algiers, to replenish
their merchantable stock.

If it were only for a vocabulary, the scholar \page{98} would be
covetous of action. Life is our dictionary. Years are well spent in
country labors; in town; in the insight into trades and manufactures;
in frank intercourse with many men and women; in science; in art; to
the one end of mastering in all their facts a language by which to
illustrate and embody our perceptions. I learn immediately from any
speaker how much he has already lived, through the poverty or the
splendor of his speech. Life lies behind us as the quarry from whence
we get tiles and copestones for the masonry of to-day. This is the way
to learn grammar. Colleges and books only copy the language which the
field and the work-yard made.

But the final value of action, like that of books, and better than
books, is that it is a resource. That great principle of Undulation in
nature, that shows itself in the inspiring and expiring of the breath;
in desire and satiety; in the ebb and flow of the sea; in day and
night; in heat and cold; and, as yet more deeply ingrained in every
atom and every fluid, is known to us under the name of
Po\-lar\-i\-ty,---these ``fits of easy transmission and reflection,''
as Newton called them, are the law of nature because they are the law
of spirit.

\page{99}The mind now thinks, now acts, and each fit reproduces the
other. When the artist has exhausted his materials, when the fancy no
longer paints, when thoughts are no longer apprehended and books are a
wear\-i\-ness,---he has always the resource \textit{to live}.
Character is higher than intellect. Thinking is the function. Living
is the functionary. The stream retreats to its source. A great soul
will be strong to live, as well as strong to think. Does he lack organ
or medium to impart his truths? He can still fall back on this
elemental force of living them. This is a total act. Thinking is a
partial act. Let the grandeur of justice shine in his affairs. Let the
beauty of affection cheer his lowly roof. Those ``far from fame,'' who
dwell and act with him, will feel the force of his constitution in the
doings and passages of the day better than it can be measured by any
public and designed display. Time shall teach him that the scholar
loses no hour which the man lives. Herein he unfolds the sacred germ
of his instinct, screened from influence. What is lost in seemliness
is gained in strength. Not out of those on whom systems of education
have exhausted their culture, comes the helpful giant to destroy the
old or to build the new, but out of unhandselled \page{100} savage
nature; out of terrible Druids and Berserkers come at last Alfred and
Shakspeare.

I hear therefore with joy whatever is beginning to be said of the
dignity and necessity of labor to every citizen. There is virtue yet
in the hoe and the spade, for learned as well as for unlearned hands.
And labor is everywhere welcome; always we are invited to work; only
be this limitation observed, that a man shall not for the sake of
wider activity sacrifice any opinion to the popular judgments and
modes of action.

\vspace{1\baselineskip}

I have now spoken of the education of the scholar by nature, by books,
and by action. It remains to say somewhat of his duties.

They are such as become Man Thinking. They may all be comprised in
self-trust. The office of the scholar is to cheer, to raise, and to
guide men by showing them facts amidst appearances. He plies the slow,
unhonored, and unpaid task of observation. Flamsteed and Herschel, in
their glazed observatories, may catalogue the stars with the praise of
all men, and the results being splendid and useful, honor is sure. But
he, in his private observatory, cataloguing obscure and nebulous stars
of the hu-\page{101}man mind, which as yet no man has thought of as
such,---watch\-ing days and months sometimes for a few facts;
correcting still his old re\-cords;---must relinquish display and
immediate fame. In the long period of his preparation he must betray
often an ignorance and shiftlessness in popular arts, incurring the
disdain of the able who shoulder him aside. Long he must stammer in
his speech; often forego the living for the dead. Worse yet, he must
ac\-cept---how often!---po\-ver\-ty and solitude. For the ease and
pleasure of treading the old road, accepting the fashions, the
education, the religion of society, he takes the cross of making his
own, and, of course, the self-accusation, the faint heart, the
frequent uncertainty and loss of time, which are the nettles and
tangling vines in the way of the self-relying and self-directed; and
the state of virtual hostility in which he seems to stand to society,
and especially to educated society. For all this loss and scorn, what
offset? He is to find consolation in exercising the highest functions
of human nature. He is one who raises himself from private
considerations and breathes and lives on public and illustrious
thoughts. He is the world's eye. He is the world's heart. He is to
resist the vulgar prosperity that retro-\page{102}grades ever to
barbarism, by preserving and communicating heroic sentiments, noble
biographies, melodious verse, and the conclusions of history.
Whatsoever oracles the human heart, in all emergencies, in all solemn
hours, has uttered as its commentary on the world of
ac\-tions,---these he shall receive and impart. And whatsoever new
verdict Reason from her inviolable seat pronounces on the passing men
and events of to-day,---this he shall hear and promulgate.

These being his functions, it becomes him to feel all confidence in
himself, and to defer never to the popular cry. He and he only knows
the world. The world of any moment is the merest appearance. Some
great decorum, some fetish of a government, some ephemeral trade, or
war, or man, is cried up by half mankind and cried down by the other
half, as if all depended on this particular up or down. The odds are
that the whole question is not worth the poorest thought which the
scholar has lost in listening to the controversy. Let him not quit his
belief that a popgun is a popgun, though the ancient and honorable of
the earth affirm it to be the crack of doom. In silence, in
steadiness, in severe abstraction, let him hold by himself; add
\page{103} observation to observation, patient of neglect, patient of
reproach, and bide his own time,---hap\-py enough if he can satisfy
himself alone that this day he has seen something truly. Success
treads on every right step. For the instinct is sure, that prompts him
to tell his brother what he thinks. He then learns that in going down
into the secrets of his own mind he has descended into the secrets of
all minds. He learns that he who has mastered any law in his private
thoughts, is master to that extent of all men whose language he
speaks, and of all into whose language his own can be translated. The
poet, in utter solitude remembering his spontaneous thoughts and
recording them, is found to have recorded that which men in crowded
cities find true for them also. The orator distrusts at first the
fitness of his frank confessions, his want of knowledge of the persons
he addresses, until he finds that he is the complement of his
hear\-ers;---that they drink his words because he fulfils for them
their own nature; the deeper he dives into his privatest, secretest
presentiment, to his wonder he finds this is the most acceptable, most
public, and universally true. The people delight in it; the better
part of every man feels, This is my music; this is myself.

\page{104}In self-trust all the virtues are comprehended. Free should
the scholar be,---free and brave. Free even to the definition of
freedom, ``without any hindrance that does not arise out of his own
constitution.'' Brave; for fear is a thing which a scholar by his very
function puts behind him. Fear always springs from ignorance. It is a
shame to him if his tranquillity, amid dangerous times, arise from the
presumption that like children and women his is a protected class; or
if he seek a temporary peace by the diversion of his thoughts from
politics or vexed questions, hiding his head like an ostrich in the
flowering bushes, peeping into microscopes, and turning rhymes, as a
boy whistles to keep his courage up. So is the danger a danger still;
so is the fear worse. Manlike let him turn and face it. Let him look
into its eye and search its nature, inspect its or\-i\-gin,---see the
whelping of this li\-on,---which lies no great way back; he will then
find in himself a perfect comprehension of its nature and extent; he
will have made his hands meet on the other side, and can henceforth
defy it and pass on superior. The world is his who can see through its
pretension. What deafness, what stone-blind custom, what overgrown
error you behold is there only by suffer-\page{105}ance,---by your
sufferance. See it to be a lie, and you have already dealt it its
mortal blow.

Yes, we are the cowed,---we the trustless. It is a mischievous notion
that we are come late into nature; that the world was finished a long
time ago. As the world was plastic and fluid in the hands of God, so
it is ever to so much of his attributes as we bring to it. To
ignorance and sin, it is flint. They adapt themselves to it as they
may; but in proportion as a man has any thing in him divine, the
firmament flows before him and takes his signet and form. Not he is
great who can alter matter, but he who can alter my state of mind.
They are the kings of the world who give the color of their present
thought to all nature and all art, and persuade men by the cheerful
serenity of their carrying the matter, that this thing which they do
is the apple which the ages have desired to pluck, now at last ripe,
and inviting nations to the harvest. The great man makes the great
thing. Wherever Macdonald sits, there is the head of the table.
Linn{\ae}us makes botany the most alluring of studies, and wins it
from the farmer and the herb-woman; Davy, chemistry; and Cuvier,
fossils. The day is always his who works in it with serenity and great
aims. The unstable estimates \page{106} of men crowd to him whose
mind is filled with a truth, as the heaped waves of the Atlantic
follow the moon.

% In the Centenary Edition, 'downtrod' is broken by a line break. In
% the Fireside Edition, where it doesn't appear at the end of a line,
% the word is 'downtrod', not 'down-trod'.

For this self-trust, the reason is deeper than can be
fath\-omed,---dark\-er than can be enlightened. I might not carry with
me the feeling of my audience in stating my own belief. But I have
already shown the ground of my hope, in adverting to the doctrine that
man is one. I believe man has been wronged; he has wronged himself. He
has almost lost the light that can lead him back to his prerogatives.
Men are become of no account. Men in history, men in the world of
to-day, are bugs, are spawn, and are called ``the mass'' and ``the
herd.'' In a century, in a millennium, one or two men; that is to say,
one or two approximations to the right state of every man. All the
rest behold in the hero or the poet their own green and crude
be\-ing,---rip\-ened; yes, and are content to be less, so
\textit{that} may attain to its full stature. What a testimony, full
of grandeur, full of pity, is borne to the demands of his own nature,
by the poor clansman, the poor partisan, who rejoices in the glory of
his chief. The poor and the low find some amends to their immense
moral capacity, for their acquiescence in a political and social
inferi-\page{107}ority. They are content to be brushed like flies from
the path of a great person, so that justice shall be done by him to
that common nature which it is the dearest desire of all to see
enlarged and glorified. They sun themselves in the great man's light,
and feel it to be their own element. They cast the dignity of man from
their downtrod selves upon the shoulders of a hero, and will perish to
add one drop of blood to make that great heart beat, those giant
sinews combat and conquer. He lives for us, and we live in him.

Men, such as they are, very naturally seek money or power; and power
because it is as good as mon\-ey,---the ``spoils,'' so called, ``of
office.'' And why not? for they aspire to the highest, and this, in
their sleep-walking, they dream is highest. Wake them and they shall
quit the false good and leap to the true, and leave governments to
clerks and desks. This revolution is to be wrought by the gradual
domestication of the idea of Culture. The main enterprise of the world
for splendor, for extent, is the upbuilding of a man. Here are the
materials strewn along the ground. The private life of one man shall
be a more illustrious monarchy, more formidable to its enemy, more
sweet and serene in its influence to its friend, than any kingdom in
history. For \page{108} a man, rightly viewed, comprehendeth the
particular natures of all men. Each philosopher, each bard, each actor
has only done for me, as by a delegate, what one day I can do for
myself. The books which once we valued more than the apple of the eye,
we have quite exhausted. What is that but saying that we have come up
with the point of view which the universal mind took through the eyes
of one scribe; we have been that man, and have passed on. First, one,
then another, we drain all cisterns, and waxing greater by all these
supplies, we crave a better and more abundant food. The man has never
lived that can feed us ever. The human mind cannot be enshrined in a
person who shall set a barrier on any one side to this unbounded,
unboundable empire. It is one central fire, which, flaming now out of
the lips of Etna, lightens the capes of Sicily, and now out of the
throat of Vesuvius, illuminates the towers and vineyards of Naples. It
is one light which beams out of a thousand stars. It is one soul which
animates all men.

\vspace{1\baselineskip}

But I have dwelt perhaps tediously upon this abstraction of the
Scholar. I ought not to delay longer to add what I have to say of
nearer reference to the time and to this country.

\page{109}Historically, there is thought to be a difference in the
ideas which predominate over successive epochs, and there are data for
marking the genius of the Classic, of the Romantic, and now of the
Reflective or Philosophical age. With the views I have intimated of
the oneness or the identity of the mind through all individuals, I do
not much dwell on these differences. In fact, I believe each
individual passes through all three. The boy is a Greek; the youth,
romantic; the adult, reflective. I deny not, however, that a
revolution in the leading idea may be distinctly enough traced.

Our age is bewailed as the age of Introversion. Must that needs be
evil? We, it seems, are critical; we are embarrassed with second
thoughts; we cannot enjoy any thing for hankering to know whereof the
pleasure consists; we are lined with eyes; we see with our feet; the
time is infected with Hamlet's unhappiness,---

\begin{center}\small``Sicklied o'er with the pale cast of
thought.''\end{center}

\noindent It is so bad then? Sight is the last thing to be pitied.
Would we be blind? Do we fear lest we should outsee nature and God,
and drink truth dry? I look upon the discontent of the literary class
as a mere announcement of the fact \page{110} that they find
themselves not in the state of mind of their fathers, and regret the
coming state as untried; as a boy dreads the water before he has
learned that he can swim. If there is any period one would desire to
be born in, is it not the age of Revolution; when the old and the new
stand side by side and admit of being compared; when the energies of
all men are searched by fear and by hope; when the historic glories of
the old can be compensated by the rich possibilities of the new era?
This time, like all times, is a very good one, if we but know what to
do with it.

I read with some joy of the auspicious signs of the coming days, as
they glimmer already through poetry and art, through philosophy and
science, through church and state.

One of these signs is the fact that the same movement which effected
the elevation of what was called the lowest class in the state,
assumed in literature a very marked and as benign an aspect. Instead
of the sublime and beautiful, the near, the low, the common, was
explored and poetized. That which had been negligently trodden under
foot by those who were harnessing and provisioning themselves for long
journeys into far countries, is suddenly found to be richer \page{111}
than all foreign parts. The literature of the poor, the feelings of
the child, the philosophy of the street, the meaning of household
life, are the topics of the time. It is a great stride. It is a
sign---is it not?---of new vigor when the extremities are made active,
when currents of warm life run into the hands and the feet. I ask not
for the great, the remote, the romantic; what is doing in Italy or
Arabia; what is Greek art, or Provencal minstrelsy; I embrace the
common, I explore and sit at the feet of the familiar, the low. Give
me insight into to-day, and you may have the antique and future
worlds. What would we really know the meaning of? The meal in the
firkin; the milk in the pan; the ballad in the street; the news of the
boat; the glance of the eye; the form and the gait of the body;---show
me the ultimate reason of these matters; show me the sublime presence
of the highest spiritual cause lurking, as always it does lurk, in
these suburbs and extremities of nature; let me see every trifle
bristling with the polarity that ranges it instantly on an eternal
law; and the shop, the plough, and the ledger referred to the like
cause by which light undulates and poets sing;---and the world lies no
longer a dull miscellany and lumber-room, \page{112} but has form and
order; there is no trifle, there is no puzzle, but one design unites
and animates the farthest pinnacle and the lowest trench.

This idea has inspired the genius of Goldsmith, Burns, Cowper, and, in
a newer time, of Goethe, Wordsworth, and Carlyle. This idea they have
differently followed and with various success. In contrast with their
writing, the style of Pope, of Johnson, of Gibbon, looks cold and
pedantic. This writing is blood-warm. Man is surprised to find that
things near are not less beautiful and wondrous than things remote.
The near explains the far. The drop is a small ocean. A man is related
to all nature. This perception of the worth of the vulgar is fruitful
in discoveries. Goethe, in this very thing the most modern of the
moderns, has shown us, as none ever did, the genius of the ancients.

There is one man of genius who has done much for this philosophy of
life, whose literary value has never yet been rightly estimated;---I
mean Emanuel Swedenborg. The most imaginative of men, yet writing with
the precision of a mathematician, he endeavored to engraft a purely
philosophical Ethics on the popular Christianity of his time. Such an
attempt of course must have difficulty which no genius \page{113}
could surmount. But he saw and showed the connection between nature
and the affections of the soul. He pierced the emblematic or spiritual
character of the visible, audible, tangible world. Especially did his
shade-loving muse hover over and interpret the lower parts of nature;
he showed the mysterious bond that allies moral evil to the foul
material forms, and has given in epical parables a theory of insanity,
of beasts, of unclean and fearful things.

Another sign of our times, also marked by an analogous political
movement, is the new importance given to the single person. Every
thing that tends to insulate the individual,---to surround him with
barriers of natural respect, so that each man shall feel the world is
his, and man shall treat with man as a sovereign state with a
sovereign state,---tends to true union as well as greatness. ``I
learned,'' said the melancholy Pestalozzi, ``that no man in God's wide
earth is either willing or able to help any other man.'' Help must
come from the bosom alone. The scholar is that man who must take up
into himself all the ability of the time, all the contributions of the
past, all the hopes of the future. He must be an university of
knowledges. If there be one lesson more than another which \page{114}
should pierce his ear, it is, The world is nothing, the man is all; in
yourself is the law of all nature, and you know not yet how a globule
of sap ascends; in yourself slumbers the whole of Reason; it is for
you to know all; it is for you to dare all. Mr. President and
Gentlemen, this confidence in the unsearched might of man belongs, by
all motives, by all prophecy, by all preparation, to the American
Scholar. We have listened too long to the courtly muses of Europe. The
spirit of the American freeman is already suspected to be timid,
imitative, tame. Public and private avarice make the air we breathe
thick and fat. The scholar is decent, indolent, complaisant. See
already the tragic consequence. The mind of this country, taught to
aim at low objects, eats upon itself. There is no work for any but the
decorous and the complaisant. Young men of the fairest promise, who
begin life upon our shores, inflated by the mountain winds, shined
upon by all the stars of God, find the earth below not in unison with
these, but are hindered from action by the disgust which the
principles on which business is managed inspire, and turn drudges,
or die of disgust, some of them suicides. What is the remedy? They did
not yet see, and thousands of young men as hopeful \page{115} now
crowding to the barriers for the career do not yet see, that if the
single man plant himself indomitably on his instincts, and there
abide, the huge world will come round to him. Patience,---pa\-tience;
with the shades of all the good and great for company; and for solace
the perspective of your own infinite life; and for work the study
and the communication of principles, the making those instincts
prevalent, the conversion of the world. Is it not the chief disgrace
in the world, not to be an u\-nit;---not to be reckoned one
char\-ac\-ter;---not to yield that peculiar fruit which each man was
created to bear, but to be reckoned in the gross, in the hundred, or
the thousand, of the party, the section, to which we belong; and our
opinion predicted geographically, as the north, or the south? Not
so, brothers and friends---please God, ours shall not be so. We will
walk on our own feet; we will work with our own hands; we will speak
our own minds. The study of letters shall be no longer a name for
pity, for doubt, and for sensual indulgence. The dread of man and the
love of man shall be a wall of defence and a wreath of joy around all.
A nation of men will for the first time exist, because each believes
himself inspired by the Divine Soul which also inspires all men.

