
\author{William Paley}
\authdate{1743--1805}
\textdate{1802}
\chapter[William Paley -- Natural Theology, chaps. 1 and 2]{Natural
Theology\\\smaller Chapters 1 and 2}

\nfootnote{\fullcite{paley1811}}

\page{1}\section*{Chapter I. State of the Argument}

In crossing a heath, suppose I pitched my foot against a
\textit{stone}, and were asked how the stone came to be there; I might
possibly answer, that, for any thing I knew to the contrary, it had
lain there for ever: nor would it perhaps be very easy to show the
absurdity of this answer. But suppose I had found a \textit{watch}
upon the ground, and it should be inquired how the watch happened to
be in that place; I should hardly think of the answer I had before
given, that, for any thing I knew, the watch might have always been
there. Yet why should not this answer serve for the watch as well as
for the stone? why is it not as admissible in the second case, as in
the first? For this reason, \page{2} and for no other, viz. that, when
we come to inspect the watch, we perceive (what we could not discover
in the stone) that its several parts are framed and put together for a
purpose, \textit{e.g.} that they are so formed and adjusted as to
produce motion, and that motion so regulated as to point out the hour
of the day; that, if the different parts had been differently shaped
from what they are, of a different size from what they are, or placed
after any other manner, or in any other order, than that in which they
are placed, either no motion at all would have been carried on in the
machine, or none which would have answered the use that is now served
by it. To reckon up a few of the plainest of these parts, and of their
offices, all tending to one re\-sult:---We see a cylindrical box
containing a coiled elastic spring, which, by its endeavour to relax
itself, turns round the box. We next observe a flexible chain
(artificially wrought for the sake of flexure), communicating the
action of the spring from the box to the fusee. We then find a series
of wheels, the teeth of which catch in, and apply to, each other,
conducting the motion from the fusee to the balance, and from the
balance to the pointer; and at the same time, by the \page{3} size and
shape of those wheels, so regulating that motion, as to terminate in
causing an index, by an equable and measured progression, to pass over
a given space in a given time. We take notice that the wheels are
made of brass in order to keep them from rust; the springs of steel,
no other metal being so elastic; that over the face of the watch there
is placed a glass, a material employed in no other part of the work,
but in the room of which, if there had been any other than a
transparent substance, the hour could not be seen without opening the
case. This mechanism being observed (it requires indeed an examination
of the instrument, and perhaps some previous knowledge of the subject,
to perceive and understand it; but being once, as we have said,
observed and understood) the inference, we think, is inevitable, that
the watch must have had a maker: that there must have existed, at some
time, and at some place or other, an artificer or artificers who
formed it for the purpose which we find it actually to answer; who
comprehended its construction, and designed its use.

I. Nor would it, I apprehend, weaken the conclusion, that we had never
seen a watch made; that we had never known an artist \page{4} capable
of making one; that we were altogether incapable of executing such a
piece of workmanship ourselves, or of understanding in what manner it
was performed; all this being no more than what is true of some
exquisite remains of ancient art, of some lost arts, and, to the
generality of mankind, of the more curious productions of modern
manufacture. Does one man in a million know how oval frames are
turned? Ignorance of this kind exalts our opinion of the unseen and
unknown, but raises no doubts in our minds of the existence and agency
of such an artist, at some former time, and in some place or other.
Nor can I perceive that it varies at all the inference, whether the
question arise concerning a human agent, or concerning an agent of a
different species, or an agent possessing, in some respects, a
different nature.

II. Neither, secondly, would it invalidate our conclusion, that the
watch sometimes went wrong, or that it seldom went exactly right. The
purpose of the machinery, the design, and the designer, might be
evident, and in the case supposed would be evident, in whatever way we
accounted for the irregu-\page{5}larity of the movement, or whether we
could account for it or not. It is not necessary that a machine be
perfect, in order to show with what design it was made: still less
necessary, where the only question is, whether it were made with any
design at all.

III. Nor, thirdly, would it bring any uncertainty into the argument,
if there were a few parts of the watch, concerning which we could not
discover, or had not yet discovered, in what manner they conduced to
the general effect; or even some parts, concerning which we could not
ascertain, whether they conduced to that effect in any manner
whatever. For, as to the first branch of the case; if by the loss, or
disorder, or decay of the parts in question, the movement of the watch
were found in fact to be stopped, or disturbed, or retarded, no doubt
would remain in our minds as to the utility or intention of these
parts, although we should be unable to investigate the manner
according to which, or the connexion by which, the ultimate effect
depended upon their action or assistance; and the more complex is the
machine, the more likely is this obscurity to arise. Then, as to the
second thing supposed, namely, that there were parts which might be
spared, without prejudice to the \page {6} movement of the watch, and
that we had proved this by experiment,---these superfluous parts, even
if we were completely assured that they were such, would not vacate
the reasoning which we had instituted concerning other parts. The
indication of contrivance remained, with respect to them, nearly as it
was before.

IV. Nor, fourthly, would any man in his senses think the existence of
the watch, with its various machinery, accounted for, by being told
that it was one out of possible combinations of material forms; that
whatever he had found in the place where he found the watch, must
have contained some internal configuration or other; and that this
configuration might be the structure now exhibited, viz. of the works
of a watch, as well as a different structure.

V. Nor, fifthly, would it yield his inquiry more satisfaction to be
answered, that there existed in things a principle of order, which had
disposed the parts of the watch into their present form and situation.
He never knew a watch made by the principle of order; nor can he even
form to himself an idea of what is meant by a principle of order,
distinct from the intelligence of the watch-maker.

VI. Sixthly, he would be surprised to hear \page{7} that the mechanism
of the watch was no proof of contrivance, only a motive to induce the
mind to think so:

VII. And not less surprised to be informed, that the watch in his hand
was nothing more than the result of the laws of \textit{metallic}
nature. It is a perversion of language to assign any laws, as the
efficient, operative cause of any thing. A law presupposes an agent;
for it is only the mode, according to which an agent proceeds: it
implies a power; for it is the order, according to which that power
acts. Without this agent, without this power, which are both distinct
from itself, the \textit{law} does nothing; is nothing. The
expression, ``the law of metallic nature,'' may sound strange and
harsh to a philosophic ear; but it seems quite as justifiable as some
others which are more familiar to him, such as ``the law of vegetable
nature,'' ``the law of animal nature,'' or indeed as ``the law of
nature'' in general, when assigned as the cause of ph{\ae}nomena, in
exclusion of agency and power; or when it is substituted into the
place of these.

VIII. Neither, lastly, would our observer be driven out of his
conclusion, or from his confidence in its truth, by being told that he
knew nothing at all about the matter. He \page{8} knows enough for his
argument: he knows the utility of the end: he knows the subserviency
and adaptation of the means to the end. These points being known, his
ignorance of other points, his doubts concerning other points,
affect not the certainty of his reasoning. The consciousness of
knowing little, need not beget a distrust of that which he does
know.

\section*{Chapter II. State of the Argument Continued}

Suppose, in the next place, that the person, who found the watch,
should, after sometime, discover that, in addition to all the
properties which he had hitherto observed in it, it possessed the
unexpected property of producing, in the course of its movement,
another watch like itself (the thing is conceivable); that it
contained within it a mechanism, a system of parts, a mould for
instance, or a complex adjustment of lathes, files, and other tools,
evidently and separately calculated for this purpose; let us
enquire, what effect ought such a discovery to have upon his former
conclusion.

\page{9}I. The first effect would be to increase his admiration of the
contrivance, and his conviction of the consummate skill of the
contriver. Whether he regarded the object of the contrivance, the
distinct apparatus, the intricate, yet in many parts intelligible
mechanism, by which it was carried on, he would perceive, in this new
observation, nothing but an additional reason for doing what he had
already done,---for referring the construction of the watch to design,
and to supreme art. If that construction \textit{without} this
property, or, which is the same thing, before this property had been
noticed, proved intention and art to have been employed about it;
still more strong would the proof appear, when he came to the
knowledge of this further property, the crown and perfection of all
the rest.

II. He would reflect, that though the watch before him were,
\textit{in some sense}, the maker of the watch, which was fabricated
in the course of its movements, yet it was in a very different sense
from that, in which a carpenter, for instance, is the maker of a
chair; the author of its contrivance, the cause of the relation of its
parts to their use. With respect to these, the first watch was no
cause at all to the second: in no such sense as this was it the author
of the constitution and order, either \page{10} of the parts which the
new watch contained, or of the parts by the aid and instrumentality of
which it was produced. We might possibly say, but with great latitude
of expression, that a stream of water ground corn: but no latitude of
expression would allow us to say, no stretch of conjecture could lead
us to think, that the stream of water built the mill, though it were
too ancient for us to know who the builder was. What the stream of
water does in the affair, is neither more nor less than this; by the
application of an unintelligent impulse to a mechanism previously
arranged, arranged independently of it, and arranged by intelligence,
an effect is produced, viz. the corn is ground. But the effect results
from the arrangement. The force of the stream cannot be said to be the
cause or author of the effect, still less of the arrangement.
Understanding and plan in the formation of the mill were not the less
necessary, for any share which the water has in grinding the corn: yet
is this share the same, as that which the watch would have contributed
to the production of the new watch, upon the supposition assumed in
the last section. Therefore,

III. Though it be now no longer probable, that the individual watch,
which our observer had found, was made immediately by the hand
\page{11} of an artificer, yet doth not this alteration in anywise
affect the inference, that an artificer had been originally employed
and concerned in the production. The argument from design remains as
it was. Marks of design and contrivance are no more accounted for now,
than they were before. In the same thing, we may ask for the cause of
different properties. We may ask for the cause of the colour of a
body, of its hardness, of its heat; and these causes may be all
different. We are now asking for the cause of that subserviency to a
use, that relation to an end, which we have remarked in the watch
before us. No answer is given to this question, by telling us that a
preceding watch produced it. There cannot be a design without a
designer; contrivance without a contriver; order without choice;
arrangement, without any thing capable of arranging; subserviency and
relation to a purpose, without that which could intend a purpose;
means suitable to an end, and executing their office in accomplishing
that end, without the end ever having been contemplated, or the means
accommodated to it. Arrangement, disposition of parts, subserviency of
means to an end, relation of instruments to a use, imply the presence
of intelligence and mind. No one, therefore, can \page{12} rationally
believe, that the insensible, inanimate watch, from which the watch
before us issued, was the proper cause of the mechanism we so much
admire in it;---could be truly said to have constructed the
instrument, disposed its parts, assigned their office, determined
their order, action, and mutual dependency, combined their several
motions into one result, and that also a result connected with the
utilities of other beings. All these properties, therefore, are as
much unaccounted for, as they were before.

IV. Nor is any thing gained by running the difficulty farther back,
\textit{i. e.} by supposing the watch before us to have been produced
from another watch, that from a former, and so on indefinitely. Our
going back ever so far, brings us no nearer to the least degree of
satisfaction upon the subject. Contrivance is still unaccounted for.
We still want a contriver. A designing mind is neither supplied by
this supposition, nor dispensed with. If the difficulty were
diminished the further we went back, by going back indefinitely we
might exhaust it. And this is the only case to which this sort of
reasoning applies. Where there is a tendency, or, as we increase the
number of terms, a continual approach towards a limit, \textit{there},
by supposing the num-\page{13}ber of terms to be what is called
infinite, we may conceive the limit to be attained: but where there is
no such tendency, or approach, nothing is effected by lengthening the
series. There is no difference as to the point in question (whatever
there may be as to many points), between one series and another;
between a series which is finite, and a series which is infinite. A
chain, composed of an infinite number of links, can no more support
itself, than a chain composed of a finite number of links. And of this
we are assured (though we never \textit{can} have tried the
experiment), because, by increasing the number of links, from ten for
instance to a hundred, from a hundred to a thousand, \&c. we make not
the smallest approach, we observe not the smallest tendency, towards
self-support. There is no difference in this respect (yet there may be
a great difference in several respects) between a chain of a greater
or less length, between one chain and another, between one that is
finite and one that is infinite. This very much resembles the case
before us. The machine which we are inspecting, demonstrates, by its
construction, contrivance and design. Contrivance must have had a
contriver; design, a designer; whether the machine immediately
proceeded from another machine or \page{14} not. That circumstance
alters not the case. That other machine may, in like manner, have
proceeded from a former machine: nor does that alter the case;
contrivance must have had a contriver. That former one from one
preceding it: no alteration still; a contriver is still necessary. No
tendency is perceived, no approach towards a diminution of this
necessity. It is the same with any and every succession of these
machines; a succession of ten, of a hundred, of a thousand; with one
series, as with another; a series which is finite, as with a series
which is infinite. In whatever other respects they may differ, in this
they do not. In all equally, contrivance and design are unaccounted
for.

The question is not simply, How came the first watch into existence?
which question, it may be pretended, is done away by supposing the
series of watches thus produced from one another to have been
infinite, and consequently to have had no such \textit{first}, for
which it was necessary to provide a cause. This, perhaps, would have
been nearly the state of the question, if nothing had been before us
but an unorganized, unmechanized substance, without mark or indication
of contrivance. It might be difficult to show that such substance
could \page{15} not have existed from eternity, either in succession
(if it were possible, which I think it is not, for unorganized bodies
to spring from one another), or by individual perpetuity. But that is
not the question now. To suppose it to be so, is to suppose that it
made no difference whether we had found a watch or a stone. As it is,
the metaphysics of that question have no place; for, in the watch
which we are examining, are seen contrivance, design; an end, a
purpose; means for the end, adaptation to the purpose. And the
question which irresistibly presses upon our thoughts, is, whence this
contrivance and design? The thing required is, the intending mind, the
adapting hand, the intelligence by which that hand was directed. This
question, this demand, is not shaken off, by increasing a number or
succession of substances, destitute of these properties; nor the more,
by increasing that number to infinity. If it be said, that, upon the
supposition of one watch being produced from another in the course of
that other's movements, and by means of the mechanism within it, we
have a cause for the watch in my hand, viz. the watch from which it
proceeded: I deny, that for the design, the contrivance, the
suitableness of means to an end, the adaptation of instruments to a
use (all which \page{16} we discover in a watch), we have any cause
whatever. It is in vain, therefore, to assign a series of such causes,
or to allege that a series may be carried back to infinity; for I do
not admit that we have yet any cause at all of the phenomena, still
less any series of causes either finite or infinite. Here is
contrivance, but no contriver; proofs of design, but no designer.

V. Our observer would further also reflect, that the maker of the
watch before him, was, in truth and reality, the maker of every watch
produced from it; there being no difference (except that the latter
manifests a more exquisite skill) between the making of another watch
with his own hands, by the mediation of files, lathes, chisels, \&c.
and the disposing, fixing, and inserting, of these instruments, or of
others equivalent to them, in the body of the watch already made in
such a manner, as to form a new watch in the course of the movements
which he had given to the old one. It is only working by one set of
tools, instead of another.

The conclusion which the \textit{first} examination of the watch, of
its works, construction, and movement, suggested, was, that it must
have had, for the cause and author of that construction, an artificer,
who understood its \page{17} mechanism, and designed its use. This
conclusion is invincible. A \textit{second} examination presents us
with a new discovery. The watch is found, in the course of its
movement, to produce another watch, similar to itself; and not only
so, but we perceive in it a system or organization, separately
calculated for that purpose. What effect would this discovery have, or
ought it to have, upon our former inference? What, as hath already
been said, but to increase, beyond measure, our admiration of the
skill, which had been employed in the formation of such a machine? Or
shall it, instead of this, all at once turn us round to an opposite
conclusion, viz. that no art or skill whatever has been concerned in
the business, although all other evidences of art and skill remain as
they were, and this last and supreme piece of art be now added to the
rest? Can this be maintained without absurdity? Yet this is atheism.

