
\author{Ren\'e Descartes}
\authdate{1596--1650}
\textdate{1641}
\chapter[Ren\'e Descartes -- Meditiations on First Philosophy, 1 and
2]{Meditations on First Philosophy\\\smaller First and Second
Meditations}

\nfootnote{\fullcite{descartes1911.1.3}}

\page{144}\section*{Meditation I.}

\begin{center}\textit{Of the things which may be brought within the
sphere of the doubtful.}\end{center}

It is now some years since I detected how many were the false beliefs
that I had from my earliest youth admitted as true, and how doubtful
was everything I had since constructed on this basis; and from that
time I was convinced that I must once for all seriously undertake to
rid myself of all the opinions which I had formerly accepted, and
commence to build anew from the foundation, if I wanted to establish
any firm and permanent structure in the sciences. But as this
enterprise appeared to be a very great one, I waited until I had
attained an age so mature that I could not hope that at any later date
I should be better fitted to execute my design. This reason caused me
to delay so long that I should feel that I was doing wrong were I to
occupy in deliberation the time that yet remains to me for action.
To-day, then, since very opportunely for the plan I have in view I
have delivered my mind from every care [and am happily agitated by no
passions] and since I have procured for myself an assured leisure in a
peaceable retirement, I shall at last seriously and freely address
myself to the general upheaval of all my former opinions.

\page{145}Now for this object it is not necessary that I should show
that all of these are false---I shall perhaps never arrive at this
end. But inasmuch as reason already persuades me that I ought no less
carefully to withhold my assent from matters which are not entirely
certain and indubitable than from those which appear to me
manifestly to be false, if I am able to find in each one some reason
to doubt, this will suffice to justify my rejecting the whole. And for
that end it will not be requisite that I should examine each in
particular, which would be an endless undertaking; for owing to the
fact that the destruction of the foundations of necessity brings with
it the downfall of the rest of the edifice, I shall only in the first
place attack those principles upon which all my former opinions
rested.

All that up to the present time I have accepted as most true and
certain I have learned either from the senses or through the senses;
but it is sometimes proved to me that these senses are deceptive, and
it is wiser not to trust entirely to anything by which we have once
been deceived.

But it may be that although the senses sometimes deceive us concerning
things which are hardly perceptible, or very far away, there are yet
many others to be met with as to which we cannot reasonably have any
doubt, although we recognise them by their means. For example, there
is the fact that I am here, seated by the fire, attired in a dressing
gown, having this paper in my hands and other similar matters. And how
could I deny that these hands and this body are mine, were it not
perhaps that I compare myself to certain persons, devoid of sense,
whose cerebella are so troubled and clouded by the violent vapours of
black bile, that they constantly assure us that they think they are
kings when they are really quite poor, or that they are clothed in
purple when they are really without covering, or who imagine that they
have an earthenware head or are nothing but pumpkins or are made of
glass. But they are mad, and I should not be any the less insane were
I to follow examples so extravagant.

At the same time I must remember that I am a man, and that
consequently I am in the habit of sleeping, and in my dreams
representing to myself the same things or sometimes even less probable
things, than do those who are insane in their waking moments. How
often has it happened to me that in the night I dreamt that I found
myself in this particular place, that I was dressed and seated near
the fire, whilst in reality I was lying \page{146} undressed in bed!
At this moment it does indeed seem to me that it is with eyes awake
that I am looking at this paper; that this head which I move is not
asleep, that it is deliberately and of set purpose that I extend my
hand and perceive it; what happens in sleep does not appear so clear
nor so distinct as does all this. But in thinking over this I remind
myself that on many occasions I have in sleep been deceived by similar
illusions, and in dwelling carefully on this reflection I see so
manifestly that there are no certain indications by which we may
clearly distinguish wakefulness from sleep that I am lost in
astonishment. And my astonishment is such that it is almost capable of
persuading me that I now dream.

Now let us assume that we are asleep and that all these particulars,
e.g. that we open our eyes, shake our head, extend our hands, and so
on, are but false delusions; and let us reflect that possibly neither
our hands nor our whole body are such as they appear to us to be. At
the same time we must at least confess that the things which are
represented to us in sleep are like painted representations which can
only have been formed as the counterparts of something real and true,
and that in this way those general things at least, i.e. eyes, a head,
hands, and a whole body, are not imaginary things, but things really
existent. For, as a matter of fact, painters, even when they study
with the greatest skill to represent sirens and satyrs by forms the
most strange and extraordinary, cannot give them natures which are
entirely new, but merely make a certain medley of the members of
different animals; or if their imagination is extravagant enough to
invent something so novel that nothing similar has ever before been
seen, and that then their work represents a thing purely fictitious
and absolutely false, it is certain all the same that the colours of
which this is composed are necessarily real. And for the same reason,
although these general things, to wit, [a body], eyes, a head, hands,
and such like, may be imaginary, we are bound at the same time to
confess that there are at least some other objects yet more simple and
more universal, which are real and true; and of these just in the same
way as with certain real colours, all these images of things which
dwell in our thoughts, whether true and real or false and fantastic,
are formed.

To such a class of things pertains corporeal nature in general, and
its extension, the figure of extended things, their quantity or
magnitude and number, as also the place in which they are, the time
which measures their duration, and so on.

\page{147}That is possibly why our reasoning is not unjust when we
conclude from this that Physics, Astronomy, Medicine and all other
sciences which have as their end the consideration of composite
things, are very dubious and uncertain; but that Arithmetic,
Geometry and other sciences of that kind which only treat of things
that are very simple and very general, without taking great trouble to
ascertain whether they are actually existent or not, contain some
measure of certainty and an element of the indubitable. For whether I
am awake or asleep, two and three together always form five, and the
square can never have more than four sides, and it does not seem
possible that truths so clear and apparent can be suspected of any
falsity [or uncertainty].

Nevertheless I have long had fixed in my mind the belief that an
all-powerful God existed by whom I have been created such as I am. But
how do I know that He has not brought it to pass that there is no
earth, no heaven, no extended body, no magnitude, no place, and that
nevertheless [I possess the perceptions of all these things and that]
they seem to me to exist just exactly as I now see them? And, besides,
as I sometimes imagine that others deceive themselves in the things
which they think they know best, how do I know that I am not deceived
every time that I add two and three, or count the sides of a square,
or judge of things yet simpler, if anything simpler can be imagined?
But possibly God has not desired that I should be thus deceived, for
He is said to be supremely good. If, however, it is contrary to His
goodness to have made me such that I constantly deceive myself, it
would also appear to be contrary to His goodness to permit me to be
sometimes deceived, and nevertheless I cannot doubt that He does
permit this.

There may indeed be those who would prefer to deny the existence of a
God so powerful, rather than believe that all other things are
uncertain. But let us not oppose them for the present, and grant that
all that is here said of a God is a fable; nevertheless in whatever
way they suppose that I have arrived at the state of being that I have
reached---whether they attribute it to fate or to accident, or make
out that it is by a continual succession of antecedents, or by some
other method---since to err and deceive oneself is a defect, it is
clear that the greater will be the probability of my being so
imperfect as to deceive myself ever, as is the Author to whom they
assign my origin the less powerful. To these reasons I have certainly
nothing to reply, but at the end I feel constrained to confess that
there is nothing in all that I formerly believed to be \page{148}
true, of which I cannot in some measure doubt, and that not merely
through want of thought or through levity, but for reasons which are
very powerful and maturely considered; so that henceforth I ought not
the less carefully to refrain from giving credence to these opinions
than to that which is manifestly false, if I desire to arrive at any
certainty [in the sciences].

But it is not sufficient to have made these remarks, we must also be
careful to keep them in mind. For these ancient and commonly held
opinions still revert frequently to my mind, long and familiar custom
having given them the right to occupy my mind against my inclination
and rendered them almost masters of my belief; nor will I ever lose
the habit of deferring to them or of placing my confidence in them, so
long as I consider them as they really are, i.e. opinions in some
measure doubtful, as I have just shown, and at the same time highly
probable, so that there is much more reason to believe in than to deny
them. That is why I consider that I shall not be acting amiss, if,
taking of set purpose a contrary belief, I allow myself to be
deceived, and for a certain time pretend that all these opinions are
entirely false and imaginary, until at last, having thus balanced my
former prejudices with my latter [so that they cannot divert my
opinions more to one side than to the other], my judgment will no
longer be dominated by bad usage or turned away from the right
knowledge of the truth. For I am assured that there can be neither
peril nor error in this course, and that I cannot at present yield too
much to distrust, since I am not considering the question of action,
but only of knowledge.

I shall then suppose, not that God who is supremely good and the
fountain of truth, but some evil genius not less powerful than
deceitful, has employed his whole energies in deceiving me; I shall
consider that the heavens, the earth, colours, figures, sound, and all
other external things are nought but the illusions and dreams of which
this genius has availed himself in order to lay traps for my
credulity; I shall consider myself as having no hands, no eyes, no
flesh, no blood, nor any senses, yet falsely believing myself to
possess all these things; I shall remain obstinately attached to this
idea, and if by this means it is not in my power to arrive at the
knowledge of any truth, I may at least do what is in my power [i.e.
suspend my judgment], and with firm purpose avoid giving credence to
any false thing, or being imposed upon by this arch deceiver, however
powerful and deceptive he may be. But this \page{149} task is a
laborious one, and insensibly a certain lassitude leads me into the
course of my ordinary life. And just as a captive who in sleep enjoys
an imaginary liberty, when he begins to suspect that his liberty is
but a dream, fears to awaken, and conspires with these agreeable
illusions that the deception may be prolonged, so insensibly of my own
accord I fall back into my former opinions, and I dread awakening
from this slumber, lest the laborious wakefulness which would follow
the tranquillity of this repose should have to be spent not in
daylight, but in the excessive darkness of the difficulties which have
just been discussed.

\section*{Meditation II.}

\begin{center}\textit{Of the Nature of the Human Mind; and that it is
more easily known than the Body.}\end{center}

The Meditation of yesterday filled my mind with so many doubts that it
is no longer in my power to forget them. And yet I do not see in what
manner I can resolve them; and, just as if I had all of a sudden
fallen into very deep water, I am so disconcerted that I can neither
make certain of setting my feet on the bottom, nor can I swim and so
support myself on the surface. I shall nevertheless make an effort and
follow anew the same path as that on which I yesterday entered, i.e. I
shall proceed by setting aside all that in which the least doubt could
be supposed to exist, just as if I had discovered that it was
absolutely false; and I shall ever follow in this road until I have
met with something which is certain, or at least, if I can do nothing
else, until I have learned for certain that there is nothing in the
world that is certain. Archimedes, in order that he might draw the
terrestrial globe out of its place, and transport it elsewhere,
demanded only that one point should be fixed and immoveable; in the
same way I shall have the right to conceive high hopes if I am happy
enough to discover one thing only which is certain and indubitable.

I suppose, then, that all the things that I see are false; I persuade
myself that nothing has ever existed of all that my fallacious memory
represents to me. I consider that I possess no senses; I imagine that
body, figure, extension, movement and place are but the fictions of my
mind. What, then, can be esteemed as true? Perhaps nothing at all,
unless that there is nothing in the world that is certain.

But how can I know there is not something different from those
\page{150} things that I have just considered, of which one cannot
have the slightest doubt? Is there not some God, or some other being
by whatever name we call it, who puts these reflections into my mind?
That is not necessary, for is it not possible that I am capable of
producing them myself? I myself, am I not at least something? But I
have already denied that I had senses and body. Yet I hesitate, for
what follows from that? Am I so dependent on body and senses that I
cannot exist without these? But I was persuaded that there was nothing
in all the world, that there was no heaven, no earth, that there were
no minds, nor any bodies: was I not then likewise persuaded that I did
not exist? Not at all; of a surety I myself did exist since I
persuaded myself of something [or merely because I thought of
something]. But there is some deceiver or other, very powerful and
very cunning, who ever employs his ingenuity in deceiving me. Then
without doubt I exist also if he deceives me, and let him deceive me
as much as he will, he can never cause me to be nothing so long as I
think that I am something. So that after having reflected well and
carefully examined all things, we must come to the definite conclusion
that this proposition: I am, I exist, is necessarily true each time
that I pronounce it, or that I mentally conceive it.

But I do not yet know clearly enough what I am, I who am certain that
I am; and hence I must be careful to see that I do not imprudently
take some other object in place of myself, and thus that I do not go
astray in respect of this knowledge that I hold to be the most certain
and most evident of all that I have formerly learned. That is why I
shall now consider anew what I believed myself to be before I embarked
upon these last reflections; and of my former opinions I shall
withdraw all that might even in a small degree be invalidated by the
reasons which I have just brought forward, in order that there may be
nothing at all left beyond what is absolutely certain and indubitable.

What then did I formerly believe myself to be? Undoubtedly I believed
myself to be a man. But what is a man? Shall I say a reasonable
animal? Certainly not; for then I should have to inquire what an
animal is, and what is reasonable; and thus from a single question I
should insensibly fall into an infinitude of others more difficult;
and I should not wish to waste the little time and leisure remaining
to me in trying to unravel subtleties like these. But I shall rather
stop here to consider the thoughts which of themselves spring up in my
mind, and which were not inspired by \page{151} anything beyond my own
nature alone when I applied myself to the consideration of my being.
In the first place, then, I considered myself as having a face, hands,
arms, and all that system of members composed of bones and flesh as
seen in a corpse which I designated by the name of body. In addition
to this I considered that I was nourished, that I walked, that I felt,
and that I thought, and I referred all these actions to the soul:
but I did not stop to consider what the soul was, or if I did stop, I
imagined that it was something extremely rare and subtle like a wind,
a flame, or an ether, which was spread throughout my grosser parts. As
to body I had no manner of doubt about its nature, but thought I had
a very clear knowledge of it; and if I had desired to explain it
according to the notions that I had then formed of it, I should have
described it thus: By the body I understand all that which can be
defined by a certain figure: something which can be confined in a
certain place, and which can fill a given space in such a way that
every other body will be excluded from it; which can be perceived
either by touch, or by sight, or by hearing, or by taste, or by smell:
which can be moved in many ways not, in truth, by itself, but by
something which is foreign to it, by which it is touched [and from
which it receives impressions]: for to have the power of
self-movement, as also of feeling or of thinking, I did not consider
to appertain to the nature of body: on the contrary, I was rather
astonished to find that faculties similar to them existed in some
bodies.

But what am I, now that I suppose that there is a certain genius which
is extremely powerful, and, if I may say so, malicious, who employs
all his powers in deceiving me? Can I affirm that I possess the least
of all those things which I have just said pertain to the nature of
body? I pause to consider, I revolve all these things in my mind, and
I find none of which I can say that it pertains to me. It would be
tedious to stop to enumerate them. Let us pass to the attributes of
soul and see if there is any one which is in me? What of nutrition or
walking [the first mentioned]? But if it is so that I have no body it
is also true that I can neither walk nor take nourishment. Another
attribute is sensation. But one cannot feel without body, and besides
I have thought I perceived many things during sleep that I recognised
in my waking moments as not having been experienced at all. What of
thinking? I find here that thought is an attribute that belongs to me;
it alone cannot be separated from me. I am, I exist, that is certain.
But how often? Just when I think; for it might possibly be the case if
I ceased \page{152} entirely to think, that I should likewise cease
altogether to exist. I do not now admit anything which is not
necessarily true: to speak accurately I am not more than a thing which
thinks, that is to say a mind or a soul, or an understanding, or a
reason, which are terms whose significance was formerly unknown to me.
I am, however, a real thing and really exist; but what thing? I have
answered: a thing which thinks.

And what more? I shall exercise my imagination [in order to see if I
am not something more]. I am not a collection of members which we call
the human body: I am not a subtle air distributed through these
members, I am not a wind, a fire, a vapour, a breath, nor anything at
all which I can imagine or conceive; because I have assumed that all
these were nothing. Without changing that supposition I find that I
only leave myself certain of the fact that I am somewhat. But perhaps
it is true that these same things which I supposed were non-existent
because they are unknown to me, are really not different from the self
which I know. I am not sure about this, I shall not dispute about it
now; I can only give judgment on things that are known to me. I know
that I exist, and I inquire what I am, I whom I know to exist. But it
is very certain that the knowledge of my existence taken in its
precise significance does not depend on things whose existence is not
yet known to me; consequently it does not depend on those which I can
feign in imagination. And indeed the very term \textit{feign} in
imagination\footnote{Or `form an image' (effingo).} proves to me my
error, for I really do this if I image myself a something, since to
imagine is nothing else than to contemplate the figure or image of a
corporeal thing. But I already know for certain that I am, and that it
may be that all these images, and, speaking generally, all things that
relate to the nature of body are nothing but dreams [and chimeras].
For this reason I see clearly that I have as little reason to say, ``I
shall stimulate my imagination in order to know more distinctly what I
am,'' than if I were to say, ``I am now awake, and I perceive somewhat
that is real and true: but because I do not yet perceive it distinctly
enough, I shall go to sleep of express purpose, so that my dreams may
represent the perception with greatest truth and evidence.'' And,
thus, I know for certain that nothing of all that I can understand by
means of my imagination belongs to this knowledge which I have of
myself, and that it is necessary to recall the mind from \page{153}
this mode of thought with the utmost diligence in order that it may be
able to know its own nature with perfect distinctness.

But what then am I? A thing which thinks. What is a thing which
thinks? It is a thing which doubts, understands, [conceives], affirms,
denies, wills, refuses, which also imagines and feels.

Certainly it is no small matter if all these things pertain to my
nature. But why should they not so pertain? Am I not that being who
now doubts nearly everything, who nevertheless understands certain
things, who affirms that one only is true, who denies all the others,
who desires to know more, is averse from being deceived, who imagines
many things, sometimes indeed despite his will, and who perceives many
likewise, as by the intervention of the bodily organs? Is there
nothing in all this which is as true as it is certain that I exist,
even though I should always sleep and though he who has given me being
employed all his ingenuity in deceiving me? Is there likewise any one
of these attributes which can be distinguished from my thought, or
which might be said to be separated from myself? For it is so evident
of itself that it is I who doubts, who understands, and who desires,
that there is no reason here to add anything to explain it. And I have
certainly the power of imagining likewise; for although it may happen
(as I formerly supposed) that none of the things which I imagine are
true, nevertheless this power of imagining does not cease to be really
in use, and it forms part of my thought. Finally, I am the same who
feels, that is to say, who perceives certain things, as by the organs
of sense, since it truth I see light, I hear noise, I feel heat. But
it will be said that these phenomena are false and that I am dreaming.
Let it be so; still it is at least quite certain that it seems to me
that I see light, that I hear noise and that I feel heat. That cannot
be false; properly speaking it is what is in me called
feeling\footnote{Sentire.}; and used in this precise sense that is no
other thing than thinking.

From this time I begin to know what I am with a little more clearness
and distinction than before; but nevertheless it still seems to me,
and I cannot prevent myself from thinking, that corporeal things,
whose images are framed by thought, which are tested by the senses,
are much more distinctly known than that obscure part of me which does
not come under the imagination. Although really it is very strange to
say that I know and understand more distinctly these things whose
existence seems to me \page{154} dubious, which are unknown to me, and
which do not belong to me, than others of the truth of which I am
convinced, which are known to me and which pertain to my real nature,
in a word, than myself. But I see clearly how the case stands: my mind
loves to wander, and cannot yet suffer itself to be retained within
the just limits of truth. Very good, let us once more give it the
freest rein, so that, when afterwards we seize the proper occasion for
pulling up, it may the more easily be regulated and controlled.

Let us begin by considering the commonest matters, those which we
believe to be the most distinctly comprehended, to wit, the bodies
which we touch and see; not indeed bodies in general, for these
general ideas are usually a little more confused, but let us consider
one body in particular. Let us take, for example, this piece of wax:
it has been taken quite freshly from the hive, and it has not yet lost
the sweetness of the honey which it contains; it still retains
somewhat of the odour of the flowers from which it has been culled;
its colour, its figure, its size are apparent; it is hard, cold,
easily handled, and if you strike it with the finger, it will emit a
sound. Finally all the things which are requisite to cause us
distinctly to recognise a body, are met with in it. But notice that
while I speak and approach the fire what remained of the taste is
exhaled, the smell evaporates, the colour alters, the figure is
destroyed, the size increases, it becomes liquid, it heats, scarcely
can one handle it, and when one strikes it, no sound is emitted. Does
the same wax remain after this change? We must confess that it
remains; none would judge otherwise. What then did I know so
distinctly in this piece of wax? It could certainly be nothing of all
that the senses brought to my notice, since all these things which
fall under taste, smell, sight, touch, and hearing, are found to be
changed, and yet the same wax remains.

Perhaps it was what I now think, viz. that this wax was not that
sweetness of honey, nor that agreeable scent of flowers, nor that
particular whiteness, nor that figure, nor that sound, but simply a
body which a little while before appeared tome as perceptible under
these forms, and which is now perceptible under others. But what,
precisely, is it that I imagine when I form such conceptions? Let us
attentively consider this, and, abstracting from all that does not
belong to the wax, let us see what remains. Certainly nothing remains
excepting a certain extended thing which is flexible and movable. But
what is the meaning of flexible and movable? Is it not that I imagine
that this piece of wax being round is capable of \page{155} becoming
square and of passing from a square to a triangular figure? No,
certainly it is not that, since I imagine it admits of an infinitude
of similar changes, and I nevertheless do not know how to compass
the infinitude by my imagination, and consequently this conception
which I have of the wax is not brought about by the faculty of
imagination. What now is this extension? Is it not also unknown? For
it becomes greater when the wax is melted, greater when it is boiled,
and greater still when the heat increases; and I should not conceive
[clearly] according to truth what wax is, if I did not think that even
this piece that we are considering is capable of receiving more
variations in extension than I have ever imagined. We must then grant
that I could not even understand through the imagination what this
piece of wax is, and that it is my mind\footnote{entendement F., mens
L.} alone which perceives it. I say this piece of wax in particular,
for as to wax in general it is yet clearer. But what is this piece of
wax which cannot be understood excepting by the [understanding or]
mind? It is certainly the same that I see, touch, imagine, and finally
it is the same which I have always believed it to be from the
beginning. But what must particularly be observed is that its
perception is neither an act of vision, nor of touch, nor of
imagination, and has never been such although it may have appeared
formerly to be so, but only an intuition\footnote{inspectio} of the
mind, which may be imperfect and confused as it was formerly, or clear
and distinct as it is at present, according as my attention is more or
less directed to the elements which are found in it, and of which it
is composed.

Yet in the meantime I am greatly astonished when I consider [the great
feebleness of mind] and its proneness to fall [insensibly] into error;
for although without giving expression to my thoughts I consider all
this in my own mind, words often impede me and I am almost deceived by
the terms of ordinary language. For we say that we see the same wax,
if it is present, and not that we simply judge that it is the same
from its having the same colour and figure. From this I should
conclude that I knew the wax by means of vision and not simply by the
intuition of the mind; unless by chance I remember that, when looking
from a window and saying I see men who pass in the street, I really do
not see them, but infer that what I see is men, just as I say that I
see wax. And yet what do I see from the window but hats and coats
which may cover automatic machines? Yet I judge these to be men. And
similarly \page{156} solely by the faculty of judgment which rests in
my mind, I comprehend that which I believed I saw with my eyes.

A man who makes it his aim to raise his knowledge above the common
should be ashamed to derive the occasion for doubting from the forms
of speech invented by the vulgar; I prefer to pass on and consider
whether I had a more evident and perfect conception of what the wax
was when I first perceived it, and when I believed I knew it by means
of the external senses or at least by the common sense\footnote{sensus
communis.} as it is called, that is to say by the imaginative faculty,
or whether my present conception is clearer now that I have most
carefully examined what it is, and in what way it can be known. It
would certainly be absurd to doubt as to this. For what was there in
this first perception which was distinct? What was there which might
not as well have been perceived by any of the animals? But when I
distinguish the wax from its external forms, and when, just as if I
had taken from it its vestments, I consider it quite naked, it is
certain that although some error may still be found in my judgment, I
can nevertheless not perceive it thus without a human mind.

But finally what shall I say of this mind, that is, of myself, for up
to this point I do not admit in myself anything but mind? What then, I
who seem to perceive this piece of wax so distinctly, do I not know
myself, not only with much more truth and certainty, but also with
much more distinctness and clearness? For if I judge that the wax is
or exists from the fact that I see it, it certainly follows much more
clearly that I am or that I exist myself from the fact that I see it.
For it may be that what I see is not really wax, it may also be that I
do not possess eyes with which to see anything; but it cannot be that
when I see, or (for I no longer take account of the distinction) when
I think I see, that I myself who think am nought. So if I judge that
the wax exists from the fact that I touch it, the same thing will
follow, to wit, that I am; and if I judge that my imagination, or some
other cause, whatever it is, persuades me that the wax exists, I shall
still conclude the same. And what I have here remarked of wax may be
applied to all other things which are external to me [and which are
met with outside of me]. And further, if the [notion or] perception of
wax has seemed to me clearer and more distinct, not only after the
sight or the touch, but also after many other causes have rendered it
quite manifest to me, with how much more [evidence] and distinctness
\page{157} must it be said that I now know myself, since all the
reasons which contribute to the knowledge of wax, or any other body
whatever, are yet better proofs of the nature of my mind! And there
are so many other things in the mind itself which may contribute to
the elucidation of its nature, that those which depend on body such as
these just mentioned, hardly merit being taken into account.

But finally here I am, having insensibly reverted to the point I
desired, for, since it is now manifest to me that even bodies are not
properly speaking known by the senses or by the faculty of
imagination, but by the understanding only, and since they are not
known from the fact that they are seen or touched, but only because
they are understood, I see clearly that there is nothing which is
easier for me to know than my mind. But because it is difficult to rid
oneself so promptly of an opinion to which one was accustomed for so
long, it will be well that I should halt a little at this point, so
that by the length of my meditation I may more deeply imprint on my
memory this new knowledge.

