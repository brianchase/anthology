
\author{Jane Addams}
\authdate{1860--1935}
\textdate{1892/1910}
\chapter{The Subjective Necessity for Social Settlements}
\source{addams1910f}

% Jane Addams delivered the speech that forms the heart of this text
% in 1892. It was published the following year under the title ``The
% Subjective Necessity for Social Settlements''

\page{113}\noindent The Ethical Culture Societies held a summer school
at Plymouth, Massachusetts, in 1892, to which they invited several
people representing the then new Settlement movement, that they might
discuss with others the general theme of Philanthropy and Social
Progress.

I venture to produce here parts of a lecture I delivered in Plymouth,
both because I have found it impossible to formulate with the same
freshness those early motives and strivings, and because, when
published with other papers given that summer, it was received by the
Settlement people themselves as a satisfactory statement.

I remember one golden summer afternoon during the sessions of the
summer school that several of us met on the shores of a pond in a pine
wood a few miles from Plymouth, to discuss our new movement. The
natural leader of the group was Robert A. Woods. He had recently
returned from a residence in Toynbee Hall, London, to open Andover
House in Boston, and had just issued a book, ``English Social
Movements,'' in which he had gathered together and focused the many
\page{114} forms of social endeavor preceding and contemporaneous with
the English Settlements. There were Miss Vida D. Scudder and Miss
Helena Dudley from the College Settlement Association, Miss Julia C.
Lathrop and myself from Hull-House. Some of us had numbered our years
as far as thirty, and we all carefully avoided the extravagance of
statement which characterizes youth, and yet I doubt if anywhere on
the continent that summer could have been found a group of people more
genuinely interested in social development or more sincerely convinced
that they had found a clue by which the conditions in crowded cities
might be understood and the agencies for social betterment developed.

We were all careful to avoid saying that we had found a ``life work,''
perhaps with an instinctive dread of expending all our energy in vows
of constancy, as so often happens; and yet it is interesting to note
that of all the people whom I have \page{115} recalled as the
enthusiasts at that little conference have remained attached to
Settlements in actual residence for longer or shorter periods each
year during the eighteen years that have elapsed since then, although
they have also been closely identified as publicists or governmental
officials with movements outside. It is as if they had discovered that
the Settlement was too valuable as a method as a way of approach to
the social question to be abandoned, although they had long since
discovered it was not a ``social movement'' in itself. This, however,
is anticipating the future, whereas the following paper on ``The
Subjective Necessity for Social Settlements'' should have a chance to
speak for itself. It is perhaps too late in the day to express regret
for its stilted title.

This paper is an attempt to analyze the motives which underlie a
movement based, not only upon conviction, but upon genuine emotion,
wherever educated young people are seeking an outlet for that
sentiment for universal brotherhood, which the best spirit of our
times is forcing from an emotion into a motive. These young people
accomplish little toward the solution of this social problem, and bear
the brunt of being cultivated into unnourished, oversensitive lives.
They have been shut off from the common labor by which they live which
is a great source of moral and physical health. They feel a fatal want
of harmony between their theory and their lives, a lack of
coördination between thought and action. I think it is hard for us to
realize how seriously many of \page{116} them are taking to the notion
of human brotherhood, how eagerly they long to give tangible
expression to the democratic ideal. These young men and women, longing
to socialize their democracy, are animated by certain hopes which may
be thus loosely formulated; that if in a democratic country nothing
can be permanently achieved save through the masses of the people, it
will be impossible to establish a higher political life than the
people themselves crave; that it is difficult to see how the notion of
a higher civic life can be fostered save through common intercourse;
that the blessings which we associate with a life of refinement and
cultivation can be made universal and must be made universal if they
are to be permanent; that the good we secure for ourselves is
precarious and uncertain, is floating in mid-air, until it is secured
for all of us and incorporated into our common life. It is easier to
state these hopes than to formulate the line of motives, which I
believe to constitute the trend of the subjective pressure toward the
Settlement. There is something primordial about these motives, but I
am perhaps overbold in designating them as a great desire to share the
race life. We all bear traces of the starvation struggle which for so
long made up the life of the race. Our very organism holds memories
and glimpses of that long life of our ancestors, which still goes on
among so many of our contemporaries. Nothing so deadens the sympathies
and shrivels the power of enjoyment as the persistent keeping away
from the great opportunities for helpfulness and a continual ignoring
of the starvation struggle which makes up the life of at least half
the race. To shut one's self away from that \page{117} half of the
race life is to shut one's self away from the most vital part of it;
it is to live out but half the humanity to which we have been born
heir and to use but half our faculties. We have all had longings for a
fuller life which should include the use of these faculties. These
longings are the physical complement of the ``Intimations of
Immortality,'' on which no ode has yet been written. To portray these
would be the work of a poet, and it is hazardous for any but a poet to
attempt it.

You may remember the forlorn feeling which occasionally seizes you
when you arrive early in the morning a stranger in a great city: the
stream of laboring people goes past you as you gaze through the
plate-glass window of your hotel; you see hard working men lifting
great burdens; you hear the driving and jostling of huge carts and
your heart sinks with a sudden sense of futility. The door opens
behind you and you turn to the man who brings you in your breakfast
with a quick sense of human fellowship. You find yourself praying that
you may never lose your hold on it all. A more poetic prayer would be
that the great mother breasts of our common humanity, with its labor
and suffering and its homely comforts, may never be withheld from you.
You turn helplessly to the waiter and feel that it would be almost
grotesque to claim from him the sympathy you crave because
civilization has placed you apart, but you resent your position with a
sudden sense of snobbery. Literature is full of portrayals of these
glimpses: they come to shipwrecked men on rafts; they overcome the
differences of an incongruous multitude when in the presence of a
great danger or \page{118} when moved by a common enthusiasm. They are
not, however, confined to such moments, and if we were in the habit of
telling them to each other, the recital would be as long as the tales
of children are, when they sit down on the green grass and confide to
each other how many times they have remembered that they lived once
before. If these childish tales are the stirring of inherited
impressions, just so surely is the other the striving of inherited
powers.

``It is true that there is nothing after disease, indigence and a
sense of guilt, so fatal to health and to life itself as the want of a
proper outlet for active faculties.'' I have seen young girls suffer
and grow sensibly lowered in vitality in the first years after they
leave school. In our attempt then to give a girl pleasure and freedom
from care we succeed, for the most part, in making her pitifully
miserable. She finds ``life'' so different from what she expected it
to be. She is besotted with innocent little ambitions, and does not
understand this apparent waste of herself, this elaborate preparation,
if no work is provided for her. There is a heritage of noble
obligation which young people accept and long to perpetuate. The
desire for action, the wish to right wrong and alleviate suffering
haunts them daily. Society smiles at it indulgently instead of making
it of value to itself. The wrong to them begins even farther back,
when we restrain the first childish desires for ``doing good'' and
tell them that they must wait until they are older and better fitted.
We intimate that social obligation begins at a fixed date, forgetting
that it begins at birth itself. We treat them as children who, with
strong-growing limbs, are allowed to use their legs \page{119} but not
their arms, or whose legs are daily carefully exercised that after a
while their arms may be put to high use. We do this in spite of the
protest of the best educators, Locke and Pestalozzi. We are fortunate
in the meantime if their unused members do not weaken and disappear.
They do sometimes. There are a few girls who, by the time they are
``educated,'' forget their old childish desires to help the world and
to play with poor little girls ``who haven't playthings.'' Parents are
often inconsistent: they deliberately expose their daughters to
knowledge of the distress in the world; they send them to hear
missionary addresses on famines in India and China; they accompany
them to lectures on the suffering in Siberia; they agitate together
over the forgotten region of East London. In addition to this, from
babyhood the altruistic tendencies of these daughters are persistently
cultivated. They are taught to be self-forgetting and
self-sacrificing, to consider the good of the whole before the good of
the ego. But when all this information and culture show results, when
the daughter comes back from college and begins to recognize her
social claim to the ``submerged tenth,'' and to evince a disposition
to fulfill it, the family claim is strenuously asserted; she is told
that she is unjustified, ill-advised in her efforts. If she persists,
the family too often are injured and unhappy unless the efforts are
called missionary and the religious zeal of the family carry them over
their sense of abuse. When this zeal does not exist, the result is
perplexing. It is a curious violation of what we would fain believe a
fundamental law---that the final return of the deed is upon the head
of the doer. The deed is that of exclusiveness \page{120} and caution,
but the return, instead of falling upon the head of the exclusive and
cautious, falls upon a young head full of generous and unselfish
plans. The girl loses something vital out of her life to which she is
entitled. She is restricted and unhappy; her elders meanwhile, are
unconscious of the situation and we have all the elements of a
tragedy.

We have in America a fast-growing number of cultivated young people
who have no recognized outlet for their active faculties. They hear
constantly of the great social maladjustment, but no way is provided
for them to change it, and their uselessness hangs about them heavily.
Huxley declares that the sense of uselessness is the severest shock
which the human system can sustain, and that if persistently
sustained, it results in atrophy of function. These young people have
had advantages of college, of European travel, and of economic study,
but they are sustaining this shock of inaction. They have pet phrases,
and they tell you that the things that make us all alike are stronger
than the things that make us different. They say that all men are
united by needs and sympathies far more permanent and radical than
anything that temporarily divides them and sets them in opposition to
each other. If they affect art, they say that the decay in artistic
expression is due to the decay in ethics, that art when shut away from
the human interests and from the great mass of humanity is
self-destructive. They tell their elders with all the bitterness of
youth that if they expect success from them in business or politics or
in whatever lines their ambition for them has run, they must let them
consult all of humanity; that they must let them \page{121} find out
what the people want and how they want it. It is only the stronger
young people, however, who formulate this. Many of them dissipate
their energies in so-called enjoyment. Others not content with that,
go on studying and go back to college for their second degrees; not
that they are especially fond of study, but because they want
something definite to do, and their powers have been trained in the
direction of mental accumulation. Many are buried beneath this mental
accumulation with lowered vitality and discontent. Walter Besant says
they have had the vision that Peter had when he saw the great sheet
let down from heaven, wherein was neither clean nor unclean. He calls
it the sense of humanity. It is not philanthropy nor benevolence, but
a thing fuller and wider than either of these.

This young life, so sincere in its emotion and good phrases and yet so
undirected, seems to me as pitiful as the other great mass of
destitute lives. One is supplementary to the other, and some method of
communication can surely be devised. Mr. Barnett, who urged the first
Settlement,---Toynbee Hall, in East London,---recognized this need of
outlet for the young men of Oxford and Cambridge, and hoped that the
Settlement would supply the communication. It is easy to see why the
Settlement movement originated in England, where the years of
education are more constrained and definite than they are here, where
class distinctions are more rigid. The necessity of it was greater
there, but we are fast feeling the pressure of the need and meeting
the necessity for Settlements in America. Our young people feel
nervously the need of \page{122} putting theory into action, and
respond quickly to the Settlement form of activity.

% In the paragraph below, 'hart' is in the original! It would be
% interesting to check the recent reprint to see whether the editor
% regarded it as a typo or a deliberate anachronism.

Other motives which I believe make toward the Settlement are the
result of a certain renaissance going forward in Christianity. The
impulse to share the lives of the poor, the desire to make social
service, irrespective of propaganda, express the spirit of Christ, is
as old as Christianity itself. We have no proof from the records
themselves that the early Roman Christians, who strained their simple
art to the point of grotesqueness in their eagerness to record a
``good news'' on the walls of the catacombs, considered this good news
a religion. Jesus had no set of truths labeled Religious. On the
contrary, his doctrine was that all truth is one, that the
appropriation of it is freedom. His teaching had no dogma to mark it
off from truth and action in general. He himself called it a
revelation---a life. These early Roman Christians received the Gospel
message, a command to love all men, with a certain joyous simplicity.
The image of the Good Shepherd is blithe and gay beyond the gentlest
shepherd of Greek mythology; the hart no longer pants, but rushes to
the water brooks. The Christians looked for the continuous revelation,
but believed what Jesus said, that this revelation, to be retained and
made manifest, must be put into terms of action; that action is the
only medium man has for receiving and appropriating truth; that the
doctrine must be known through the will.

That Christianity has to be revealed and embodied in the line of
social progress is a corollary to the simple proposition, that man's
action is found in his social \page{123} relationships in the way in
which he connects with his fellows; that his motives for action are
the zeal and affection with which he regards his fellows. By this
simple process was created a deep enthusiasm for humanity; which
regarded man as at once the organ and the object of revelation; and by
this process came about the wonderful fellowship, the true democracy
of the early Church, that so captivates the imagination. The early
Christians were pre\"{e}minently nonresistant. They believed in love
as a cosmic force. There was no iconoclasm during the minor peace of
the Church. They did not yet denounce nor tear down temples, nor
preach the end of the world. They grew to a mighty number, but it
never occurred to them, either in their weakness or in their strength,
to regard other men for an instant as their foes or as aliens. The
spectacle of the Christians loving all men was the most astounding
Rome had ever seen. They were eager to sacrifice themselves for the
weak, for children, and for the aged; they identified themselves with
slaves and did not avoid the plague; they longed to share the common
lot that they might receive the constant revelation. It was a new
treasure which the early Christians added to the sum of all treasures,
a joy hitherto unknown in the world---the joy of finding the Christ
which lieth in each man, but which no man can unfold save in
fellowship. A happiness ranging from the heroic to the pastoral
enveloped them. They were to possess a revelation as long as life had
new meaning to unfold, new action to propose.

I believe that there is a distinct turning among many young men and
women toward this simple acceptance of Christ's message. They resent
the \page{124} assumption that Christianity is a set of ideas which
belong to the religious consciousness, whatever that may be. They
insist that it cannot be proclaimed and instituted apart from the
social life of the community and that it must seek a simple and
natural expression in the social organism itself. The Settlement
movement is only one manifestation of that wider humanitarian movement
which throughout Christendom, but pre-eminently in England, is
endeavoring to embody itself, not in a sect, but in society itself.

I believe that this turning, this renaissance of the early Christian
humanitarianism, is going on in America, in Chicago, if you please,
without leaders who write or philosophize, without much speaking, but
with a bent to express in social service and in terms of action the
spirit of Christ. Certain it is that spiritual force is found in the
Settlement movement, and it is also true that this force must be
evoked and must be called into play before the success of any
Settlement is assured. There must be the overmastering belief that all
that is noblest in life is common to men as men, in order to
accentuate the likenesses and ignore the differences which are found
among the people whom the Settlement constantly brings into
juxtaposition. It may be true, as the Positivists insist, that the
very religious fervor of man can be turned into love for his race, and
his desire for a future life into content to live in the echo of his
deeds; Paul's formula of seeking for the Christ which lieth in each
man and founding our likenesses on him, seems a simpler formula to
many of us.

In a thousand voices singing the Hallelujah Chorus in Handel's
``Messiah,'' it is possible to distinguish \page{125} the leading
voices, but the differences of training and cultivation between them
and the voices in the chorus, are lost in the unity of purpose and in
the fact that they are all human voices lifted by a high motive. This
is a weak illustration of what a Settlement attempts to do. It aims,
in a measure, to develop whatever of social life its neighborhood may
afford, to focus and give form to that life, to bring to bear upon it
the results of cultivation and training; but it receives in exchange
for the music of isolated voices the volume and strength of the
chorus. It is quite impossible for me to say in what proportion or
degree the subjective necessity which led to the opening of Hull-House
combined the three trends: first, the desire to interpret democracy in
social terms; secondly, the impulse beating at the very source of our
lives, urging us to aid in the race progress; and, thirdly, the
Christian movement toward humanitarianism. It is difficult to analyze
a living thing; the analysis is at best imperfect. Many more motives
may blend with the three trends; possibly the desire for a new form of
social success due to the nicety of imagination, which refuses worldly
pleasures unmixed with the joys of self-sacrifice; possibly a love of
approbation, so vast that it is not content with the treble clapping
of delicate hands, but wishes also to hear the bass notes from
toughened palms, may mingle with these.

The Settlement, then, is an experimental effort to aid in the solution
of the social and industrial problems which are engendered by the
modern conditions of life in a great city. It insists that these
problems are not confined to any one portion of a city. It is an
attempt to relieve, at the same time, the overaccumulation at
\page{126} one end of society and the destitution at the other; but it
assumes that this overaccumulation and destitution is most sorely felt
in the things that pertain to social and educational privileges. From
its very nature it can stand for no political or social propaganda. It
must, in a sense, give the warm welcome of an inn to all such
propaganda, if perchance one of them be found an angel. The only thing
to be dreaded in the Settlement is that it lose its flexibility, its
power of quick adaptation, its readiness to change its methods as its
environment may demand. It must be open to conviction and must have a
deep and abiding sense of tolerance. It must be hospitable and ready
for experiment. It should demand from its residents a scientific
patience in the accumulation of facts and the steady holding of their
sympathies as one of the best instruments for that accumulation. It
must be grounded in a philosophy whose foundation is on the solidarity
of the human race, a philosophy which will not waver when the race
happens to be represented by a drunken woman or an idiot boy. Its
residents must be emptied of all conceit of opinion and all
self-assertion, and ready to arouse and interpret the public opinion
of their neighborhood. They must be content to live quietly side by
side with their neighbors, until they grow into a sense of
relationship and mutual interests. Their neighbors are held apart by
differences of race and language which the residents can more easily
overcome. They are bound to see the needs of their neighborhood as a
whole, to furnish data for legislation, and to use their influence to
secure it. In short, residents are pledged to devote themselves to the
duties of good citizenship and to the arousing of the social energies
\page{127} which too largely lie dormant in every neighborhood given
over to industrialism. They are bound to regard the entire life of
their city as organic, to make an effort to unify it, and to protest
against its over-differentiation.

It is always easy to make all philosophy point one particular moral
and all history adorn one particular tale; but I may be forgiven the
reminder that the best speculative philosophy sets forth the
solidarity of the human race; that the highest moralists have taught
that without the advance and improvement of the whole, no man can hope
for any lasting improvement in his own moral or material individual
condition; and that the subjective necessity for Social Settlements is
therefore identical with that necessity, which urges us on toward
social and individual salvation.

