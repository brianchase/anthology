
\author{Aulus Gellius}
\authdate{b. ca. 125}
\textdate{ca. 160s}
\addon{Book 5.10}
\chapter[Attic Nights, bk. 5.10]{Attic Nights}
\source*{gellius1927.1}

\page{405}\begin{center}{\small On the arguments which by the Greeks
are called \grk{ἀντιστρέφοντα}, and in Latin may be termed
\textit{reciproca}.}\end{center}

\noindent Among fallacious arguments the one which the Greeks call
\grk{ἀντιστρέφον} seems to be by far the most fallacious. Such
arguments some of our own philosophers have rather appropriately
termed \textit{reciproca}, or ``convertible.'' The fallacy arises from
the fact that the argument that is presented may be turned in the
opposite direction and used against the one who has offered it, and is
equally strong for both sides of the question. An example is the
well-known argument which Protagoras, the keenest of all sophists, is
said to have used against his pupil Euathlus.

\page{407} For a dispute arose between them and an altercation as to
the fee which had been agreed upon, as follows: Euathlus, a wealthy
young man, was desirous of instruction in oratory and the pleading of
causes. He became a pupil of Protagoras and promised to pay him a
large sum of money, as much as Protagoras had demanded. He paid half
of the amount at once, before beginning his lessons, and agreed to pay
the remaining half on the day when he first pleaded before jurors and
won his case. Afterwards, when he had been for some little time a
pupil and follower of Protagoras, and had in fact made considerable
progress in the study of oratory, he nevertheless did not undertake
any cases. And when the time was already getting long, and he seemed
to be acting thus in order not to pay the rest of the fee, Protagoras
formed what seemed to him at the time a wily scheme; he determined to
demand his pay according to the contract, and brought suit against
Euathlus.

And when they had appeared before the jurors to bring forward and to
contest the case, Protagoras began as follows: ``Let me tell you, most
foolish of youths, that in either event you will have to pay what I am
demanding, whether judgment be pronounced for or against you. For if
the case goes against you, the money will be due me in accordance with
the verdict, because I have won; but if the decision be in your
favour, the money will be due me according to our contract, since you
will have won a case.''

To this Euatlllus replied: ``I might have met this sophism of yours,
tricky as it is, by not pleading my own cause but employing another as
my advocate. But I take greater satisfaction in a victory in which
\page{409} I defeat you, not only in the suit, but also in this
argument of yours. So let me tell you in turn, wisest of masters, that
in either event I shall not have to pay what you demand, whether
judgment be pronounced for or against me. For if the jurors decide in
my favour, according to their verdict nothing will be due you, because
I have won; but if they give judgment against me, by the terms of our
contract I shall owe you nothing, because I have not won a case.''

Then the jurors, thinking that the plea on both sides was uncertain
and insoluble, for fear that their decision, for whichever side it was
rendered, might annul itself, left the matter undecided and postponed
the case to a distant day. Thus a celebrated master of oratory was
refuted by his youthful pupil with his own argument, and his cleverly
devised sophism failed.

