
\author{Plato}
\authdate{ca. 427--348/347 \BCE}
%\textdate{}
\chapter{Euthyphro}
\source{plato1919.1}

\page{7}\textsc{Euthyphro}. What strange thing has happened, Socrates,
that you have left your accustomed haunts in the Lyceum and are now
haunting the portico where the king archon sits? For it cannot be that
you have an action before the king, as I have.

\textsc{Socrates}. Our Athenians, Euthyphro, do not call it an action,
but an indictment.

\textsc{Euthyphro}. What? Somebody has, it seems, brought an
indictment against you; for I don't accuse you of having brought one
against anyone else.

\textsc{Socrates}. Certainly not.

\textsc{Euthyphro}. But someone else against you?

\textsc{Socrates}. Quite so.

\textsc{Euthyphro}. Who is he?

\textsc{Socrates}. I don't know the man very well myself, Euthyphro,
for he seems to be a young and unknown person. His name, however, is
Meletus, I believe. And he is of the deme of Pitthus, if you remember
any Pitthian Meletus, with long hair and only a little beard, but with
a hooked nose.

\page{9}\textsc{Euthyphro}. I don't remember him, Socrates. But what
sort of an indictment has he brought against you?

\textsc{Socrates}. What sort? No mean one, it seems to me; for the
fact that, young as he is, he has apprehended so important a matter
reflects no small credit upon him. For he says he knows how the youth
are corrupted and who those are who corrupt them. He must be a wise
man; who, seeing my lack of wisdom and that I am corrupting his
fellows, comes to the State, as a boy runs to his mother, to accuse
me. And he seems to me to be the only one of the public men who begins
in the right way; for the right way is to take care of the young men
first, to make them as good as possible, just as a good husbandman
will naturally take care of the young plants first and afterwards of
the rest. And so Meletus, perhaps, is first clearing away us who
corrupt the young plants, as he says; then after this, when he has
turned his attention to the older men, he will bring countless most
precious blessings upon the State,---at least, that is the natural
outcome of the beginning he has made.

\textsc{Euthyphro}. I hope it may be so, Socrates; but I fear the
opposite may result. For it seems to me that he begins by injuring the
State at its very heart, when he undertakes to harm you. Now tell me,
what does he say you do that corrupts the young?

\textsc{Socrates}. Absurd things, my friend, at first hearing. For he
says I am a maker of gods; and because I make new gods and do not
believe in the old ones, he indicted me for the sake of these old
ones, as he says.

\textsc{Euthyphro}. I understand, Socrates; it is because \page{11}
you say the divine monitor keeps coming to you. So he has brought the
indictment against you for making innovations in religion, and he is
going into court to slander you, knowing that slanders on such
subjects are readily accepted by the people. Why, they even laugh at
me and say I am crazy when I say anything in the assembly about divine
things and foretell the future to them. And yet there is not one of
the things I have foretold that is not true; but they are jealous of
all such men as you and I are. However, we must not be disturbed, but
must come to close quarters with them.

\textsc{Socrates}. My dear Euthyphro, their ridicule is perhaps of no
consequence. For the Athenians, I fancy, are not much concerned, if
they think a man is clever, provided he does not impart his clever
notions to others; but when they think he makes others to be like
himself, they are angry with him, either through jealousy, as you say,
or for some other reason.

\textsc{Euthyphro}. I don't much desire to test their sentiments
toward me in this matter.

\textsc{Socrates}. No, for perhaps they think that you are reserved
and unwilling to impart your wisdom. But I fear that because of my
love of men they think that I not only pour myself out copiously to
anyone and everyone without payment, but that I would even pay
something myself, if anyone would listen to me. Now if, as I was
saying just now, they were to laugh at me, as you say they do at you,
it would not be at all unpleasant to pass the time in the court with
jests and laughter; but if they are in earnest, then only soothsayers
like you can tell how this will end.

\page{13}\textsc{Euthyphro}. Well, Socrates, perhaps it won't amount
to much, and you will bring your case to a satisfactory ending, as I
think I shall mine.

\textsc{Socrates}. What is your case, Euthyphro? Are you defending or
prosecuting?

\textsc{Euthyphro}. Prosecuting.

\textsc{Socrates}. Whom?

\textsc{Euthyphro}. Such a man that they think I am insane because I
am prosecuting\footnote{The Greek word has much the same meaning as
the Latin \textit{prosequor}, from which the English `prosecute' is
derived, `follow,' `pursue,' and is at the same time the technical
term for `prosecute.'} him.

\textsc{Socrates}. Why? Are you prosecuting one who has wings to fly
away with?

\textsc{Euthyphro}. No flying for him at his ripe old age.

\textsc{Socrates}. Who is he?

\textsc{Euthyphro}. My father.

\textsc{Socrates}. Your father, my dear man?

\textsc{Euthyphro}. Certainly.

\textsc{Socrates}. But what is the charge, and what is the suit about?

\textsc{Euthyphro}. Murder, Socrates.

\textsc{Socrates}. Heracles! Surely, Euthyphro, most people do not
know where the right lies; for I fancy it is not everyone who can
rightly do what you are doing, but only one who is already very far
advanced in wisdom.

\textsc{Euthyphro}. Very far, indeed, Socrates, by Zeus.

\textsc{Socrates}. Is the one who was killed by your father a
relative? But of course he was; for you would not bring a charge of
murder against him on a stranger's account.

\textsc{Euthyphro}. It is ridiculous, Socrates, that you think it
matters whether the man who was killed \page{15} was a stranger or a
relative, and do not see that the only thing to consider is whether
the action of the slayer was justified or not, and that if it was
justified one ought to let him alone, and if not, one ought to
proceed against him, even if he share one's hearth and eat at one's
table. For the pollution is the same if you associate knowingly with
such a man and do not purify yourself and him by proceeding against
him. In this case, the man who was killed was a hired workman of mine,
and when we were farming at Naxos, he was working there on our land.
Now he got drunk, got angry with one of our house slaves, and
butchered him. So my father bound him hand and foot, threw him into a
ditch, and sent a man here to Athens to ask the religious adviser what
he ought to do. In the meantime he paid no attention to the man as he
lay there bound, and neglected him, thinking that he was a murderer
and it did not matter if he were to die. And that is just what
happened to him. For he died of hunger and cold and his bonds before
the messenger came back from the adviser. Now my father and the rest
of my relatives are angry with me, because for the sake of this
murderer I am prosecuting my father for murder. For they say he did
not kill him, and if he had killed him never so much, yet since the
dead man was a murderer, I ought not to trouble myself about such a
fellow, because it is unholy for a son to prosecute his father for
murder. Which shows how little they know what the divine law is in
regard to holiness and unholiness.

\textsc{Socrates}. But, in the name of Zeus, Euthyphro, do you think
your knowledge about divine laws and \page{17} holiness and unholiness
is so exact that, when the facts are as you say, you are not afraid of
doing something unholy yourself in prosecuting your father for
murder?

\textsc{Euthyphro}. I should be of no use, Socrates, and Euthyphro
would be in no way different from other men, if I did not have exact
knowledge about all such things.

\textsc{Socrates}. Then the best thing for me, my admirable Euthyphro,
is to become your pupil and, before the suit with Meletus comes on, to
challenge him and say that I always thought it very important before
to know about divine matters and that now, since he says I am doing
wrong by acting carelessly and making innovations in matters of
religion, I have become your pupil. And ``Meletus,'' I should say,
``if you acknowledge that Euthyphro is wise in such matters, then
believe that I also hold correct opinions, and do not bring me to
trial; and if you do not acknowledge that, then bring a suit against
him, my teacher, rather than against me, and charge him with
corrupting the old, namely, his father and me, which he does by
teaching me and by correcting and punishing his father.'' And if he
does not do as I ask and does not release me from the indictment or
bring it against you in my stead, I could say in the court the same
things I said in my challenge to him, could I not?

\textsc{Euthyphro}. By Zeus, Socrates, if he should undertake to
indict me, I fancy I should find his weak spot, and it would be much
more a question about him in court than about me.

\textsc{Socrates}. And I, my dear friend, perceiving this, wish to
become your pupil; for I know that neither \page{19} this fellow
Meletus, nor anyone else, seems to notice you at all, but he has seen
through me so sharply and so easily that he has indicted me for
impiety. Now in the name of Zeus, tell me what you just now asserted
that you knew so well. What do you say is the nature of piety and
impiety, both in relation to murder and to other things? Is not
holiness always the same with itself in every action, and, on the
other hand, is not unholiness the opposite of all holiness, always the
same with itself and whatever is to be unholy possessing some one
characteristic quality?

\textsc{Euthyphro}. Certainly, Socrates.

\textsc{Socrates}. Tell me then, what do you say holiness is, and what
unholiness?

\textsc{Euthyphro}. Well then, I say that holiness is doing what I am
doing now, prosecuting the wrongdoer who commits murder or steals from
the temples or does any such thing, whether he be your father, or your
mother or anyone else, and not prosecuting him is unholy. And,
Socrates, see what a sure proof I offer you,---a proof I have already
given to others,---that this is established and right and that we
ought not to let him who acts impiously go unpunished, no matter who
he may be. Men believe that Zeus is the best and most just of the
gods, and they acknowledge that he put his father in bonds because he
wickedly devoured his children, and he in turn had mutilated his
father for similar reasons; but they are incensed against me because I
proceed against my father when he has done wrong, and so they are
\page{21} inconsistent in what they say about the gods and about me.

\textsc{Socrates}. Is not this, Euthyphro, the reason why I am being
prosecuted, because when people tell such stories about the gods I
find it hard to accept them? And therefore, probably, people will say
I am wrong. Now if you, who know so much about such things, accept
these tales, I suppose I too must give way. For what am I to say, who
confess frankly that I know nothing about them? But tell me, in the
name of Zeus, the god of friendship, do you really believe these
things happened?

\textsc{Euthyphro}. Yes, and still more wonderful things than these,
Socrates, which most people do not know.

\textsc{Socrates}. And so you believe that there was really war
between the gods, and fearful enmities and battles and other things of
the sort, such as are told of by the poets and represented in varied
designs by the great artists in our sacred places and especially on
the robe which is carried up to the Acropolis at the great
Panathenaea? for this is covered with such representations. Shall we
agree that these things are true, Euthyphro?

\textsc{Euthyphro}. Not only these things, Socrates; but, as I said
just now, I will, if you like, tell you many other things about the
gods, which I am sure will amaze you when you hear them.

\textsc{Socrates}. I dare say. But you can tell me those things at
your leisure some other time. At present try to tell more clearly what
I asked you just now. For, my friend, you did not give me sufficient
information before, when I asked what holiness was, but you told me
that this was holy \page{23} which you are now doing, prosecuting your
father for murder.

\textsc{Euthyphro}. Well, what I said was true, Socrates.

\textsc{Socrates}. Perhaps. But, Euthyphro, you say that many other
things are holy, do you not?

\textsc{Euthyphro}. Why, so they are.

\textsc{Socrates}. Now call to mind that this is not what I asked you,
to tell me one or two of the many holy acts, but to tell the essential
aspect, by which all holy acts are holy; for you said that all unholy
acts were unholy and all holy ones holy by one aspect. Or don't you
remember?

\textsc{Euthyphro}. I remember.

\textsc{Socrates}. Tell me then what this aspect is, that I may keep
my eye fixed upon it and employ it as a model and, if anything you or
anyone else does agrees with it, may say that the act is holy, and if
not, that it is unholy.

\textsc{Euthyphro}. If you wish me to explain in that way, I will do
so.

\textsc{Socrates}. I do wish it.

\textsc{Euthyphro}. Well then, what is dear to the gods is holy, and
what is not dear to them is unholy.

\textsc{Socrates}. Excellent, Euthyphro; now you have answered as I
asked you to answer. However, whether it is true, I am not yet sure;
but you will, of course, show that what you say is true.

\textsc{Euthyphro}. Certainly.

\textsc{Socrates}. Come then, let us examine our words. The thing and
the person that are dear to the gods are holy, and the thing and the
person that are hateful to the gods are unholy; and the two are not
the same, but the holy and the unholy are the \page{25} exact
opposites of each other. Is not this what we have said?

\textsc{Euthyphro}. Yes, just this.

\textsc{Socrates}. And it seems to be correct?

\textsc{Euthyphro}. I think so, Socrates.

\textsc{Socrates}. Well then, have we said this also, that the gods,
Euthyphro, quarrel and disagree with each other, and that there is
enmity between them?

\textsc{Euthyphro}. Yes, we have said that.

\textsc{Socrates}. But what things is the disagreement about, which
causes enmity and anger? Let us look at it in this way. If you and I
were to disagree about number, for instance, which of two numbers were
the greater, would the disagreement about these matters make us
enemies and make us angry with each other, or should we not quickly
settle it by resorting to arithmetic?

\textsc{Euthyphro}. Of course we should.

\textsc{Socrates}. Then, too, if we were to disagree about the
relative size of things, we should quickly put an end to the
disagreement by measuring?

\textsc{Euthyphro}. Yes.

\textsc{Socrates}. And we should, I suppose, come to terms about
relative weights by weighing?

\textsc{Euthyphro}. Of course.

\textsc{Socrates}. But about what would a disagreement be, which we
could not settle and which would cause us to be enemies and be angry
with each other? Perhaps you cannot give an answer offhand; but let
\page{27} me suggest it. Is it not about right and wrong, and noble
and disgraceful, and good and bad? Are not these the questions about
which you and I and other people become enemies, when we do become
enemies, because we differ about them and cannot reach any
satisfactory agreement?

\textsc{Euthyphro}. Yes, Socrates, these are the questions about which
we should become enemies.

\textsc{Socrates}. And how about the gods, Euthyphro? If they
disagree, would they not disagree about these questions?

\textsc{Euthyphro}. Necessarily.

\textsc{Socrates}. Then, my noble Euthyphro, according to what you
say, some of the gods too think some things are right or wrong and
noble or disgraceful, and good or bad, and others disagree; for they
would not quarrel with each other if they did not disagree about these
matters. Is that the case?

\textsc{Euthyphro}. You are right.

\textsc{Socrates}. Then the gods in each group love the things which
they consider good and right and hate the opposites of these things?

\textsc{Euthyphro}. Certainly.

\textsc{Socrates}. But you say that the same things are considered
right by some of them and wrong by others; and it is because they
disagree about these things that they quarrel and wage war with each
other. Is not this what you said?

\textsc{Euthyphro}. It is.

\textsc{Socrates}. Then, as it seems, the same things are hated and
loved by the gods, and the same things would be dear and hateful to
the gods.

\textsc{Euthyphro}. So it seems.

\page{29}\textsc{Socrates}. And then the same things would be both
holy and unholy, Euthyphro, according to this statement.

\textsc{Euthyphro}. I suppose so.

\textsc{Socrates}. Then you did not answer my question, my friend. For
I did not ask you what is at once holy and unholy; but, judging from
your reply, what is dear to the gods is also hateful to the gods. And
so, Euthyphro, it would not be surprising if, in punishing your father
as you are doing, you were performing an act that is pleasing to Zeus,
but hateful to Cronus and Uranus, and pleasing to Hephaestus, but
hateful to Hera, and so forth in respect to the other gods, if any
disagree with any other about it.

\textsc{Euthyphro}. But I think, Socrates, that none of the gods
disagrees with any other about this, or holds that he who kills anyone
wrongfully ought not to pay the penalty.

\textsc{Socrates}. Well, Euthyphro, to return to men, did you ever
hear anybody arguing that he who had killed anyone wrongfully, or had
done anything else whatever wrongfully, ought not to pay the penalty?

\textsc{Euthyphro}. Why, they are always arguing these points,
especially in the law courts. For they do very many wrong things; and
then there is nothing they will not do or say, in defending
themselves, to avoid the penalty.

\textsc{Socrates}. Yes, but do they acknowledge, Euthyphro, that they
have done wrong and, although they acknowledge it, nevertheless say
that they ought not to pay the penalty?

\textsc{Euthyphro}. Oh, no, they don't do that.

\page{31}\textsc{Socrates}. Then there is something they do not do and
say. For they do not, I fancy, dare to say and argue that, if they
have really done wrong, they ought not to pay the penalty; but, I
think, they say they have not done wrong; do they not?

\textsc{Euthyphro}. You are right.

\textsc{Socrates}. Then they do not argue this point, that the
wrongdoer must not pay the penalty; but perhaps they argue about this,
who is a wrongdoer, and what he did, and when.

\textsc{Euthyphro}. That is true.

\textsc{Socrates}. Then is not the same thing true of the gods, if
they quarrel about right and wrong, as you say, and some say others
have done wrong, and some say they have not? For surely, my friend, no
one, either of gods or men, has the face to say that he who does wrong
ought not to pay the penalty.

\textsc{Euthyphro}. Yes, you are right about this, Socrates, in the
main.

\textsc{Socrates}. But I think, Euthyphro, those who dispute, both men
and gods, if the gods do dispute, dispute about each separate act.
When they differ with one another about any act, some say it was right
and others that it was wrong. Is it not so?

\textsc{Euthyphro}. Certainly.

\textsc{Socrates}. Come now, my dear Euthyphro, inform me, that I may
be made wiser, what proof you have that all the gods think that the
man lost his life wrongfully, who, when he was a servant, committed
\page{33} murder, was bound by the master of the man he killed, and
died as a result of his bonds before the master who had bound him
found out from the advisers what he ought to do with him, and that it
is right on account of such a man for a son to proceed against his
father and accuse him of murder. Come, try to show me clearly about
this, that the gods surely believe that this conduct is right; and if
you show it to my satisfaction, I will glorify your wisdom as long
as I live.

\textsc{Euthyphro}. But perhaps this is no small task, Socrates;
though I could show you quite clearly.

\textsc{Socrates}. I understand; it is because you think I am slower
to understand than the judges; since it is plain that you will show
them that such acts are wrong and that all the gods hate them.

\textsc{Euthyphro}. Quite clearly, Socrates; that is, if they listen
to me.

\textsc{Socrates}. They will listen, if they find that you are a good
speaker. But this occurred to me while you were talking, and I said to
myself: ``If Euthyphro should prove to me no matter how clearly that
all the gods think such a death is wrongful, what have I learned from
Euthyphro about the question, what is holiness and what is unholiness?
For this act would, as it seems, be hateful to the gods; but we saw
just now that holiness and its opposite are not defined in this way;
for we saw that what is hateful to the gods is also dear to them; and
so I let you off any discussion of this point, Euthyphro. If you like,
all the gods may think it wrong and may hate it. But shall we now
emend our definition and \page{35} say that whatever all the gods hate
is unholy and whatever they all love is holy, and what some love and
others hate is neither or both? Do you wish this now to be our
definition of holiness and unholiness?

\textsc{Euthyphro}. What is to hinder, Socrates?

\textsc{Socrates}. Nothing, so far as I am concerned, Euthyphro, but
consider your own position, whether by adopting this definition you
will most easily teach me what you promised.

\textsc{Euthyphro}. Well, I should say that what all the gods love is
holy and, on the other hand, what they all hate is unholy.

\textsc{Socrates}. Then shall we examine this again, Euthyphro, to see
if it is correct, or shall we let it go and accept our own statement,
and those of others, agreeing that it is so, if anyone merely says
that it is? Or ought we to inquire into the correctness of the
statement?

\textsc{Euthyphro}. We ought to inquire. However, I think this is now
correct.

\textsc{Socrates}. We shall soon know more about this, my friend. Just
consider this question:---Is that which is holy loved by the gods
because it is holy, or is it holy because it is loved by the gods?

\textsc{Euthyphro}. I don't know what you mean, Socrates.

\textsc{Socrates}. Then I will try to speak more clearly. We speak of
being carried and of carrying, of being led and of leading, of being
seen and of seeing; and you understand---do you not?---that in all
such expressions the two parts differ one from the other in meaning,
and how they differ.

\textsc{Euthyphro}. I think I understand.

\page{37}\textsc{Socrates}. Then, too, we conceive of a thing being
loved and of a thing loving, and the two are different?

\textsc{Euthyphro}. Of course.

\textsc{Socrates}. Now tell me, is a thing which is carried a carried
thing because one carries it, or for some other reason?

\textsc{Euthyphro}. No, for that reason.

\textsc{Socrates}. And a thing which is led is led because one leads
it, and a thing which is seen is so because one sees it?

\textsc{Euthyphro}. Certainly.

\textsc{Socrates}. Then one does not see it because its a seen thing,
but, on the contrary, it is a seen thing because one sees it; and one
does not lead it because it is a led thing, but it is a led thing
because one leads it; and one does not carry it because it is a
carried thing, but it is a carried thing because one carries it. Is it
clear, Euthyphro, what I am trying to say? I am trying to say this,
that if anything becomes or undergoes, it does not become because it
is in a state of becoming, but it is in a state of becoming because it
becomes, and it does not undergo because it is a thing which
undergoes, but because it undergoes it is a thing which undergoes; or
do you not agree to this?

\textsc{Euthyphro}. I agree.

\textsc{Socrates}. Is not that which is beloved a thing which is
either becoming or undergoing something?

\textsc{Euthyphro}. Certainly.

\textsc{Socrates}. And is this case like the former ones: those who
love it do not love it because it is a beloved thing, but it is a
beloved thing because they love it?

\textsc{Euthyphro}. Obviously.

\page{39}\textsc{Socrates}. Now what do you say about that which is
holy, Euthyphro? It is loved by all the gods, is it not, according to
what you said?

\textsc{Euthyphro}. Yes.

\textsc{Socrates}. For this reason, because it is holy, or for some
other reason?

\textsc{Euthyphro}. No, for this reason.

\textsc{Socrates}. It is loved because it is holy, not holy because it
is loved?

\textsc{Euthyphro}. I think so.

\textsc{Socrates}. But that which is dear to the gods is dear to them
and beloved by them because they love it.

\textsc{Euthyphro}. Of course.

\textsc{Socrates}. Then that which is dear to the gods and that which
is holy are not identical, but differ one from the other.

\textsc{Euthyphro}. How so, Socrates?

\textsc{Socrates}. Because we are agreed that the holy is loved
because it is holy and that it is not holy because it is loved; are we
not?

\textsc{Euthyphro}. Yes.

\textsc{Socrates}. But we are agreed that what is dear to the gods is
dear to them because they love it, that is, by reason of this love,
not that they love it because it is dear.

\textsc{Euthyphro}. Very true.

\textsc{Socrates}. But if that which is dear to the gods and that
which is holy were identical, my dear Euthyphro, then if the holy were
loved because it is holy, that which is dear to the gods would be
loved because it is dear, and if that which is dear to the gods is
dear because it is loved, then that which is holy would be holy
because \page{41} it is loved; but now you see that the opposite is
the case, showing that the two are different from each other. For the
one becomes lovable from the fact that it is loved, whereas the other
is loved because it is in itself lovable. And, Euthyphro, it seems
that when you were asked what holiness is you were unwilling to make
plain its essence, but you mentioned something that has happened to
this holiness, namely, that it is loved by the gods. But you did not
tell as yet what it really is. So, if you please, do not hide it from
me, but begin over again and tell me what holiness is, no matter
whether it is loved by the gods or anything else happens it; for we
shall not quarrel about that. But tell me frankly, What is holiness,
and what is unholiness?

\textsc{Euthyphro}. But, Socrates, I do not know how to say what I
mean. For whatever statement we advance, somehow or other it moves
about and won't stay where we put it.

\textsc{Socrates}. Your statements, Euthyphro, are like works of
my\footnote{Socrates was the son of a sculptor and was himself
educated to be a sculptor. This is doubtless the reason for his
reference to Daedalus as an ancestor. Daedalus was a half mythical
personage whose statues were said to have been so lifelike that they
moved their eyes and walked about.} ancestor Daedalus, and if I were
the one who made or advanced them, you might laugh at me and say that
on account of my relationship to him my works in words run away and
won't stay where they are put. But now---well, the statements are
yours; so some other jest is demanded; for they stay fixed, as you
yourself see.

\textsc{Euthyphro}. I think the jest does very well as it \page{43}
is; for I am not the one who makes these statements move about and not
stay in the same place, but you are the Daedalus; for they would have
stayed, so far as I am concerned.

\textsc{Socrates}. Apparently then, my friend, I am a more clever
artist than Daedalus, inasmuch as he made only his own works move,
whereas I, as it seems, give motion to the works of others as well as
to my own. And the most exquisite thing about my art is that I am
clever against my will; for I would rather have my words stay fixed
and stable than possess the wisdom of Daedalus and the wealth of
Tantalus besides. But enough of this. Since you seem to be indolent, I
will aid you myself, so that you may instruct me about holiness. And
do not give it up beforehand. Just see whether you do not think that
everything that is holy is right.

\textsc{Euthyphro}. I do.

\textsc{Socrates}. But is everything that is right also holy? Or is
all which is holy right, and not all which is right holy, but part of
it holy and part something else?

\textsc{Euthyphro}. I can't follow you, Socrates.

\textsc{Socrates}. And yet you are as much younger than I as you are
wiser; but, as I said, you are indolent on account of your wealth of
wisdom. But exert \page{45} yourself, my friend; for it is not hard to
understand what I mean. What I mean is the opposite of what the
poet\footnote{Stasinus, author of the ``Cypria'' (Fragm. 20, ed.
Kinkel).} said, who wrote: ``Zeus the creator, him who made all
things, thou wilt not name; for where fear is, there also is
reverence.'' Now I disagree with the poet. Shall I tell you how?

\textsc{Euthyphro}. By all means.

\textsc{Socrates}. It does not seem to me true that where fear is,
there also is reverence; for many who fear diseases and poverty and
other such things seem to me to fear, but not to reverence at all
these things which they fear. Don't you think so, too?

\textsc{Euthyphro}. Certainly.

\textsc{Socrates}. But I think that where reverence is, there also is
fear; for does not everyone who has a feeling of reverence and shame
about any act also dread and fear the reputation for wickedness?

\textsc{Euthyphro}. Yes, he does fear.

\textsc{Socrates}. Then it is not correct to say ``where fear is,
there also is reverence.'' On the contrary, where reverence is, there
also is fear; but reverence is not everywhere where fear is, since, as
I think, fear is more comprehensive than reverence; for reverence is a
part of fear, just as the odd is a part of number, so that it is not
true that where number is, there also is the odd, but that where the
odd is, there also is number. Perhaps you follow me now?

\textsc{Euthyphro}. Perfectly.

\textsc{Socrates}. It was something of this sort that I meant before,
when I asked whether where the right is, there also is holiness, or
where holiness is, \page{47} there also is the right; but holiness is
not everywhere where the right is, for holiness is a part of the
right. Do we agree to this, or do you dissent?

\textsc{Euthyphro}. No, I agree; for I think the statement is correct.

\textsc{Socrates}. Now observe the next point. If holiness is a part
of the right, we must, apparently, find out what part of the right
holiness is. Now if you asked me about one of the things I just
mentioned, as, for example, what part of number the even was, and what
kind of a number it was I should say, ``that which is not indivisible
by two, but divisible by two''; or don't you agree?

\textsc{Euthyphro}. I agree.

\textsc{Socrates}. Now try in your turn to teach me what part of the
right holiness is, that I may tell Meletus not to wrong me any more or
bring suits against me for impiety, since I have now been duly
instructed by you about what is, and what is not, pious and holy.

\textsc{Euthyphro}. This then is my opinion, Socrates, that the part
of the right which has to do with attention to the gods constitutes
piety and holiness, and that the remaining part of the right is that
which has to do with the service of men.

\textsc{Socrates}. I think you are correct, Euthyphro; but there is
one little point about which I still want information, for I do not
yet understand what you mean by ``attention.'' I don't suppose you
mean the same kind of attention to the gods which is paid to other
things. We say, for example, that not everyone knows how to attend to
horses, but only he who is skilled in horsemanship, do we not?

\page{49}\textsc{Euthyphro}. Certainly.

\textsc{Socrates}. Then horsemanship is the art of attending to
horses?

\textsc{Euthyphro}. Yes.

\textsc{Socrates}. And not everyone knows how to attend to dogs, but
only the huntsman?

\textsc{Euthyphro}. That is so.

\textsc{Socrates}. Then the huntsman's art is the art of attending to
dogs?

\textsc{Euthyphro}. Yes.

\textsc{Socrates}. And the oxherd's art is that of attending to oxen?

\textsc{Euthyphro}. Certainly.

\textsc{Socrates}. And holiness and piety is the art of attending to
the gods? Is that what you mean, Euthyphro?

\textsc{Euthyphro}. Yes.

\textsc{Socrates}. Now does attention always aim to accomplish the
same end? I mean something like this: It aims at some good or benefit
to the one to whom it is given, as you see that horses, when attended
to by the horseman's art are benefited and made better; or don't you
think so?

\textsc{Euthyphro}. Yes, I do.

\textsc{Socrates}. And dogs are benefited by the huntsman's art and
oxen by the oxherd's and everything else in the same way? Or do you
think care and attention are ever meant for the injury of that which
is cared for?

\textsc{Euthyphro}. No, by Zeus, I do not.

\textsc{Socrates}. But for its benefit?

\textsc{Euthyphro}. Of course.

\textsc{Socrates}. Then holiness, since it is the art of attending to
the gods, is a benefit to the gods, and \page{51} makes them better?
And you would agree that when you do a holy or pious act you are
making one of the gods better?

\textsc{Euthyphro}. No, by Zeus, not I.

\textsc{Socrates}. Nor do I, Euthyphro, think that is what you meant.
Far from it. But I asked what you meant by ``attention to the gods''
just because I did not think you meant anything like that.

\textsc{Euthyphro}. You are right, Socrates; that is not what I mean.

\textsc{Socrates}. Well, what kind of attention to the gods is
holiness?

\textsc{Euthyphro}. The kind, Socrates, that servants pay to their
masters.

\textsc{Socrates}. I understand. It is, you mean, a kind of service to
the gods?

\textsc{Euthyphro}. Exactly.

\textsc{Socrates}. Now can you tell me what result the art that serves
the physician serves to produce? Is it not health?

\textsc{Euthyphro}. Yes.

\textsc{Socrates}. Well then; what is it which the art that serves
shipbuilders serves to produce?

\textsc{Euthyphro}. Evidently, Socrates, a ship.

\textsc{Socrates}. And that which serves housebuilders serves to build
a house?

\textsc{Euthyphro}. Yes.

\textsc{Socrates}. Then tell me, my friend; what would the art which
serves the gods serve to accomplish? For it is evident that you know,
since you say you know more than any other man about matters which
have to do with the gods.

\textsc{Euthyphro}. And what I say is true, Socrates.

\textsc{Socrates}. Then, in the name of Zeus, tell me, \page{53} what
is that glorious result which the gods accomplish by using us as
servants?

\textsc{Euthyphro}. They accomplish many fine results, Socrates.

\textsc{Socrates}. Yes, and so do generals, my friend; but
nevertheless, you could easily tell the chief of them, namely, that
they bring about victory in war. Is that not the case?

\textsc{Euthyphro}. Of course.

\textsc{Socrates}. And farmers also, I think, accomplish many fine
results; but still the chief result of their work is food from the
land?

\textsc{Euthyphro}. Certainly.

\textsc{Socrates}. But how about the many fine results the gods
accomplish? What is the chief result of their work?

\textsc{Euthyphro}. I told you a while ago, Socrates, that it is a
long task to learn accurately all about these things. However, I say
simply that when one knows how to say and do what is gratifying to the
gods, in praying and sacrificing, that is holiness, and such things
bring salvation to individual families and to states; and the opposite
of what is gratifying to the gods is impious, and that overturns and
destroys everything.

% NOTE: added missing comma after 'Euthyphro'

\textsc{Socrates}. You might, if you wished, Euthyphro, have answered
much more briefly the chief part of my question. But it is plain that
you do not care to instruct me. For now, when you were close upon it
you turned aside; and if you had answered it, I should already have
obtained from you all the instruction I need about holiness. But, as
things are, the questioner must follow the one questioned wherever he
leads. What do you say the holy, or \page{55} holiness, is? Do you not
say that it is a kind of science of sacrificing and praying?

\textsc{Euthyphro}. Yes.

\textsc{Socrates}. And sacrificing is making gifts to the gods and
praying is asking from them?

\textsc{Euthyphro}. Exactly, Socrates.

\textsc{Socrates}. Then holiness, according to this definition, would
be a science of giving and asking.

\textsc{Euthyphro}. You understand perfectly what I said, Socrates.

\textsc{Socrates}. Yes, my friend, for I am eager for your wisdom, and
give my mind to it, so that nothing you say shall fall to the ground.
But tell me, what is this service of the gods? Do you say that it
consists in asking from them and giving to them?

\textsc{Euthyphro}. Yes.

\textsc{Socrates}. Would not the right way of asking be to ask of them
what we need from them?

\textsc{Euthyphro}. What else?

\textsc{Socrates}. And the right way of giving, to present them with
what they need from us? For it would not be scientific giving to give
anyone what he does not need.

\textsc{Euthyphro}. You are right, Socrates.

\textsc{Socrates}. Then holiness would be an art of barter between
gods and men?

\textsc{Euthyphro}. Yes, of barter, if you like to call it so.

\textsc{Socrates}. I don't like to call it so, if it is not true. But
tell me, what advantage accrues to the gods from \page{57} the gifts
they get from us? For everybody knows what they give, since we have
nothing good which they do not give. But what advantage do they derive
from what they get from us? Or have we so much the better of them in
our bartering that we get all good things from them and they nothing
from us?

\textsc{Euthyphro}. Why you don't suppose, Socrates, that the gods
gain any advantage from what they get from us, do you?

\textsc{Socrates}. Well then, what would those gifts of ours to the
gods be?

\textsc{Euthyphro}. What else than honour and praise, and, as I said
before, gratitude?

\textsc{Socrates}. Then, Euthyphro, holiness is grateful to the gods,
but not advantageous or precious to the gods?

\textsc{Euthyphro}. I think it is precious, above all things.

\textsc{Socrates}. Then again, it seems, holiness is that which is
precious to the gods.

\textsc{Euthyphro}. Certainly.

\textsc{Socrates}. Then will you be surprised, since you say this, if
your words do not remain fixed but walk about, and will you accuse me
of being the Daedalus who makes them walk, when you are yourself much
more skilful than Daedalus and make them go round in a circle? Or do
you not see that our definition has come round to the point from which
it started? For you remember, I suppose, that a while ago we found
that holiness and what is dear to the gods were not the same, but
different from each other; or do you not remember?

\textsc{Euthyphro}. Yes, I remember.

\textsc{Socrates}. Then don't you see that now you say \page{59} that
what is precious to the gods is holy? And is not this what is dear to
the gods?

\textsc{Euthyphro}. Certainly.

\textsc{Socrates}. Then either our agreement a while ago was wrong, or
if that was right, we are wrong now.

\textsc{Euthyphro}. So it seems.

\textsc{Socrates}. Then we must begin again at the beginning and ask
what holiness is. Since I shall not willingly give up until I learn.
And do not scorn me, but by all means apply your mind now to the
utmost and tell me the truth; for you know, if anyone does, and like
Proteus, you must be held until you speak. For if you had not clear
knowledge of holiness and unholiness, you would surely not have
undertaken to prosecute your aged father for murder for the sake of a
servant. You would have been afraid to risk the anger of the gods, in
case your conduct should be wrong, and would have been ashamed in the
sight of men. But now I am sure you think you know what is holy and
what is not. So tell me, most excellent Euthyphro, and do not conceal
your thought.

\textsc{Euthyphro}. Some other time, Socrates. Now I am in a hurry and
it is time for me to go.

\textsc{Socrates}. Oh my friend, what are you doing? You go away and
leave me cast down from the high hope I had that I should learn from
you what is holy, and what is not, and should get rid of Meletus's
indictment by showing him that I have been made wise by Euthyphro
about divine matters and am no longer through ignorance acting
carelessly and making innovations in respect to them, and that I shall
live a better life henceforth.

