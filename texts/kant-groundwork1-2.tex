
\author{Immanuel Kant}
\authdate{1724--1804}
\textdate{1785}
\addon{First and Second Sections, excerpt}
\chapter[Groundwork of the Metaphysics of Morals, excerpt]{Groundwork
of the Metaphysics of Morals}
\source{kant1909.1}

\page{9}\section{First Section.\\\smaller Transition from the Common
Rational Knowledge of Morality to the Philosophical.}

Nothing can possibly be conceived in the world, or even out of it,
which can be called good, without qualification, except a Good Will.
Intelligence, wit, judgment, and the other \textit{talents} of the
mind, however they may be named, or courage, resolution, perseverance,
as qualities of temperament, are undoubtedly good and desirable in
many respects; but these gifts of nature may also become extremely bad
and mischievous if the will which is to make use of them, and which,
therefore, constitutes what is called \textit{character}, is not good.
It is the same with the \textit{gifts of fortune}. Power, riches,
honour, even health, and the general well-being and contentment with
one's condition which is called \textit{happiness}, inspire pride, and
often presumption, if there is not a good will to correct the
influence of these on the mind, and with this also to rectify the
whole principle of acting, and adapt it to its end. The sight of a
being who is not adorned with a single feature of a pure and good
will, enjoying unbroken prosperity, can never give pleasure to an
impartial rational spectator. Thus a good will appears to constitute
the indispensable condition even of being worthy of happiness.

There are even some qualities which are of service to this good will
itself, and may facilitate its action, yet which have no intrinsic
unconditional value, but always presuppose a good will, and this
qualifies the esteem that we justly have for them, and does not permit
us to regard them as absolutely good. Moderation in the affections and
passions, self-control, and calm deliberation are not only good in
many respects, but even seem to constitute part of the intrinsic worth
of the person; but they are far from deserving to be called good
without \page{10} qualification, although they have been so
unconditionally praised by the ancients. For without the principles of
a good will, they may become extremely bad; and the coolness of a
villain not only makes him far more dangerous, but also directly makes
him more abominable in our eyes than he would have been without it.

A good will is good not because of what it performs or effects, not by
its aptness for the attainment of some proposed end, but simply by
virtue of the volition, that is, it is good in itself, and considered
by itself is to be esteemed much higher than all that can be brought
about by it in favour of any inclination, nay, even of the sum-total
of all inclinations. Even if it should happen that, owing to special
disfavour of fortune, or the niggardly provision of a step-motherly
nature, this will should wholly lack power to accomplish its purpose,
if with its greatest efforts it should yet achieve nothing, and there
should remain only the good will (not, to be sure, a mere wish, but
the summoning of all means in our power), then, like a jewel, it would
still shine by its own light, as a thing which has its whole value in
itself. Its usefulness or fruitlessness can neither add to nor take
away anything from this value. It would be, as it were, only the
setting to enable us to handle it the more conveniently in common
commerce, or to attract to it the attention of those who are not yet
connoisseurs, but not to recommend it to true connoisseurs, or to
determine its value.

\snip

\page{12}We have then to develop the notion of a will which deserves
to be highly esteemed for itself, and is good without a view to
\page{13} anything further, a notion which exists already in the sound
natural understanding, requiring rather to be cleared up than to be
taught, and which in estimating the value of our actions always takes
the first place, and constitutes the condition of all the rest. In
order to do this, we will take the notion of duty, which includes that
of a good will, although implying certain subjective restrictions and
hindrances. These, however, far from concealing it, or rendering it
unrecognizable, rather bring it out by contrast, and make it shine
forth so much the brighter.

I omit here all actions which are already recognized as inconsistent
with duty, although they may be useful for this or that purpose, for
with these the question whether they are done \textit{from duty}
cannot arise at all, since they even conflict with it. I also set
aside those actions which really conform to duty, but to which men
have \textit{no} direct \textit{inclination}, performing them because
they are impelled thereto by some other inclination. For in this
case we can readily distinguish whether the action which agrees with
duty is done \textit{from duty}, or from a selfish view. It is much
harder to make this distinction when the action accords with duty, and
the subject has besides a \textit{direct} inclination to it. For
example, it is always a matter of duty that a dealer should not
overcharge an inexperienced purchaser; and wherever there is much
commerce the prudent tradesman does not overcharge, but keeps a fixed
price for everyone, so that a child buys of him as well as any other.
Men are thus \textit{honestly} served; but this is not enough to make
us believe that the tradesman has so acted from duty and from
principles of honesty: his own advantage required it; it is out of the
question in this case to suppose that he might besides have a direct
inclination in favour of the buyers, so that, as it were, from love he
should give no advantage to one over another. Accordingly the action
was done neither from duty nor from direct inclination, but merely
with a selfish view.

On the other hand, it is a duty to maintain one's life; and, in
addition, everyone has also a direct inclination to do so. But on this
account the often anxious care which most men take for \page{14} it
has no intrinsic worth, and their maxim has no moral import. They
preserve their life \textit{as duty requires}, no doubt, but not
\textit{because duty requires}. On the other hand, if adversity and
hopeless sorrow have completely taken away the relish for life; if the
unfortunate one, strong in mind, indignant at his fate rather than
desponding or dejected, wishes for death, and yet preserves his life
without loving it---not from inclination or fear, but from duty---then
his maxim has a moral worth.

To be beneficent when we can is a duty; and besides this, there are
many minds so sympathetically constituted that, without any other
motive of vanity or self-interest, they find a pleasure in spreading
joy around them, and can take delight in the satisfaction of others so
far as it is their own work. But I maintain that in such a case an
action of this kind, however proper, however amiable it may be, has
nevertheless no true moral worth, but is on a level with other
inclinations, \textit{e.g.} the inclination to honour, which, if it is
happily directed to that which is in fact of public utility and
accordant with duty, and consequently honourable, deserves praise and
encouragement, but not esteem. For the maxim lacks the moral import,
namely, that such actions be done \textit{from duty}, not from
inclination. Put the case that the mind of that philanthropist was
clouded by sorrow of his own, extinguishing all sympathy with the lot
of others, and that while he still has the power to benefit others in
distress, he is not touched by their trouble because he is absorbed
with his own; and now suppose that he tears himself out of this dead
insensibility, and performs the action without any inclination to it,
but simply from duty, then first has his action its genuine moral
worth. Further still; if nature has put little sympathy in the heart
of this or that man; if he, supposed to be an upright man, is by
temperament cold and indifferent to the sufferings of others, perhaps
because in respect of his own he is provided with the special gift of
patience and fortitude, and supposes, or even requires, that others
should have the same---and such a man would certainly not be the
meanest product of nature---but if nature had not specially framed him
for a philanthropist, would he not still find in himself a source
\page{15} from whence to give himself a far higher worth than that of
a good-natured temperament could be? Unquestionably. It is just in
this that the moral worth of the character is brought out which is
incomparably the highest of all, namely, that he is beneficent, not
from inclination, but from duty.

% NOTE: the source has 'trangression' below

To secure one's own happiness is a duty, at least indirectly; for
discontent with one's condition, under a pressure of many anxieties
and amidst unsatisfied wants, might easily become a great
\textit{temptation to transgression of duty}. But here again, without
looking to duty, all men have already the strongest and most intimate
inclination to happiness, because it is just in this idea that all
inclinations are combined in one total. But the precept of happiness
is often of such a sort that it greatly interferes with some
inclinations, and yet a man cannot form any definite and certain
conception of the sum of satisfaction of all of them which is called
happiness. It is not then to be wondered at that a single inclination,
definite both as to what it promises and as to the time within which
it can be gratified, is often able to overcome such a fluctuating
idea, and that a gouty patient, for instance, can choose to enjoy what
he likes, and to suffer what he may, since, according to his
calculation, on this occasion at least, he has [only] not sacrificed
the enjoyment of the present moment to a possibly mistaken expectation
of a happiness which is supposed to be found in health. But even in
this case, if the general desire for happiness did not influence his
will, and supposing that in his particular case health was not a
necessary element in this calculation, there yet remains in this, as
in all other cases, this law, namely, that he should promote his
happiness not from inclination but from duty, and by this would his
conduct first acquire true moral worth.

It is in this manner, undoubtedly, that we are to understand those
passages of Scripture also in which we are commanded to love our
neighbour, even our enemy. For love, as an affection, cannot be
commanded, but beneficence for duty's sake may; even though we are not
impelled to it by any inclination---nay, are even repelled by a
natural and unconquerable aversion. This is \textit{practical} love,
and not \textit{pathological}---a love which is seated in \page{16}
the will, and not in the propensions of sense---in principles of
action and not of tender sympathy; and it is this love alone which can
be commanded.

The second proposition is: That an action done from duty derives its
moral worth, \textit{not from the purpose} which is to be attained by
it, but from the maxim by which it is determined, and therefore does
not depend on the realization of the object of the action, but merely
on the \textit{principle of volition} by which the action has taken
place, without regard to any object of desire. It is clear from what
precedes that the purposes which we may have in view in our actions,
or their effects regarded as ends and springs of the will, cannot give
to actions any unconditional or moral worth. In what, then, can their
worth lie, if it is not to consist in the will and in reference to its
expected effect? It cannot lie anywhere but in the \textit{principle
of the will} without regard to the ends which can be attained by the
action. For the will stands between its \textit{\`a priori} principle,
which is formal, and its \textit{\`a posteriori} spring, which is
material, as between two roads, and as it must be determined by
something, it follows that it must be determined by the formal
principle of volition when an action is done from duty, in which case
every material principle has been withdrawn from it.

The third proposition, which is a consequence of the two preceding, I
would express thus: \textit{Duty is the necessity of acting from
respect for the law}. I may have \textit{inclination} for an object as
the effect of my proposed action, but I cannot have \textit{respect}
for it, just for this reason, that it is an effect and not an energy
of will. Similarly, I cannot have respect for inclination, whether my
own or another's; I can at most, if my own, approve it; if another's,
sometimes even love it; \textit{i.e.} look on it as favourable to my
own interest. It is only what is connected with my will as a
principle, by no means as an effect---what does not subserve my
inclination, but overpowers it, or at least in case of choice excludes
it from its calculation---in other words, simply the law \page{17} of
itself, which can be an object of respect, and hence a command. Now an
action done from duty must wholly exclude the influence of
inclination, and with it every object of the will, so that nothing
remains which can determine the will except objectively the
\textit{law}, and subjectively \textit{pure respect} for this
practical law, and consequently the maxim\footnote{A \textit{maxim} is
the subjective principle of volition. The objective principle
(\textit{i.e.} that which would also serve subjectively as a practical
principle to all rational beings if reason had full power over the
faculty of desire) is the practical law.} that I should follow this
law even to the thwarting of all my inclinations.

Thus the moral worth of an action does not lie in the effect expected
from it, nor in any principle of action which requires to borrow its
motive from this expected effect. For all these
effects---agreeableness of one's condition, and even the promotion of
the happiness of others---could have been also brought about by other
causes, so that for this there would have been no need of the will of
a rational being; whereas it is in this alone that the supreme and
unconditional good can be found. The pre-eminent good which we call
moral can therefore consist in nothing else than the
\textit{conception of law} in itself, \textit{which certainly is only
possible in a rational being}, in so far as this conception, and not
the expected effect, determines the will. This is a good which is
already present in the person who acts accordingly, and we have not to
wait for it to appear first in the result\footnote{It might be here
objected to me that I take refuge behind the word \textit{respect} in
an obscure feeling, instead of giving a distinct solution of the
question by a concept of the reason. But although respect is a
feeling, it is not a feeling \textit{received} through influence, but
is \textit{self-wrought} by a rational concept, and, therefore, is
specifically distinct from all feelings of the former kind, which may
be referred either to inclination or fear. What I recognize
immediately as a law for me, I recognize with respect. This merely
signifies the consciousness that my will is \textit{subordinate} to a
law, without the intervention of other influences on my sense. The
immediate determination of the will by the law, and the consciousness
of this, is called \textit{respect}, so that this is regarded as an
\textit{effect} of the law on the subject, and not as the
\textit{cause} of it. Respect is properly the conception of a worth
which thwarts my self-love. Accordingly it is something which is
considered neither as an object of inclination nor of fear, although
it has something analogous to both. The \textit{object} of respect is
the \textit{law} only, and that, the law which we impose on
\textit{ourselves}, and yet recognize as necessary in itself. As a
law, we are subjected to it without consulting self-love; as imposed
by us on ourselves, it is a result of our will. In the former aspect
it has an analogy to fear, in the latter to inclination. Respect for a
person is properly only respect for the law (of honesty, \&c.) of
which he gives us an example. Since we also look on the improvement of
our talents as a duty, we consider that we see in a person of talents,
as it were, the \textit{example of a law} (viz. to become like him in
this by exercise), and this constitutes our respect. All so-called
moral \textit{interest} consists simply in \textit{respect} for the
law.}.

\section{Second Section.\\\smaller Transition from Popular Moral
Philosophy to the Metaphysic of Morals.}

% NOTE: the source has 'independent on inclination' below

\page{29}Everything in nature works according to laws. Rational beings
alone have the faculty of acting according \textit{to the conception}
of laws, that is according to principles, \textit{i.e.} have a
\textit{will}. Since the deduction of actions from principles requires
\textit{reason}, the will is nothing but practical reason. If reason
infallibly determines the will, then the actions of such a being which
are recognized as objectively necessary are subjectively necessary
also, \textit{i.e.} the will is a faculty to choose \textit{that only}
which reason independent of inclination recognizes as practically
necessary, \textit{i.e.} as good. But if reason of itself does not
sufficiently determine the will, if the latter is subject also to
subjective conditions (particular impulses) which do not always
coincide with the objective conditions; in a word, if the will does
not \textit{in itself} completely accord with reason (which is
actually the case with men), then the actions which objectively are
recognized as necessary are subjectively contingent, and the
determination of \page{30} such a will according to objective laws is
\textit{obligation}, that is to say, the relation of the objective
laws to a will that is not thoroughly good is conceived as the
determination of the will of a rational being by principles of reason,
but which the will from its nature does not of necessity follow.

The conception of an objective principle, in so far as it is
obligatory for a will, is called a command (of reason), and the
formula of the command is called an Imperative.

All imperatives are expressed by the word \textit{ought} [or
\textit{shall}], and thereby indicate the relation of an objective law
of reason to a will, which from its subjective constitution is not
necessarily determined by it (an obligation). They say that something
would be good to do or to forbear, but they say it to a will which
does not always do a thing because it is conceived to be good to do
it. That is practically \textit{good}, however, which determines the
will by means of the conceptions of reason, and consequently not from
subjective causes, but objectively, that is on principles which are
valid for every rational being as such. It is distinguished from the
\textit{pleasant}, as that which influences the will only by means of
sensation from merely subjective causes, valid only for the sense of
this or that one, and not as a principle of reason, which holds for
every one.\footnote{The dependence of the desires on sensations is
called inclination, and this accordingly always indicates a
\textit{want}. The dependence of a contingently determinable will on
principles of reason is called an \textit{interest}. This, therefore,
is found only in the case of a dependent will which does not always of
itself conform to reason; in the Divine will we cannot conceive any
interest. But the human will can also \textit{take an interest} in a
thing without therefore acting \textit{from interest}. The former
signifies the \textit{practical} interest in the action, the latter
the \textit{pathological} in the object of the action. The former
indicates only dependence of the will on principles of reason in
themselves; the second, dependence on principles of reason for the
sake of inclination, reason supplying only the practical rules how the
requirement of the inclination may be satisfied. In the first case the
action interests me; in the second the object of the action (because
it is pleasant to me). We have seen in the first section that in an
action done from duty we must look not to the interest in the object,
but only to that in the action itself, and in its rational principle
(viz. the law).}

\page{31}A perfectly good will would therefore be equally subject to
objective laws (viz. laws of good), but could not be conceived as
\textit{obliged} thereby to act lawfully, because of itself from its
subjective constitution it can only be determined by the conception of
good. Therefore no imperatives hold for the Divine will, or in general
for a \textit{holy} will; \textit{ought} is here out of place, because
the volition is already of itself necessarily in unison with the law.
Therefore imperatives are only formul\ae{} to express the relation of
objective laws of all volition to the subjective imperfection of the
will of this or that rational being, \textit{e.g.} the human will.

Now all \textit{imperatives} command either \textit{hypothetically} or
\textit{categorically}. The former represent the practical necessity
of a possible action as means to something else that is willed (or at
least which one might possibly will). The categorical imperative would
be that which represented an action as necessary of itself without
reference to another end, \textit{i.e.}, as objectively necessary.

Since every practical law represents a possible action as good, and on
this account, for a subject who is practically determinable by reason,
necessary, all imperatives are formul\ae{} determining an action which
is necessary according to the principle of a will good in some
respects. If now the action is good only as a means \textit{to
something else}, then the imperative is \textit{hypothetical}; if it
is conceived as good \textit{in itself} and consequently as being
necessarily the principle of a will which of itself conforms to
reason, then it is \textit{categorical}.

\snip

\page{34}Now arises the question, how are all these imperatives
possible? This question does not seek to know how we can conceive the
accomplishment of the action which the imperative ordains, but merely
how we can conceive the obligation of the will which the imperative
expresses. No special explanation is needed to show how an imperative
of skill is possible. Whoever wills the end, wills also (so far as
reason decides his conduct) the means in his power which are
indispensably necessary thereto. This proposition is, as regards the
volition, analytical; for, in willing an object as my effect, there is
already thought the causality of myself as an acting cause, that is to
say, the use of the means; and the imperative educes from the
conception of volition of an end the conception of actions necessary
to this end. Synthetical propositions must no doubt be employed in
defining the means to a proposed end; but they do not concern the
principle, the act of the will, but the object and its realization.
\textit{Ex. gr.}, that in order to bisect a line on an unerring
principle I must draw from its extremities two intersecting arcs; this
no doubt is taught by mathematics only in synthetical propositions;
but if I know that it is only by this process that the intended
operation can be performed, then to say that if I fully will the
operation, I also will the action required for it, is an analytical
proposition; for it is one and the same thing to conceive something as
an effect which I can \page{35} produce in a certain way, and to
conceive myself as acting in this way.

\snip

\page{36}On the other hand, the question, how the imperative of
\textit{morality} is possible, is undoubtedly one, the only one,
demanding a solution, as this is not at all hypothetical, and the
objective necessity which it presents cannot rest on any hypothesis,
as is the ease with the hypothetical imperatives. Only here we must
never leave out of consideration that we \textit{cannot} make out
\textit{by any example}, in other words empirically, whether there is
such an imperative at all; but it is rather to be feared that all
those which seem to be categorical may yet be at bottom hypothetical.
For instance, when the precept is: Thou shalt not promise deceitfully;
and it is assumed that the necessity of this is not a mere counsel to
avoid some other evil, so that it should mean: Thou shalt not make a
lying promise, lest if it become known thou shouldst destroy thy
credit, but that an action of this kind must be regarded as evil in
itself, so that the imperative of the prohibition is categorical; then
we cannot \page{37} show with certainty in any example that the will
was determined merely by the law, without any other spring of action,
although it may appear to be so. For it is always possible that fear
of disgrace, perhaps also obscure dread of other dangers, may have a
secret influence on the will. Who can prove by experience the
non-existence of a cause when all that experience tells us is that we
do not perceive it? But in such a case the so-called moral imperative,
which as such appears to be categorical and unconditional, would in
reality be only a pragmatic precept, drawing our attention to our own
interests, and merely teaching us to take these into consideration.

We shall therefore have to investigate \textit{\`a priori} the
possibility of a categorical imperative, as we have not in this case
the advantage of its reality being given in experience, so that [the
elucidation of] its possibility should be requisite only for its
explanation, not for its establishment. In the meantime it may be
discerned beforehand that the categorical imperative alone has the
purport of a practical law: all the rest may indeed be called
\textit{principles} of the will but not laws, since whatever is only
necessary for the attainment of some arbitrary purpose may be
considered as in itself contingent, and we can at any time be free
from the precept if we give up the purpose: on the contrary, the
unconditional command leaves the will no liberty to choose the
opposite; consequently it alone carries with it that necessity which
we require in a law.

\snip

\page{38}When I conceive a hypothetical imperative, in general I do
not know beforehand what it will contain until I am given the
condition. But when I conceive a categorical imperative, I know at
once what it contains. For as the imperative contains besides the law
only the necessity that the maxims\footnote{A \textsc{Maxim} is a
subjective principle of action, and must be distinguished from the
\textit{objective principle}, namely, practical law. The former
contains the practical rule set by reason according to the conditions
of the subject (often its ignorance or its inclinations), so that it
is the principle on which the subject \textit{acts}; but the law is
the objective principle valid for every rational being, and is the
principle on which it \textit{ought to act} that is an imperative.}
shall conform to this law, while the law contains no conditions
restricting it, there remains nothing but the general statement that
the maxim of the action should conform to a universal law, and it is
this conformity alone that the imperative properly represents, as
necessary.

There is therefore but one categorical imperative, namely, this:
\textit{Act only on that maxim whereby thou canst at the same time
will that it should become a universal law}.

Now if all imperatives of duty can be deduced from this one imperative
as from their principle, then, although it should remain undecided
whether what is called duty is not merely a \page{39} vain notion, yet
at least we shall be able to show what we understand by it and what
this notion means.

Since the universality of the law according to which effects are
produced constitutes what is properly called \textit{nature} in the
most general sense (as to form), that is the existence of things so
far as it is determined by general laws, the imperative of duty may be
expressed thus: \textit{Act as if the maxim of thy action were to
become by thy will a universal law of nature}.

We will now enumerate a few duties, adopting the usual division of
them into duties to ourselves and to others, and into perfect and
imperfect duties.\footnote{It must be noted here that I reserve the
division of duties for a future \textit{metaphysic of morals}; so that
I give it here only as an arbitrary one (in order to arrange my
examples). For the rest, I understand by a perfect duty one that
admits no exception in favour of inclination, and then I have not
merely external but also internal perfect duties. This is contrary to
the use of the word adopted in the schools; but I do not intend to
justify it here, as it is all one for my purpose whether it is
admitted or not.}

1. A man reduced to despair by a series of misfortunes feels wearied
of life, but is still so far in possession of his reason that he can
ask himself whether it would not be contrary to his duty to himself to
take his own life. Now he inquires whether the maxim of his action
could become a universal law of nature. His maxim is: From self-love I
adopt it as a principle to shorten my life when its longer duration is
likely to bring more evil than satisfaction. It is asked then simply
whether this principle founded on self-love can become a universal law
of nature. Now we see at once that a system of nature of which it
should be a law to destroy life by means of the very feeling whose
special nature it is to impel to the improvement of life would
contradict itself, and therefore could not exist as a system of
nature; hence that maxim cannot possibly exist as a universal law of
nature, and consequently \page{40} would be wholly inconsistent with
the supreme principle of all duty.

2. Another finds himself forced by necessity to borrow money. He knows
that he will not be able to repay it, but sees also that nothing will
be lent to him, unless he promises stoutly to repay it in a definite
time. He desires to make this promise, but he has still so much
conscience as to ask himself: Is it not unlawful and inconsistent with
duty to get out of a difficulty in this way? Suppose, however, that he
resolves to do so, then the maxim of his action would be expressed
thus: When I think myself in want of money, I will borrow money and
promise to repay it, although I know that I never can do so. Now this
principle of self-love or of one's own advantage may perhaps be
consistent with my whole future welfare; but the question now is, Is
it right? I change then the suggestion of self-love into a universal
law, and state the question thus: How would it be if my maxim were a
universal law? Then I see at once that it could never hold as a
universal law of nature, but would necessarily contradict itself. For
supposing it to be a universal law that everyone when he thinks
himself in a difficulty should be able to promise whatever he pleases,
with the purpose of not keeping his promise, the promise itself would
become impossible, as well as the end that one might have in view in
it, since no one would consider that anything was promised to him, but
would ridicule all such statements as vain pretences.

3. A third finds in himself a talent which with the help of some
culture might make him a useful man in many respects. But he finds
himself in comfortable circumstances, and prefers to indulge in
pleasure rather than to take pains in enlarging and improving his
happy natural capacities. He asks, however, whether his maxim of
neglect of his natural gifts, besides agreeing with his inclination to
indulgence, agrees also with what is called duty. He sees then that a
system of nature could indeed subsist with such a universal law
although men \page{41} (like the South Sea islanders) should let their
talents rest, and resolve to devote their lives merely to idleness,
amusement, and propagation of their species---in a word, to enjoyment;
but he cannot possibly \textit{will} that this should be a universal
law of nature, or be implanted in us as such by a natural instinct.
For, as a rational being, he necessarily wills that his faculties be
developed, since they serve him, and have been given him, for all
sorts of possible purposes.

4. A fourth, who is in prosperity, while he sees that others have to
contend with great wretchedness and that he could help them, thinks:
What concern is it of mine? Let everyone be as happy as Heaven
pleases, or as he can make himself; I will take nothing from him nor
even envy him, only I do not wish to contribute anything to his
welfare or to his assistance in distress! Now no doubt if such a mode
of thinking were a universal law, the human race might very well
subsist, and doubtless even better than in a state in which everyone
talks of sympathy and good-will, or even takes care occasionally to
put it into practice, but, on the other side, also cheats when he can,
betrays the rights of men, or otherwise violates them. But although it
is possible that a universal law of nature might exist in accordance
with that maxim, it is impossible to \textit{will} that such a
principle should have the universal validity of a law of nature. For a
will which resolved this would contradict itself, inasmuch as many
cases might occur in which one would have need of the love and
sympathy of others, and in which, by such a law of nature, sprung from
his own will, he would deprive himself of all hope of the aid he
desires.

These are a few of the many actual duties, or at least what we regard
as such, which obviously fall into two classes on the one principle
that we have laid down. We must be \textit{able to will} that a maxim
of our action should be a universal law. This is the canon of the
moral appreciation of the action generally. Some actions are of such a
character that their maxim cannot without contradiction be even
\textit{conceived} as a universal law of nature, far from it being
possible that we should \textit{will} that it \textit{should} be so.
In others this intrinsic impossibility is not \page{42} found, but
still it is impossible to \textit{will} that their maxim should be
raised to the universality of a law of nature, since such a will would
contradict itself. It is easily seen that the former violate strict or
rigorous (inflexible) duty; the latter only laxer (meritorious) duty.
Thus it has been completely shown by these examples how all duties
depend as regards the nature of the obligation (not the object of the
action) on the same principle.

If now we attend to ourselves on occasion of any transgression of
duty, we shall find that we in fact do not will that our maxim should
be a universal law, for that is impossible for us; on the contrary, we
will that the opposite should remain a universal law, only we assume
the liberty of making an \textit{exception} in our own favour or (just
for this time only) in favour of our inclination. Consequently if we
considered all cases from one and the same point of view, namely, that
of reason, we should find a contradiction in our own will, namely,
that a certain principle should be objectively necessary as a
universal law, and yet subjectively should not be universal, but admit
of exceptions. As, however, we at one moment regard our action from
the point of view of a will wholly conformed to reason, and then again
look at the same action from the point of view of a will affected by
inclination, there is not really any contradiction, but an antagonism
of inclination to the precept of reason, whereby the universality of
the principle is changed into a mere generality, so that the practical
principle of reason shall meet the maxim half way. Now, although this
cannot be justified in our own impartial judgment, yet it proves that
we do really recognize the validity of the categorical imperative and
(with all respect for it) only allow ourselves a few exceptions, which
we think unimportant and forced from us.

We have thus established at least this much, that if duty is a
conception which is to have any import and real legislative authority
for our actions, it can only be expressed in categorical, and not at
all in hypothetical imperatives. We have also, which is of great
importance, exhibited clearly and definitely for every practical
application the content of the \page{43} categorical imperative, which
must contain the principle of all duty if there is such a thing at
all. We have not yet, however, advanced so far as to prove \textit{\`a
priori} that there actually is such an imperative, that there is a
practical law which commands absolutely of itself, and without any
other impulse, and that the following of this law is duty.

With the view of attaining to this it is of extreme importance to
remember that we must not allow ourselves to think of deducing the
reality of this principle from the \textit{particular attributes of
human nature}. For duty is to be a practical, unconditional necessity
of action; it must therefore hold for all rational beings (to whom an
imperative can apply at all), and \textit{for this reason only} be
also a law for all human wills. On the contrary, whatever is deduced
from the particular natural characteristics of humanity, from certain
feelings and propensions, nay, even, if possible, from any particular
tendency proper to human reason, and which need not necessarily hold
for the will of every rational being; this may indeed supply us with a
maxim, but not with a law; with a subjective principle on which we may
have a propension and inclination to act, but not with an objective
principle on which we should be \textit{enjoined} to act, even though
all our propensions, inclinations, and natural dispositions were
opposed to it. In fact, the sublimity and intrinsic dignity of the
command in duty are so much the more evident, the less the subjective
impulses favour it and the more they oppose it, without being able in
the slightest degree to weaken the obligation of the law or to
diminish its validity.

\snip

\page{44}The question then is this: Is it a necessary law \textit{for
all rational beings} that they should always judge of their actions by
maxims of which they can themselves will that they should serve as
universal laws? If it is so, then it must be connected (altogether
\textit{\`a priori}) with the very conception of the will of a
rational being generally. But in order to discover this connexion we
must, however reluctantly, take a step into metaphysic, although into
a domain of it which is distinct from speculative philosophy, namely,
the metaphysic of morals. In \page{45} a practical philosophy, where
it is not the reasons of what \textit{happens} that we have to
ascertain, but the laws of what \textit{ought to happen}, even
although it never does, \textit{i.e.} objective practical laws, there
it is not necessary to inquire into the reasons why anything pleases
or displeases, how the pleasure of mere sensation differs from taste,
and whether the latter is distinct from a general satisfaction of
reason; on what the feeling of pleasure or pain rests, and how from it
desires and inclinations arise, and from these again maxims by the
co-operation of reason: for all this belongs to an empirical
psychology, which would constitute the second part of physics, if we
regard physics as the \textit{philosophy of nature}, so far as it is
based on \textit{empirical laws}. But here we are concerned with
objective practical laws, and consequently with the relation of the
will to itself so far as it determined by reason alone, in which case
whatever has reference to anything empirical is necessarily excluded;
since if \textit{reason of itself alone} determines the conduct (and
it is the possibility of this that we are now investigating), it must
necessarily do so \textit{\`a priori}.

The will is conceived as a faculty of determining oneself to action
\textit{in accordance with the conception of certain laws}. And such a
faculty can be found only in rational beings. Now that which serves
the will as the objective ground of its self-determination is the
\textit{end}, and if this is assigned by reason alone, it must hold
for all rational beings. On the other hand, that which merely contains
the ground of possibility of the action of which the effect is the
end, this is called the \textit{means}. The subjective ground of the
desire is the \textit{spring}, the objective ground of the volition is
the \textit{motive}; hence the distinction between subjective ends
which rest on springs, and objective ends which depend on motives
valid for every rational being. Practical principles are
\textit{formal} when they abstract from all subjective ends; they are
\textit{material} when they assume these, and therefore particular
springs of action. The ends which a rational being proposes to himself
at pleasure as \textit{effects} of his actions (material ends) are all
only relative, for it is only their relation to the particular desires
of the subject that gives them their worth, \page{46} which therefore
cannot furnish principles universal and necessasary for all rational
beings and for every volition, that is to say practical laws. Hence
all these relative ends can give rise only to hypothetical
imperatives.

Supposing, however, that there were something \textit{whose existence}
has \textit{in itself} an absolute worth, something which, being
\textit{an end in itself}, could be a source of definite laws, then in
this and this alone would lie the source of a possible categorical
imperative, \textit{i.e.} a practical law.

Now I say: man and generally any rational being \textit{exists} as an
end in himself, \textit{not merely as a means} to be arbitrarily used
by this or that will, but in all his actions, whether they concern
himself or other rational beings, must be always regarded at same time
as an end. All objects of the inclinations have only conditional
worth; for if the inclinations and the wants founded on them did not
exist, then their object would be without value. But the inclinations
themselves being sources of want, are so far from having an absolute
worth for which they should be desired, that on the contrary it must
be the universal wish of every rational being to be wholly free from
them. Thus the worth of any object which is \textit{to be acquired} by
our action is always conditional. Beings whose existence depends not
on our will but on nature's, have nevertheless, if they are rational
beings, only a relative value as means, and are therefore called
\textit{things}; rational beings, on the contrary, are called
\textit{persons}, because their very nature points them out as ends in
themselves, that is as something which must not be used merely as
means, and so far therefore restricts freedom of action (and is an
object of respect). These, therefore, are not merely subjective ends
whose existence has a worth \textit{for us} as an effect of our
action, but \textit{objective ends}, that is things whose existence is
an end in itself: an end moreover for which no other can be
substituted, which they should subserve \textit{merely} as means, for
otherwise nothing whatever would possess \textit{absolute worth}; but
if all worth were conditioned and therefore contingent, then there
would be no supreme practical principle of reason whatever.

If then there is a supreme practical principle or, in respect of
\page{47} the human will, a categorical imperative, it must be one
which, being drawn from the conception of that which is necessarily an
end for everyone because it is \textit{an end in itself}, constitutes
an \textit{objective} principle of will, and can therefore serve as a
universal practical law. The foundation of this principle is:
\textit{rational nature exists as an end in itself}. Man necessarily
conceives his own existence as being so: so far then this is a
\textit{subjective} principle of human actions. But every other
rational being regards its existence similarly, just on the same
rational principle that holds for me\footnote{This proposition is here
stated as a postulate. The ground of it will be found in the
concluding section.}: so that it is at the same time an objective
principle, from which as a supreme practical law all laws of the will
must be capable of being deduced. Accordingly the practical imperative
will be as follows: \textit{So act as to treat humanity, whether in
thine own person or in that of any other, in every case as an end
withal, never as means only}. We will now inquire whether this can be
practically carried out.

To abide by the previous examples:

\textit{Firstly}, under the head of necessary duty to oneself: He who
contemplates suicide should ask himself whether his action can be
consistent with the idea of humanity \textit{as an end in itself}. If
he destroys himself in order to escape from painful circumstances, he
uses a person merely as \textit{a mean} to maintain a tolerable
condition up to the end of life. But a man is not a thing, that is to
say, something which can be used merely as means, but must in all his
actions be always considered as an end in himself. I cannot,
therefore, dispose in any way of a man in my own person so as to
mutilate him, to damage or kill him. (It belongs to ethics proper to
define this principle more precisely, so as to avoid all
misunderstanding, \textit{e.g.} as to the amputation of the limbs in
order to preserve myself; as to exposing my life to danger with a view
to preserve it, \&c. This question is therefore omitted here.)

\textit{Secondly}, as regards necessary duties, or those of strict
obligation, towards others; he who is thinking of making a lying
\page{48} promise to others will see at once that he would be using
another man \textit{merely as a mean}, without the latter containing
at the same time the end in himself. For he whom I propose by such a
promise to use for my own purposes cannot possibly assent to my mode
of acting towards him, and therefore cannot himself contain the end of
this action. This violation of the principle of humanity in other men
is more obvious if we take in examples of attacks on the freedom and
property of others. For then it is clear that he who transgresses the
rights of men intends to use the person of others merely as means,
without considering that as rational beings they ought always to be
esteemed also as ends, that is, as beings who must be capable of
containing in themselves the end of the very same action.\footnote{Let
it not be thought that the common: \textit{quod tibi non vis fieri,
\&c.}, could serve here as the rule or principle. For it is only a
deduction from the former, though with several limitations; it cannot
be a universal law, for it does not contain the principle of duties to
oneself, nor of the duties of benevolence to others (for many a one
would gladly consent that others should not benefit him, provided only
that he might be excused from showing benevolence to them), nor
finally that of duties of strict obligation to one another, for on
this principle the criminal might argue against the judge who punishes
him, and so on.}

\textit{Thirdly}, as regards contingent (meritorious) duties to
oneself; it is not enough that the action does not violate humanity in
our own person as an end in itself, it must also \textit{harmonize
with} it. Now there are in humanity capacities of greater perfection
which belong to the end that nature has in view in regard to humanity
in ourselves as the subject: to neglect these might perhaps be
consistent with the \textit{maintenance} of humanity as an end in
itself, but not with the \textit{advancement} of this end.

\textit{Fourthly}, as regards meritorious duties towards others: the
natural end which all men have is their own happiness. Now humanity
might indeed subsist, although no one should contribute anything to
the happiness of others, provided he did not intentionally withdraw
anything from it; but after all, this would only harmonize negatively,
not positively, with \textit{humanity \page{49} as an end in itself},
if everyone does not also endeavour, as far as in him lies, to forward
the ends of others. For the ends of any subject which is an end in
himself, ought as far as possible to be \textit{my} ends also, if that
conception is to have its \textit{full} effect with me.

This principle, that humanity and generally every rational nature is
\textit{an end in itself} (which is the supreme limiting condition of
every man's freedom of action), is not borrowed from experience,
\textit{firstly}, because it is universal, applying as it does to all
rational beings whatever, and experience is not capable of determining
anything about them; \textit{secondly}, because it does not present
humanity as an end to men (subjectively), that is as an object which
men do of themselves actually adopt as an end; but as an objective
end, which must as a law constitute the supreme limiting condition of
all our subjective ends, let them be what we will; it must therefore
spring from pure reason. In fact the objective principle of all
practical legislation lies (according to the first principle) in
\textit{the rule} and its form of universality which makes it capable
of being a law (say, \textit{e.g.}, a law of nature); but the
\textit{subjective} principle is in the \textit{end}; now by the
second principle the subject of all ends is each rational being
inasmuch as it is an end in itself. Hence follows the third practical
principle of the will, which is the ultimate condition of its harmony
with the universal practical reason, viz.: the idea of \textit{the
will of every rational being as a universally legislative will}.

On this principle all maxims are rejected which are inconsistent with
the will being itself universal legislator. Thus the will is not
subject simply to the law, but so subject that it must be regarded
\textit{as itself giving the law}, and on this ground only, subject to
the law (of which it can regard itself as the author).

In the previous imperatives, namely, that based on the conception of
the conformity of actions to general laws, as in a \textit{physical
system of nature}, and that based on the universal
\textit{prerogative} of rational beings as \textit{ends} in
themselves---these imperatives just because they were conceived as
categorical, excluded \page{50} from any share in their authority all
admixture of any interest as a spring of action; they were, however,
only \textit{assumed} to be categorical, because such an assumption
was necessary to explain the conception of duty. But we could not
prove independently that there are practical propositions which
command categorically, nor can it be proved in this section; one
thing, however, could be done, namely, to indicate in the imperative
itself by some determinate expression, that in the case of volition
from duty all interest is renounced, which is the specific criterion
of categorical as distinguished from hypothetical imperatives. This is
done in the present (third) formula of the principle, namely, in the
idea of the will of every rational being as a \textit{universally
legislating will}.

For although a will \textit{which is subject to laws} may be attached
to this law by means of an interest, yet a will which is itself a
supreme lawgiver so far as it is such cannot possibly depend on any
interest, since a will so dependent would itself still need another
law restricting the interest of its self-love by the condition that it
should be valid as universal law.

Thus the \textit{principle} that every human will is \textit{a will
which in all its maxims gives universal laws}\footnote{I may be
excused from adducing examples to elucidate this principle, as those
which have already been used to elucidate the categorical imperative
and its formula would all serve for the like purpose here.}, provided
it be otherwise justified, would be very \textit{well adapted} to be
the categorical imperative, in this respect, namely, that just because
of the idea of universal legislation it is \textit{not based on any
interest}, and therefore it alone among all possible imperatives can
be \textit{unconditional}. Or still better, converting the
proposition, if there is a categorical imperative (\textit{i.e.}, a
law for the will of every rational being), it can only command that
everything be done from maxims of one's will regarded as a will which
could at the same time will that it should itself give universal laws,
for in that case only the practical principle and the imperative which
it obeys are unconditional, since they cannot be based on any
interest.

\snip

\page{51}The conception of every rational being as one which must
consider itself as giving in all the maxims of its will universal
laws, so as to judge itself and its actions from this point of
view---this conception leads to another which depends on it and is
very fruitful, namely, that of a \textit{kingdom of ends}.

By a \textit{kingdom} I understand the union of different rational
beings in a system by common laws. Now since it is by laws that ends
are determined as regards their universal validity, hence, if we
abstract from the personal differences of rational beings, and
likewise from all the content of their private ends, we shall be able
to conceive all ends combined in a systematic whole (including both
rational beings as ends in themselves, and also the special ends which
each may propose to himself), that is to say, we can conceive a
kingdom of ends, which on the preceding principles is possible.

\page{52}For all rational beings come under the \textit{law} that each
of them must treat itself and all others \textit{never merely as
means}, but in every case \textit{at the same time as ends in
themselves}. Hence results a systematic union of rational beings by
common objective laws, \textit{i.e.}, a kingdom which may be called a
kingdom of ends, since what these laws have in view is just the
relation of these beings to one another as ends and means. It is
certainly only an ideal.

\snip

\page{53}In the kingdom of ends everything has either Value or
Dignity. Whatever has a value can be replaced by something else which
is \textit{equivalent}; whatever, on the other hand, is above all
value, and therefore admits of no equivalent, has a dignity.

Whatever has reference to the general inclinations and wants of
mankind has a \textit{market value}; whatever, without presupposing a
want, corresponds to a certain taste, that is to a satisfaction in the
mere purposeless play of our faculties, has a \textit{fancy value};
but that which constitutes the condition under which alone anything
can be an end in itself, this has not merely a relative worth,
\textit{i.e.} value, but an intrinsic worth, that is \textit{dignity}.

Now morality is the condition under which alone a rational being can
be an end in himself, since by this alone it is possible that he
should be a legislating member in the kingdom of ends. Thus morality,
and humanity as capable of it, is that which alone has dignity. Skill
and diligence in labour have a market value; wit, lively imagination,
and humour, have fancy value; on the other hand, fidelity to promises,
benevolence from principle (not from instinct), have an intrinsic
worth. Neither nature nor art contains anything which in default of
these it could put in their place, for their worth consists not in the
effects which spring from them, not in the use and advantage which
they secure, but in the disposition of mind, that is, the maxims of
the will which are ready to manifest themselves in such actions, even
though they should not have the desired effect. These actions also
need no recommendation from any subjective taste or sentiment, that
they may be looked on with immediate favour and satisfaction: they
need no immediate propension or feeling for them; they exhibit the
will that performs them as an object of an immediate respect,
\page{54} and nothing but reason is required to \textit{impose} them
on the will; not to \textit{flatter} it into them, which, in the case
of duties, would be a contradiction. This estimation therefore shows
that the worth of such a disposition is dignity, and places it
infinitely above all value, with which it cannot for a moment be
brought into comparison or competition without as it were violating
its sanctity.

What then is it which justifies virtue or the morally good
disposition, in making such lofty claims? It is nothing less than the
privilege it secures to the rational being of participating in the
giving of universal laws, by which it qualifies him to be a member of
a possible kingdom of ends, a privilege to which he was already
destined by his own nature as being an end in himself, and on that
account legislating in the kingdom of ends; free as regards all laws
of physical nature, and obeying those only which he himself gives, and
by which his maxims can belong to a system of universal law, to which
at the same time he submits himself. For nothing has any worth except
what the law assigns it. Now the legislation itself which assigns the
worth of everything must for that very reason possess dignity, that is
an unconditional incomparable worth; and the word \textit{respect}
alone supplies a becoming expression for the esteem which a rational
being must have for it. \textit{Autonomy} then is the basis of the
dignity of human and of every rational nature.

The three modes of presenting the principle of morality that have been
adduced are at bottom only so many formul\ae{} of the very same law,
and each of itself involves the other two. There is, however, a
difference in them, but it is rather subjectively than objectively
practical, intended namely to bring an idea of the reason nearer to
intuition (by means of a certain analogy), and thereby nearer to
feeling. All maxims, in fact, have---

1. A \textit{form}, consisting in universality; and in this view the
formula of the moral imperative is expressed thus, that the maxims
must be so chosen as if they were to serve as universal laws of
nature.

\page{55}2. A \textit{matter}, namely, an end, and here the formula
says that the rational being, as it is an end by its own nature and
therefore an end in itself, must in every maxim serve as the condition
limiting all merely relative and arbitrary ends.

3. A \textit{complete characterisation} of all maxims by means of that
formula, namely, that all maxims ought by their own legislation to
harmonize with a possible kingdom of ends as with a kingdom of
nature\footnote{Teleology considers nature as a kingdom of ends;
Ethics regards a possible kingdom of ends as a kingdom of nature. In
the first case, the kingdom of ends is a theoretical idea, adopted to
explain what actually is. In the latter it is a practical idea,
adopted to bring about that which is not yet, but which can be
realized by our conduct, namely, if it conforms to this idea.}. There
is a progress here in the order of the categories of \textit{unity} of
the form of the will (its universality), \textit{plurality} of the
matter (the objects, \textit{i.e.} the ends), and \textit{totality} of
the system of these. In forming our moral \textit{judgment} of actions
it is better to proceed always on the strict method, and start from
the general formula of the categorical imperative: \textit{Act
according to a maxim which can at the same time make itself a
universal law}. If, however, we wish to gain an \textit{entrance} for
the moral law, it is very useful to bring one and the same action
under the three specified conceptions, and thereby as far as possible
to bring it nearer to intuition.

We can now end where we started at the beginning, namely, with the
conception of a will unconditionally good. \textit{That will} is
\textit{absolutely good} which cannot be evil---in other words, whose
maxim, if made a universal law, could never contradict itself. This
principle, then, is its supreme law: Act always on such a maxim as
thou canst at the same time will to be a universal law; this is the
sole condition under which a will can never contradict itself; and
such an imperative is categorical. Since the validity of the will as a
universal law for possible actions is analogous to the universal
connexion of the existence of things by general laws, which is the
formal notion of nature in general, \page{56} the categorical
imperative can also be expressed thus: \textit{Act on maxims which can
at the same time have for their object themselves as universal laws of
nature}. Such then is the formula of an absolutely good will.

Rational nature is distinguished from the rest of nature by this, that
it sets before itself an end. This end would be the matter of every
good will. But since in the idea of a will that is absolutely good
without being limited by any condition (of attaining this or that end)
we must abstract wholly from every end \textit{to be effected} (since
this would make every will only relatively good), it follows that in
this case the end must be conceived, not as an end to be effected, but
as an \textit{independently} existing end. Consequently it is
conceived only negatively, \textit{i.e.}, as that which we must never
act against, and which, therefore, must never be regarded merely as
means, but must in every volition be esteemed as an end likewise. Now
this end can be nothing but the subject of all possible ends, since
this is also the subject of a possible absolutely good will; for such
a will cannot without contradiction be postponed to any other object.
This principle: So act in regard to every rational being (thyself and
others), that he may always have place in thy maxim as an end in
himself, is accordingly essentially identical with this other: Act
upon a maxim which, at the same time, involves its own universal
validity for every rational being. For that in using means for every
end I should limit my maxim by the condition of its holding good as a
law for every subject, this comes to the same thing as that the
fundamental principle of all maxims of action must be that the subject
of all ends, \textit{i.e.}, the rational being himself, be never
employed merely as means, but as the supreme condition restricting the
use of all means, that is in every case as an end likewise.

It follows incontestably that, to whatever laws any rational being may
be subject, he being an end in himself must be able to regard himself
as also legislating universally in respect of these same laws, since
it is just this fitness of his maxims for universal legislation that
distinguishes him as an end in himself; \page{57} also it follows that
this implies his dignity (prerogative) above all mere physical beings,
that he must always take his maxims from the point of view which
regards himself, and likewise every other rational being, as lawgiving
beings (on which account they are called persons). In this way a world
of rational beings (\textit{mundus intelligibilis}) is possible as a
kingdom of ends, and this by virtue of the legislation proper to all
persons as members. Therefore every rational being must so act as if
he were by his maxims in every case a legislating member in the
universal kingdom of ends. The formal principle of these maxims is: So
act as if thy maxim were to serve likewise as the universal law (of
all rational beings). A kingdom of ends is thus only possible on the
analogy of a kingdom of nature, the former, however, only by maxims,
that is self-imposed rules, the latter only by the laws of efficient
causes acting under necessitation from without. Nevertheless, although
the system of nature is looked upon as a machine, yet so far as it has
reference to rational beings as its ends, it is given on this account
the name of a kingdom of nature. Now such a kingdom of ends would be
actually realized by means of maxims conforming to the canon which the
categorical imperative prescribes to all rational beings, \textit{if
they were universally followed}. But although a rational being, even
if he punctually follows this maxim himself, cannot reckon upon all
others being therefore true to the same, nor expect that the kingdom
of nature and its orderly arrangements shall be in harmony with him as
a fitting member, so as to form a kingdom of ends to which he himself
contributes, that is to say, that it shall favour his expectation of
happiness, still that law: Act according to the maxims of a member of
a merely possible kingdom of ends legislating in it universally,
remains in its full force, inasmuch as it commands categorically. And
it is just in this that the paradox lies; that the mere dignity of man
as a rational creature, without any other end or advantage to be
attained thereby, in other words, respect for a mere idea, should yet
serve as an inflexible precept of the will, and that it is precisely
in this independence of the maxim on all such springs of \page{58}
action that its sublimity consists; and it is this that makes every
rational subject worthy to be a legislative member in the kingdom of
ends: for otherwise he would have to be conceived only as subject to
the physical law of his wants. And although we should suppose the
kingdom of nature and the kingdom of ends to be united under one
sovereign, so that the latter kingdom thereby ceased to be a mere idea
and acquired true reality, then it would no doubt gain the accession
of a strong spring, but by no means any increase of its intrinsic
worth. For this sole absolute lawgiver must, notwithstanding this, be
always conceived as estimating the worth of rational beings only by
their disinterested behaviour, as prescribed to themselves from that
idea [the dignity of man] alone. The essence of things is not altered
by their external relations, and that which, abstracting from these,
alone constitutes the absolute worth of man, is also that by which he
must be judged, whoever the judge may be, and even by the Supreme
Being. \textit{Morality}, then, is the relation of actions to the
autonomy of the will, that is, to the potential universal legislation
by its maxims. An action that is consistent with the autonomy of the
will is \textit{permitted}; one that does not agree therewith is
\textit{forbidden}. A will whose maxims necessarily coincide with the
laws of autonomy is a \textit{holy} will, good absolutely. The
dependence of a will not absolutely good on the principle of autonomy
(moral necessitation) is obligation. This, then, cannot be applied to
a holy being. The objective necessity of actions from obligation is
called \textit{duty}.

From what has just been said, it is easy to see how it happens that
although the conception of duty implies subjection to the law, we yet
ascribe a certain \textit{dignity} and sublimity to the person who
fulfils all his duties. There is not, indeed, any sublimity in him, so
far as he is \textit{subject} to the moral law; but inasmuch as in
regard to that very law he is likewise a \textit{legislator}, and on
that account alone subject to it, he has sublimity. We have also shown
above that neither fear nor inclination, but simply respect for the
law, is the spring which can give actions a moral worth. Our own will,
so far as we \page{59} suppose it to act only under the condition that
its maxims are potentially universal laws, this ideal will which is
possible to us is the proper object of respect; and the dignity of
humanity consists just in this capacity of being universally
legislative, though with the condition that it is itself subject to
this same legislation.

