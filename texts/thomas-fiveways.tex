
\author{Thomas Aquinas}
\authdate{ca. 1225--1274}
\textdate{ca. 1265--1274, excerpt}
\chapter[Thomas Aquinas -- Five Ways]{Five Ways\\\smaller Summa
Theologica\\\smaller Part I, Question 2, Article 3}

\nfootnote{\fullcite*[24--27]{thomas1920}}

\page{24}\textit{I answer that}, The existence of God can be proved in
five ways.

The first and more manifest way is the argument from motion. It is
certain, and evident to our senses, that in the world some things are
in motion. Now whatever is in motion is put in motion by another, for
nothing can be in motion except it is in potentiality to that towards
which it is in motion; whereas a thing moves inasmuch as it is in act.
For motion is nothing else than the reduction of something from
potentiality to actuality. But nothing can be reduced from
potentiality to actuality, except by something in a state of
actuality. Thus that which is actually hot, as fire, makes wood, which
is potentially hot, to be actually hot, and thereby moves and changes
it. Now it is not possible that the same thing should be at once in
actuality and potentiality in the same \page {25} respect, but only in
different respects. For what is actually hot cannot simultaneously be
potentially hot; but it is simultaneously potentially cold. It is
therefore impossible that in the same respect and in the same way a
thing should be both mover and moved, \textit{i.e.}, that it should
move itself. Therefore, whatever is in motion must be put in motion by
another. If that by which it is put in motion be itself put in motion,
then this also must needs be put in motion by another, and that by
another again. But this cannot go on to infinity, because then there
would be no first mover, and, consequently, no other mover; seeing
that subsequent movers move only inasmuch as they are put in motion by
the first mover; as the staff moves only because it is put in motion
by the hand. Therefore it is necessary to arrive at a first mover, put
in motion by no other; and this everyone understands to be God.

The second way is from the nature of the efficient cause. In the world
of sense we find there is an order of efficient causes. There is no
case known (neither is it, indeed, possible) in which a thing is found
to be the efficient cause of itself; for so it would be prior to
itself, which is impossible. Now in efficient causes it is not
possible to go on to infinity, because in all efficient causes
following in order, the first is the cause of the intermediate cause,
and the intermediate is the cause of the ultimate cause, whether the
intermediate cause be several, or only one. Now to take away the cause
is to take away the effect. Therefore, if there be no first cause
among efficient causes, there will be no ultimate, nor any
intermediate cause. But if in efficient causes it is possible to go on
to infinity, there will be no first efficient cause, neither will
there be an ultimate effect, nor any intermediate efficient causes;
all of which is plainly false. Therefore it is necessary to admit a
first efficient cause, to which everyone gives the name of God.

The third way is taken from possibility and necessity, and runs thus.
We find in nature things that are possible to be and not to be, since
they are found to be generated, and \page{26} to corrupt, and
consequently, they are possible to be and not to be. But it is
impossible for these always to exist, for that which is possible not
to be at some time is not. Therefore, if everything is possible not
to be, then at one time there could have been nothing in existence.
Now if this were true, even now there would be nothing in existence,
because that which does not exist only begins to exist by something
already existing. Therefore, if at one time nothing was in
existence, it would have been impossible for anything to have begun to
exist; and thus even now nothing would be in existence---which is
absurd. Therefore, not all beings are merely possible, but there must
exist something the existence of which is necessary. But every
necessary thing either has its necessity caused by another, or not.
Now it is impossible to go on to infinity in necessary things which
have their necessity caused by another, as has been already proved in
regard to efficient causes. Therefore we cannot but postulate the
existence of some being having of itself its own necessity, and not
receiving it from another, but rather causing in others their
necessity. This all men speak of as God.

The fourth way is taken from the gradation to be found in things.
Among beings there are some more and some less good, true, noble, and
the like. But `more' and `less' are predicated of different things,
according as they resemble in their different ways something which is
the maximum, as a thing is said to be hotter according as it more
nearly resembles that which is hottest; so that there is something
which is truest, something best, something noblest, and, consequently,
something which is uttermost being; for those things that are greatest
in truth are greatest in being, as it is written in \textit{Metaph}.
ii. Now the maximum in any genus is the cause of all in that genus; as
fire, which is the maximum heat, is the cause of all hot things.
Therefore there must also be something which is to all beings the
cause of their being, goodness, and every other perfection; and this
we call God.

The fifth way is taken from the governance of the world. \page{27} We
see that things which lack intelligence, such as natural bodies, act
for an end, and this is evident from their acting always, or nearly
always, in the same way, so as to obtain the best result. Hence it
is plain that not fortuitously, but designedly, do they achieve their
end. Now whatever lacks intelligence cannot move towards an end,
unless it be directed by some being endowed with knowledge and
intelligence; as the arrow is shot to its mark by the archer.
Therefore some intelligent being exists by whom all natural things are
directed to their end; and this being we call God.

