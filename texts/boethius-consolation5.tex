
%\author{Anicius Manlius Severinus Boethius}
\author{Boethius}
%\authdate{ca. 480--524/5}
\authdate{ca. 480 -- ca. 526}
%\textdate{524/5}
\textdate{ca. 524--526}
\addon{Book 5}
\chapter[The Consolation of Philosophy, bk. 5]{The Consolation of
Philosophy}
\source{boethius1902}

\page{140}\section{Prose I}

Here she made an end and was for turning the course of her speaking to
the handling and explaining of other subjects. Then said I: `Your
encouragement is right and most worthy in truth of your name and
weight. But I am learning by experience what you just now said of
Providence; that the question is bound up in others. I would ask you
whether you think that Chance exists at all, and what you think it
is?'

Then she answered: `I am eager to fulfil my promised debt, and to shew
you the path by which you may seek your home. But these things, though
all-expedient for knowledge, are none the less rather apart from our
path, and we must be careful lest you become wearied by our turnings
aside, and so be not strong enough to complete the straight journey.'

`Have no fear at all thereof,' said I. `It will be restful to know
these things in which I have so great a pleasure; and when every view
of your reasoning has stood firm with unshaken credit, so let there be
no doubt of what shall follow.'

`I will do your pleasure,' she made answer, and thus she began to
speak:

\page{141}`If chance is defined as an outcome of random influence,
produced by no sequence of causes, I am sure that there is no such
thing as chance, and I consider that it is but an empty word, beyond
shewing the meaning of the matter which we have in hand. For what
place can be left for anything happening at random, so long as God
controls everything in order? It is a true saying that nothing can
come out of nothing. None of the old philosophers has denied that,
though they did not apply it to the effective principle, but to the
matter operated upon---that is to say, to nature; and this was the
foundation upon which they built all their reasoning. If anything
arises from no causes, it will appear to have risen out of nothing.
But if this is impossible, then chance also cannot be anything of
that sort, which is stated in the definition which we mentioned.'

`Then is there nothing which can be justly called chance, nor anything
``by chance''?' I asked. `Or is there anything which common people
know not, but which those words do suit?'

`My philosopher, Aristotle, defined it in his
\textit{Physics}\footnote{Aristotle, \textit{Physics}, ii. 3.} shortly
and well-nigh truly.'

`How?' I asked.

`Whenever anything is done with one intention, but something else,
other than was intended, results from certain causes, that is called
chance: as, for instance, if a man digs \page{142} the ground for the
sake of cultivating it, and finds a heap of buried gold. Such a thing
is believed to have happened by chance, but it does not come from
nothing, for it has its own causes, whose unforeseen and unexpected
coincidence seem to have brought about a chance. For if the cultivator
did not dig the ground, if the owner had not buried his money, the
gold would not have been found. These are the causes of the chance
piece of good fortune, which comes about from the causes which meet
it, and move along with it, not from the intention of the actor. For
neither the burier nor the tiller intended that the gold should be
found; but, as I said, it was a coincidence, and it happened that the
one dug up what the other buried. We may therefore define chance as an
unexpected result from the coincidence of certain causes in matters
where there was another purpose. The order of the universe, advancing
with its inevitable sequences, brings about this coincidence of
causes. This order itself emanates from its source, which is
Providence, and disposes all things in their proper time and place.

\section{Meter I}

`In the land where the Parthian, as he turns in flight, shoots his
arrows into the pursuer's breast, from the rocks of the crag of
Achæmenia, the Tigris and Euphrates flow from out one source, but
quickly with divided streams are separate. If they should come
together and again be joined in a single course, all, that \page{143}
the two streams bear along, would flow in one together. Boats would
meet boats, and trees meet trees torn up by the currents, and the
mingled waters would together entwine their streams by chance; but
their sloping beds restrain these chances vague, and the downward
order of the falling torrent guides their courses. Thus does chance,
which seems to rush onward without rein, bear the bit, and take its
way by rule.'

\section{Prose II}

`I have listened to you,' I said, `and agree that it is as you say.
But in this close sequence of causes, is there any freedom for our
judgment, or does this chain of fate bind the very feelings of our
minds too?'

`There is free will,' she answered. `Nor could there be any reasoning
nature without freedom of judgment. For any being that can use its
reason by nature, has a power of judgment by which it can without
further aid decide each point, and so distinguish between objects to
be desired and objects to be shunned. Each therefore seeks what it
deems desirable, and flies from what it considers should be shunned.
Wherefore all who have reason have also freedom of desiring and
refusing in themselves. But I do not lay down that this is equal in
all beings. Heavenly and divine beings have with them a judgment of
great insight, an imperturbable will, and a power which can effect
their desires. But human \page{144} spirits must be more free when
they keep themselves safe in the contemplation of the mind of God; but
less free when they sink into bodies, and less still when they are
bound by their earthly members. The last stage is mere slavery, when
the spirit is given over to vices and has fallen away from the
possession of its reason. For when the mind turns its eyes from the
light of truth on high to lower darkness, soon they are dimmed by the
clouds of ignorance, and become turbid through ruinous passions; by
yielding to these passions and consenting to them, men increase the
slavery which they have brought upon themselves, and their true
liberty is lost in captivity. But God, looking upon all out of the
infinite, perceives the views of Providence, and disposes each as its
destiny has already fated for it according to its merits: ``He looketh
over all and heareth all.''\footnote{A phrase from Homer
(\textit{Illiad}, iii. 277, and \textit{Odyssey}, xi. 109), where it
is said of the sun.}

\section{Meter II}

`Homer with his honeyed lips sang of the bright sun's clear light; yet
the sun cannot burst with his feeble rays the bowels of the earth or
the depths of the sea. Not so with the Creator of this great sphere.
No masses of earth can block His vision as He looks over all. Night's
cloudy darkness cannot resist Him. With one glance of His intelligence
He sees all that has been, that is, and that is to come. \page{145} He
alone can see all things, so truly He may be called the
Sun.'\footnote{This sentence, besides referring to the application of
Homer's words used above, contains also a play on words in the Latin,
which can only be clumsily reproduced in English by some such words as
`The sole power which can see all is justly to be called the
solar.'}

\section{Prose III}

Then said I, `Again am I plunged in yet more doubt and difficulty.'

`What are they,' she asked, `though I have already my idea of what
your trouble consists?'

`There seems to me,' I said, `to be such incompatibility between the
existence of God's universal foreknowledge and that of any freedom of
judgment. For if God foresees all things and cannot in anything be
mistaken, that, which His Providence sees will happen, must result.
Wherefore if it knows beforehand not only men's deeds but even their
designs and wishes, there will be no freedom of judgment. For there
can neither be any deed done, nor wish formed, except such as the
infallible Providence of God has foreseen. For if matters could ever
so be turned that they resulted otherwise than was foreseen of
Providence, this foreknowledge would cease to be sure. But, rather
than knowledge, it is opinion which is uncertain; and that, I deem, is
not applicable to God. And, further, I cannot approve of an argument
by which some men think that they can cut this knot; for they say that
a result does not come \page{146} to pass for the reason that
Providence has foreseen it, but the opposite rather, namely, that
because it is about to come to pass, therefore it cannot be hidden
from God's Providence. In that way it seems to me that the argument
must resolve itself into an argument on the other side. For in that
case it is not necessary that that should happen which is foreseen,
but that that which is about to happen should be foreseen; as though,
indeed, our doubt was whether God's foreknowledge is the certain
cause of future events, or the certainty of future events is the cause
of Providence. But let our aim be to prove that, whatever be the shape
which this series of causes takes, the fulfilment of God's
foreknowledge is necessary, even if this knowledge may not seem to
induce the necessity for the occurrence of future events. For
instance, if a man sits down, it must be that the opinion, which
conjectures that he is sitting, is true; but conversely, if the
opinion concerning the man is true because he is sitting, he must be
sitting down. There is therefore necessity in both cases: the man must
be sitting, and the opinion must be true. But he does not sit
because the opinion is true, but rather the opinion is true because
his sitting down has preceded it. Thus, though the cause of the truth
of the opinion proceeds from the other fact, yet there is a common
necessity on both parts. In like manner we must reason of Providence
and future events. For even though they are foreseen because they are
about \page{146} to happen, yet they do not happen because they are
foreseen. None the less it is necessary that either what is about to
happen should be foreseen of God, or that what has been foreseen
should happen; and this alone is enough to destroy all free will.

`Yet how absurd it is that we should say that the result of temporal
affairs is the cause of eternal foreknowledge! And to think that God
foresees future events because they are about to happen, is nothing
else than to hold events of past time to be the cause of that highest
Providence. Besides, just as, when I know a present fact, that fact
must be so; so also when I know of something that will happen, that
must come to pass. Thus it follows that the fulfilment of a foreknown
event must be inevitable.

`Lastly, if any one believes that any matter is otherwise than the
fact is, he not only has not knowledge, but his opinion is false also,
and that is very far from the truth of knowledge. Wherefore, if any
future event is such that its fulfilment is not sure or necessary, how
can it possibly be known beforehand that it will occur? For just as
absolute knowledge has no taint of falsity, so also that which is
conceived by knowledge cannot be otherwise than as it is conceived.
That is the reason why knowledge cannot lie, because each matter must
be just as knowledge knows that it is. What then? How can God know
beforehand these uncertain future events? For if He thinks inevitable
the \page{148} fulfilment of such things as may possibly not result,
He is wrong; and that we may not believe, nor even utter, rightly. But
if He perceives that they will result as they are in such a manner
that He only knows that they may or may not occur, equally, how is
this foreknowledge, this which knows nothing for sure, nothing
absolutely? How is such a foreknowledge different from the absurd
prophecy which Horace puts in the mouth of Tiresias: ``Whatever I
shall say, will either come to pass, or it will
not?''\footnote{Horace, \textit{Satires}, \textsc{ii}. \textsc{v}.
59.} How, too, would God's Providence be better than man's opinion,
if, as men do, He only sees to be uncertain such things as have an
uncertain result? But if there can be no uncertainty with God, the
most sure source of all things, then the fulfilment of all that He has
surely foreknown, is certain. Thus we are led to see that there is no
freedom for the intentions or actions of men; for the mind of God,
foreseeing all things without error or deception, binds all together
and controls their results. And when we have once allowed this, it is
plain how complete is the fall of all human actions in consequence. In
vain are rewards or punishments set before good or bad, for there is
no free or voluntary action of the mind to deserve them; and what we
just now determined was most fair, will prove to be most unfair of
all, namely to punish the dishonest or reward the honest, since their
own will does not put them in the way of \page{149} honesty or
dishonesty, but the unfailing necessity of development constrains
them. Wherefore neither virtues nor vices are anything, but there is
rather an indiscriminate confusion of all deserts. And nothing could
be more vicious than this; since the whole order of all comes from
Providence, and nothing is left to human intention, it follows that
our crimes, as well as our good deeds, must all be held due to the
author of all good. Hence it is unreasonable to hope for or pray
against aught. For what could any man hope for or pray against, if an
undeviating chain links together all that we can desire? Thus will
the only understanding between God and man, the right of prayer, be
taken away. We suppose that at the price of our deservedly humbling
ourselves before Him we may win a right to the inestimable reward of
His divine grace: this is the only manner in which men can seem to
deal with God, so to speak, and by virtue of prayer to join ourselves
to that inaccessible light, before it is granted to us; but if we
allow the inevitability of the future, and believe that we have no
power, what means shall we have to join ourselves to the Lord of all,
or how can we cling to Him? Wherefore, as you sang but a little while
ago,\footnote{\textit{Supra}, Book \textsc{iv}. Met. vi. p. 135.}
the human race must be cut off from its source and ever fall away.

\section{Meter III}

`What cause of discord is it breaks the \page{150} bonds of agreement
here? What heavenly power has set such strife between two truths?
Thus, though apart each brings no doubt, yet can they not be linked
together. Comes there no discord between these truths? Stand they for
ever sure by one another? Yes, 'tis the mind, o'erwhelmed by the
body's blindness, which cannot see by the light of that dimmed
brightness the finest threads that bind the truth. But wherefore burns
the spirit with so strong desire to learn the hidden signs of truth?
Knows it the very object of its careful search? Then why seeks it to
learn anew what it already knows? If it knows it not, why searches it
in blindness? For who would desire aught unwitting? Or who could seek
after that which is unknown? How should he find it, or recognise its
form when found, if he knows it not? And when the mind of man
perceived the mind of God, did it then know the whole and parts alike?
Now is the mind buried in the cloudy darkness of the body, yet has not
altogether forgotten its own self, and keeps the whole though it has
lost the parts. Whosoever, therefore, seeks the truth, is not wholly
in ignorance, nor yet has knowledge wholly; for he knows not all, yet
is not ignorant of all. He takes thought for the whole which he keeps
in memory, handling again what he saw on high, so that he may add to
that which he has kept, that which he has forgotten.'

\page{151}\section{Prose IV}

Then said she, `This is the old plaint concerning Providence which was
so strongly urged Philosophy by Cicero when treating of
Divination,\footnote{Cicero, \textit{De Divinatione}, \textsc{ii}.}
and you yourself have often and at length questioned the same subject.
But so far, none of you have explained it with enough diligence or
certainty. The cause of this obscurity is that the working of human
reason cannot approach the directness of divine foreknowledge. If this
could be understood at all, there would be no doubt left. And this
especially will I try to make plain, if I can first explain your
difficulties.

`Tell me why you think abortive the reasoning of those who solve the
question thus; they argue that foreknowledge cannot be held to be a
cause for the necessity of future results, and therefore free will is
not in any way shackled by foreknowledge.\footnote{Refering to
Boethius's words in Prose iii of this book, p. 145.} Whence do you
draw your proof of the necessity of future results if not from the
fact that such things as are known beforehand cannot but come to pass?
If, then (as you yourself admitted just now), foreknowledge brings no
necessity to bear upon future events, how is it that the voluntary
results of such events are bound to find a fixed end? Now for the sake
of the argument, that you may turn your attention to what follows, let
us state that there is no foreknowledge at all. Then are the events
which are decided by free will, bound by any necessity, so far as this
goes? \page{152} Of course not. Secondly, let us state that
foreknowledge exists, but brings no necessity to bear upon events;
then, I think, the same free will will be left, intact and absolute.
``But,'' you will say, ``though foreknowledge is no necessity for a
result in the future, yet it is a sign that it will necessarily come
to pass.'' Thus, therefore, even if there had been no foreknowledge,
it would be plain that future results were under necessity; for every
sign can only shew what it is that it points out; it does not bring it
to pass. Wherefore we must first prove that nothing happens but of
necessity, in order that it may be plain that foreknowledge is a sign
of this necessity. Otherwise, if there is no necessity, then
foreknowledge will not be a sign of that which does not exist. Now it
is allowed that proof rests upon firm reasoning, not upon signs or
external arguments; it must be deduced from suitable and binding
causes. How can it possibly be that things, which are foreseen as
about to happen, should not occur? That would be as though we were to
believe that events would not occur which Providence foreknows as
about to occur, and as though we did not rather think this, that
though they occur, yet they have had no necessity in their own natures
which brought them about. We can see many actions developing before
our eyes; just as chariot drivers see the development of their actions
as they control and guide their chariots, and many other things
likewise. Does any necessity compel any of those things \page{153} to
occur as they do? Of course not. All art, craft, and intention would
be in vain, if everything took place by compulsion. Therefore, if
things have no necessity for coming to pass when they do, they cannot
have any necessity to be about to come to pass before they do.
Wherefore there are things whose results are entirely free from
necessity. For I think not that there is any man who will say this,
that things, which are done in the present, were not about to be done
in the past, before they are done. Thus these foreknown events have
their free results. Just as foreknowledge of present things brings no
necessity to bear upon them as they come to pass, so also
foreknowledge of future things brings no necessity to bear upon things
which are to come.

`But you will say that there is no doubt of this too, whether there
can be any foreknowledge of things which have not results bounden by
necessity. For they do seem to lack harmony: and you think that if
they are foreseen, the necessity follows; if there is no necessity,
then they cannot be foreseen; nothing can be perceived certainly by
knowledge, unless it be certain. But if things have uncertainty of
result, but are foreseen as though certain, this is plainly the
obscurity of opinion, and not the truth of knowledge. For you believe
that to think aught other than it is, is the opposite of true
knowledge. The cause of this error is that every man believes that all
the subjects, that he knows, are known by their own force or
\page{154} nature alone, which are known; but it is quite the
opposite. For every subject, that is known, is comprehended not
according to its own force, but rather according to the nature of
those who know it. Let me make this plain to you by a brief example:
the roundness of a body may be known in one way by sight, in another
way by touch. Sight can take in the whole body at once from a distance
by judging its radii, while touch clings, as it were, to the outside
of the sphere, and from close at hand perceives through the material
parts the roundness of the body as it passes over the actual
circumference. A man himself is differently comprehended by the
senses, by imagination, by reason, and by intelligence. For the senses
distinguish the form as set in the matter operated upon by the form;
imagination distinguishes the appearance alone without the matter.
Reason goes even further than imagination; by a general and universal
contemplation it investigates the actual kind which is represented in
individual specimens. Higher still is the view of the intelligence,
which reaches above the sphere of the universal, and with the
unsullied eye of the mind gazes upon that very form of the kind in its
absolute simplicity. Herein the chief point for our consideration is
this: the higher power of understanding includes the lower, but the
lower never rises to the higher. For the senses are capable of
understanding naught but the matter; imagination cannot look upon
universal or natural kinds; reason cannot comprehend \page{155} the
absolute form; whereas the intelligence seems to look down from above
and comprehend the form, and distinguishes all that lie below, but in
such a way that it grasps the very form which could not be known to
any other than itself. For it perceives and knows the general kind, as
does reason; the appearance, as does the imagination; and the matter,
as do the senses, but with one grasp of the mind it looks upon all
with a clear conception of the whole. And reason too, as it views
general kinds, does not make use of the imagination nor the senses,
but yet does perceive the objects both of the imagination and of the
senses. It is reason which thus defines a general kind according to
its conception: Man, for instance, is an animal, biped and reasoning.
This is a general notion of a natural kind, but no man denies that the
subject can be approached by the imagination and by the senses, just
because reason investigates it by a reasonable conception and not by
the imagination or senses. Likewise, though imagination takes its
beginning of seeing and forming appearances from the senses, yet
without their aid it surveys each subject by an imaginative faculty of
distinguishing, not by the distinguishing faculty of the senses.

`Do you see then, how in knowledge of all things, the subject uses its
own standard of capability, and not those of the objects known? And
this is but reasonable, for every judgment formed is an act of the
person who judges, and therefore each man must of necessity perform
\page{156} his own action from his own capability and not the
capability of any other.

\section{Meter IV}

`In days of old the Porch at Athens\footnote{Zeno, of Citium (342--270
\textsc{b.c.}), the founder of the Stoic school, taught in the Stoa
Poekile, whence the name of the school. The following lines refer to
their doctrine of presentations and impressions.} gave us men, seeing
dimly as in old age, who could believe that the feelings of the senses
and the imagination were but impressions on the mind from bodies
without them, just as the old custom was to impress with swift-running
pens letters upon the surface of a waxen tablet which bore no marks
before. But if the mind with its own force can bring forth naught by
its own exertions; if it does but lie passive and subject to the marks
of other bodies; if it reflects, as does, forsooth, a mirror, the vain
reflections of other things; whence thrives there in the soul an
all-seeing power of knowledge? What is the force that sees the single
parts, or which distinguishes the facts it knows? What is the force
that gathers up the parts it has distinguished, that takes its course
in order due, now rises to mingle with the things on high, and now
sinks down among the things below, and then to itself brings back
itself, and, so examining, refutes the false with truth? This is a
cause of greater power, of more effective force by far than that which
only receives the impressions of material bodies. Yet does the passive
reception come first, rousing and stirring \page{157} all the strength
of the mind in the living body. When the eyes are smitten with a
light, or the ears are struck with a voice's sound, then is the
spirit's energy aroused, and, thus moved, calls upon like forms, such
as it holds within itself, fits them to signs without and mingles the
forms of its imagination with those which it has stored within.

\section{Prose V}

`With regard to feeling the effects of bodies, natures which are
brought into contact from without may affect the organs of the senses,
and the body's passive affection may precede the active energy of the
spirit, and call forth to itself the activity of the mind; if then,
when the effects of bodies are felt, the mind is not marked in any way
by its passive reception thereof, but declares that reception subject
to the body of its own force, how much less do those subjects, which
are free from all affections of bodies, follow external objects in
their perceptions, and how much more do they make clear the way for
the action of their mind? By this argument many different manners of
understanding have fallen to widely different natures of things. For
the senses are incapable of any knowledge but their own, and they
alone fall to those living beings which are incapable of motion, as
are sea shell-fish, and other low forms of life which live by clinging
to rocks; while imagination is granted to animals with the power of
motion, who seem to be affected by some desire to seek or avoid
certain things. \page{158} But reason belongs to the human race alone,
just as the true intelligence is God's alone. Wherefore that manner of
knowledge is better than others, for it can comprehend of its own
nature not only the subject peculiar to itself, but also the subjects
of the other kinds of knowledge. Suppose that the senses and
imagination thus oppose reasoning, saying, ``The universal natural
kinds, which reason believes that it can perceive, are nothing; for
what is comprehensible to the senses and the imagination cannot be
universal: therefore either the judgment of reason is true, and that
which can be perceived by the senses is nothing; or, since reason
knows well that there are many subjects comprehensible to the senses
and imagination, the conception of reason is vain, for it holds to be
universal what is an individual matter comprehensible to the senses.''
To this reason might answer, that ``it sees from a general point of
view what is comprehensible to the senses and the imagination, but
they cannot aspire to a knowledge of universals, since their manner of
knowledge cannot go further than material or bodily appearances; and
in the matter of knowledge it is better to trust to the stronger and
more nearly perfect judgment.'' If such a trial of argument occurred,
should not we, who have within us the force of reasoning as well as
the powers of the senses and imagination, approve of the cause of
reason rather than that of the others? It is in like manner that human
reason thinks that \page{159} the divine intelligence cannot perceive
the things of the future except as it conceives them itself. For you
argue thus: ``If there are events which do not appear to have sure or
necessary results, their results cannot be known for certain
beforehand: therefore there can be no foreknowledge of these events;
for if we believe that there is any foreknowledge thereof, there can
exist nothing but such as is brought forth of necessity.'' If
therefore we, who have our share in possession of reason, could go
further and possess the judgment of the mind of God, we should then
think it most just that human reason should yield itself to the mind
of God, just as we have determined that the senses and imagination
ought to yield to reason.

`Let us therefore raise ourselves, if so be that we can, to that
height of the loftiest intelligence. For there reason will see what it
cannot of itself perceive, and that is to know how even such things as
have uncertain results are perceived definitely and for certain by
foreknowledge; and such foreknowledge will not be mere opinion, but
rather the single and direct form of the highest knowledge unlimited
by any finite bounds.

\section{Meter V}

`In what different shapes do living beings move upon the earth! Some
make flat their bodies, sweeping through the dust and using their
strength to make therein a furrow without break; some flit here and
there upon light wings \page{160} which beat the breeze, and they
float through vast tracks of air in their easy flight. 'Tis others'
wont to plant their footsteps on the ground, and pass with their paces
over green fields or under trees. Though all these thou seest move in
different shapes, yet all have their faces downward along the ground,
and this doth draw downward and dull their senses. Alone of all, the
human race lifts up its head on high, and stands in easy balance with
the body upright, and so looks down to spurn the earth. If thou art
not too earthly by an evil folly, this pose is as a lesson. Thy glance
is upward, and thou dost carry high thy head, and thus thy search is
heavenward: then lead thy soul too upward, lest while the body is
higher raised, the mind sink lower to the earth.

\section{Prose VI}

`Since then all that is known is apprehended, as we just now shewed,
not according to its nature but according to the nature of the knower,
let us examine, so far as we lawfully may, the character of the divine
nature, so that we may be able to learn what its knowledge is.

`The common opinion, according to all men living, is that God is
eternal. Let us therefore consider what is eternity. For eternity
will, I think, make clear to us at the same time the divine nature and
knowledge.

`Eternity is the simultaneous and complete possession of infinite
life. This will appear more clearly if we compare it with temporal
\page{161} things. All that lives under the conditions of time moves
through the present from the past to the future; there is nothing set
in time which can at one moment grasp the whole space of its lifetime.
It cannot yet comprehend to-morrow; yesterday it has already lost. And
in this life of to-day your life is no more than a changing, passing
moment. And as Aristotle\footnote{Aristotle, \textit{De C\ae lo},
\textsc{i}.} said of the universe, so it is of all that is subject to
time; though it never began to be, nor will ever cease, and its life
is co-extensive with the infinity of time, yet it is not such as can
be held to be eternal. For though it apprehends and grasps a space of
infinite lifetime, it does not embrace the whole simultaneously; it
has not yet experienced the future. What we should rightly call
eternal is that which grasps and possesses wholly and simultaneously
the fulness of unending life, which lacks naught of the future, and
has lost naught of the fleeting past; and such an existence must be
ever present in itself to control and aid itself, and also must keep
present with itself the infinity of changing time. Therefore, people
who hear that Plato thought that this universe had no beginning of
time and will have no end, are not right in thinking that in this way
the created world is co-eternal with its creator.\footnote{Boethius
speaks of people who `hear that Plato thought, etc.,' because this was
the teaching of some of Plato's successors at the Academy. Plato
himself thought otherwise, as may be seen in the \textit{Tim\ae us},
\textit{e.g.} ch. xi. 38 \textsc{b}., `Time then has come into being
along with the universe, that being generated together, together they
may be dissolved, should a dissolution of them ever come to pass; and
it was made after the pattern of the eternal nature that it might be
as like to it as possible. For the pattern is existent for all
eternity, but the copy has been, and is, and shall be, throughout all
time continually.' (Mr. Archer Hind's translation.)} \page{162} For to
pass through unending life, the attribute which Plato ascribes to the
universe is one thing; but it is another thing to grasp simultaneously
the whole of unending life in the present; this is plainly a peculiar
property of the mind of God.

`And further, God should not be regarded as older than His creations
by any period of time, but rather by the peculiar property of His own
single nature. For the infinite changing of temporal things tries to
imitate the ever simultaneously present immutability of His life: it
cannot succeed in imitating or equalling this, but sinks from
immutability into change, and falls from the single directness of the
present into an infinite space of future and past. And since this
temporal state cannot possess its life completely and simultaneously,
but it does in the same manner exist for ever without ceasing, it
therefore seems to try in some degree to rival that which it cannot
fulfil or represent, for it binds itself to some sort of present time
out of this small and fleeting moment; but inasmuch as this temporal
present bears a certain appearance of that abiding present, it somehow
makes \page{163} those, to whom it comes, seem to be in truth what
they imitate. But since this imitation could not be abiding, the
unending march of time has swept it away, and thus we find that it has
bound together, as it passes, a chain of life, which it could not by
abiding embrace in its fulness. And thus if we would apply proper
epithets to those subjects, we can say, following Plato, that God is
eternal, but the universe is continual.

`Since then all judgment apprehends the subjects of its thought
according to its own nature, and God has a condition of ever-present
eternity, His knowledge, which passes over every change of time,
embracing infinite lengths of past and future, views in its own direct
comprehension everything as though it were taking place in the
present. If you would weigh the foreknowledge by which God
distinguishes all things, you will more rightly hold it to be a
knowledge of a never-failing constancy in the present, than a
foreknowledge of the future. Whence Providence is more rightly to be
understood as a looking forth than a looking forward, because it is
set far from low matters and looks forth upon all things as from a
lofty mountain-top above all. Why then do you demand that all things
occur by necessity, if divine light rests upon them, while men do not
render necessary such things as they can see? Because you can see
things of the present, does your sight therefore put upon them any
necessity? \page{164} Surely not. If one may not unworthily compare
this present time with the divine, just as you can see things in this
your temporal present, so God sees all things in His eternal present.
Wherefore this divine foreknowledge does not change the nature or
individual qualities of things: it sees things present in its
understanding just as they will result some time in the future. It
makes no confusion in its distinctions, and with one view of its mind
it discerns all that shall come to pass whether of necessity or not.
For instance, when you see at the same time a man walking on the earth
and the sun rising in the heavens, you see each sight simultaneously,
yet you distinguish between them, and decide that one is moving
voluntarily, the other of necessity. In like manner the perception of
God looks down upon all things without disturbing at all their nature,
though they are present to Him but future under the conditions of
time. Wherefore this foreknowledge is not opinion but knowledge
resting upon truth, since He knows that a future event is, though He
knows too that it will not occur of necessity. If you answer here that
what God sees about to happen, cannot but happen, and that what cannot
but happen is bound by necessity, you fasten me down to the word
necessity, I will grant that we have a matter of most firm truth, but
it is one to which scarce any man can approach unless he be a
contemplator of the divine. For I shall answer that such a thing
\page{165} will occur of necessity, when it is viewed from the point
of divine knowledge; but when it is examined in its own nature, it
seems perfectly free and unrestrained. For there are two kinds of
necessities; one is simple: for instance, a necessary fact, ``all men
are mortal''; the other is conditional; for instance, if you know that
a man is walking, he must be walking: for what each man knows cannot
be otherwise than it is known to be; but the conditional one is by no
means followed by this simple and direct necessity; for there is no
necessity to compel a voluntary walker to proceed, though it is
necessary that, if he walks, he should be proceeding. In the same way,
if Providence sees an event in its present, that thing must be, though
it has no necessity of its own nature. And God looks in His present
upon those future things which come to pass through free will.
Therefore if these things be looked at from the point of view of God's
insight, they come to pass of necessity under the condition of divine
knowledge; if, on the other hand, they are viewed by themselves, they
do not lose the perfect freedom of their nature. Without doubt, then,
all things that God foreknows do come to pass, but some of them
proceed from free will; and though they result by coming into
existence, yet they do not lose their own nature, because before they
came to pass they could also not have come to pass.

```What then,'' you may ask, ``is the differ-\page{166}ence in their
not being bound by necessity, since they result under all
circumstances as by necessity, on account of the condition of divine
knowledge?'' This is the difference, as I just now put forward: take
the sun rising and a man walking; while these operations are
occurring, they cannot but occur: but the one was bound to occur
before it did; the other was not so bound. What God has in His
present, does exist without doubt; but of such things some follow by
necessity, others by their authors' wills. Wherefore I was justified
in saying that if these things be regarded from the view of divine
knowledge, they are necessary, but if they are viewed by themselves,
they are perfectly free from all ties of necessity: just as when you
refer all, that is clear to the senses, to the reason, it becomes
general truth, but it remains particular if regarded by itself.
``But,'' you will say, ``if it is in my power to change a purpose of
mine, I will disregard Providence, since I may change what Providence
foresees.'' To which I answer, ``You can change your purpose, but
since the truth of Providence knows in its present that you can do so,
and whether you do so, and in what direction you may change it,
therefore you cannot escape that divine foreknowledge: just as you
cannot avoid the glance of a present eye, though you may by your free
will turn yourself to all kinds of different actions.'' ``What?'' you
will say, ``can I by my own action change \page{167} divine knowledge,
so that if I choose now one thing, now another, Providence too will
seem to change its knowledge?'' No; divine insight precedes all future
things, turning them back and recalling them to the present time of
its own peculiar knowledge. It does not change, as you may think,
between this and that alternation of foreknowledge. It is constant in
preceding and embracing by one glance all your changes. And God does
not receive this ever-present grasp of all things and vision of the
present at the occurrence of future events, but from His own peculiar
directness. Whence also is that difficulty solved which you laid down
a little while ago, that it was not worthy to say that our future
events were the cause of God's knowledge. For this power of knowledge,
ever in the present and embracing all things in its perception, does
itself constrain all things, and owes naught to following events from
which it has received naught. Thus, therefore, mortal men have their
freedom of judgment intact. And since their wills are freed from all
binding necessity, laws do not set rewards or punishments unjustly.
God is ever the constant foreknowing overseer, and the ever-present
eternity of His sight moves in harmony with the future nature of our
actions, as it dispenses rewards to the good, and punishments to the
bad. Hopes are not vainly put in God, nor prayers in vain offered: if
these are right, they cannot but be answered. Turn \page{168}
therefore from vice: ensue virtue: raise your soul to upright hopes:
send up on high your prayers from this earth. If you would be honest,
great is the necessity enjoined upon your goodness, since all you do
is done before the eyes of an all-seeing Judge.'

