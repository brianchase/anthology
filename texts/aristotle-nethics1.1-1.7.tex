
\author{Aristotle}
\authdate{384--322 \BCE}
%\textdate{}
\addon{Book 1, Chapters 1 through 7}
\chapter[Nicomachean Ethics, bk. 1, chaps. 1--7]{Nicomachean Ethics}
\source{aristotle1906}

\page{1}\textbf{1}. Every art and every kind of inquiry, and likewise
every act and purpose, seems to aim at some good: and so it has been
well said that the good is that at which everything aims.

But a difference is observable among these aims or ends. What is aimed
at is sometimes the exercise of a faculty, sometimes a certain result
beyond that exercise. And where there is an end beyond the act, there
the result is better than the exercise of the faculty.

Now since there are many kinds of actions and many arts and sciences,
it follows that there are many ends also; \textit{e.g.} health is the
end of medicine, ships of shipbuilding, victory of the art of war, and
wealth of economy.

But when several of these are subordinated to \page{2} some one art or
science,---as the making of bridles and other trappings to the art of
horsemanship, and this in turn, along with all else that the soldier
does, to the art of war, and so on,\footnote{Reading \grk{τὸν αὐτὸν
δέ}.}---then the end of the master-art is always more desired than the
ends of the subordinate arts, since these are pursued for its sake.
And this is equally true whether the end in view be the mere exercise
of a faculty or something beyond that, as in the above instances.

\textbf{2}. If then in what we do there be some end which we wish for
on its own account, choosing all the others as means to this, but not
every end without exception as a means to something else (for so we
should go on \textit{ad infinitum}, and desire would be left void and
objectless),---this evidently will be the good or the best of all
things. And surely from a practical point of view it much concerns us
to know this good; for then, like archers shooting at a definite mark,
we shall be more likely to attain what we want.

If this be so, we must try to indicate roughly what it is, and first
of all to which of the arts or sciences it belongs.

It would seem to belong to the supreme art or science, that one which
most of all deserves the name of master-art or master-science.

Now Politics\footnote{To Aristotle Politics is a much wider term than
to us; it covers the whole field of human life, since man is
essentially social (7, 6); it has to determine (1) what is the
good?---the question of this treatise (\S9)---and (2) what can law do
to promote this good?---the question of the sequel, which is specially
called ``The Politics;'' \textit{cf.} X. 9.} seems to answer to this
description. \page{3} For it prescribes which of the sciences a state
needs, and which each man shall study, and up to what point; and to it
we see subordinated even the highest arts, such as economy, rhetoric,
and the art of war.

Since then it makes use of the other practical sciences, and since it
further ordains what men are to do and from what to refrain, its end
must include the ends of the others, and must be the proper good of
man.

For though this good is the same for the individual and the state, yet
the good of the state seems a grander and more perfect thing both to
attain and to secure; and glad as one would be to do this service for
a single individual, to do it for a people and for a number of states
is nobler and more divine.

This then is the aim of the present inquiry, which is a sort of
political inquiry.\footnote{\textit{i.e.} covers a part of the ground
only: see preceding note}

\textbf{3}. We must be content if we can attain to so much precision
in our statement as the subject before us admits of; for the same
degree of accuracy is no more to be expected in all kinds of reasoning
than in all kinds of handicraft.

Now the things that are noble and just (with which Politics deals) are
so various and so uncertain, that some think these are merely
conventional and not natural distinctions.

There is a similar uncertainty also about what is good, because good
things often do people harm: men have before now been ruined by
wealth, and have lost their lives through courage.

Our subject, then, and our data being of this \page{4} nature, we must
be content if we can indicate the truth roughly and in outline, and
if, in dealing with matters that are not amenable to immutable laws,
and reasoning from premises that are but probable, we can arrive at
probable conclusions.\footnote{The expression \grk{τὰ ὡς ἐπὶ τὸ πολύ}
covers both (1) what is generally thought not universally true, and
(2) what is probable though not certain.}

The reader, on his part, should take each of my statements in the same
spirit; for it is the mark of an educated man to require, in each kind
of inquiry, just so much exactness as the subject admits of: it is
equally absurd to accept probable reasoning from a mathematician, and
to demand scientific proof from an orator.

But each man can form a judgment about what he knows, and is called
``a good judge'' of that---of any special matter when he has received
a special education therein, ``a good judge'' (without any qualifying
epithet) when he has received a universal education. And hence a young
man is not qualified to be a student of Politics; for he lacks
experience of the affairs of life, which form the data and the
subject-matter of Politics.

Further, since he is apt to be swayed by his feelings, he will derive
no benefit from a study whose aim is not speculative but practical.

But in this respect young in character counts the same as young in
years; for the young man's disqualification is not a matter of time,
but is due to the fact that feeling rules his life and directs all his
desires. Men of this character turn the knowledge \page{5} they get to
no account in practice, as we see with those we call incontinent; but
those who direct their desires and actions by reason will gain much
profit from the knowledge of these matters.

So much then by way of preface as to the student, and the spirit in
which he must accept what we say, and the object which we propose to
ourselves.

\textbf{4}. Since---to re\-sume---all knowledge and all purpose aims
at some good, what is this which we say is the aim of Politics; or, in
other words, what is the highest of all realizable goods?

As to its name, I suppose nearly all men are agreed; for the masses
and the men of culture alike declare that it is happiness, and hold
that to ``live well'' or to ``do well'' is the same as to be
``happy.''

But they differ as to what this happiness is, and the masses do not
give the same account of it as the philosophers.

The former take it to be something palpable and plain, as pleasure or
wealth or fame; one man holds it to be this, and another that, and
often the same man is of different minds at different times,---af\-ter
sickness it is health, and in poverty it is wealth; while when they
are impressed with the consciousness of their ignorance, they admire
most those who say grand things that are above their comprehension.

Some philosophers, on the other hand, have thought that, beside these
several good things, there is an ``absolute'' good which is the cause
of their goodness.

As it would hardly be worth while to review all the opinions that have
been held, we will confine ourselves to those which are most popular,
or which seem to have some foundation in reason.

\page{6}But we must not omit to notice the distinction that is drawn
between the method of proceeding from your starting-points or
principles, and the method of working up to them. Plato used with
fitness to raise this question, and to ask whether the right way is
from or to your starting-points, as in the race-course you may run
from the judges to the boundary, or \textit{vice vers\^a}.

Well, we must start from what is known.

But ``what is known'' may mean two things: ``what is known to us,''
which is one thing, or ``what is known'' simply, which is another.

I think it is safe to say that \textit{we} must start from what is
known to \textit{us}.

And on this account nothing but a good moral training can qualify a
man to study what is noble and just---in a word, to study questions of
Politics. For the undemonstrated fact is here the starting-point, and
if this undemonstrated fact be sufficiently evident to a man, he will
not require a ``reason why.'' Now the man who has had a good moral
training either has already arrived at starting-points or principles
of action, or will easily accept them when pointed out. But he who
neither has them nor will accept them may hear what Hesiod
says\footnote{``Works and Days,'' 291--295.}---

\begin{verse}
``The best is he who of himself doth know;\\
Good too is he who listens to the wise;\\
But he who neither knows himself nor heeds\\
The words of others, is a useless man.''
\end{verse}

\textbf{5}. Let us now take up the discussion at the point from which
we digressed.

\page{7}It seems that men not unreasonably take their notions of the
good or happiness from the lives actually led, and that the masses who
are the least refined suppose it to be pleasure, which is the reason
why they aim at nothing higher than the life of enjoyment.

For the most conspicuous kinds of life are three: this life of
enjoyment, the life of the statesman, and, thirdly, the contemplative
life.

The mass of men show themselves utterly slavish in their preference
for the life of brute beasts, but their views receive consideration
because many of those in high places have the tastes of Sardanapalus.

Men of refinement with a practical turn prefer honour; for I suppose
we may say that honour is the aim of the statesman's life.

But this seems too superficial to be the good we are seeking: for it
appears to depend upon those who give rather than upon those who
receive it; while we have a presentiment that the good is something
that is peculiarly a man's own and can scarce be taken away from him.

Moreover, these men seem to pursue honour in order that they may be
assured of their own excellence,---at least, they wish to be honoured by
men of sense, and by those who know them, and on the ground of their
virtue or excellence. It is plain, then, that in their view, at any
rate, virtue or excellence is better than honour; and perhaps we
should take this to be the end of the statesman's life, rather than
honour.

But virtue or excellence also appears too incomplete to be what we
want; for it seems that a man \page{8} might have virtue and yet be
asleep or be inactive all his life, and, moreover, might meet with the
greatest disasters and misfortunes; and no one would maintain that
such a man is happy, except for argument's sake. But we will not dwell
on these matters now, for they are sufficiently discussed in the
popular treatises.

The third kind of life is the life of contemplation: we will treat of
it further on.\footnote{\textit{Cf.} VI. 7, 12, and X. 7, 8.}

As for the money-making life, it is something quite contrary to
nature; and wealth evidently is not the good of which we are in
search, for it is merely useful as a means to something else. So we
might rather take pleasure and virtue or excellence to be ends than
wealth; for they are chosen on their own account. But it seems that
not even they are the end, though much breath has been wasted in
attempts to show that they are.

\textbf{6}. Dismissing these views, then, we have now to consider the
``universal good,'' and to state the difficulties which it presents;
though such an inquiry is not a pleasant task in view of our
friendship for the authors of the doctrine of ideas. But we venture to
think that this is the right course, and that in the interests of
truth we ought to sacrifice even what is nearest to us, especially as
we call ourselves philosophers. Both are dear to us, but it is a
sacred duty to give the preference to truth.

In the first place, the authors of this theory themselves did not
assert a common idea in the case of things of which one is prior to
the other; and for this \page{9} reason they did not hold one common
idea of numbers. Now the predicate good is applied to substances and
also to qualities and relations. But that which has independent
existence, what we call ``substance,'' is logically prior to that
which is relative; for the latter is an offshoot as it were, or [in
logical language] an accident of a thing or substance. So [by their
own showing] there cannot be one common idea of these goods.

Secondly, the term good is used in as many different ways as the term
``is'' or ``being:'' we apply the term to substances or independent
existences, as God, reason; to qualities, as the virtues; to quantity,
as the moderate or due amount; to relatives, as the useful; to time,
as opportunity; to place, as habitation, and so on. It is evident,
therefore, that the word good cannot stand for one and the same notion
in all these various applications; for if it did, the term could not
be applied in all the categories, but in one only.

Thirdly, if the notion were one, since there is but one science of all
the things that come under one idea, there would be but one science of
all goods; but as it is, there are many sciences even of the goods
that come under one category; as, for instance, the science which
deals with opportunity in war is strategy, but in disease is medicine;
and the science of the due amount in the matter of food is medicine,
but in the matter of exercise is the science of gymnastic.

Fourthly, one might ask what they mean by the ``absolute:'' in
``absolute man'' and ``man'' the word ``man'' has one and the same
sense; for in respect of manhood there will be no difference between
them; \page{10} and if so, neither will there be any difference in
respect of goodness between ``absolute good'' and ``good.''

Fifthly, they do not make the good any more good by making it eternal;
a white thing that lasts a long while is no whiter than what lasts but
a day.

There seems to be more plausibility in the doctrine of the
Pythagoreans, who [in their table of opposites] place the one on the
same side with the good things [instead of reducing all goods to
unity]; and even Speusippus\footnote{Plato's nephew and successor.}
seems to follow them in this.

However, these points may be reserved for another occasion; but
objection may be taken to what I have said on the ground that the
Platonists do not speak in this way of all goods indiscriminately, but
hold that those that are pursued and welcomed on their own account are
called good by reference to one common form or type, while those
things that tend to produce or preserve these goods, or to prevent
their opposites, are called good only as means to these, and in a
different sense.

It is evident that there will thus be two classes of goods: one good
in themselves, the other good as means to the former. Let us separate
then from the things that are merely useful those that are good in
themselves, and inquire if they are called good by reference to one
common idea or type.

Now what kind of things would one call ``good in themselves''?

Surely those things that we pursue even apart from their consequences,
such as wisdom and sight \page{11} and certain pleasures and certain
honours; for although we sometimes pursue these things as means, no
one could refuse to rank them among the things that are good in
themselves.

If these be excluded, nothing is good in itself except the idea; and
then the type or form will be meaningless.\footnote{For there is no
meaning in a form which is a form of nothing, in a universal which has
no particulars under it.}

If however, these are ranked among the things that are good in
themselves, then it must be shown that the goodness of all of them can
be defined in the same terms, as white has the same meaning when
applied to snow and to white lead.

But, in fact, we have to give a separate and different account of the
goodness of honour and wisdom and pleasure.

Good, then, is not a term that is applied to all these things alike in
the same sense or with reference to one common idea or form.

But how then do these things come to be called good? for they do not
appear to have received the same name by chance merely. Perhaps it is
because they all proceed from one source, or all conduce to one end;
or perhaps it is rather in virtue of some analogy, just as we call the
reason the eye of the soul because it bears the same relation to the
soul that the eye does to the body, and so on.

But we may dismiss these questions at present; for to discuss them in
detail belongs more properly to another branch of philosophy.

And for the same reason we may dismiss the \page{12} further
consideration of the idea; for even granting that this term good,
which is applied to all these different things, has one and the same
meaning throughout, or that there is an absolute good apart from
these particulars, it is evident that this good will not be anything
that man can realize or attain: but it is a good of this kind that we
are now seeking.

It might, perhaps, be thought that it would nevertheless be well to
make ourselves acquainted with this universal good, with a view to the
goods that are attainable and realizable. With this for a pattern, it
may be said, we shall more readily discern our own good, and
discerning achieve it.

There certainly is some plausibility in this argument, but it seems to
be at variance with the existing sciences; for though they are all
aiming at some good and striving to make up their deficiencies, they
neglect to inquire about this universal good. And yet it is scarce
likely that the professors of the several arts and sciences should not
know, nor even look for, what would help them so much.

And indeed I am at a loss to know how the weaver or the carpenter
would be furthered in his art by a knowledge of this absolute good, or
how a man would be rendered more able to heal the sick or to command
an army by contemplation of the pure form or idea. For it seems to me
that the physician does not even seek for health in this abstract way,
but seeks for the health of man, or rather of some particular man, for
it is individuals that he has to heal.

\textbf{7}. Leaving these matters, then, let us return once \page{13}
more to the question, what this good can be of which we are in search.

It seems to be different in different kinds of action and in different
arts,---one thing in medicine and another in war, and so on. What then
is the good in each of these cases? Surely that for the sake of which
all else is done. And that in medicine is health, in war is victory,
in building is a house,---a different thing in each different case,
but always, in whatever we do and in whatever we choose, the end. For
it is always for the sake of the end that all else is done.

If then there be one end of all that man does, this end will be the
realizable good,---or these ends, if there be more than one.

By this generalization our argument is brought to the same point as
before.\footnote{2, 1. See Stewart.} This point we must try to explain
more clearly.

We see that there are many ends. But some of these are chosen only as
means, as wealth, flutes, and the whole class of instruments. And so
it is plain that not all ends are final.

But the best of all things must, we conceive, be something final.

If then there be only one final end, this will be what we are
seeking,---or if there be more than one, then the most final of them.

Now that which is pursued as an end in itself is more final than that
which is pursued as means to something else, and that which is never
chosen as means than that which is chosen both as an end in itself and
as means, and that is strictly final which \page{14} is always chosen
as an end in itself and never as means.

Happiness seems more than anything else to answer to this description:
for we always choose it for itself, and never for the sake of
something else; while honour and pleasure and reason, and all virtue
or excellence, we choose partly indeed for themselves (for, apart from
any result, we should choose each of them), but partly also for the
sake of happiness, supposing that they will help to make us happy. But
no one chooses happiness for the sake of these things, or as a means
to anything else at all.

We seem to be led to the same conclusion when we start from the notion
of self-sufficiency.

The final good is thought to be self-sufficing [or all-sufficing]. In
applying this term we do not regard a man as an individual leading a
solitary life, but we also take account of parents, children, wife,
and, in short, friends and fellow-citizens generally, since man is
naturally a social being. Some limit must indeed be set to this; for
if you go on to parents and descendants and friends of friends, you
will never come to a stop. But this we will consider further on: for
the present we will take self-sufficing to mean what by itself makes
life desirable and in want of nothing. And happiness is believed to
answer to this description.

And further, happiness is believed to be the most desirable thing in
the world, and that not merely as one among other good things: if it
were merely one among other good things [so that other things could be
added to it], it is plain that the addition of the least \page{15} of
other goods must make it more desirable; for the addition becomes a
surplus of good, and of two goods the greater is always more
desirable.

Thus it seems that happiness is something final and self-sufficing,
and is the end of all that man does.

But perhaps the reader thinks that though no one will dispute the
statement that happiness is the best thing in the world, yet a still
more precise definition of it is needed.

This will best be gained, I think, by asking, What is the function of
man? For as the goodness and the excellence of a piper or a sculptor,
or the practiser of any art, and generally of those who have any
function or business to do, lies in that function, so man's good would
seem to lie in his function, if he has one.

But can we suppose that, while a carpenter and a cobbler has a
function and a business of his own, man has no business and no
function assigned him by nature? Nay, surely as his several members,
eye and hand and foot, plainly have each his own function, so we must
suppose that man also has some function over and above all these.

What then is it?

Life evidently he has in common even with the plants, but we want that
which is peculiar to him. We must exclude, therefore, the life of mere
nutrition and growth.

Next to this comes the life of sense; but this too he plainly shares
with horses and cattle and all kinds of animals.

There remains then the life whereby he acts---the \page{16} life of
his rational nature,\footnote{\grk{πρακτική τις τοῦ λόγον ἔχοντος}.
Aristotle frequently uses the terms \grk{πρᾶξις}, \grk{πρακτός},
\grk{πρακτικός} in this wide sense, covering all that man does,
\textit{i.e.} all that part of man's life that is within the control
of his will, or that is consciously directed to an end, including
therefore speculation as well as action.} with its two sides or
divisions, one rational as obeying reason, the other rational as
having and exercising reason.

But as this expression is ambiguous,\footnote{For it might mean either
the mere possession of the vital faculties, or their exersice.} we
must be understood to mean thereby the life that consists in the
exercise of the faculties; for this seems to be more properly entitled
to the name.

The function of man, then, is exercise of his vital faculties [or
soul] on one side in obedience to reason, and on the other side with
reason.

But what is called the function of a man of any profession and the
function of a man who is good in that profession are generically the
same, \textit{e.g.} of a harper and of a good harper; and this holds
in all cases without exception, only that in the case of the latter
his superior excellence at his work is added; for we say a harper's
function is to harp, and a good harper's to harp well.

(Man's function then being, as we say, a kind of life---that is to
say, exercise of his faculties and action of various kinds with
reason---the good man's function is to do this well and beautifully
[or nobly]. But the function of anything is done well when it is done
in accordance with the proper excellence of that thing.)\footnote{This
paragraph seems to be a repetition (I would rather say a re-writing)
of the previous paragraph. See note VII. 3, 2.}

\page{17}If this be so the result is that the good of man is exercise
of his faculties in accordance with excellence or virtue, or, if there
be more than one, in accordance with the best and most complete
virtue.\footnote{This ``best and most complete excellence or virtue''
is the trained faculty for philosophic speculation, and the
contemplative life is man's highest happiness. \textit{Cf.} X. 7, 1.}

But there must also be a full term of years for this
exercise;\footnote{\textit{Cf.} 9, 11.} for one swallow or one fine
day does not make a spring, nor does one day or any small space of
time make a blessed or happy man.

This, then, may be taken as a rough outline of the good; for this, I
think, is the proper method,---first to sketch the outline, and then
to fill in the details. But it would seem that, the outline once
fairly drawn, any one can carry on the work and fit in the several
items which time reveals to us or helps us to find. And this indeed is
the way in which the arts and sciences have grown; for it requires no
extraordinary genius to fill up the gaps.

We must bear in mind, however, what was said above, and not demand the
same degree of accuracy in all branches of study, but in each case so
much as the subject-matter admits of and as is proper to that kind of
inquiry. The carpenter and the geometer both look for the right angle,
but in different ways: the former only wants such an approximation to
it as his work requires, but the latter wants to know what constitutes
a right angle, or what is its special quality; his aim is to find out
the truth. And so in other cases we must follow the same course, lest
we spend more \page{18} time on what is immaterial than on the real
business in hand.

Nor must we in all cases alike demand the reason why; sometimes it is
enough if the undemonstrated fact be fairly pointed out, as in the
case of the starting-points or principles of a science. Undemonstrated
facts always form the first step or starting-point of a science; and
these starting-points or principles are arrived at some in one way,
some in another---some by induction, others by perception, others
again by some kind of training. But in each case we must try to
apprehend them in the proper way, and do our best to define them
clearly; for they have great influence upon the subsequent course of
an inquiry. A good start is more than half the race, I think, and our
starting-point or principle, once found, clears up a number of our
difficulties.

