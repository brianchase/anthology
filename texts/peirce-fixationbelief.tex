
\author{Charles Sanders Peirce}
\authdate{1839--1914}
\textdate{1877}
\chapter[Charles Sanders Peirce -- The Fixation of Belief]{The
Fixation of Belief}

\nfootnote{\fullcite{peirce1877}}

% FIX: Perhaps proofread again.

\page{1}\section*{I.}

Few persons care to study logic, because everybody conceives himself
to be proficient enough in the art of reasoning already. But I observe
that this satisfaction is limited to one's own ratiocination, and does
not extend to that of other men.

We come to the full possession of our power of drawing inferences the
last of all our faculties, for it is not so much a natural gift as a
long and difficult art. The history of its practice would make a grand
subject for a book. The medieval schoolmen, following the Romans, made
logic the earliest of a boy's studies after grammar, as being very
easy. So it was, as they understood it. Its fundamental principle,
according to them, was, that all knowledge rests either on authority
or reason; but that whatever is deduced by reason depends ultimately
on a premise derived from authority. Accordingly, as soon as a boy
was perfect in the syllogistic procedure, his intellectual kit of
tools was held to be complete.

To Roger Bacon, that remarkable mind who in the middle of the
thirteenth century was almost a scientific man, the schoolmen's
conception of reasoning appeared only an obstacle to truth. He saw
that experience alone teaches anything---a proposition which to us
seems easy to understand, because a distinct conception of experience
has been handed down to us from former generations; which to him
likewise seemed perfectly clear, because its difficulties had not yet
unfolded themselves. Of all kinds of experience, the best, he thought,
was interior illumination, which teaches many things about Nature
\page{2} which the external senses could never discover, such as the
transubstantiation of bread.

Four centuries later, the more celebrated Bacon, in the first book of
his ``Novum Organum,'' gave his clear account of experience as
something which must be open to verification and reexamination. But,
superior as Lord Bacon's conception is to earlier notions, a modern
reader who is not in awe of his grandiloquence is chiefly struck by
the inadequacy of his view of scientific procedure. That we have only
to make some crude experiments, to draw up briefs of the results in
certain blank forms, to go through these by rule, checking off
everything disproved and setting down the alternatives, and that thus
in a few years physical science would be finished up---what an idea!
``He wrote on science like a Lord Chancellor,'' indeed, as Harvey, a
genuine man of science said.

The early scientists, Copernicus, Tycho Brahe, Kepler, Galileo, and
Gilbert, had methods more like those of their modern brethren. Kepler
undertook to draw a curve through the places of Mars;\footnote{Not
quite so, but as nearly so as can be told in a few words.} and his
greatest service to science was in impressing on men's minds that this
was the thing to be done if they wished to improve astronomy; that
they were not to content themselves with inquiring whether one system
of epicycles was better than another, but that they were to sit down
to the figures and find out what the curve, in truth, was. He
accomplished this by his incomparable energy and courage, blundering
along in the most inconceivable way (to us), from one irrational
hypothesis to another, until, after trying twenty-two of these, he
fell, by the mere exhaustion of his invention, upon the orbit which a
mind well furnished with the weapons of modern logic would have tried
almost at the outset.

In the same way, every work of science great enough to be well
remembered for a few generations affords some exemplification of the
defective state of the art of reasoning of the time when it was
written; and each chief step in science has been a lesson in logic. It
was so when Lavoisier and his contemporaries took up the study of
Chemistry. The old chemist's maxim had been, ``\textit{Lege, lege,
lege, labora, ora, et relege.}'' Lavoisier's method was not to read
and pray, not to dream that some long and complicated chemical process
would have a certain effect, to put it into practice with dull
patience, after its inevitable failure to dream that with some
modification it would have another result, and to end by publishing
the last dream as a fact: his way was to carry his mind into his
laboratory, and literally to make of his alembics and cucurbits
instruments of thought, giving a new conception of reasoning as
something which was to be done with one's eyes open, in manipulating
real things instead of words and fancies.

The Darwinian controversy is, in large part, a question of logic. Mr.
Darwin proposed to apply the statistical method to biology. The same
thing has been done in a widely different branch of science, the
\page{3} theory of gases. Though unable to say what the movements of
any particular molecule of gas would be on a certain hypothesis
regarding the constitution of this class of bodies, Clausius and
Maxwell were yet able, by the application of the doctrine of
probabilities, to predict that in the long run such and such a
proportion of the molecules would, under given circumstances, acquire
such and such velocities; that there would take place, every second,
such and such a relative number of collisions, etc.; and from these
propositions were able to deduce certain properties of gases,
especially in regard to their heat-relations. In like manner, Darwin,
while unable to say what the operation of variation and natural
selection in any individual case will be, demonstrates that in the
long run they will adapt animals to their circumstances. Whether or
not existing animal forms are due to such action, or what position the
theory ought to take, forms the subject of a discussion in which
questions of fact and questions of logic are curiously interlaced.

\section*{II.}

The object of reasoning is to find out, from the consideration of what
we already know, something else which we do not know. Consequently,
reasoning is good if it be such as to give a true conclusion from true
premises, and not otherwise. Thus, the question of validity is purely
one of fact and not of thinking. A being the premises and B the
conclusion, the question is, whether these facts are really so related
that if A is B is. If so, the inference is valid; if not, not. It is
not in the least the question whether, when the premises are accepted
by the mind, we feel an impulse to accept the conclusion also. It is
true that we do generally reason correctly by nature. But that is an
accident; the true conclusion would remain true if we had no impulse
to accept it; and the false one would remain false, though we could
not resist the tendency to believe in it.

We are, doubtless, in the main logical animals, but we are not
perfectly so. Most of us, for example, are naturally more sanguine and
hopeful than logic would justify. We seem to be so constituted that in
the absence of any facts to go upon we are happy and self-satisfied;
so that the effect of experience is continually to contract our hopes
and aspirations. Yet a lifetime of the application of this corrective
does not usually eradicate our sanguine disposition. Where hope is
unchecked by any experience, it is likely that our optimism is
extravagant. Logicality in regard to practical matters is the most
useful quality an animal can possess, and might, therefore, result
from the action of natural selection; but outside of these it is
probably of more advantage to the animal to have his mind filled
with pleasing and encouraging visions, independently of their truth;
and thus, upon unpractical subjects, natural selection might occasion
a fallacious tendency of thought.

\page{4}That which determines us, from given premises, to draw one
inference rather than another, is some habit of mind, whether it be
constitutional or acquired. The habit is good or otherwise, according
as it produces true conclusions from true premises or not; and an
inference is regarded as valid or not, without reference to the truth
or falsity of its conclusion specially, but according as the habit
which determines it is such as to produce true conclusions in general
or not. The particular habit of mind which governs this or that
inference may be formulated in a proposition whose truth depends on
the validity of the inferences which the habit determines; and such a
formula is called a \textit{guiding principle} of inference. Suppose,
for example, that we observe that a rotating disk of copper quickly
comes to rest when placed between the poles of a magnet, and we infer
that this will happen with every disk of copper. The guiding principle
is, that what is true of one piece of copper is true of another. Such
a guiding principle with regard to copper would be much safer than
with regard to many other sub\-stanc\-es---brass, for example.

A book might be written to signalize all the most important of these
guiding principles of reasoning. It would probably be, we must
confess, of no service to a person whose thought is directed wholly to
practical subjects, and whose activity moves along thoroughly-beaten
paths. The problems that present themselves to such a mind are matters
of routine which he has learned once for all to handle in learning his
business. But let a man venture into an unfamiliar field, or where his
results are not continually checked by experience, and all history
shows that the most masculine intellect will ofttimes lose his
orientation and waste his efforts in directions which bring him no
nearer to his goal, or even carry him entirely astray. He is like a
ship in the open sea, with no one on board who understands the rules
of navigation. And in such a case some general study of the guiding
principles of reasoning would be sure to be found useful.

The subject could hardly be treated, however, without being first
limited; since almost any fact may serve as a guiding principle. But
it so happens that there exists a division among facts, such that in
one class are all those which are absolutely essential as guiding
principles, while in the others are all which have any other interest
as objects of research. This division is between those which are
necessarily taken for granted in asking why a certain conclusion
follows from certain premises, and those which are not implied in such
a question. A moment's thought will show that a variety of facts are
already assumed when the logical question is first asked. It is
implied, for instance, that there are such states of mind as doubt and
be\-lief---that a passage from one to the other is possible, the
object of thought remaining the same, and that this transition is
subject to some rules which all minds are alike bound by. As these are
facts \page{5} which we must already know before we can have any clear
conception of reasoning at all, it cannot be supposed to be any longer
of much interest to inquire into their truth or falsity. On the other
hand, it is easy to believe that those rules of reasoning which are
deduced from the very idea of the process are the ones which are the
most essential; and, indeed, that so long as it conforms to these it
will, at least, not lead to false conclusions from true premises. In
point of fact, the importance of what may be deduced from the
assumptions involved in the logical question turns out to be greater
than might be supposed, and this for reasons which it is difficult to
exhibit at the outset. The only one which I shall here mention is,
that conceptions which are really products of logical reflection,
without being readily seen to be so, mingle with our ordinary
thoughts, and are frequently the causes of great confusion. This is
the case, for example, with the conception of quality. A quality as
such is never an object of observation. We can see that a thing is
blue or green, but the quality of being blue and the quality of being
green are not things which we see; they are products of logical
reflection. The truth is, that common-sense, or thought as it first
emerges above the level of the narrowly practical, is deeply imbued
with that bad logical quality to which the epithet
\textit{metaphysical} is commonly applied; and nothing can clear it up
but a severe course of logic.

\section*{III.}

We generally know when we wish to ask a question and when we wish to
pronounce a judgment, for there is a dissimilarity between the
sensation of doubting and that of believing.

But this is not all which distinguishes doubt from belief. There is a
practical difference. Our beliefs guide our desires and shape our
actions. The Assassins, or followers of the Old Man of the Mountain,
used to rush into death at his least command, because they believed
that obedience to him would insure everlasting felicity. Had they
doubted this, they would not have acted as they did. So it is with
every belief, according to its degree. The feeling of believing is a
more or less sure indication of there being established in our nature
some habit which will determine our actions. Doubt never has such an
effect.

Nor must we overlook a third point of difference. Doubt is an uneasy
and dissatisfied state from which we struggle to free ourselves and
pass into the state of belief; while the latter is a calm and
satisfactory state which we do not wish to avoid, or to change to a
belief in anything else.\footnote{I am not speaking of secondary
effects occasionally produced by the interference of other impulses.}
On the contrary, we cling tenaciously, not merely to believing, but to
believing just what we do believe.

\page{6}Thus, both doubt and belief have positive effects upon us,
though very different ones. Belief does not make us act at once, but
puts us into such a condition that we shall behave in a certain way,
when the occasion arises. Doubt has not the least effect of this sort,
but stimulates us to action until it is destroyed. This reminds us of
the irritation of a nerve and the reflex action produced thereby;
while for the analogue of belief, in the nervous system, we must look
to what are called nervous as\-so\-ci\-a\-tions---for example, to
that habit of the nerves in consequence of which the smell of a peach
will make the mouth water.

\section*{IV.}

The irritation of doubt causes a struggle to attain a state of belief.
I shall term this struggle \textit{inquiry}, though it must be admitted
that this is sometimes not a very apt designation.

The irritation of doubt is the only immediate motive for the struggle
to attain belief. It is certainly best for us that our beliefs should
be such as may truly guide our actions so as to satisfy our desires;
and this reflection will make us reject any belief which does not seem
to have been so formed as to insure this result. But it will only do
so by creating a doubt in the place of that belief. With the doubt,
therefore, the struggle begins, and with the cessation of doubt it
ends. Hence, the sole object of inquiry is the settlement of opinion.
We may fancy that this is not enough for us, and that we seek, not
merely an opinion, but a true opinion. But put this fancy to the
test, and it proves groundless; for as soon as a firm belief is
reached we are entirely satisfied, whether the belief be true or
false. And it is clear that nothing out of the sphere of our knowledge
can be our object, for nothing which does not affect the mind can be
the motive for mental effort. The most that can be maintained is, that
we seek for a belief that we shall \textit{think} to be true. But we
think each one of our beliefs to be true, and, indeed, it is mere
tautology to say so.

That the settlement of opinion is the sole end of inquiry is a very
important proposition. It sweeps away, at once, various vague and
erroneous conceptions of proof. A few of these may be noticed here.

1. Some philosophers have imagined that to start an inquiry it was
only necessary to utter a question or set it down upon paper, and have
even recommended us to begin our studies with questioning everything!
But the mere putting of a proposition into the interrogative form does
not stimulate the mind to any struggle after belief. There must be a
real and living doubt, and without this all discussion is idle.

2. It is a very common idea that a demonstration must rest on some
ultimate and absolutely indubitable propositions. These, according to
one school, are first principles of a general nature; according to
another, are first sensations. But, in point of fact, an inquiry,
\page{7} to have that completely satisfactory result called
demonstration, has only to start with propositions perfectly free from
all actual doubt. If the premises are not in fact doubted at all, they
cannot be more satisfactory than they are.

3. Some people seem to love to argue a point after all the world is
fully convinced of it. But no further advance can be made. When doubt
ceases, mental action on the subject comes to an end; and, if it did
go on, it would be without a purpose.

\section*{V.}

If the settlement of opinion is the sole object of inquiry, and if
belief is of the nature of a habit, why should we not attain the
desired end, by taking as answer to a question any we may fancy, and
constantly reiterating it to ourselves, dwelling on all which may
conduce to that belief, and learning to turn with contempt and hatred
from anything that might disturb it? This simple and direct method is
really pursued by many men. I remember once being entreated not to
read a certain newspaper lest it might change my opinion upon
free-trade. ``Lest I might be entrapped by its fallacies and
misstatements,'' was the form of expression. ``You are not,'' my
friend said, ``a special student of political economy. You might,
therefore, easily be deceived by fallacious arguments upon the
subject. You might, then, if you read this paper, be led to believe in
protection. But you admit that free-trade is the true doctrine; and
you do not wish to believe what is not true.'' I have often known this
system to be deliberately adopted. Still oftener, the instinctive
dislike of an undecided state of mind, exaggerated into a vague
dread of doubt, makes men cling spasmodically to the views they
already take. The man feels that, if he only holds to his belief
without wavering, it will be entirely satisfactory. Nor can it be
denied that a steady and immovable faith yields great peace of mind.
It may, indeed, give rise to inconveniences, as if a man should
resolutely continue to believe that fire would not burn him, or that
he would be eternally damned if he received his \textit{ingesta}
otherwise than through a stomach-pump. But then the man who adopts
this method will not allow that its inconveniences are greater than
its advantages. He will say, ``I hold steadfastly to the truth, and
the truth is always wholesome.'' And in many cases it may very well be
that the pleasure he derives from his calm faith overbalances any
inconveniences resulting from its deceptive character. Thus, if it be
true that death is annihilation, then the man who believes that he
will certainly go straight to heaven when he dies, provided he have
fulfilled certain simple observances in this life, has a cheap
pleasure which will not be followed by the least disappointment. A
similar consideration seems to have weight with many persons in
religious topics, for we frequently hear it said, ``Oh, \page{8} I
could not believe so-and-so, because I should be wretched if I did.''
When an ostrich buries its head in the sand as danger approaches, it
very likely takes the happiest course. It hides the danger, and then
calmly says there is no danger; and, if it feels perfectly sure there
is none, why should it raise its head to see? A man may go through
life, systematically keeping out of view all that might cause a change
in his opinions, and if he only suc\-ceeds---basing his method, as he
does, on two fundamental psychological laws---I do not see what can be
said against his doing so. It would be an egotistical impertinence to
object that his procedure is irrational, for that only amounts to
saying that his method of settling belief is not ours. He does not
propose to himself to be rational, and, indeed, will often talk with
scorn of man's weak and illusive reason. So let him think as he
pleases.

But this method of fixing belief, which may be called the method of
tenacity, will be unable to hold its ground in practice. The social
impulse is against it. The man who adopts it will find that other men
think differently from him, and it will be apt to occur to him, in
some saner moment, that their opinions are quite as good as his own,
and this will shake his confidence in his belief. This conception,
that another man's thought or sentiment may be equivalent to one's
own, is a distinctly new step, and a highly important one. It arises
from an impulse too strong in man to be suppressed, without danger of
destroying the human species. Unless we make ourselves hermits, we
shall necessarily influence each other's opinions; so that the problem
becomes how to fix belief, not in the individual merely, but in the
community.

Let the will of the state act, then, instead of that of the
individual. Let an institution be created which shall have for its
object to keep correct doctrines before the attention of the people,
to reiterate them perpetually, and to teach them to the young; having
at the same time power to prevent contrary doctrines from being
taught, advocated, or expressed. Let all possible causes of a change
of mind be removed from men's apprehensions. Let them be kept
ignorant, lest they should learn of some reason to think otherwise
than they do. Let their passions be enlisted, so that they may regard
private and unusual opinions with hatred and horror. Then, let all men
who reject the established belief be terrified into silence. Let the
people turn out and tar-and-feather such men, or let inquisitions be
made into the manner of thinking of suspected persons, and, when they
are found guilty of forbidden beliefs, let them be subjected to some
signal punishment. When complete agreement could not otherwise be
reached, a general massacre of all who have not thought in a certain
way has proved a very effective means of settling opinion in a
country. If the power to do this be wanting, let a list of opinions be
drawn up, to which no man of the least independence of thought can
\page{9} assent, and let the faithful be required to accept all these
propositions, in order to segregate them as radically as possible from
the influence of the rest of the world.

This method has, from the earliest times, been one of the chief means
of upholding correct theological and political doctrines, and of
preserving their universal or catholic character. In Rome, especially,
it has been practised from the days of Numa Pompilius to those of Pius
Nonus. This is the most perfect example in history; but wherever there
is a priest\-hood---and no religion has been without one---this method
has been more or less made use of. Wherever there is an aristocracy,
or a guild, or any association of a class of men whose interests
depend on certain propositions, there will be inevitably found some
traces of this natural product of social feeling. Cruelties always
accompany this system; and when it is consistently carried out, they
become atrocities of the most horrible kind in the eyes of any
rational man. Nor should this occasion surprise, for the officer of a
society does not feel justified in surrendering the interests of that
society for the sake of mercy, as he might his own private
interests. It is natural, therefore, that sympathy and fellowship
should thus produce a most ruthless power.

In judging this method of fixing belief, which may be called the
method of authority, we must, in the first place, allow its
immeasurable mental and moral superiority to the method of tenacity.
Its success is proportionately greater; and, in fact, it has over and
over again worked the most majestic results. The mere structures of
stone which it has caused to be put to\-geth\-er---in Siam, for
example, in Egypt, and in Eu\-rope---have many of them a sublimity
hardly more than rivaled by the greatest works of Nature. And, except
the geological epochs, there are no periods of time so vast as those
which are measured by some of these organized faiths. If we scrutinize
the matter closely, we shall find that there has not been one of their
creeds which has remained always the same; yet the change is so slow
as to be imperceptible during one person's life, so that individual
belief remains sensibly fixed. For the mass of mankind, then, there is
perhaps no better method than this. If it is their highest impulse to
be intellectual slaves, then slaves they ought to remain.

But no institution can undertake to regulate opinions upon every
subject. Only the most important ones can be attended to, and on the
rest men's minds must be left to the action of natural causes. This
imperfection will be no source of weakness so long as men are in such
a state of culture that one opinion does not influence
an\-oth\-er---that is, so long as they cannot put two and two
together. But in the most priestridden states some individuals will be
found who are raised above that condition. These men possess a wider
sort of social \page{10} feeling; they see that men in other countries
and in other ages have held to very different doctrines from those
which they themselves have been brought up to believe; and they cannot
help seeing that it is the mere accident of their having been taught
as they have, and of their having been surrounded with the manners and
associations they have, that has caused them to believe as they do and
not far differently. And their candour cannot resist the reflection
that there is no reason to rate their own views at a higher value than
those of other nations and other centuries; thus giving rise to doubts
in their minds.

% proofread to here

They will further perceive that such doubts as these must exist in
their minds with reference to every belief which seems to be
determined by the caprice either of themselves or of those who
originated the popular opinions. The willful adherence to a belief,
and the arbitrary forcing of it upon others, must, therefore, both be
given up, and a new method of settling opinions must be adopted, which
shall not only produce an impulse to believe, but shall also decide
what proposition it is which is to be believed. Let the action of
natural preferences be unimpeded, then, and under their influence
let men, conversing together and regarding matters in different
lights, gradually develop beliefs in harmony with natural causes. This
method resembles that by which conceptions of art have been brought to
maturity. The most perfect example of it is to be found in the
history of metaphysical philosophy. Systems of this sort have not
usually rested upon any observed facts, at least not in any great
degree. They have been chiefly adopted because their fundamental
propositions seemed ``agreeable to reason.'' This is an apt
expression; it does not mean that which agrees with experience, but
that which we find ourselves inclined to believe. Plato, for example,
finds it agreeable to reason that the distances of the celestial
spheres from one another should be proportional to the different
lengths of strings which produce harmonious chords. Many philosophers
have been led to their main conclusions by considerations like this;
but this is the lowest and least developed form which the method
takes, for it is clear that another man might find Kepler's theory,
that the celestial spheres are proportional to the inscribed and
circumscribed spheres of the different regular solids, more agreeable
to \textit{his} reason. But the shock of opinions will soon lead men
to rest on preferences of a far more universal nature. Take, for
example, the doctrine that man only acts self\-ish\-ly---that is, from
the consideration that acting in one way will afford him more pleasure
than acting in another. This rests on no fact in the world, but it has
had a wide acceptance as being the only reasonable theory.

This method is far more intellectual and respectable from the point of
view of reason than either of the others which we have noticed. But
its failure has been the most manifest. It makes of inquiry \page{11}
something similar to the development of taste; but taste,
unfortunately, is always more or less a matter of fashion, and
accordingly metaphysicians have never come to any fixed agreement, but
the pendulum has swung backward and forward between a more material
and a more spiritual philosophy, from the earliest times to the
latest. And so from this, which has been called the \textit{a priori}
method, we are driven, in Lord Bacon's phrase, to a true induction. We
have examined into this \textit{a priori} method as something which
promised to deliver our opinions from their accidental and capricious
element. But development, while it is a process which eliminates the
effect of some casual circumstances, only magnifies that of others.
This method, therefore, does not differ in a very essential way from
that of authority. The government may not have lifted its finger to
influence my convictions; I may have been left outwardly quite free to
choose, we will say, between monogamy and polygamy, and, appealing to
my conscience only, I may have concluded that the latter practice is
in itself licentious. But when I come to see that the chief obstacle
to the spread of Christianity among a people of as high culture as the
Hindoos has been a conviction of the immorality of our way of treating
women, I cannot help seeing that, though governments do not interfere,
sentiments in their development will be very greatly determined by
accidental causes. Now, there are some people, among whom I must
suppose that my reader is to be found, who, when they see that any
belief of theirs is determined by any circumstance extraneous to the
facts, will from that moment not merely admit in words that that
belief is doubtful, but will experience a real doubt of it, so that it
ceases to be a belief.

To satisfy our doubts, therefore, it is necessary that a method should
be found by which our beliefs may be caused by nothing human, but by
some external per\-ma\-nen\-cy---by something upon which our thinking
has no effect. Some mystics imagine that they have such a method in a
private inspiration from on high. But that is only a form of the
method of tenacity, in which the conception of truth as something
public is not yet developed. Our external permanency would not be
external, in our sense, if it was restricted in its influence to one
individual. It must be something which affects, or might affect, every
man. And, though these affections are necessarily as various as are
individual conditions, yet the method must be such that the ultimate
conclusion of every man shall be the same. Such is the method of
science. Its fundamental hypothesis, restated in more familiar
language, is this: There are real things, whose characters are
entirely independent of our opinions about them; those realities
affect our senses according to regular laws, and, though our
sensations are as different as our relations to the objects, yet, by
taking advantage of the laws of perception, we can ascertain by
reasoning how things really are, and any man, if he have sufficient
experience \page{12} and reason enough about it, will be led to the
one true conclusion. The new conception here involved is that of
reality. It may be asked how I know that there are any realities. If
this hypothesis is the sole support of my method of inquiry, my method
of inquiry must not be used to support my hypothesis. The reply is
this: 1. If investigation cannot be regarded as proving that there are
real things, it at least does not lead to a contrary conclusion; but
the method and the conception on which it is based remain ever in
harmony. No doubts of the method, therefore, necessarily arise from
its practice, as is the case with all the others. 2. The feeling which
gives rise to any method of fixing belief is a dissatisfaction at two
repugnant propositions. But here already is a vague concession that
there is some \textit{one} thing which a proposition should conform.
Nobody, therefore, can really doubt that there are realities, or, if
he did, doubt would not be a source of dissatisfaction. The
hypothesis, therefore, is one which every mind admits. So that the
social impulse does not cause me to doubt it. 3. Everybody uses the
scientific method about a great many things, and only ceases to use it
when he does not know how to apply it. 4. Experience of the method has
not led me to doubt it, but, on the contrary, scientific investigation
has had the most wonderful triumphs in the way of settling opinion.
These afford the explanation of my not doubting the method or the
hypothesis which it supposes; and not having any doubt, nor believing
that anybody else whom I could influence has, it would be the merest
babble for me to say more about it. If there be anybody with a living
doubt upon the subject, let him consider it.

To describe the method of scientific investigation is the object of
this series of papers. At present I have only room to notice some
points of contrast between it and other methods of fixing belief.

This is the only one of the four methods which presents any
distinction of a right and a wrong way. If I adopt the method of
tenacity and shut myself out from all influences, whatever I think
necessary to doing this is necessary according to that method. So with
the method of authority: the state may try to put down heresy by means
which, from a scientific point of view, seem very ill-calculated to
accomplish its purposes; but the only test \textit{on that method} is
what the state thinks, so that it cannot pursue the method wrongly. So
with the \textit{a priori} method. The very essence of it is to think
as one is inclined to think. All metaphysicians will be sure to do
that, however they may be inclined to judge each other to be
perversely wrong. The Hegelian system recognizes every natural
tendency of thought as logical, although it be certain to be abolished
by counter-tendencies. Hegel thinks there is a regular system in the
succession of these tendencies, in consequence of which, after
drifting one way and the other for a long time, opinion will at last
go right. \page{13} And it is true that metaphysicians do get the
right ideas at last; Hegel's system of Nature represents tolerably the
science of that day; and one may be sure that whatever scientific
investigation shall have put out of doubt will presently receive
\textit{a priori} demonstration on the part of the metaphysicians. But
with the scientific method the case is different. I may start with
known and observed facts to proceed to the unknown; and yet the rules
which I follow in doing so may not be such as investigation would
approve. The test of whether I am truly following the method is not
an immediate appeal to my feelings and purposes, but, on the contrary,
itself involves the application of the method. Hence it is that bad
reasoning as well as good reasoning is possible; and this fact is the
foundation of the practical side of logic.

It is not to be supposed that the first three methods of settling
opinion present no advantage whatever over the scientific method. On
the contrary, each has some peculiar convenience of its own. The \textit{a
priori} method is distinguished for its comfortable conclusions. It
is the nature of the process to adopt whatever belief we are inclined
to, and there are certain flatteries to the vanity of man which we all
believe by nature, until we are awakened from our pleasing dream by
some rough facts. The method of authority will always govern the mass
of mankind; and those who wield the various forms of organized force
in the state will never be convinced that dangerous reasoning ought
not to be suppressed in some way. If liberty of speech is to be
untrammeled from the grosser forms of constraint, then uniformity of
opinion will be secured by a moral terrorism to which the
respectability of society will give its thorough approval. Following
the method of authority is the path of peace. Certain non-conformities
are permitted; certain others (considered unsafe) are forbidden. These
are different in different countries and in different ages; but,
wherever you are, let it be known that you seriously hold a tabooed
belief, and you may be perfectly sure of being treated with a cruelty
less brutal but more refined than hunting you like a wolf. Thus, the
greatest intellectual benefactors of mankind have never dared, and
dare not now, to utter the whole of their thought; and thus a shade of
\textit{prima facie} doubt is cast upon every proposition which is
considered essential to the security of society. Singularly enough,
the persecution does not all come from without; but a man torments
himself and is oftentimes most distressed at finding himself believing
propositions which he has been brought up to regard with aversion. The
peaceful and sympathetic man will, therefore, find it hard to resist
the temptation to submit his opinions to authority. But most of all I
admire the method of tenacity for its strength, simplicity, and
directness. Men who pursue it are distinguished for their decision of
character, which becomes very easy with such a mental rule. They do
not waste time in trying to make up their minds what they want,
\page{14} but, fastening like lightning upon whatever alternative
comes first, they hold to it to the end, whatever happens, without an
instant's irresolution. This is one of the splendid qualities which
generally accompany brilliant, unlasting success. It is impossible not
to envy the man who can dismiss reason, although we know how it must
turn out at last.

Such are the advantages which the other methods of settling opinion
have over scientific investigation. A man should consider well of
them; and then he should consider that, after all, he wishes his
opinions to coincide with the fact, and that there is no reason why
the results of those three first methods should do so. To bring about
this effect is the prerogative of the method of science. Upon such
considerations he has to make his choice---a choice which is far more
than the adoption of any intellectual opinion, which is one of the
ruling decisions of his life, to which, when once made, he is bound to
adhere. The force of habit will sometimes cause a man to hold on to
old beliefs, after he is in a condition to see that they have no sound
basis. But reflection upon the state of the case will overcome these
habits, and he ought to allow reflection its full weight. People
sometimes shrink from doing this, having an idea that beliefs are
wholesome which they cannot help feeling rest on nothing. But let such
persons suppose an analogous though different case from their own. Let
them ask themselves what they would say to a reformed Mussulman who
should hesitate to give up his old notions in regard to the relations
of the sexes; or to a reformed Catholic who should still shrink from
reading the Bible. Would they not say that these persons ought to
consider the matter fully, and clearly understand the new doctrine,
and then ought to embrace it, in its entirety? But, above all, let it
be considered that what is more wholesome than any particular belief
is integrity of belief, and that to avoid looking into the support of
any belief from a fear that it may turn out rotten is quite as immoral
as it is disadvantageous. The person who confesses that there is such
a thing as truth, which is distinguished from falsehood simply by
this, that if acted on it will carry us to the point we aim at and not
astray, and then, though convinced of this, dares not know the truth
and seeks to avoid it, is in a sorry state of mind indeed.

Yes, the other methods do have their merits: a clear logical
conscience does cost some\-thing---just as any virtue, just as all
that we cherish, costs us dear. But we should not desire it to be
otherwise. The genius of a man's logical method should be loved and
reverenced as his bride, whom he has chosen from all the world. He
need not contemn the others; on the contrary, he may honor them
deeply, and in doing so he only honors her the more. But she is the
one that he has chosen, and he knows that he was right in making that
choice. And having made it, he will work and fight for her, and will
not com-\page{15}plain that there are blows to take, hoping that there
may be as many and as hard to give, and will strive to be the worthy
knight and champion of her from the blaze of whose splendors he draws
his inspiration and his courage.

