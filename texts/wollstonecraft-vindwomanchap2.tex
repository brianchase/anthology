
\author{Mary Wollstonecraft}
\authdate{1759--1797}
\textdate{1792}
\addon{Chapter 2}
\chapter[A Vindication of the Rights of Woman, chap. 2]{A Vindication
of the Rights of Woman}
\source{wollstonecraft1796.2}

%\page{32}\section*{The Prevailing Opinion of a Sexual Character
%Discussed}

\page{32}To account for, and excuse the tyranny of man, many ingenious
arguments have been brought forward to prove, that the two sexes, in
the acquirement of virtue, ought to aim at attaining a very different
character; or, to speak explicitly, women are not allowed to have
sufficient strength of mind to acquire what really deserves the name
of virtue. Yet it should seem, allowing them to have souls, that there
is but one way appointed by Providence to lead \textit{mankind} to
either virtue or happiness.

If then women are not a swarm of ephemeron triflers, why should they
be kept in ignorance under the specious name of innocence? Men
complain, and with reason, of the follies and caprices of our sex,
when they do not keenly satirize our headstrong passions and groveling
vices. \page{33} ---Be\-hold, I should answer, the natural effect of
ignorance! The mind will ever be unstable that has only prejudices to
rest on, and the current will run with destructive fury when there are
no barriers to break its force. Women are told from their infancy, and
taught by the example of their mothers, that a little knowledge of
human weakness, justly termed cunning, softness of temper,
\textit{outward} obedience, and a scrupulous attention to a puerile
kind of propriety, will obtain for them the protection of man; and
should they be beautiful, every thing else is needless, for, at least
twenty years of their lives.

Thus Milton describes our first frail mother; though when he tells us
that women are formed for softness and sweet attractive grace, I
cannot comprehend his meaning, unless, in the true Mahometan strain,
he meant to deprive us of souls, and insinuate that we were beings
only designed by sweet attractive grace, and docile blind obedience,
to gratify the senses of man when he can no longer soar on the wing of
contemplation.

How grossly do they insult us who thus advise us only to render
ourselves gentle, domestic \page{34} brutes! For instance, the winning
softness so warmly, and frequently, recommended, that governs by
obeying. What childish expressions, and how insignificant is the
be\-ing---can it be an immortal one? who will condescend to govern by
such sinister methods! `Certainly,' says Lord Bacon, `man is of kin to
the beasts by his body; and if he be not of kin to God by his spirit,
he is a base and ignoble creature!' Men, indeed, appear to me to act
in a very unphilosophical manner when they try to secure the good
conduct of women by attempting to keep them always in a state of
childhood. Rousseau was more consistent when he wished to stop the
progress of reason in both sexes, for if men eat of the tree of
knowledge, women will come in for a taste; but, from the imperfect
cultivation which their understandings now receive, they only attain a
knowledge of evil.

Children, I grant, should be innocent; but when the epithet is applied
to men, or women, it is but a civil term for weakness. For if it be
allowed that women were destined by Providence to acquire human
virtues, and by the exercise of their understandings, that stability
of character which is the firmest \page{35} ground to rest our future
hopes upon, they must be permitted to turn to the fountain of light,
and not forced to shape their course by the twinkling of a mere
satellite. Milton, I grant, was of a very different opinion; for he
only bends to the indefeasible right of beauty, though it would be
difficult to render two passages which I now mean to contrast,
consistent. But into similar inconsistencies are great men often led
by their senses.

\begin{verse}
`To whom thus Eve with \textit{perfect beauty} adorn'd.\\
My Author and Disposer, what thou bidst\\
\textit{Unargued} I obey; so God ordains;\\
God is \textit{thy law, thou mine}: to know no more\\
Is Woman's \textit{happiest} knowledge and her \textit{praise}.'
\end{verse}

These are exactly the arguments that I have used to children; but I
have added, your reason is now gaining strength, and, till it arrives
at some degree of maturity, you must look up to me for ad\-vice---then
you ought to \textit{think}, and only rely on God.

Yet in the following lines Milton seems to coincide with me; when he
makes Adam thus expostulate with his Maker.

\begin{verse}
`Hast thou not made me here thy substitute,\\
And these inferior far beneath me set?\\
\page{36}Among \textit{unequals} what society\\
Can sort, what harmony or delight?\\
Which must be mutual, in proportion due\\
Giv'n and receiv'd; but in \textit{disparity}\\
The one intense, the other still remiss\\
Cannot well suit with either, but soon prove\\
Tedious alike: of \textit{fellowship} I speak\\
Such as I seek, fit to participate\\
All rational delight---'
\end{verse}

In treating, therefore, of the manners of women, let us, disregarding
sensual arguments, trace what we should endeavour to make them in
order to co-operate, if the expression be not too bold, with the
supreme Being.

By individual education, I mean, for the sense of the word is not
precisely defined, such an attention to a child as will slowly sharpen
the senses, form the temper, regulate the passions as they begin to
ferment, and set the understanding to work before the body arrives at
maturity; so that the man may only have to proceed, not to begin, the
important task of learning to think and reason.

To prevent any misconstruction, I must add, that I do not believe that
a private education can work the wonders which some sanguine writers
have attributed to it. Men and women must be educated, in a great
degree, by the opinions and manners of the \page{37} society they live
in. In every age there has been a stream of popular opinion that has
carried all before it, and given a family character, as it were, to
the century. It may then fairly be inferred, that, till society be
differently constituted, much cannot be expected from education. It
is, however, sufficient for my present purpose to assert, that,
whatever effect circumstances have on the abilities, every being may
become virtuous by the exercise of its own reason; for if but one
being was created with vicious inclinations, that is positively bad,
what can save us from atheism? or if we worship a God, is not that God
a devil?

Consequently, the most perfect education, in my opinion, is such an
exercise of the understanding as is best calculated to strengthen the
body and form the heart. Or, in other words, to enable the individual
to attain such habits of virtue as will render it independent. In
fact, it is a farce to call any being virtuous whose virtues do not
result from the exercise of its own reason. This was Rousseau's
opinion respecting men: I extend it to women, and confidently assert
that they have been drawn out of their sphere by false refinement, and
not by an endeavour to \page{38} acquire masculine qualities. Still
the regal homage which they receive is so intoxicating, that till the
manners of the times are changed, and formed on more reasonable
principles, it may be impossible to convince them that the
illegitimate power, which they obtain, by degrading themselves, is a
curse, and that they must return to nature and equality, if they wish
to secure the placid satisfaction that unsophisticated affections
impart. But for this epoch we must wait---wait, perhaps, till kings
and nobles, enlightened by reason, and, preferring the real dignity of
man to childish state, throw off their gaudy hereditary trappings: and
if then women do not resign the arbitrary power of beauty---they will
prove that they have \textit{less} mind than man.

I may be accused of arrogance; still I must declare what I firmly
believe, that all the writers who have written on the subject of
female education and manners, from Rousseau to Dr. Gregory, have
contributed to render women more artificial, weak characters, than
they would otherwise have been; and consequently, more useless members
of society. I might have expressed this conviction in a lower key; but
I am afraid it would have \page{39} been the whine of affectation, and
not the faithful expression of my feelings, of the clear result which
experience and reflection have led me to draw. When I come to that
division of the subject, I shall advert to the passages that I more
particularly disapprove of, in the works of the authors I have just
alluded to; but it is first necessary to observe, that my objection
extends to the whole purport of those books, which tend, in my
opinion, to degrade one half of the human species, and render women
pleasing at the expense of every solid virtue.

Though, to reason on Rousseau's ground, if man did attain a degree of
perfection of mind when his body arrived at maturity, it might be
proper, in order to make a man and his wife \textit{one}, that she
should rely entirely on his understanding; and the graceful ivy,
clasping the oak that supported it, would form a whole in which
strength and beauty would be equally conspicuous. But, alas! husbands,
as well as their helpmates, are often only overgrown children; nay,
thanks to early debauchery, scarcely men in their outward form---and
if the blind lead the blind, one need not come from heaven to tell us
the consequence.

\page{40}Many are the causes that, in the present corrupt state of
society, contribute to enslave women by cramping their understandings
and sharpening their senses. One, perhaps, that silently does more
mischief than all the rest, is their disregard of order.

To do every thing in an orderly manner, is a most important precept,
which women, who, generally speaking, receive only a disorderly kind
of education, seldom attend to with that degree of exactness that men,
who from their infancy are broken into method, observe. This negligent
kind of guess-work, for what other epithet can be used to point out
the random exertions of a sort of instinctive common sense, never
brought to the test of reason? prevents their generalizing matters of
fact---so they do to-day, what they did yesterday, merely because they
did it yesterday.

This contempt of the understanding in early life has more baneful
consequences than is commonly supposed; for the little knowledge which
women of strong minds attain, is, from various circumstances, of a
more desultory kind than the knowledge of men, and it is acquired more
by sheer observations on real \page{41} life, than from comparing what
has been individually observed with the results of experience
generalized by speculation. Led by their dependent situation and
domestic employments more into society, what they learn is rather by
snatches; and as learning is with them, in general, only a secondary
thing, they do not pursue any one branch with that persevering ardour
necessary to give vigour to the faculties, and clearness to the
judgment. In the present state of society, a little learning is
required to support the character of a gentleman; and boys are obliged
to submit to a few years of discipline. But in the education of women,
the cultivation of the understanding is always subordinate to the
acquirement of some corporeal accomplishment; even while enervated by
confinement and false notions of modesty, the body is prevented from
attaining that grace and beauty which relaxed half-formed limbs never
exhibit. Besides, in youth their faculties are not brought forward by
emulation; and having no serious scientific study, if they have
natural sagacity it is turned too soon on life and manners. They dwell
on effects, and modifications, without tracing them back to causes;
and compli-\page{42}cated rules to adjust behaviour are a weak
substitute for simple principles.

As a proof that education gives this appearance of weakness to
females, we may instance the example of military men, who are, like
them, sent into the world before their minds have been stored with
knowledge or fortified by principles. The consequences are similar;
soldiers acquire a little superficial knowledge, snatched from the
muddy current of conversation, and, from continually mixing with
society, they gain, what is termed a knowledge of the world; and this
acquaintance with manners and customs has frequently been confounded
with a knowledge of the human heart. But can the crude fruit of casual
observation, never brought to the test of judgment, formed by
comparing speculation and experience, deserve such a distinction?
Soldiers, as well as women, practice the minor virtues with
punctilious politeness. Where is then the sexual difference, when the
education has been the same? All the difference that I can discern,
arises from the superior advantage of liberty, which enables the
former to see more of life.

It is wandering from my present subject, \page{43} perhaps, to make a
political remark; but, as it was produced naturally by the train of my
reflections, I shall not pass it silently over.

Standing armies can never consist of resolute robust men; they may be
well disciplined machines, but they will seldom contain men under the
influence of strong passions, or with very vigorous faculties. And as
for any depth of understanding, I will venture to affirm, that it is
as rarely to be found in the army as amongst women; and the cause, I
maintain, is the same. It may be further observed, that officers are
also particularly attentive to their persons, fond of dancing, crowded
rooms, adventures, and ridicule.\footnote{Why should women be censured
wit petulant acrimony, because they seem to have a passion for a
scarlet coat? Has not education placed them more on a level with
soldiers than any other class of men?} Like the \textit{fair} sex, the
business of their lives is gallantry.---They were taught to please,
and they only live to please. Yet they do not lose their rank in the
distinction of sexes, for they are still reckoned superior to women,
though in what their superiority consists, beyond what I have just
mentioned, it is difficult to discover.

The great misfortune is this, that they \page{44} both acquire manners
before morals, and a knowledge of life before they have, from
reflection, any acquaintance with the grand ideal outline of human
nature. The consequence is natural; satisfied with common nature, they
become a prey to prejudices, and taking all their opinions on credit,
they blindly submit to authority. So that, if they have any sense, it
is a kind of instinctive glance, that catches proportions, and decides
with respect to manners; but fails when arguments are to be pursued
below the surface, or opinions analyzed.

May not the same remark be applied to women? Nay, the argument may be
carried still further, for they are both thrown out of a useful
station by the unnatural distinctions established in civilized life.
Riches and hereditary honours have made cyphers of women to give
consequence to the numerical figure; and idleness has produced a
mixture of gallantry and despotism in society, which leads the very
men who are the slaves of their mistresses to tyrannize over their
sisters, wives, and daughters. This is only keeping them in rank and
file, it is true. Strengthen the female mind by enlarging it, and
there will be an end to blind obedience; but, as blind obedience is
ever \page{45} sought for by power, tyrants and sensualists are in the
right when they endeavour to keep women in the dark, because the
former only want slaves, and the latter a play-thing. The
sensualist, indeed, has been the most dangerous of tyrants, and women
have been duped by their lovers, as princes by their ministers, whilst
dreaming that they reigned over them.

I now principally allude to Rousseau, for his character of Sophia is,
undoubtedly, a captivating one, though it appears to me grossly
unnatural; however, it is not the superstructure, but the foundation
of her character, the principles on which her education was built,
that I mean to attack; nay, warmly as I admire the genius of that able
writer, whose opinions I shall often have occasion to cite,
indignation always takes place of admiration, and the rigid frown of
insulted virtue effaces the smile of complacency, which his eloquent
periods are wont to raise, when I read his voluptuous reveries. Is
this the man, who, in his ardour for virtue, would banish all the soft
arts of peace, and almost carry us back to Spartan discipline? Is this
the man who delights to paint the useful struggles of passion, the
triumphs \page{46} of good dispositions, and the heroic flights which
carry the glowing soul out of itself?---How are these mighty
sentiments lowered when he describes the pretty foot and enticing airs
of his little favourite! But, for the present, I waive the subject,
and, instead of severely reprehending the transient effusions of
overweening sensibility, I shall only observe, that whoever has cast
a benevolent eye on society, must often have been gratified by the
sight of humble mutual love, not dignified by sentiment, or
strengthened by a union in intellectual pursuits. The domestic trifles
of the day have afforded matters for cheerful converse, and innocent
caresses have softened toils which did not require great exercise of
mind or stretch of thought: yet, has not the sight of this moderate
felicity excited more tenderness than respect? An emotion similar to
what we feel when children are playing, or animals
sporting\footnote{Similar feelings has Milton's pleasing picture of
paradisiacal happiness ever raised in my mind; yet, instead of envying
the lovely pair, I have, with conscious dignity, or Satanic pride,
turned to hell for sublimer objects. In the same style, when viewing
some noble monument of human art, I have traced the emanation of the
Deity in the order I admired, till, descending from that giddy height,
I have caught myself contemplating the gradest of all human
sights;---for fancy quickly placed, in some solitary recess, an
outcast of fortune, rising superior to passion and discontent.},
whilst the contempla-\page{47}tion of the noble struggles of suffering
merit has raised admiration, and carried our thoughts to that world
where sensation will give place to reason.

Women are, therefore, to be considered either as moral beings, or so
weak that they must be entirely subjected to the superior faculties of
men.

Let us examine this question. Rousseau declares that a woman should
never, for a moment, feel herself independent, that she should be
governed by fear to exercise her \textit{natural} cunning, and made a
coquetish slave in order to render her a more alluring object of
desire, a \textit{sweeter} companion to man, whenever he chooses to
relax himself. He carries the arguments, which he pretends to draw
from the indications of nature, still further, and insinuates that
truth and fortitude, the corner stones of all human virtue, should be
cultivated with certain restrictions, because, with respect to the
female character, obedience is the grand lesson which ought to be
impressed with unrelenting rigour.

% NOTE: 'when', not 'When', is correct below

What nonsense! when will a great man arise with sufficient strength of
mind to puff \page{48} away the fumes which pride and sensuality have
thus spread over the subject! If women are by nature inferior to men,
their virtues must be the same in quality, if not in degree, or
virtue is a relative idea; consequently, their conduct should be
founded on the same principles, and have the same aim.

Connected with man as daughters, wives, and mothers, their moral
character may be estimated by their manner of fulfilling those simple
duties; but the end, the grand end of their exertions should be to
unfold their own faculties and acquire the dignity of conscious
virtue. They may try to render their road pleasant; but ought never to
forget, in common with man, that life yields not the felicity which
can satisfy an immortal soul. I do not mean to insinuate that either
sex should be so lost in abstract reflections or distant views, as to
forget the affections and duties that lie before them, and are, in
truth, the means appointed to produce the fruit of life: on the
contrary, I would warmly recommend them, even while I assert, that
they afford most satisfaction when they are considered in their true,
sober light.

Probably the prevailing opinion, that woman was created for man, may
have taken \page{49} its rise from Moses's poetical story; yet, as
very few, it is presumed, who have bestowed any serious thought on the
subject, ever supposed that Eve was, literally speaking, one of Adam's
ribs, the deduction must be allowed to fall to the ground; or, only be
so far admitted as it proves that man, from the remotest antiquity,
found it convenient to exert his strength to subjugate his companion,
and his invention to shew that she ought to have her neck bent under
the yoke, because the whole creation was only created for his
convenience or pleasure.

Let it not be concluded that I wish to invert the order of things; I
have already granted, that, from the constitution of their bodies, men
seem to be designed by Providence to attain a greater degree of
virtue. I speak collectively of the whole sex; but I see not the
shadow of a reason to conclude that their virtues should differ in
respect to their nature. In fact, how can they, if virtue has only one
eternal standard? I must therefore, if I reason consequentially, as
strenuously maintain that they have the same simple direction, as that
there is a God.

\page{50}It follows then that cunning should not be opposed to wisdom,
little cares to great exertions, nor insipid softness, varnished over
with the name of gentleness, to that fortitude which grand views alone
can inspire.

I shall be told that woman would then lose many of her peculiar
graces, and the opinion of a well known poet might be quoted to refute
my unqualified assertions. For Pope has said, in the name of the whole
male sex,

\begin{verse}
`Yet ne'er so sure our passion to create,\\
As when she touch'd the brink of all we hate.'
\end{verse}

In what light this sally places men and women, I shall leave to the
judicious to determine; meanwhile I shall content myself with
observing, that I cannot discover why, unless they are mortal, females
should always be degraded by being made subservient to love or lust.

To speak disrespectfully of love is, I know, high treason against
sentiment and fine feelings; but I wish to speak the simple language
of truth, and rather to address the head than the heart. To endeavour
to reason love out of the world, would be to out Quixote
Cer-\page{51}vantes, and equally offend against common sense; but an
endeavour to restrain this tumultuous passion, and to prove that it
should not be allowed to dethrone superior powers, or to usurp the
sceptre which the understanding should ever coolly wield, appears less
wild.

Youth is the season for love in both sexes; but in those days of
thoughtless enjoyment provision should be made for the more important
years of life, when reflection takes place of sensation. But Rousseau,
and most of the male writers who have followed his steps, have warmly
inculcated that the whole tendency of female education ought to be
directed to one point:---to render them pleasing.

Let me reason with the supporters of this opinion who have any
knowledge of human nature, do they imagine that marriage can eradicate
the habitude of life? The woman who has only been taught to please
will soon find that her charms are oblique sunbeams, and that they
cannot have much effect on her husband's heart when they are seen
every day, when the summer is past and gone. Will she then have
sufficient native energy to look into herself for comfort, and
cultivate \page{52} her dormant faculties? or, is it not more rational
to expect that she will try to please other men; and, in the emotions
raised by the expectation of new conquests, endeavour to forget the
mortification her love or pride has received? When the husband ceases
to be a lover---and the time will inevitably come, her desire of
pleasing will then grow languid, or become a spring of bitterness; and
love, perhaps, the most evanescent of all passions, gives place to
jealousy or vanity.

% NOTE: 'it' after 'study?' is correct

I now speak of women who are restrained by principle or prejudice;
such women, though they would shrink from an intrigue with real
abhorrence, yet, nevertheless, wish to be convinced by the homage of
gallantry that they are cruelly neglected by their husbands; or, days
and weeks are spent in dreaming of the happiness enjoyed by congenial
souls till the health is undermined and the spirits broken by
discontent. How then can the great art of pleasing be such a necessary
study? it is only useful to a mistress; the chaste wife, and serious
mother, should only consider her power to please as the polish of her
virtues, and the affection of her husband as one of the comforts that
render her task less difficult and \page{53} her life happier.---But,
whether she be loved or neglected, her first wish should be to make
herself respectable, and not to rely for all her happiness on a being
subject to like infirmities with herself.

The amiable Dr. Gregory fell into a similar error. I respect his
heart; but entirely disapprove of his celebrated Legacy to his
Daughters.

He advises them to cultivate a fondness for dress, because a fondness
for dress, he asserts, is natural to them. I am unable to comprehend
what either he or Rousseau mean, when they frequently use this
indefinite term. If they told us that in a pre-existent state the
soul was fond of dress, and brought this inclination with it into a
new body, I should listen to them with a half smile, as I often do
when I hear a rant about innate elegance.---But if he only meant to
say that the exercise of the faculties will produce this fondness---I
deny it.---It is not natural; but arises, like false ambition in men,
from a love of power.

Dr. Gregory goes much further; he actually recommends dissimulation,
and advises an innocent girl to give the lie to her feelings, and not
dance with spirit, when gaiety of \page{54} heart would make her feet
eloquent without making her gestures immodest. In the name of truth
and common sense, why should not one woman acknowledge that she can
take more exercise than another? or, in other words, that she has a
sound constitution; and why, to damp innocent vivacity, is she darkly
to be told that men will draw conclusions which she little thinks
of?---Let the libertine draw what inference he pleases; but, I hope,
that no sensible mother will restrain the natural frankness of youth
by instilling such indecent cautions. Out of the abundance of the
heart the mouth speaketh; and a wiser than Solomon hath said, that the
heart should be made clean, and not trivial ceremonies observed, which
it is not very difficult to fulfil with scrupulous exactness when vice
reigns in the heart.

Women ought to endeavour to purify their heart; but can they do so
when their uncultivated understandings make them entirely dependent on
their senses for employment and amusement, when no noble pursuit sets
them above the little vanities of the day, or enables them to curb the
wild emotions that agitate a reed over which every passing breeze has
\page{55} power? To gain the affections of a virtuous man, is
affectation necessary? Nature has given woman a weaker frame than man;
but, to ensure her husband's affections, must a wife, who by the
exercise of her mind and body whilst she was discharging the duties of
a daughter, wife, and mother, has allowed her constitution to retain
its natural strength, and her nerves a healthy tone, is she, I say, to
condescend to use art and feign a sickly delicacy in order to secure
her husband's affection? Weakness may excite tenderness, and gratify
the arrogant pride of man; but the lordly caresses of a protector will
not gratify a noble mind that pants for, and deserves to be respected.
Fondness is a poor substitute for friendship!

In a seraglio, I grant, that all these arts are necessary; the epicure
must have his palate tickled, or he will sink into apathy; but have
women so little ambition as to be satisfied with such a condition? Can
they supinely dream life away in the lap of pleasure, or the languor
of weariness, rather than assert their claim to pursue reasonable
pleasures and render themselves conspicuous by practising the virtues
which dignify mankind? Surely she has not an immortal soul who can
loiter life away \page{56} merely employed to adorn her person, that
she may amuse the languid hours, and soften the cares of a
fellow-creature who is willing to be enlivened by her smiles and
tricks, when the serious business of life is over.

Besides, the woman who strengthens her body and exercises her mind
will, by managing her family and practising various virtues, become
the friend, and not the humble dependent of her husband; and if she,
by possessing such substantial qualities, merit his regard, she will
not find it necessary to conceal her affection, nor to pretend to an
unnatural coldness of constitution to excite her husband's passions.
In fact, if we revert to history, we shall find that the women who
have distinguished themselves have neither been the most beautiful nor
the most gentle of their sex.

Nature, or to speak with strict propriety God, has made all things
right; but man has sought him out many inventions to mar the work. I
now allude to that part of Dr. Gregory's treatise, where he advises a
wife never to let her husband know the extent of her sensibility or
affection. Voluptuous precaution, and as ineffectual as
absurd.---Love, from its very nature, must be transitory. To \page{57}
seek for a secret that would render it constant, would be as wild a
search as for the philosopher's stone, or the grand panacea: and the
discovery would be equally useless, or rather pernicious, to mankind.
The most holy band of society is friendship. It has been well said, by
a shrewd satirist, ``that rare as true love is, true friendship is
still rarer.''

This is an obvious truth, and the cause not lying deep, will not elude
a slight glance of inquiry.

Love, the common passion, in which chance and sensation take place of
choice and reason, is, in some degree, felt by the mass of mankind;
for it is not necessary to speak, at present, of the emotions that
rise above or sink below love. This passion, naturally increased by
suspense and difficulties, draws the mind out of its accustomed state,
and exalts the affections; but the security of marriage, allowing the
fever of love to subside, a healthy temperature is thought insipid,
only by those who have not sufficient intellect to substitute the calm
tenderness of friendship, the confidence of respect, instead of blind
admiration, and the sensual emotions of fondness.

\page{58}This is, must be, the course of nature---friendship or
indifference inevitably succeeds love.---And this constitution seems
perfectly to harmonize with the system of government which prevails in
the moral world. Passions are spurs to action, and open the mind; but
they sink into mere appetites, become a personal momentary
gratification, when the object is gained, and the satisfied mind rests
in enjoyment. The man who had some virtue whilst he was struggling for
a crown, often becomes a voluptuous tyrant when it graces his brow;
and, when the lover is not lost in the husband, the dotard, a prey to
childish caprices, and fond jealousies, neglects the serious duties of
life, and the caresses which should excite confidence in his children
are lavished on the overgrown child, his wife.

In order to fulfil the duties of life, and to be able to pursue with
vigour the various employments which form the moral character, a
master and mistress of a family ought not to continue to love each
other with passion. I mean to say, that they ought not to indulge
those emotions which disturb the order of society, and engross the
thoughts that should \page{59} be otherwise employed. The mind that
has never been engrossed by one object wants vigour---if it can long
be so, it is weak.

A mistaken education, a narrow, uncultivated mind, and many sexual
prejudices, tend to make women more constant than men; but, for the
present, I shall not touch on this branch of the subject. I will go
still further, and advance, without dreaming of a paradox, that an
unhappy marriage is often very advantageous to a family, and that the
neglected wife is, in general, the best mother. And this would almost
always be the consequence if the female mind was more enlarged: for,
it seems to be the common dispensation of Providence, that what we
gain in present enjoyment should be deducted from the treasure of
life, experience; and that when we are gathering the flowers of the
day and revelling in pleasure, the solid fruit of toil and wisdom
should not be caught at the same time. The way lies before us, we must
turn to the right or left; and he who will pass life away in bounding
from one pleasure to another, must not complain if he acquire neither
wisdom nor respectability of character.

\page{60}Supposing, for a moment, that the soul is not immortal, and
that man was only created for the present scene,---I think we should
have reason to complain that love, infantine fondness, ever grew
insipid and palled upon the sense. Let us eat, drink, and love, for
to-morrow we die, would be, in fact, the language of reason, the
morality of life; and who but a fool would part with a reality for a
fleeting shadow? But, if awed by observing the improvable powers of
the mind, we disdain to confine our wishes or thoughts to such a
comparatively mean field of action; that only appears grand and
important, as it is connected with a boundless prospect and sublime
hopes, what necessity is there for falsehood in conduct, and why must
the sacred majesty of truth be violated to detain a deceitful good
that saps the very foundation of virtue? Why must the female mind be
tainted by coquetish arts to gratify the sensualist, and prevent love
from subsiding into friendship, or compassionate tenderness, when
there are not qualities on which friendship can be built? Let the
honest heart show itself, and \textit{reason} teach passion to submit
to necessity; or, let \page{61} the dignified pursuit of virtue and
knowledge raise the mind above those emotions which rather imbitter
than sweeten the cup of life, when they are not restrained within due
bounds.

I do not mean to allude to the romantic passion, which is the
concomitant of genius.---Who can clip its wing? But that grand passion
not proportioned to the puny enjoyments of life, is only true to the
sentiment, and feeds on itself. The passions which have been
celebrated for their durability have always been unfortunate. They
have acquired strength by absence and constitutional melancholy.---The
fancy has hovered round a form of beauty dimly seen---but familiarity
might have turned admiration into disgust; or, at least, into
indifference, and allowed the imagination leisure to start fresh
game. With perfect propriety, according to this view of things, does
Rousseau make the mistress of his soul, Eloisa, love St. Preux, when
life was fading before her; but this is no proof of the immortality of
the passion.

Of the same complexion is Dr. Gregory's advice respecting delicacy of
sentiment, which he advises a woman not to acquire, if she has
determined to marry. This determination, \page{62} however, perfectly
consistent with his former advice, he calls \textit{indelicate}, and
earnestly persuades his daughters to conceal it, though it may govern
their conduct:---as if it were indelicate to have the common appetites
of human nature.

Noble morality! and consistent with the cautious prudence of a little
soul that cannot extend its views beyond the present minute division
of existence. If all the faculties of woman's mind are only to be
cultivated as they respect her dependence on man; if, when a husband
be obtained, she have arrived at her goal, and meanly proud rests
satisfied with such a paltry crown, let her grovel contentedly,
scarcely raised by her employments above the animal kingdom; but, if,
struggling for the prize of her high calling, she look beyond the
present scene, let her cultivate her understanding without stopping to
consider what character the husband may have whom she is destined to
marry. Let her only determine, without being too anxious about present
happiness, to acquire the qualities that ennoble a rational being, and
a rough inelegant husband may shock her taste without destroying her
peace of mind. She will not model her soul to suit \page{63} the
frailties of her companion, but to bear with them: his character may
be a trial, but not an impediment to virtue.

If Dr. Gregory confined his remark to romantic expectations of
constant love and congenial feelings, he should have recollected that
experience will banish what advice can never make us cease to wish
for, when the imagination is kept alive at the expence of reason.

I own it frequently happens that women who have fostered a romantic
unnatural delicacy of feeling, waste their\footnote{For example, the
herd of Novelists.} lives in \textit{imagining} how happy they should
have been with a husband who could love them with a fervid
increasing affection every day, and all day. But they might as well
pine married as single---and would not be a jot more unhappy with a
bad husband than longing for a good one. That a proper education; or,
to speak with more precision, a well stored mind, would enable a woman
to support a single life with dignity, I grant; but that she should
avoid cultivating her taste, lest her husband should occasionally
shock it, is quitting a substance for a shadow. To say the truth, I do
not know of what use is an improved taste, if \page{64} the individual
be not rendered more independent of the casualties of life; if new
sources of enjoyment, only dependent on the solitary operations of the
mind, are not opened. People of taste, married or single, without
distinction, will ever be disgusted by various things that touch not
less observing minds. On this conclusion the argument must not be
allowed to hinge; but in the whole sum of enjoyment is taste to be
denominated a blessing?

The question is, whether it procures most pain or pleasure? The answer
will decide the propriety of Dr. Gregory's advice, and shew how absurd
and tyrannic it is thus to lay down a system of slavery; or to attempt
to educate moral beings by any other rules than those deduced from
pure reason, which apply to the whole species.

Gentleness of manners, forbearance and long-suffering, are such
amiable Godlike qualities, that in sublime poetic strains the Deity
has been invested with them; and, perhaps, no representation of his
goodness so strongly fastens on the human affections as those that
represent him abundant in mercy and willing to pardon. Gentleness,
con-\page{65}sidered in this point of view, bears on its front all the
characteristics of grandeur, combined with the winning graces of
condescension; but what a different aspect it assumes when it is the
submissive demeanour of dependence, the support of weakness that
loves, because it wants protection; and is forbearing, because it must
silently endure injuries; smiling under the lash at which it dare not
snarl. Abject as this picture appears, it is the portrait of an
accomplished woman, according to the received opinion of female
excellence, separated by specious reasoners from human excellence. Or,
they\footnote{Vide Rousseau, and Swedenborg.} kindly restore the rib,
and make one moral being of a man and woman; not forgetting to give
her all the `submissive charms.'

How women are to exist in that state where there is to be neither
marrying nor giving in marriage, we are not told. For though moralists
have agreed that the tenor of life seems to prove that \textit{man} is
prepared by various circumstances for a future state, they constantly
concur in advising \textit{woman} only to provide for the present.
Gentleness, docility, and a spaniel-like affection are, on this
ground, consistently recommended as the cardinal virtues \page{66} of
the sex; and, disregarding the arbitrary economy of nature, one writer
has declared that it is masculine for a woman to be melancholy. She
was created to be the toy of man, his rattle, and it must jingle in
his ears, whenever, dismissing reason, he chooses to be amused.

% NOTE: 'individuals' is correct below

To recommend gentleness, indeed, on a broad basis is strictly
philosophical. A frail being should labour to be gentle. But when
forbearance confounds right and wrong, it ceases to be a virtue; and,
however convenient it may be found in a com\-pan\-ion---that companion
will ever be considered as an inferior, and only inspire a vapid
tenderness, which easily degenerates into contempt. Still, if advice
could really make a being gentle, whose natural disposition admitted
not of such a fine polish, something towards the advancement of order
would be attained; but if, as might quickly be demonstrated, only
affectation be produced by this indiscriminate counsel, which throws a
stumbling-block in the way of gradual improvement, and true
melioration of temper, the sex is not much benefited by sacrificing
solid virtues to the attainment of superficial graces, though for a
few years \page{67} they may procure the individuals regal sway.

As a philosopher, I read with indignation the plausible epithets which
men use to soften their insults; and, as a moralist, I ask what is
meant by such heterogeneous associations, as fair defects, amiable
weaknesses, \&c.? If there is but one criterion of morals, but one
archetype for man, women appear to be suspended by destiny, according
to the vulgar tale of Mahomet's coffin; they have neither the unerring
instinct of brutes, nor are allowed to fix the eye of reason on a
perfect model. They were made to be loved, and must not aim at
respect, lest they should be hunted out of society as masculine.

But to view the subject in another point of view. Do passive indolent
women make the best wives? Confining our discussion to the present
moment of existence, let us see how such weak creatures perform their
part? Do the women who, by the attainment of a few superficial
accomplishments, have strengthened the prevailing prejudice, merely
contribute to the happiness of their husbands? Do they display their
charms merely to amuse them? And have women, who have early imbibed
\page{68} notions of passive obedience, sufficient character to manage
a family or educate children? So far from it, that, after surveying
the history of woman, I cannot help, agreeing with the severest
satirist, considering the sex as the weakest as well as the most
oppressed half of the species. What does history disclose but marks of
inferiority, and how few women have emancipated themselves from the
galling yoke of sovereign man?---So few, that the exceptions remind me
of an ingenious conjecture respecting Newton: that he was probably a
being of a superior order, accidentally caged in a human body.
Following the same train of thinking, I have been led to imagine that
the few extraordinary women who have rushed in eccentrical directions
out of the orbit prescribed to their sex, were \textit{male} spirits,
confined by mistake in female frames. But if it be not philosophical
to think of sex when the soul is mentioned, the inferiority must
depend on the organs; or the heavenly fire, which is to ferment the
clay, is not given in equal portions.

But avoiding, as I have hitherto done, any direct comparison of the
two sexes collectively, or frankly acknowledging the
in-\page{69}feriority of woman, according to the present appearance of
things, I shall only insist that men have increased that inferiority
till women are almost sunk below the standard of rational creatures.
Let their faculties have room to unfold, and their virtues to gain
strength, and then determine where the whole sex must stand in the
intellectual scale. Yet let it be remembered, that for a small number
of distinguished women I do not ask a place.

It is difficult for us purblind mortals to say to what height human
discoveries and improvements may arrive when the gloom of despotism
subsides, which makes us stumble at every step; but, when morality
shall be settled on a more solid basis, then, without being gifted
with a prophetic spirit, I will venture to predict that woman will be
either the friend or slave of man. We shall not, as at present, doubt
whether she is a moral agent, or the link which unites man with
brutes. But, should it then appear, that like the brutes they were
principally created for the use of man, he will let them patiently
bite the bridle, and not mock them with empty praise; or, should their
rationality be proved, \page{70} he will not impede their improvement
merely to gratify his sensual appetites. He will not, with all the
graces of rhetoric, advise them to submit implicitly their
understanding to the guidance of man. He will not, when he treats of
the education of women, assert that they ought never to have the free
use of reason, nor would he recommend cunning and dissimulation to
beings who are acquiring, in like manner as himself, the virtues of
humanity.

Surely there can be but one rule of right, if morality has an eternal
foundation, and whoever sacrifices virtue, strictly so called, to
present convenience, or whose \textit{duy} it is to act in such a
manner, lives only for the passing day, and cannot be an accountable
creature.

The poet then should have dropped his sneer when he says,

% NOTE: double quotes below are correct, though verses above use
% single quotes

\begin{verse}
``If weak women go astray,\\
The stars are more in fault than they.''
\end{verse}

\noindent For that they are bound by the adamantine chain of destiny
is most certain, if it be proved that they are never to exercise their
own reason, never to be independent, never to rise \page{71} above
opinion, or to feel the dignity of a rational will that only bows to
God, and often forgets that the universe contains any being but itself
and the model of perfection to which its ardent gaze is turned, to
adore attributes that, softened into virtues, may be imitated in kind,
though the degree overwhelms the enraptured mind.

If, I say, for I would not impress by declamation when Reason offers
her sober light, if they are really capable of acting like rational
creatures, let them not be treated like slaves; or, like the brutes
who are dependent on the reason of man, when they associate with him;
but cultivate their minds, give them the salutary, sublime curb of
principle, and let them attain conscious dignity by feeling themselves
only dependent on God. Teach them, in common with man, to submit to
necessity, instead of giving, to render them more pleasing, a sex to
morals.

% Without '\linebreak', an overfull hbox warning:

Further, should experience prove that they cannot attain the same
degree of \linebreak[4] strength of mind, perseverance, and fortitude,
let their virtues be the same in kind, though they may vainly struggle
for the same degree; and the superiority of man will be equally clear,
if not \page{72} clearer; and truth, as it is a simple principle,
which admits of no modification, would be common to both. Nay, the
order of society as it is at present regulated, would not be inverted,
for woman would then only have the rank that reason assigned her, and
arts could not be practised to bring the balance even, much less to
turn it.

These may be termed Utopian dreams.---Thanks to that Being who
impressed them on my soul, and gave me sufficient strength of mind to
dare to exert my own reason, till, becoming dependent only on him for
the support of my virtue, I view, with indignation, the mistaken
notions that enslave my sex.

I love man as my fellow; but his sceptre, real, or usurped, extends
not to me, unless the reason of an individual demands my homage; and
even then the submission is to reason, and not to man. In fact, the
conduct of an accountable being must be regulated by the operations of
its own reason; or on what foundation rests the throne of God?

It appears to me necessary to dwell on these obvious truths, because
females have been insulted, as it were; and, while they have been
stripped of the virtues that should clothe \page{73} humanity, they
have been decked with artificial graces that enable them to exercise a
short-lived tyranny. Love, in their bosoms, taking place of every
nobler passion, their sole ambition is to be fair, to raise emotion
instead of inspiring respect; and this ignoble desire, like the
servility in absolute monarchies, destroys all strength of character.
Liberty is the mother of virtue, and if women be, by their very
constitution, slaves, and not allowed to breathe the sharp
invigorating air of freedom, they must ever languish like exotics, and
be reckoned beautiful flaws in nature.

As to the argument respecting the subjection in which the sex has ever
been held, it retorts on man. The many have always been enthralled by
the few; and monsters, who have scarcely shewn any discernment of
human excellence, have tyrannized over thousands of their
fellow-creatures. Why have men of superior endowments submitted to
such degradation? For, is it not universally acknowledged that kings,
viewed collectively, have ever been inferior, in abilities and virtue,
to the same number of men taken from the common mass of mankind---yet,
have they not, and are they not still treated with a de-\page{74}gree
of reverence that is an insult to reason? China is not the only
country where a living man has been made a God. \textit{Men} have
submitted to superior strength to enjoy with impunity the pleasure of
the mo\-ment---\textit{women} have only done the same, and therefore
till it is proved that the courtier, who servilely resigns the
birthright of a man, is not a moral agent, it cannot be demonstrated
that woman is essentially inferior to man because she has always been
subjugated.

Brutal force has hitherto governed the world, and that the science of
politics is in its infancy, is evident from philosophers scrupling to
give the knowledge most useful to man that determinate distinction.

I shall not pursue this argument any further than to establish an
obvious inference, that as sound politics diffuse liberty, mankind,
including woman, will become more wise and virtuous.

