
\author{Fyodor Dostoevsky}
\authdate{1821--1881}
\textdate{1880}
\addon{Part 2, Book 5, Chapters 4 and 5}
\chapter[The Brothers Karamazov, excerpt]{The Brothers Karamazov}
\source{dostoevsky1922}

\page{248}\section*{Chapter IV. Rebellion}

``I must make one confession,'' Ivan began. ``I could never understand
how one can love one's neighbours. It's just one's neighbours, to my
mind, that one can't love, though one might \page{249} love those at a
distance. I once read somewhere of John the Merciful, a saint, that
when a hungry, frozen beggar came to him, he took him into his bed,
held him in his arms, and began breathing into his mouth, which was
putrid and loathsome from some awful disease. I am convinced that he
did that from `self-laceration,' from the self-laceration of falsity,
for the sake of the charity imposed by duty, as a penance laid on him.
For anyone to love a man, he must be hidden, for as soon as he shows
his face, love is gone.''

``Father Zossima has talked of that more than once,'' observed
Alyosha; ``he, too, said that the face of a man often hinders many
people not practised in love, from loving him. But yet there's a great
deal of love in mankind, and almost Christ-like love. I know that
myself, Ivan.''

``Well, I know nothing of it so far, and can't understand it, and the
innumerable mass of mankind are with me there. The question is,
whether that's due to men's bad qualities or whether it's inherent in
their nature. To my thinking, Christ-like love for men is a miracle
impossible on earth. He was God. But we are not gods. Suppose I, for
instance, suffer intensely. Another can never know how much I suffer,
because he is another and not I. And what's more, a man is rarely
ready to admit another's suffering (as though it were a distinction).
Why won't he admit it, do you think? Because I smell unpleasant,
because I have a stupid face, because I once trod on his foot. Besides
there is suffering and suffering; degrading, humiliating suffering
such as humbles me---hun\-ger, for instance,---my benefactor will
perhaps allow me; but when you come to higher suffering---for an idea,
for in\-stance---he will very rarely admit that, perhaps because my
face strikes him as not at all what he fancies a man should have who
suffers for an idea. And so he deprives me instantly of his favour,
and not at all from badness of heart. Beggars, especially genteel
beggars, ought never to show themselves, but to ask for charity
through the newspapers. One can love one's neighbours in the abstract,
or even at a distance, but at close quarters it's almost impossible.
If it were as on the stage, in the ballet, where if beggars come in,
they wear silken rags and tattered lace and beg for alms dancing
gracefully, then one might like looking at them. But even then we
should not love them. But enough of that. I simply wanted to show you
my point of view. I meant to speak of the suffering of mankind
generally, but we had better confine ourselves to the sufferings
\page{250} of the children. That reduces the scope of my argument to a
tenth of what it would be. Still we'd better keep to the children,
though it does weaken my case. But, in the first place, children can
be loved even at close quarters, even when they are dirty, even when
they are ugly (I fancy, though, children never are ugly). The second
reason why I won't speak of grown-up people is that, besides being
disgusting and unworthy of love, they have a
com\-pen\-sa\-tion---they've eaten the apple and know good and evil,
and they have become `like gods.' They go on eating it still. But the
children haven't eaten anything, and are so far innocent. Are you fond
of children, Alyosha? I know you are, and you will understand why I
prefer to speak of them. If they, too, suffer horribly on earth, they
must suffer for their fathers' sins, they must be punished for their
fathers, who have eaten the apple; but that reasoning is of the other
world and is incomprehensible for the heart of man here on earth. The
innocent must not suffer for another's sins, and especially such
innocents! You may be surprised at me, Alyosha, but I am awfully fond
of children, too. And observe, cruel people, the violent, the
rapacious, the Karamazovs are sometimes very fond of children.
Children while they are quite lit\-tle---up to seven, for
in\-stance---are so remote from grown-up people; they are different
creatures, as it were, of a different species. I knew a criminal in
prison who had, in the course of his career as a burglar, murdered
whole families, including several children. But when he was in prison,
he had a strange affection for them. He spent all his time at his
window, watching the children playing in the prison yard. He trained
one little boy to come up to his window and made great friends with
him.... You don't know why I am telling you all this, Alyosha? My head
aches and I am sad.''

``You speak with a strange air,'' observed Alyosha uneasily, ``as
though you were not quite yourself.''

``By the way, a Bulgarian I met lately in Moscow,'' Ivan went on,
seeming not to hear his brother's words, ``told me about the crimes
committed by Turks and Circassians in all parts of Bulgaria through
fear of a general rising of the Slavs. They burn villages, murder,
outrage women and children, they nail their prisoners by the ears to
the fences, leave them so till morning, and in the morning they hang
them---all sorts of things you can't imagine. People talk sometimes of
bestial cruelty, but that's a great injustice and insult to the
beasts; a beast can never be so cruel as a man, so artistically cruel.
The tiger only \page{251} tears and gnaws, that's all he can do. He
would never think of nailing people by the ears, even if he were able
to do it. These Turks took a pleasure in torturing children, too;
cutting the unborn child from the mothers womb, and tossing babies
up in the air and catching them on the points of their bayonets before
their mother's eyes. Doing it before the mother's eyes was what gave
zest to the amusement. Here is another scene that I thought very
interesting. Imagine a trembling mother with her baby in her arms, a
circle of invading Turks around her. They've planned a diversion: they
pet the baby, laugh to make it laugh. They succeed, the baby laughs.
At that moment a Turk points a pistol four inches from the baby's
face. The baby laughs with glee, holds out its little hands to the
pistol, and he pulls the trigger in the baby's face and blows out its
brains. Artistic, wasn't it? By the way, Turks are particularly fond
of sweet things, they say.''

``Brother, what are you driving at?'' asked Alyosha.

``I think if the devil doesn't exist, but man has created him, he has
created him in his own image and likeness.''

``Just as he did God, then?'' observed Alyosha.

```It's wonderful how you can turn words,' as Polonius says in
\textit{Hamlet},'' laughed Ivan. ``You turn my words against me. Well,
I am glad. Yours must be a fine God, if man created Him in His image
and likeness. You asked just now what I was driving at. You see, I am
fond of collecting certain facts, and, would you believe, I even copy
anecdotes of a certain sort from newspapers and books, and I've
already got a fine collection. The Turks, of course, have gone into
it, but they are foreigners. I have specimens from home that are even
better than the Turks. You know we prefer beat\-ing---rods and
scour\-ges---that's our national institution. Nailing ears is
unthinkable for us, for we are, after all, Europeans. But the rod and
the scourge we have always with us and they cannot be taken from us.
Abroad now they scarcely do any beating. Manners are more humane, or
laws have been passed, so that they don't dare to flog men now. But
they make up for it in another way just as national as ours. And so
national that it would be practically impossible among us, though I
believe we are being inoculated with it, since the religious movement
began in our aristocracy. I have a charming pamphlet, translated from
the French, describing how, quite recently, five years ago, a
murderer, Richard, was ex\-e\-cut\-ed---a young man, I believe, of
three and twenty, who repented and was \page{252} converted to the
Christian faith at the very scaffold. This Richard was an illegitimate
child who was given as a child of six by his parents to some shepherds
on the Swiss mountains. They brought him up to work for them. He
grew up like a little wild beast among them. The shepherds taught him
nothing, and scarcely fed or clothed him, but sent him out at seven to
herd the flock in cold and wet, and no one hesitated or scrupled to
treat him so. Quite the contrary, they thought they had every right,
for Richard had been given to them as a chattel, and they did not even
see the necessity of feeding him. Richard himself describes how in
those years, like the Prodigal Son in the Gospel, he longed to eat
of the mash given to the pigs, which were fattened for sale. But they
wouldn't even give that, and beat him when he stole from the pigs. And
that was how he spent all his childhood and his youth, till he grew
up and was strong enough to go away and be a thief. The savage began
to earn his living as a day labourer in Geneva. He drank what he
earned, he lived like a brute, and finished by killing and robbing an
old man. He was caught, tried, and condemned to death. They are not
sentimentalists there. And in prison he was immediately surrounded by
pastors, members of Christian brotherhoods, philanthropic ladies, and
the like. They taught him to read and write in prison, and expounded
the Gospel to him. They exhorted him, worked upon him, drummed at him
incessantly, till at last he solemnly confessed his crime. He was
converted. He wrote to the court himself that he was a monster, but
that in the end God had vouchsafed him light and shown grace. All
Geneva was in excitement about him---all philanthropic and religious
Geneva. All the aristocratic and well-bred society of the town rushed
to the prison, kissed Richard and embraced him; `You are our brother,
you have found grace.' And Richard does nothing but weep with emotion,
`Yes, I've found grace! All my youth and childhood I was glad of pigs'
food, but now even I have found grace. I am dying in the Lord.' `Yes,
Richard, die in the Lord; you have shed blood and must die. Though
it's not your fault that you knew not the Lord, when you coveted the
pigs' food and were beaten for stealing it (which was very wrong of
you, for stealing is forbidden); but you've shed blood and you must
die.' And on the last day, Richard, perfectly limp, did nothing but
cry and repeat every minute: `This is my happiest day. I am going to
the Lord.' `Yes,' cry the pastors and the judges and philanthropic
ladies. `This is the happiest day of your life, for \page{253} you are
going to the Lord!' They all walk or drive to the scaffold in
procession behind the prison van. At the scaffold they call to
Richard: `Die, brother, die in the Lord, for even thou hast found
grace!' And so, covered with his brothers' kisses, Richard is dragged
on to the scaffold, and led to the guillotine. And they chopped off
his head in brotherly fashion, because he had found grace. Yes, that's
characteristic. That pamphlet is translated into Russian by some
Russian philanthropists of aristocratic rank and evangelical
aspirations, and has been distributed gratis for the enlightenment of
the people. The case of Richard is interesting because it's national.
Though to us it's absurd to cut off a man's head, because he has
become our brother and has found grace, yet we have our own
speciality, which is all but worse. Our historical pastime is the
direct satisfaction of inflicting pain. There are lines in Nekrassov
describing how a peasant lashes a horse on the eyes, `on its meek
eyes,' every one must have seen it. It's peculiarly Russian. He
describes how a feeble little nag has foundered under too heavy a load
and cannot move. The peasant beats it, beats it savagely, beats it at
last not knowing what he is doing in the intoxication of cruelty,
thrashes it mercilessly over and over again. `However weak you are,
you must pull, if you die for it.' The nag strains, and then he begins
lashing the poor defenceless creature on its weeping, on its `meek
eyes.' The frantic beast tugs and draws the load, trembling all over,
gasping for breath, moving sideways, with a sort of unnatural
spasmodic ac\-tion---it's awful in Nekrassov. But that only a horse,
and God has horses to be beaten. So the Tatars have taught us, and
they left us the knout as a remembrance of it. But men, too, can be
beaten. A well-educated, cultured gentleman and his wife beat their
own child with a birch-rod, a girl of seven. I have an exact account
of it. The papa was glad that the birch was covered with twigs. `It
stings more,' said he, and so be began stinging his daughter. I know
for a fact there are people who at every blow are worked up to
sensuality, to literal sensuality, which increases progressively at
every blow they inflict. They beat for a minute, for five minutes, for
ten minutes, more often and more savagely. The child screams. At last
the child cannot scream, it gasps, `Daddy daddy!' By some diabolical
unseemly chance the case was brought into court. A counsel is engaged.
The Russian people have long called a barrister `a conscience for
hire.' The counsel protests in his client's defence. `It's such
\page{254} a simple thing,' he says, `an everyday domestic event. A
father corrects his child. To our shame be it said, it is brought into
court.' The jury, convinced by him, give a favourable verdict. The
public roars with delight that the torturer is acquitted. Ah, pity I
wasn't there! I would have proposed to raise a subscription in his
honour!\ldots Charming pictures.

``But I've still better things about children. I've collected a great,
great deal about Russian children, Alyosha. There was a little girl of
five who was hated by her father and mother, `most worthy and
respectable people, of good education and breeding.' You see, I must
repeat again, it is a peculiar characteristic of many people, this
love of torturing children, and children only. To all other types of
humanity these torturers behave mildly and benevolently, like
cultivated and humane Europeans; but they are very fond of tormenting
children, even fond of children themselves in that sense. It's just
their defencelessness that tempts the tormentor, just the angelic
confidence of the child who has no refuge and no appeal, that sets his
vile blood on fire. In every man, of course, a demon lies
hid\-den---the demon of rage, the demon of lustful heat at the screams
of the tortured victim, the demon of lawlessness let off the chain,
the demon of diseases that follow on vice, gout, kidney disease, and
so on.

``This poor child of five was subjected to every possible torture by
those cultivated parents. They beat her, thrashed her, kicked her for
no reason till her body was one bruise. Then, they went to greater
refinements of cru\-el\-ty---shut her up all night in the cold and
frost in a privy, and because she didn't ask to be taken up at night
(as though a child of five sleeping its angelic, sound sleep could be
trained to wake and ask), they smeared her face and filled her mouth
with excrement, and it was her mother, her mother did this. And that
mother could sleep, hearing the poor child's groans! Can you
understand why a little creature, who can't even understand what's
done to her, should beat her little aching heart with her tiny fist in
the dark and the cold, and weep her meek unresentful tears to dear,
kind God to protect her? Do you understand that, friend and brother,
you pious and humble novice? Do you understand why this infamy must be
and is permitted? Without it, I am told, man could not have existed on
earth, for he could not have known good and evil. Why should he know
that diabolical good and evil when it costs so much? Why, the whole
world of knowledge is not worth that child's prayer to `dear, kind
God'! I say nothing of the sufferings \page{255} of grown-up people,
they have eaten the apple, damn them, and the devil take them all! But
these little ones! I am making you suffer, Alyosha, you are not
yourself. I'll leave off if you like.''

``Never mind. I want to suffer too,'' muttered Alyosha.

``One picture, only one more, because it's so curious, so
characteristic, and I have only just read it in some collection of
Russian antiquities. I've forgotten the name. I must look it up. It
was in the darkest days of serfdom at the beginning of the century,
and long live the Liberator of the People! There was in those days a
general of aristocratic connections, the owner of great estates, one
of those men---some\-what exceptional, I believe, even then---who,
retiring from the service into a life of leisure, are convinced that
they've earned absolute power over the lives of their subjects. There
were such men then. So our general, settled on his property of two
thousand souls, lives in pomp, and domineers over his poor neighbours
as though they were dependents and buffoons. He has kennels of
hundreds of hounds and nearly a hundred dog-boys---all mounted, and in
uniform. One day a serf-boy, a little child of eight, threw a stone in
play and hurt the paw of the general's favourite hound. `Why is my
favourite dog lame?' He is told that the boy threw a stone that hurt
the dog's paw. `So you did it.' The general looked the child up and
down. `Take him.' He was tak\-en---tak\-en from his mother and kept
shut up all night. Early that morning the general comes out on
horseback, with the hounds, his dependents, dog-boys, and huntsmen,
all mounted around him in full hunting parade. The servants are
summoned for their edification, and in front of them all stands the
mother of the child. The child is brought from the lock-up. It's a
gloomy, cold, foggy autumn day, a capital day for hunting. The general
orders the child to be undressed; the child is stripped naked. He
shivers, numb with terror, not daring to cry.... `Make him run,'
commands the general. `Run! run!' shout the dog-boys. The boy runs....
`At him!' yells the general, and he sets the whole pack of hounds on
the child. The hounds catch him, and tear him to pieces before his
mother's eyes!\ldots I believe the general was afterwards declared
incapable of administering his estates. Well---what did he deserve? To
be shot? To be shot for the satisfaction of our moral feelings? Speak,
Alyosha!''

``To be shot,'' murmured Alyosha, lifting his eyes to Ivan with a
pale, twisted smile.

\page{256}``Bravo!'' cried Ivan delighted. ``If even you say so\ldots
You're a pretty monk! So there is a little devil sitting in your
heart, Alyosha Karamazov!''

``What I said was absurd, but---''

``That's just the point, that `but'!'' cried Ivan. ``Let me tell you,
novice, that the absurd is only too necessary on earth. The world
stands on absurdities, and perhaps nothing would have come to pass in
it without them. We know what we know!''

``What do you know?''

``I understand nothing,'' Ivan went on, as though in delirium. ``I
don't want to understand anything now. I want to stick to the fact. I
made up my mind long ago not to understand. If I try to understand
anything, I shall be false to the fact and I have determined to stick
to the fact.''

``Why are you trying me?'' Alyosha cried, with sudden distress. ``Will
you say what you mean at last?''

``Of course, I will; that's what I've been leading up to. You are dear
to me, I don't want to let you go, and I won't give you up to your
Zossima.''

Ivan for a minute was silent, his face became all at once very sad.

``Listen! I took the case of children only to make my case clearer. Of
the other tears of humanity with which the earth is soaked from its
crust to its centre, I will say nothing. I have narrowed my subject on
purpose. I am a bug, and I recognise in all humility that I cannot
understand why the world is arranged as it is. Men are themselves to
blame, I suppose; they were given paradise, they wanted freedom, and
stole fire from heaven, though they knew they would become unhappy, so
there is no need to pity them. With my pitiful, earthly, Euclidian
understanding, all I know is that there is suffering and that there
are none guilty; that cause follows effect, simply and directly; that
everything flows and finds its lev\-el---but that's only Euclidian
nonsense, I know that, and I can't consent to live by it! What comfort
is it to me that there are none guilty and that cause follows effect
simply and directly, and that I know it---I must have justice, or I
will destroy myself. And not justice in some remote infinite time and
space, but here on earth, and that I could see myself. I have believed
in it. I want to see it, and if I am dead by then, let me rise again,
for if it all happens without me, it will be too unfair. Surely I
haven't suffered, simply that I, my crimes and my sufferings, may
manure the soil of the future \page{257} harmony for somebody else. I
want to see with my own eyes the hind lie down with the lion and the
victim rise up and embrace his murderer. I want to be there when
everyone suddenly understands what it has all been for. All the
religions of the world are built on this longing, and I am a believer.
But then there are the children, and what am I to do about them?
That's a question I can't answer. For the hundredth time I repeat,
there are numbers of questions, but I've only taken the children,
because in their case what I mean is so unanswerably clear. Listen! If
all must suffer to pay for the eternal harmony, what have children to
do with it, tell me, please? It's beyond all comprehension why they
should suffer, and why they should pay for the harmony. Why should
they, too, furnish material to enrich the soil for the harmony of the
future? I understand solidarity in sin among men. I understand
solidarity in retribution, too; but there can be no such solidarity
with children. And if it is really true that they must share
responsibility for all their fathers' crimes, such a truth is not of
this world and is beyond my comprehension. Some jester will say,
perhaps, that the child would have grown up and have sinned, but you
see he didn't grow up, he was torn to pieces by the dogs, at eight
years old. Oh, Alyosha, I am not blaspheming! I understand, of course,
what an upheaval of the universe it will be, when everything in heaven
and earth blends in one hymn of praise and everything that lives and
has lived cries aloud: `Thou art just, O Lord, for Thy ways are
revealed.' When the mother embraces the fiend who threw her child to
the dogs, and all three cry aloud with tears, `Thou art just, O Lord!'
then, of course, the crown of knowledge will be reached and all will
be made clear. But what pulls me up here is that I can't accept that
harmony. And while I am on earth, I make haste to take my own
measures. You see, Alyosha, perhaps it really may happen that if I
live to that moment, or rise again to see it, I, too, perhaps, may cry
aloud with the rest, looking at the mother embracing the child's
torturer, `Thou art just, O Lord!' but I don't want to cry aloud then.
While there is still time, I hasten to protect myself and so I
renounce the higher harmony altogether. It's not worth the tears of
that one tortured child who beat itself on the breast with its little
fist and prayed in its stinking outhouse, with its unexpiated tears to
`dear, kind God'! It's not worth it, because those tears are unatoned
for. They must be atoned for, or there can be no harmony. But how? How
are you going to \page{258} atone for them? Is it possible? By their
being avenged? But what do I care for avenging them? What do I care
for a hell for oppressors? What good can hell do, since those children
have already been tortured? And what becomes of harmony, if there is
hell? I want to forgive. I want to embrace. I don't want more
suffering. And if the sufferings of children go to swell the sum of
sufferings which was necessary to pay for truth, then I protest that
the truth is not worth such a price. I don't want the mother to
embrace the oppressor who threw her son to the dogs! She dare not
forgive him! Let her forgive him for herself, if she will, let her
forgive the torturer for the immeasurable suffering of her mother's
heart. But the sufferings of her tortured child she has no right to
forgive; she dare not forgive the torturer, even if the child were to
forgive him! And if that is so, if they dare not forgive, what becomes
of harmony? Is there in the whole world a being who would have the
right to forgive and could forgive? I don't want harmony. From love
for humanity I don't want it. I would rather be left with the
unavenged suffering. I would rather remain with my unavenged suffering
and unsatisfied indignation, \textit{even if I were wrong}. Besides,
too high a price is asked for harmony; it's beyond our means to pay so
much to enter on it. And so I hasten to give back my entrance ticket,
and if I am an honest man I am bound to give it back as soon as
possible. And that I am doing. It's not God that I don't accept,
Alyosha, only I most respectfully return Him the ticket.''

``That's rebellion,'' murmered Alyosha, looking down.

``Rebellion? I am sorry you call it that,'' said Ivan earnestly. ``One
can hardly live in rebellion, and I want to live. Tell me yourself, I
challenge you---answer. Imagine that you are creating a fabric of
human destiny with the object of making men happy in the end, giving
them peace and rest at last, but that it was essential and inevitable
to torture to death only one tiny crea\-ture---that baby beating its
breast with its fist, for in\-stance---and to found that edifice on
its unavenged tears, would you consent to be the architect on those
conditions? Tell me, and tell the truth.''

``No, I wouldn't consent,'' said Alyosha softly.

``And can you admit the idea that men for whom you are building it
would agree to accept their happiness on the foundation of the
unexpiated blood of a little victim? And accepting it would remain
happy for ever?''

\page{259}``No, I can't admit it. Brother,'' said Alyosha suddenly,
with flashing eyes, ``you said just now, is there a being in the whole
world who would have the right to forgive and could forgive? But there
is a Being and He can forgive everything, all and for all, because He
gave His innocent blood for all and everything. You have forgotten
Him, and on Him is built the edifice, and it is to Him they cry aloud,
`Thou art just, O Lord, for Thy ways are revealed!'

``Ah! the One without sin and His blood! No, I have not forgotten Him;
on the contrary I've been wondering all the time how it was you did
not bring Him in before, for usually all arguments on your side put
Him in the foreground. Do you know, Al\-yosh\-a---don't laugh! I made
a poem about a year ago. If you can waste another ten minutes on me,
I'll tell it to you.''

``You wrote a poem?''

``Oh, no, I didn't write it,'' laughed Ivan, ``and I've never written
two lines of poetry in my life. But I made up this poem in prose and I
remembered it. I was carried away when I made it up. You will be my
first read\-er---that is listener. Why should an author forego even
one listener?'' smiled Ivan. ``Shall I tell it to you?''

``I am all attention.'' said Alyosha.

``My poem is called `The Grand Inquisitor'; it's a ridiculous thing,
but I want to tell it to you.''

\section*{Chapter V. The Grand Inquisitor}

% NOTE: dropped period after 'Louis XI'

``Even this must have a preface---that is, a literary preface,''
laughed Ivan, ``and I am a poor hand at making one. You see, my action
takes place in the sixteenth century, and at that time, as you
probably learnt at school, it was customary in poetry to bring down
heavenly powers on earth. Not to speak of Dante, in France, clerks, as
well as the monks in the monasteries, used to give regular
performances in which the Madonna, the saints, the angels, Christ, and
God Himself were brought on the stage. In those days it was done in
all simplicity. In Victor Hugo's `Notre Dame de Paris' an edifying and
gratuitous spectacle was provided for the people in the Hotel de Ville
of Paris in the reign of Louis XI in honour of the birth of the
dauphin. It was \page{260} called \textit{Le bon jugement de la tr\'es
sainte et gracieuse Vierge Marie}, and she appears herself on the
stage and pronounces her \textit{bon jugement}. Similar plays, chiefly
from the Old Testament, were occasionally performed in Moscow too, up
to the times of Peter the Great. But besides plays there were all
sorts of legends and ballads scattered about the world, in which the
saints and angels and all the powers of Heaven took part when
required. In our monasteries the monks busied themselves in
translating, copying, and even composing such po\-ems---and even under
the Tatars. There is, for instance, one such poem (of course, from the
Greek), `The Wanderings of Our Lady through Hell,' with descriptions
as bold as Dante's. Our Lady visits Hell, and the Archangel Michael
leads her through the torments. She sees the sinners and their
punishment. There she sees among others one noteworthy set of sinners
in a burning lake; some of them sink to the bottom of the lake so that
they can't swim out, and `these God forgets'---an expression of
extraordinary depth and force. And so Our Lady, shocked and weeping,
falls before the throne of God and begs for mercy for all in
Hell---for all she has seen there, indiscriminately. Her conversation
with God is immensely interesting. She beseeches Him, she will not
desist, and when God points to the hands and feet of her Son, nailed
to the Cross, and asks, `How can I forgive His tormentors?' she bids
all the saints, all the martyrs, all the angels and archangels to fall
down with her and pray for mercy on all without distinction. It ends
by her winning from God a respite of suffering every year from Good
Friday till Trinity Day, and the sinners at once raise a cry of
thankfulness from Hell, chanting, `Thou art just, O Lord, in this
judgment.' Well, my poem would have been of that kind if it had
appeared at that time. He comes on the scene in my poem, but He says
nothing, only appears and passes on. Fifteen centuries have passed
since He promised to come in His glory, fifteen centuries since His
prophet wrote, `Behold, I come quickly;' `Of that day and that hour
knoweth no man, neither the Son, but the Father,' as He Himself
predicted on earth. But humanity awaits him with the same faith and
with the same love. Oh, with greater faith, for it is fifteen
centuries since man has ceased to see signs from heaven.

\begin{verse}
\textit{No signs from heaven come to-day\\
To add to what the heart doth say.}
\end{verse}

\noindent There was nothing left but faith in what the heart doth say.
\page{261} It is true there were many miracles in those days. There
were saints who performed miraculous cures; some holy people,
according to their biographies, were visited by the Queen of Heaven
herself. But the devil did not slumber, and doubts were already
arising among men of the truth of these miracles. And just then there
appeared in the north of Germany a terrible new heresy. `A huge star
like to a torch' (that is, to a church) `fell on the sources of the
waters and they became bitter.' These heretics began blasphemously
denying miracles. But those who remained faithful were all the more
ardent in their faith. The tears of humanity rose up to Him as before,
awaited His coming, loved Him, hoped for Him, yearned to suffer and
die for Him as before. And so many ages mankind had prayed with faith
and fervour, `O Lord our God, hasten Thy coming,' so many ages called
upon Him, that in His infinite mercy He deigned to come down to His
servants. Before that day He had come down, He had visited some holy
men, martyrs and hermits, as is written in their `Lives.' Among us,
Tyutchev, with absolute faith in the truth of his words, bore witness
that

\begin{verse}
\textit{Bearing the Cross, in slavish dress\\
Weary and worn, the Heavenly King\\
Our mother, Russia, came to bless,\\
And through our land went wandering}.
\end{verse}

\noindent And that certainly was so, I assure you.

``And behold, He deigned to appear for a moment to the people, to the
tortured, suffering people, sunk in iniquity, but loving Him like
children. My story is laid in Spain, in Seville, in the most terrible
time of the Inquisition, when fires were lighted every day to the
glory of God, and `in the splendid \textit{auto da f\'e} the wicked
heretics were burnt.' Oh, of course, this was not the coming in which
He will appear according to His promise at the end of time in all His
heavenly glory, and which will be sudden `as lightning flashing from
east to west.' No, He visited His children only for a moment, and
there where the flames were crackling round the heretics. In His
infinite mercy He came once more among men in that human shape in
which He walked among men for three years fifteen centuries ago. He
came down to the `hot pavements' of the southern town in which on the
day before almost a hundred heretics had, \textit{ad majorem gloriam
Dei}, been burnt by the cardinal, the Grand Inquisitor, in a
magnificent \textit{auto da f\'e}, in the presence of the king,
\page{262} the court, the knights, the cardinals, the most charming
ladies of the court, and the whole population of Seville.

``He came softly, unobserved, and yet, strange to say, every one
recognised Him. That might be one of the best passages in the poem. I
mean, why they recognised Him. The people are irresistibly drawn to
Him, they surround Him, they flock about Him, follow Him. He moves
silently in their midst with a gentle smile of infinite compassion.
The sun of love burns in His heart, light and power shine from His
eyes, and their radiance, shed on the people, stirs their hearts with
responsive love. He holds out His hands to them, blesses them, and a
healing virtue comes from contact with Him, even with His garments. An
old man in the crowd, blind from childhood, cries out, `O Lord, heal
me and I shall see Thee!' and, as it were, scales fall from his eyes
and the blind man sees Him. The crowd weeps and kisses the earth under
His feet. Children throw flowers before Him, sing, and cry hosannah.
`It is He---it is He!' all repeat. `It must be He, it can be no one
but Him!' He stops at the steps of the Seville cathedral at the moment
when the weeping mourners are bringing in a little open white coffin.
In it lies a child of seven, the only daughter of a prominent citizen.
The dead child lies hidden in flowers. `He will raise your child,' the
crowd shouts to the weeping mother. The priest, coming to meet the
coffin, looks perplexed, and frowns, but the mother of the dead child
throws herself at His feet with a wail. `If it is Thou, raise my
child!' she cries, holding out her hands to Him. The procession halts,
the coffin is laid on the steps at His feet. He looks with compassion,
and His lips once more softly pronounce, `Maiden, arise!' and the
maiden arises. The little girl sits up in the coffin and looks round,
smiling with wide-open wondering eyes, holding a bunch of white roses
they had put in her hand.

``There are cries, sobs, confusion among the people, and at that
moment the cardinal himself, the Grand Inquisitor, passes by the
cathedral. He is an old man, almost ninety, tall and erect, with a
withered face and sunken eyes, in which there is still a gleam of
light. He is not dressed in his gorgeous cardinal's robes, as he was
the day before, when he was burning the enemies of the Roman
Church---at this moment he is wearing his coarse, old, monk's cassock.
At a distance behind him come his gloomy assistants and slaves and the
`holy guard.' He stops at the sight of the crowd and watches it from a
distance. He sees \page{263} everything; he sees them set the coffin
down at His feet, sees the child rise up, and his face darkens. He
knits his thick grey brows and his eyes gleam with a sinister fire. He
holds out his finger and bids the guards take Him. And such is his
power, so completely are the people cowed into submission and
trembling obedience to him, that the crowd immediately make way for
the guards, and in the midst of deathlike silence they lay hands on
Him and lead him away. The crowd instantly bows down to the earth,
like one man, before the old inquisitor. He blesses the people in
silence and passes on. The guards lead their prisoner to the close,
gloomy vaulted prison in the ancient palace of the Holy Inquisition
and shut Him in it. The day passes and is followed by the dark,
burning, `breathless' night of Seville. The air is `fragrant with
laurel and lemon.' In the pitch darkness the iron door of the prison
is suddenly opened and the Grand Inquisitor himself comes in with a
light in his hand. He is alone; the door is closed at once behind him.
He stands in the doorway and for a minute or two gazes into His face.
At last he goes up slowly, sets the light on the table and speaks.

``'Is it Thou? Thou?' but receiving no answer, he adds at once. `Don't
answer, be silent. What canst Thou say, indeed? I know too well what
Thou wouldst say. And Thou hast no right to add anything to what Thou
hadst said of old. Why, then, art Thou come to hinder us? For Thou
hast come to hinder us, and Thou knowest that. But dost Thou know what
will be to-morrow? I know not who Thou art and care not to know
whether it is Thou or only a semblance of Him, but to-morrow I shall
condemn Thee and burn Thee at the stake as the worst of heretics. And
the very people who have to-day kissed Thy feet, to-morrow at the
faintest sign from me will rush to heap up the embers of Thy fire.
Knowest Thou that? Yes, maybe Thou knowest it,' he added with
thoughtful penetration, never for a moment taking his eyes off the
Prisoner.''

``I don't quite understand, Ivan. What does it mean?'' Alyosha, who
had been listening in silence, said with a smile. ``Is it simply a
wild fantasy, or a mistake on the part of the old man---some
impossible \textit{qui pro quo}?''

``Take it as the last,'' said Ivan, laughing, ``if you are so
corrupted by modern realism and can't stand anything fantastic. If you
like it to be a case of mistaken identity, let it be so. It is true,''
he went on, laughing, ``the old man was ninety, and \page{264} he
might well be crazy over his set idea. He might have been struck by
the appearance of the Prisoner. It might, in fact, be simply his
ravings, the delusion of an old man of ninety, over-excited by the
\textit{auto da f\'e} of a hundred heretics the day before. But does
it matter to us after all whether it was a mistake of identity or a
wild fantasy? All that matters is that the old man should speak out,
should speak openly of what he has thought in silence for ninety
years.''

``And the Prisoner too is silent? Does He look at him and not say a
word?''

``That's inevitable in any case,'' Ivan laughed again. ``The old man
has told Him He hasn't the right to add anything to what He has said
of old. One may say it is the most fundamental feature of Roman
Catholicism, in my opinion at least. `All has been given by Thee to
the Pope,' they say, `and all, therefore, is still in the Pope's
hands, and there is no need for Thee to come now at all. Thou must not
meddle for the time, at least.' That's how they speak and write
too---the Jesuits, at any rate. I have read it myself in the works of
their theologians. `Hast Thou the right to reveal to us one of the
mysteries of that world from which Thou hast come?' my old man asks
Him, and answers the question for Him. `No, Thou hast not; that Thou
mayest not add to what has been said of old, and mayest not take from
men the freedom which Thou didst exalt when Thou wast on earth.
Whatsoever Thou revealest anew will encroach on men's freedom of
faith; for it will be manifest as a miracle, and the freedom of their
faith was dearer to Thee than anything in those days fifteen hundred
years ago. Didst Thou not often say then, ``I will make you free''?
But now Thou hast seen these ``free'' men,' the old man adds suddenly,
with a pensive smile. `Yes, we've paid dearly for it,' he goes on,
looking sternly at Him, `but at last we have completed that work in
Thy name. For fifteen centuries we have been wrestling with Thy
freedom, but now it is ended and over for good. Dost Thou not believe
that it's over for good? Thou lookest meekly at me and deignest not
even to be wroth with me. But let me tell Thee that now, to-day,
people are more persuaded than ever that they have perfect freedom,
yet they have brought their freedom to us and laid it humbly at our
feet. But that has been our doing. Was this what Thou didst? Was this
Thy freedom?'''

``I don't understand again.'' Alyosha broke in. ``Is he ironical, is
he jesting?''

\page{265}``Not a bit of it! He claims it as a merit for himself and
his Church that at last they have vanquished freedom and have done so
to make men happy. `For now' (he is speaking of the Inquisition, of
course) `for the first time it has become possible to think of the
happiness of men. Man was created a rebel; and how can rebels be
happy? Thou wast warned,' he says to Him. `Thou hast had no lack of
admonitions and warnings, but Thou didst not listen to those warnings;
Thou didst reject the only way by which men might be made happy. But,
fortunately, departing Thou didst hand on the work to us. Thou hast
promised, Thou hast established by Thy word, Thou hast given to us the
right to bind and to unbind, and now, of course, Thou canst not think
of taking it away. Why, then, hast Thou come to hinder us?'''

``And what's the meaning of `no lack of admonitions and warnings'?''
asked Alyosha.

``Why, that's the chief part of what the old man must say.''

```The wise and dread Spirit, the spirit of self-destruction and
non-ex\-is\-tence,' the old man goes on, `the great spirit talked with
Thee in the wilderness, and we are told in the books that he
``tempted'' Thee. Is that so? And could anything truer be said than
what he revealed to Thee in three questions and what Thou didst
reject, and what in the books is called ``the temptation''? And yet if
there has ever been on earth a real stupendous miracle, it took place
on that day, on the day of the three temptations. The statement of
those three questions was itself the miracle. If it were possible to
imagine simply for the sake of argument that those three questions of
the dread spirit had perished utterly from the books, and that we had
to restore them and to invent them anew, and to do so had gathered
together all the wise men of the earth---rulers, chief priests,
learned men, philosophers, poets---and had set them the task to invent
three questions, such as would not only fit the occasion, but express
in three words, three human phrases, the whole future history of the
world and of hu\-man\-i\-ty---dost Thou believe that all the wisdom of
the earth united could have invented anything in depth and force equal
to the three questions which were actually put to Thee then by the
wise and mighty spirit in the wilderness? From those questions alone,
from the miracle of their statement, we can see that we have here to
do not with the fleeting human intelligence, but with the absolute and
eternal. For in those three questions the whole subsequent history of
mankind is, as it were, brought together into one \page{266} whole,
and foretold, and in them are united all the unsolved historical
contradictions of human nature. At the time it could not be so clear,
since the future was unknown; but now that fifteen hundred years have
passed, we see that everything in those three questions was so justly
divined and foretold, and has been so truly fulfilled, that nothing
can be added to them or taken from them.

```Judge Thyself who was right---Thou or he who questioned Thee then?
Remember the first question; its meaning, in other words, was this:
``Thou wouldst go into the world, and art going with empty hands, with
some promise of freedom which men in their simplicity and their
natural unruliness cannot even understand, which they fear and
dread---for nothing has ever been more insupportable for a man and a
human society than freedom. But seest Thou these stones in this
parched and barren wilderness? Turn them into bread, and mankind will
run after Thee like a flock of sheep, grateful and obedient, though
for ever trembling, lest Thou withdraw Thy hand and deny them Thy
bread.'' But Thou wouldst not deprive man of freedom and didst reject
the offer, thinking, what is that freedom worth, if obedience is
bought with bread? Thou didst reply that man lives not by bread alone.
But dost Thou know that for the sake of that earthly bread the spirit
of the earth will rise up against Thee and will strive with Thee and
overcome Thee, and all will follow him, crying, ``Who can compare with
this beast? He has given us fire from heaven!'' Dost Thou know that
the ages will pass, and humanity will proclaim by the lips of their
sages that there is no crime, and therefore no sin; there is only
hunger? ``Feed men, and then ask of them virtue!'' that's what they'll
write on the banner, which they will raise against Thee, and with
which they will destroy Thy temple. Where Thy temple stood will rise a
new building; the terrible tower of Babel will be built again, and
though, like the one of old, it will not be finished, yet Thou
mightest have prevented that new tower and have cut short the
sufferings of men for a thousand years; for they will come back to us
after a thousand years of agony with their tower. They will seek us
again, hidden underground in the catacombs, for we shall be again
persecuted and tortured. They will find us and cry to us, ``Feed us,
for those who have promised us fire from heaven haven't given it!''
And then we shall finish building their tower, for he finishes the
building who feeds them. And we alone shall feed them in Thy name,
declaring \page{267} falsely that it is in Thy name. Oh, never, never
can they feed themselves without us! No science will give them bread
so long as they remain free. In the end they will lay their freedom at
our feet, and say to us, ``Make us your slaves, but feed us.'' They
will understand themselves, at last, that freedom and bread enough for
all are inconceivable together, for never, never will they be able to
share between them! They will be convinced, too, that they can never
be free, for they are weak, vicious, worthless and rebellious. Thou
didst promise them the bread of Heaven, but, I repeat again, can it
compare with earthly bread in the eyes of the weak, ever sinful and
ignoble race of man? And if for the sake of the bread of Heaven
thousands shall follow Thee, what is to become of the millions and
tens of thousands of millions of creatures who will not have the
strength to forego the earthly bread for the sake of the heavenly? Or
dost Thou care only for the tens of thousands of the great and strong,
while the millions, numerous as the sands of the sea, who are weak but
love Thee, must exist only for the sake of the great and strong? No,
we care for the weak too. They are sinful and rebellious, but in the
end they too will become obedient. They will marvel at us and look on
us as gods, because we are ready to endure the freedom which they have
found so dreadful and to rule over them---so awful it will seem to
them to be free. But we shall tell them that we are Thy servants and
rule them in Thy name. We shall deceive them again, for we will not
let Thee come to us again. That deception will be our suffering, for
we shall be forced to lie.

```This is the significance of the first question in the wilderness,
and this is what Thou hast rejected for the sake of that freedom which
Thou hast exalted above everything. Yet in this question lies hid the
great secret of this world. Choosing ``bread,'' Thou wouldst have
satisfied the universal and everlasting craving of hu\-man\-i\-ty---to
find some one to worship. So long as man remains free he strives for
nothing so incessantly and so painfully as to find some one to
worship. But man seeks to worship what is established beyond dispute,
so that all men would agree at once to worship it. For these pitiful
creatures are concerned not only to find what one or the other can
worship, but to find something that all would believe in and worship;
what is essential is that all may be \textit{together} in it. This
craving for \textit{community} of worship is the chief misery of every
man individually and of all humanity from the beginning of time.
\page{268} For the sake of common worship they've slain each other
with the sword. They have set up gods and challenged one another,
``Put away your gods and come and worship ours, or we will kill you
and your gods!'' And so it will be to the end of the world, even when
gods disappear from the earth; they will fall down before idols just
the same. Thou didst know, Thou couldst not but have known, this
fundamental secret of human nature, but Thou didst reject the one
infallible banner which was offered Thee to make all men bow down to
Thee a\-lone---the banner of earthly bread; and Thou hast rejected it
for the sake of freedom and the bread of Heaven. Behold what Thou
didst further. And all again in the name of freedom! I tell Thee that
man is tormented by no greater anxiety than to find some one quickly
to whom he can hand over that gift of freedom with which the ill-fated
creature is born. But only one who can appease their conscience can
take over their freedom. In bread there was offered Thee an invincible
banner; give bread, and man will worship thee, for nothing is more
certain than bread. But if some one else gains possession of his
con\-science---oh! then he will cast away Thy bread and follow after
him who has ensnared his conscience. In that Thou wast right. For the
secret of man's being is not only to live but to have something to
live for. Without a stable conception of the object of life, man would
not consent to go on living, and would rather destroy himself than
remain on earth, though he had bread in abundance. That is true. But
what happened? Instead of taking men's freedom from them, Thou didst
make it greater than ever! Didst Thou forget that man prefers peace,
and even death, to freedom of choice in the knowledge of good and
evil? Nothing is more seductive for man than his freedom of
conscience, but nothing is a greater cause of suffering. And behold,
instead of giving a firm foundation for setting the conscience of man
at rest for ever, Thou didst choose all that is exceptional, vague and
enigmatic; Thou didst choose what was utterly beyond the strength of
men, acting as though Thou didst not love them at all---Thou who didst
come to give Thy life for them! Instead of taking possession of men's
freedom, Thou didst increase it, and burdened the spiritual kingdom of
mankind with its sufferings forever. Thou didst desire man's free
love, that he should follow Thee freely, enticed and taken captive by
Thee. In place of the rigid ancient law, man must hereafter with free
heart decide for himself what is good and what \page{269} is evil,
having only Thy image before him as his guide. But didst Thou not know
that he would at last reject even Thy image and Thy truth, if he is
weighed down with the fearful burden of free choice? They will cry
aloud at last that the truth is not in Thee, for they could not have
been left in greater confusion and suffering than Thou hast caused,
laying upon them so many cares and unanswerable problems.

```So that, in truth, Thou didst Thyself lay the foundation for the
destruction of Thy kingdom, and no one is more to blame for it. Yet
what was offered Thee? There are three powers, three powers alone,
able to conquer and to hold captive for ever the conscience of these
impotent rebels for their hap\-pi\-ness---those forces are miracle,
mystery and authority. Thou hast rejected all three and hast set the
example for doing so. When the wise and dread spirit set Thee on the
pinnacle of the temple and said to Thee, ``If Thou wouldst know
whether Thou art the Son of God then cast Thyself down, for it is
written: the angels shall hold him up lest he fall and bruise himself,
and Thou shalt know then whether Thou art the Son of God and shalt
prove then how great is Thy faith in Thy Father.'' But Thou didst
refuse and wouldst not cast Thyself down. Oh! of course, Thou didst
proudly and well, like God; but the weak, unruly race of men, are they
gods? Oh, Thou didst know then that in taking one step, in making one
movement to cast Thyself down, Thou wouldst be tempting God and have
lost all Thy faith in Him, and wouldst have been dashed to pieces
against that earth which Thou didst come to save. And the wise spirit
that tempted Thee would have rejoiced. But I ask again, are there many
like Thee? And couldst Thou believe for one moment that men, too,
could face such a temptation? Is the nature of men such, that they can
reject miracle, and at the great moments of their life, the moments of
their deepest, most agonising spiritual difficulties, cling only to
the free verdict of the heart? Oh, Thou didst know that Thy deed would
be recorded in books, would be handed down to remote times and the
utmost ends of the earth, and Thou didst hope that man, following
Thee, would cling to God and not ask for a miracle. But Thou didst not
know that when man rejects miracle he rejects God too; for man seeks
not so much God as the miraculous. And as man cannot bear to be
without the miraculous, he will create new miracles of his own for
himself, and will worship deeds of sorcery and witchcraft, though he
might be a hundred times over a rebel, heretic and infidel. Thou didst
\page{270} not come down from the Cross when they shouted to Thee,
mocking and reviling Thee, ``Come down from the cross and we will
believe that Thou art He.'' Thou didst not come down, for again Thou
wouldst not enslave man by a miracle, and didst crave faith given
freely, not based on miracle. Thou didst crave for free love and not
the base raptures of the slave before the might that has overawed him
for ever. But Thou didst think too highly of men therein, for they are
slaves, of course, though rebellious by nature. Look round and judge;
fifteen centuries have passed, look upon them. Whom hast Thou raised
up to Thyself? I swear, man is weaker and baser by nature than Thou
hast believed him! Can he, can he do what Thou didst? By showing him
so much respect, Thou didst, as it were, cease to feel for him, for
Thou didst ask far too much from him---Thou who hast loved him more
than Thyself! Respecting him less, Thou wouldst have asked less of
him. That would have been more like love, for his burden would have
been lighter. He is weak and vile. What though he is everywhere now
rebelling against our power, and proud of his rebellion? It is the
pride of a child and a schoolboy. They are little children rioting and
barring out the teacher at school. But their childish delight will
end; it will cost them dear. They will cast down temples and drench
the earth with blood. But they will see at last, the foolish children,
that, though they are rebels, they are impotent rebels, unable to keep
up their own rebellion. Bathed in their foolish tears, they will
recognise at last that He who created them rebels must have meant to
mock at them. They will say this in despair, and their utterance will
be a blasphemy which will make them more unhappy still, for man's
nature cannot bear blasphemy, and in the end always avenges it on
itself. And so unrest, confusion and un\-hap\-pi\-ness---that is the
present lot of man after Thou didst bear so much for their freedom!
They great prophet tells in vision and in image, that he saw all those
who took part in the first resurrection and that there were of each
tribe twelve thousand. But if there were so many of them, they must
have been not men but gods. They had borne Thy cross, they had endured
scores of years in the barren, hungry wilderness, living upon locusts
and roots---and Though mayest indeed point with pride at those
children of freedom, of free love, of free and splendid sacrifice for
Thy name. But remember that they were only some thousands; and what of
the rest? And how are the other weak ones to blame, because they could
not endure what the strong have endured? How is the \page{271} weak
soul to blame that it is unable to receive such terrible gifts? Canst
Thou have simply come to the elect and for the elect? But if so, it is
a mystery and we cannot understand it. And if it is a mystery, we too
have a right to preach a mystery, and to teach them that it's not the
free judgment of their hearts, not love that matters, but a mystery
which they must follow blindly, even against their conscience. So we
have done. We have corrected Thy work and have founded it upon
\textit{miracle}, \textit{mystery} and \textit{authority}. And men
rejoiced that they were again led like sheep, and that the terrible
gift that had brought them such suffering was, at last, lifted from
their hearts. Were we right teaching them this? Speak! Did we not love
mankind, so meekly acknowledging their feebleness, lovingly lightening
their burden, and permitting their weak nature even sin with our
sanction? Why hast Thou come now to hinder us? And why dost Thou look
silently and searchingly at me with Thy mild eyes? Be angry. I don't
want Thy love, for I love Thee not. And what use is it for me to hide
anything from Thee? Don't I know to Whom I am speaking? All that I can
say is known to Thee already. And is it for me to conceal from Thee
our mystery? Perhaps it is Thy will to hear it from my lips. Listen,
then. We are not working with Thee, but with \textit{him}---that is
our mystery. It's long---eight cen\-tu\-ries---since we have been on
\textit{his} side and not on Thine. Just eight centuries ago, we took
from him what Thou didst reject with scorn, that last gift he offered
Thee, showing Thee all the kingdoms of the earth. We took from him
Rome and the sword of C\ae sar, and proclaimed ourselves sole rulers
of the earth, though hitherto we have not been able to complete our
work. But whose fault is that? Oh, the work is only beginning, but it
has begun. It has long to await completion and the earth has yet much
to suffer, but we shall triumph and shall be C\ae sars, and then we
shall plan the universal happiness of man. But Thou mightest have
taken even then the sword of C\ae sar. Why didst Thou reject that last
gift? Hadst Thou accepted that last counsel of the mighty spirit, Thou
wouldst have accomplished all that man seeks on earth---that is, some
one to worship, some one to keep his conscience, and some means of
uniting all in one unanimous and harmonious ant-heap, for the craving
for universal unity is the third and last anguish of men. Mankind as a
whole has always striven to organise a universal state. There have
been many great nations with great histories, but the more highly they
were developed the more unhappy they were, for \page{272} they felt
more acutely than other people the craving for world-wide union. The
great conquerors, Timours and Ghenghis-Khans, whirled like hurricanes
over the face of the earth striving to subdue its people, and they too
were but the unconscious expression of the same craving for universal
unity. Hadst Thou taken the world and C\ae sar's purple, Thou wouldst
have founded the universal state and have given universal peace. For
who can rule men if not he who holds their conscience and their bread
in his hands? We have taken the sword of C\ae sar, and in taking it,
of course, have rejected Thee and followed \textit{him}. Oh, ages are
yet to come of the confusion of free thought, of their science and
cannibalism. For having begun to build their tower of Babel without
us, they will end, of course, with cannibalism. But then the beast
will crawl to us and lick our feet and spatter them with tears of
blood. And we shall sit upon the beast and raise the cup, and on it
will be written, ``Mystery.'' But then, and only then, the reign of
peace and happiness will come for men. Thou art proud of Thine elect,
but Thou hast only the elect, while we give rest to all. And besides,
how many of those elect, those mighty ones who could become elect,
have grown weary waiting for Thee, and have transferred and will
transfer the powers of their spirit and the warmth of their heart to
the other camp, and end by raising their \textit{free} banner against
Thee. Thou didst Thyself lift up that banner. But with us all will be
happy and will no more rebel nor destroy one another as under Thy
freedom. Oh, we shall persuade them that they will only become free
when they renounce their freedom to us and submit to us. And shall we
be right or shall we be lying? They will be convinced that we are
right, for they will remember the horrors of slavery and confusion to
which Thy freedom brought them. Freedom, free thought, and science,
will lead them into such straits and will bring them face to face with
such marvels and insoluble mysteries, that some of them, the fierce
and rebellious, will destroy themselves, others, rebellious but weak,
will destroy one another, while the rest, weak and unhappy, will crawl
fawning to our feet and whine to us: ``Yes, you were right, you alone
possess His mystery, and we come back to you, save us from
ourselves!''

```Receiving bread from us, they will see clearly that we take the
bread made by their hands from them, to give it to them, without any
miracle. They will see that we do not change the stones to bread, but
in truth they will be more thankful for taking \page{273} it from our
hands than for the bread itself! For they will remember only too well
that in old days, without our help, even the bread they made turned to
stones in their hands, while since they have come back to us, the very
stones have turned to bread in their hands. Too, too well will they
know the value of complete submission! And until men know that, they
will be unhappy. Who is most to blame for their not knowing it, speak?
Who scattered the flock and sent it astray on unknown paths? But the
flock will come together again and will submit once more, and then it
will be once for all. Then we shall give them the quiet humble
happiness of weak creatures such as they are by nature. Oh, we shall
persuade them at last not to be proud, for Thou didst lift them up and
thereby taught them to be proud. We shall show them that they are
weak, that they are only pitiful children, but that childlike
happiness is the sweetest of all. They will become timid and will look
to us and huddle close to us in fear, as chicks to the hen. They will
marvel at us and will be awe-stricken before us, and will be proud at
our being so powerful and clever, that we have been able to subdue
such a turbulent flock of thousands of millions. They will tremble
impotently before our wrath, their minds will grow fearful, they will
be quick to shed tears like women and children, but they will be just
as ready at a sign from us to pass to laughter and rejoicing, to happy
mirth and childish song. Yes, we shall set them to work, but in their
leisure hours we shall make their life like a child's game, with
children's songs and innocent dance. Oh, we shall allow them even sin,
they are weak and helpless, and they will love us like children
because we allow them to sin. We shall tell them that every sin will
be expiated, if it is done with our permission, that we allow them to
sin because we love them, and the punishment for these sins we take
upon ourselves. And we shall take it upon ourselves, and they will
adore us as their saviours who have taken on themselves their sins
before God. And they will have no secrets from us. We shall allow or
forbid them to live with their wives and mistresses, to have or not to
have chil\-dren---ac\-cord\-ing to whether they have been obedient or
dis\-o\-be\-di\-ent---and they will submit to us gladly and
cheerfully. The most painful secrets of their conscience, all, all
they will bring to us, and we shall have an answer for all. And they
will be glad to believe our answer, for it will save them from the
great anxiety and terrible agony they endure at present in making a
free decision for themselves. And all will be happy, all the
\page{274} millions of creatures except the hundred thousand who rule
over them. For only we, we who guard the mystery, shall be unhappy.
There will be thousands of millions of happy babes, and a hundred
thousand sufferers who have taken upon themselves the curse of the
knowledge of good and evil. Peacefully they will die, peacefully they
will expire in Thy name, and beyond the grave they will find nothing
but death. But we shall keep the secret, and for their happiness we
shall allure them with the reward of heaven and eternity. Though if
there were anything in the other world, it certainly would not be for
such as they. It is prophesied that Thou wilt come again in victory,
Thou wilt come with Thy chosen, the proud and strong, but we will say
that they have only saved themselves, but we have saved all. We are
told that the harlot who sits upon the beast, and holds in her hands
the \textit{mystery}, shall be put to shame, that the weak will rise
up again, and will rend her royal purple and will strip naked her
loathsome body. But then I will stand up and point out to Thee the
thousand millions of happy children who have known no sin. And we who
have taken their sins upon us for their happiness will stand up before
Thee and say: ``Judge us if Thou canst and darest.'' Know that I fear
Thee not. Know that I too have been in the wilderness, I too have
lived on roots and locusts, I too prized the freedom with which Thou
hast blessed men, and I too was striving to stand among Thy elect,
among the strong and powerful, thirsting ``to make up the number.''
But I awakened and would not serve madness. I turned back and joined
the ranks of those \textit{who have corrected Thy work}. I left the
proud and went back to the humble, for the happiness of the humble.
What I say to Thee will come to pass, and our dominion will be built
up. I repeat, to-morrow Thou shalt see that obedient flock who at a
sign from me will hasten to heap up the hot cinders about the pile on
which I shall burn Thee for coming to hinder us. For if anyone has
ever deserved our fires, it is Thou. To-morrow I shall burn Thee.
Dixi.'''

Ivan stopped. He was carried away as he talked and spoke with
excitement; when he had finished, he suddenly smiled.

% Without '\linebreak', an overfull hbox warning:

Alyosha had listened in silence; towards the end he was greatly moved
and \linebreak[4] seemed several times on the point of interrupting,
but restrained himself. Now his words came with a rush.

``But\ldots that's absurd!'' he cried, flushing. ``Your poem is in
praise of Jesus, not in blame of Him---as you meant it \page{275} to
be. And who will believe you about freedom? Is that the way to
understand it? That's not the idea of it in the Orthodox Church....
That's Rome, and not even the whole of Rome, it's false---those are
the worst of the Catholics, the Inquisitors, the Jesuits!\ldots And
there could not be such a fantastic creature as your Inquisitor. What
are these sins of mankind they take on themselves? Who are these
keepers of the mystery who have taken some curse upon themselves for
the happiness of mankind? When have they been seen? We know the
Jesuits, they are spoken ill of, but surely they are not what you
describe? They are not that at all, not at all.... They are simply the
Romish army for the earthly sovereignty of the world in the future,
with the Pontiff of Rome for Emperor\ldots that's their ideal, but
there's no sort of mystery or lofty melancholy about it.... It's
simple lust of power, of filthy earthly gain, of
dom\-i\-na\-tion---some\-thing like a universal serfdom with them as
mas\-ters---that's all they stand for. They don't even believe in God
perhaps. Your suffering Inquisitor is a mere fantasy.''

``Stay, stay,'' laughed Ivan. ``how hot you are! A fantasy you say,
let it be so! Of course it's a fantasy. But allow me to say: do you
really think that the Roman Catholic movement of the last centuries is
actually nothing but the lust of power, of filthy earthly gain? Is
that Father Pa\"{i}ssy's teaching?''

``No, no, on the contrary, Father Pa\"{i}ssy did once say something
rather the same as you\ldots but of course it's not the same, not a
bit the same,'' Alyosha hastily corrected himself.

``A precious admission, in spite of your `not a bit the same.' I ask
you why your Jesuits and Inquisitors have united simply for vile
material gain? Why can there not be among them one martyr oppressed by
great sorrow and loving humanity? You see, only suppose that there was
one such man among all those who desire nothing but filthy material
gain---if there's only one like my old Inquisitor, who had himself
eaten roots in the desert and made frenzied efforts to subdue his
flesh to make himself free and perfect. But yet all his life he loved
humanity, and suddenly his eyes were opened, and he saw that it is no
great moral blessedness to attain perfection and freedom, if at the
same time one gains the conviction that millions of God's creatures
have been created as a mockery, that they will never be capable of
using their freedom, that these poor rebels can never turn into giants
to complete the tower, that it was not for such geese that the great
idealist dreamt his dream of harmony. Seeing \page{276} all that he
turned back and joined---the clever people. Surely that could have
happened?''

``Joined whom, what clever people?'' cried Alyosha, completely carried
away. ``They have no such great cleverness and no mysteries and
secrets.... Perhaps nothing but Atheism, that's all their secret. Your
inquisitor does not believe in God, that's his secret!''

% NOTE: text has 'council' for 'counsel'

``What if it is so! At last you have guessed it. It's perfectly true,
it's true that that's the whole secret, but isn't that suffering, at
least for a man like that, who has wasted his whole life in the desert
and yet could not shake off his incurable love of humanity? In his old
age he reached the clear conviction that nothing but the advice of the
great dread spirit could build up any tolerable sort of life for the
feeble, unruly, `incomplete, empirical creatures created in jest.' And
so, convinced of this, he sees that he must follow the counsel of the
wise spirit, the dread spirit of death and destruction, and therefore
accept lying and deception, and lead men consciously to death and
destruction, and yet deceive them all the way so that they may not
notice where they are being led, that the poor blind creatures may at
least on the way think themselves happy. And note, the deception is in
the name of Him in Whose ideal the old man had so fervently believed
all his life long. Is not that tragic? And if only one such stood at
the head of the whole army `filled with the lust of power only for the
sake of filthy gain'---would not one such be enough to make a tragedy?
More than that, one such standing at the head is enough to create the
actual leading idea of the Roman Church with all its armies and
Jesuits, its highest idea. I tell you frankly that I firmly believe
that there has always been such a man among those who stood at the
head of the movement. Who knows, there may have been some such even
among the Roman Popes. Who knows, perhaps the spirit of that accursed
old man who loves mankind so obstinately in his own way, is to be
found even now in a whole multitude of such old men, existing not by
chance but by agreement, as a secret league formed long ago for the
guarding of the mystery, to guard it from the weak and the unhappy, so
as to make them happy. No doubt it is so, and so it must be indeed. I
fancy that even among the Masons there's something of the same mystery
at the bottom, and that that's why the Catholics so detest the Masons
as their rivals breaking up the unity of the idea, while it is so
essential that there should be one flock and one shepherd....
\page{277} But from the way I defend my idea I might be an author
impatient of your criticism. Enough of it.''

``You are perhaps a Mason yourself!'' broke suddenly from Alyosha.
``You don't believe in God,'' he added, speaking this time very
sorrowfully. He fancied besides that his brother was looking at him
ironically. ``How does your poem end?'' he asked, suddenly looking
down. ``Or was it the end?''

``I meant to end it like this. When the Inquisitor ceased speaking he
waited some time for his Prisoner to answer him. His silence weighed
down upon him. He saw that the Prisoner had listened intently all the
time, looking gently in his face and evidently not wishing to reply.
The old man longed for Him to say something, however bitter and
terrible. But He suddenly approached the old man in silence and softly
kissed him on his bloodless aged lips. That was all his answer. The
old man shuddered. His lips moved. He went to the door, opened it, and
said to Him: `Go, and come no more\ldots come not at all, never,
never!' And he let Him out into the dark alleys of the town. The
Prisoner went away.''

``And the old man?''

``The kiss glows in his heart, but the old man adheres to his idea.''

``And you with him, you too?'' cried Alyosha, mournfully.

Ivan laughed.

``Why, it's all nonsense, Alyosha. It's only a senseless poem of a
senseless student, who could never write two lines of verse. Why do
you take it so seriously? Surely you don't suppose I am going straight
off to the Jesuits, to join the men who are correcting His work? Good
Lord, it's no business of mine. I told you, all I want is to live on
to thirty, and then\ldots dash the cup to the ground!''

``But the little sticky leaves, and the precious tombs, and the blue
sky, and the woman you love! How will you live, how will you love
them?'' Alyosha cried sorrowfully. ``With such a hell in your heart
and your head, how can you? No, that's just what you are going away
for, to join them\ldots if not, you will kill yourself, you can't
endure it!''

``There is a strength to endure everything,'' Ivan said with a cold
smile.

``The strength of the Karamazovs---the strength of the Karamazov
baseness.''

\page{278}``To sink into debauchery, to stifle your soul with
corruption, yes?''

``Possibly even that\ldots only perhaps till I am thirty I shall
escape it, and then---''

``How will you escape it? By what will you escape it? That's
impossible with your ideas.''

``In the Karamazov way, again.''

```Everything is lawful,' you mean? Everything is lawful, is that
it?''

Ivan scowled, and all at once turned strangely pale.

``Ah, you've caught up yesterday's phrase, which so offended
Mi\"{u}sov---and which Dmitri pounced upon so na\"{i}vely and
paraphrased!'' he smiled queerly. ``Yes, if you like, `everything is
lawful' since the word has been said. I won't deny it. And Mitya's
version isn't bad.''

Alyosha looked at him in silence.

``I thought that going away from here I have you at least,'' Ivan said
suddenly, with unexpected feeling; ``but now I see that there is no
place for me even in your heart, my dear hermit. The formula, `all is
lawful,' I won't re\-nounce---will you renounce me for that, yes?''

Alyosha got up, went to him and softly kissed him on the lips.

``That's plagiarism,'' cried Ivan, highly delighted. ``You stole that
from my poem. Thank you though. Get up, Alyosha, it's time we were
going, both of us.''

They went out, but stopped when they reached the entrance of the
restaurant.

``Listen, Alyosha,'' Ivan began in a resolute voice, ``if I am really
able to care for the sticky little leaves I shall only love them,
remembering you. It's enough for me that you are somewhere here, and I
shan't lose my desire for life yet. Is that enough for you? Take it as
a declaration of love if you like. And now you go to the right and I
to the left. And it's enough, do you hear, enough. I mean even if I
don't go away to-morrow (I think I certainly shall go) and we meet
again, don't say a word more on these subjects. I beg that
particularly. And about Dmitri too, I ask you specially never speak to
me again,'' he added, with sudden irritation; ``it's all exhausted, it
has all been said over and over again, hasn't it? And I'll make you
one promise in return for it. When at thirty, I want to `dash the cup
to the ground,' wherever I may be I'll come to have one more talk with
you, even though it were from America, \page{279} you may be sure of
that. I'll come on purpose. It will be very interesting to have a look
at you, to see what you'll be by that time. It's rather a solemn
promise, you see. And we really may be parting for seven years or ten.
Come, go now to your Pater Seraphicus, he is dying. If he dies without
you, you will be angry with me for having kept you. Good-bye, kiss me
once more; that's right, now go.''

Ivan turned suddenly and went his way without looking back. It was
just as Dmitri had left Alyosha the day before, though the parting had
been very different. The strange resemblance flashed like an arrow
through Alyosha's mind in the distress and dejection of that moment.
He waited a little, looking after his brother. He suddenly noticed
that Ivan swayed as he walked and that his right shoulder looked lower
than his left. He had never noticed it before. But all at once he
turned too, and almost ran to the monastery. It was nearly dark, and
he felt almost frightened; something new was growing up in him for
which he could not account. The wind had risen again as on the
previous evening, and the ancient pines murmured gloomily about him
when he entered the hermitage copse. He almost ran. ``Pater
Se\-ra\-phi\-cus---he got that name from some\-where---where from?''
Alyosha wondered. ``Ivan, poor Ivan, and when shall I see you
again?\ldots Here is the hermitage. Yes, yes, that he is, Pater
Seraphicus, he will save me---from him and for ever!''

Several times afterwards he wondered how he could on leaving Ivan so
completely forget his brother Dmitri, though he had that morning, only
a few hours before, so firmly resolved to find him and not to give up
doing so, even should he be unable to return to the monastery that
night.

