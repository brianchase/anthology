
\author{Margaret Fuller}
\authdate{1810--1850}
\textdate{1845}
\addon{Excerpt}
\chapter[Woman in the Nineteenth Century, excerpt]{Woman in the
Nineteenth Century}
\source{fuller1893a}

\page{15}\begin{verse}
``Frailty, thy name is \textsc{Woman}.''\\
``The Earth waits for her Queen.''
\end{verse}

\noindent The connection between these quotations may not be obvious,
but it is strict. Yet would any contradict us, if we made them
applicable to the other side, and began also,

\begin{verse}
Frailty, thy name is \textsc{Man}.\\
The Earth waits for its King?
\end{verse}

\noindent Yet Man, if not yet fully installed in his powers, has given
much earnest of his claims. Frail he is indeed,---how frail! how
impure! Yet often has the vein of gold displayed itself amid the baser
ores, and Man has appeared before us in princely promise worthy of his
future.

If, oftentimes, we see the prodigal son feeding on the husks in the
fair field no more his own, anon we raise the eyelids, heavy from
bitter tears, to behold in him the radiant apparition of genius and
love, demanding not less than the all of goodness, power and beauty.
We see that in him the largest claim finds a due foundation. \page{16}
That claim is for no partial sway, no exclusive possession. He cannot
be satisfied with any one gift of life, any one department of
knowledge or telescopic peep at the heavens. He feels himself called
to understand and aid Nature, that she may, through his
intelligence, be raised and interpreted; to be a student of, and
servant to, the universe-spirit; and king of his planet, that, as an
angelic minister, he may bring it into conscious harmony with the law
of that spirit.

In clear, triumphant moments, many times, has rung through the spheres
the prophecy of his jubilee; and those moments, though past in time,
have been translated into eternity by thought; the bright signs they
left hang in the heavens, as single stars or constellations, and,
already, a thickly sown radiance consoles the wanderer in the darkest
night. Other heroes since Hercules have fulfilled the zodiac of
beneficent labors, and then given up their mortal part to the fire
without a murmur; while no God dared deny that they should have their
reward,

\begin{verse}
\hspace{1.1em} Siquis tamen, Hercule, siquis\\
Forte Deo doliturus erit, data pr\ae mia nollet,\\
Sed meruise dari sciet, invitua que probabit,\\
\hspace{1.1em} Assensere Dei.
\end{verse}

Sages and lawgivers have bent their whole nature to the search for
truth, and thought themselves happy if they could buy, with the
sacrifice of all temporal ease and pleasure, one seed for the future
Eden. Poets and priests have strung the lyre with the heart-strings,
poured out their best blood upon the altar, which, reared anew
\page{17} from age to age, shall at last sustain the flame pure enough
to rise to highest heaven. Shall we not name with as deep a
benediction those who, if not so immediately, or so consciously, in
connection with the eternal truth, yet, led and fashioned by a divine
instinct, serve no less to develop and interpret the open secret of
love passing into life, energy creating for the purpose of happiness;
the artist whose hand, drawn by a pre\"{e}xistent harmony to a certain
medium, moulds it to forms of life more highly and completely
organized than are seen elsewhere, and, by carrying out the intention
of nature, reveals her meaning to those who are not yet wise enough to
divine it; the philosopher who listens steadily for laws and causes,
and from those obvious infers those yet unknown; the historian who, in
faith that all events must have their reason and their aim, records
them, and thus fills archives from which the youth of prophets may be
fed; the man of science dissecting the statements, testing the facts
and demonstrating order, even where he cannot its purpose?

Lives, too, which bear none of these names, have yielded tones of no
less significance. The candlestick set in a low place has given light
as faithfully, where it was needed, as that upon the hill. In close
alleys, in dismal nooks, the Word has been read as distinctly, as when
shown by angels to holy men in the dark prison. Those who till a spot
of earth scarcely larger than is wanted for a grave, have deserved
that the sun should shine upon its sod till violets answer.

So great has been, from time to time, the promise, \page{18} that, in
all ages, men have said the gods themselves came down to dwell with
them; that the All-Creating wandered on the earth to taste, in a
limited nature, the sweetness of virtue; that the All-Sustaining
incarnated himself to guard, in space and time, the destinies of this
world; that heavenly genius dwelt among the shepherds, to sing to them
and teach them how to sing. Indeed,

\begin{verse}
``Der stets den Hirten gnadig sich bewies.''
\end{verse}

\noindent ``He has constantly shown himself favorable to
shepherds.''

And the dwellers in green pastures and natural students of the stars
were selected to hail, first among men, the holy child, whose life and
death were to present the type of excellence, which has sustained the
heart of so large a portion of mankind in these later generations.

Such marks have been made by the footsteps of \textit{man} (still,
alas! to be spoken of as the \textit{ideal} man), wherever he has
passed through the wilderness of \textit{men}, and whenever the
pigmies stepped in one of those, they felt dilate within the breast
somewhat that promised nobler stature and purer blood. They were
impelled to forsake their evil ways of decrepit scepticism and
covetousness of corruptible possessions. Convictions flowed in upon
them. They, too, raised the cry: God is living, now, to-day; and all
beings are brothers, for they are his children. Simple words enough,
yet which only angelic natures can use or hear in their full, free
sense.

These were the triumphant moments; but soon the lower nature took its
turn, and the era of a truly human life was postponed.

\page{19}Thus is man still a stranger to his inheritance, still a
pleader, still a pilgrim. Yet his happiness is secure in the end. And
now, no more a glimmering consciousness, but assurance begins to be
felt and spoken, that the highest ideal Man can form of his own powers
is that which he is destined to attain. Whatever the soul knows how to
seek, it cannot fail to obtain. This is the Law and the Prophets.
Knock and it shall be opened; seek and ye shall find. It is
demonstrated; it is a maxim. Man no longer paints his proper nature in
some form, and says, ``Prometheus had it; it is God-like;'' but ``Man
must have it; it is human.'' However disputed by many, however
ignorantly used, or falsified by those who do receive it, the fact of
an universal, unceasing revelation has been too clearly stated in
words to be lost sight of in thought; and sermons preached from the
text, ``Be ye perfect,'' are the only sermons of a pervasive and
deep-searching influence.

But, among those who meditate upon this text, there is a great
difference of view as to the way in which perfection shall be sought.

``Through the intellect,'' say some. ``Gather from every growth of
life its seed of thought; look behind every symbol for its law; if
thou canst \textit{see} clearly, the rest will follow.''

``Through the life,'' say others. ``Do the best thou knowest to-day.
Shrink not from frequent error in this gradual, fragmentary state.
Follow thy light for as much as it will show thee; be faithful as far
as thou canst, in hope that faith presently will lead to sight. Help
\page{20} others, without blaming their need of thy help. Love much,
and be forgiven.''

``It needs not intellect, needs not experience,'' says a third. ``If
you took the true way, your destiny would be accomplished in a purer
and more natural order. You would not learn through facts of thought
or action, but express through them the certainties of wisdom. In
quietness yield thy soul to the causal soul. Do not disturb thy
apprenticeship by premature effort; neither check the tide of
instruction by methods of thy own. Be still; seek not, but wait in
obedience. Thy commission will be given.''

Could we indeed say what we want, could we give a description of the
child that is lost, he would be found. As soon as the soul can affirm
clearly that a certain demonstration is wanted, it is at hand. When
the Jewish prophet described the Lamb, as the expression of what was
required by the coming era, the time drew nigh. But we say not, see
not as yet, clearly, what we would. Those who call for a more
triumphant expression of love, a love that cannot be crucified, show
not a perfect sense of what has already been given. Love has already
been expressed, that made all things new, that gave the worm its place
and ministry as well as the eagle; a love to which it was alike to
descend into the depths of hell, or to sit at the right hand of the
Father.

Yet, no doubt, a new manifestation is at hand, a new hour in the day
of Man. We cannot expect to see any one sample of completed being,
when the mass of men still lie engaged in the sod, or use the freedom
of their \page{21} limbs only with wolfish energy. The tree cannot
come to flower till its root be free from the cankering worm, and its
whole growth open to air and light. While any one is base, none can be
entirely free and noble. Yet something new shall presently be shown
of the life of man, for hearts crave, if minds do not know how to ask
it.

Among the strains of prophecy, the following, by an earnest mind of a
foreign land, written some thirty years ago, is not yet outgrown; and
it has the merit of being a positive appeal from the heart, instead of
a critical declaration what Man should \textit{not} do.

``The ministry of Man implies that he must be filled from the divine
fountains which are being engendered through all eternity, so that, at
the mere name of his master, he may be able to cast all his enemies
into the abyss; that he may deliver all parts of nature from the
barriers that imprison them; that he may purge the terrestrial
atmosphere from the poisons that infect it; that he may preserve the
bodies of men from the corrupt influences that surround, and the
maladies that afflict them; still more, that he may keep their souls
pure from the malignant insinuations which pollute, and the gloomy
images that obscure them; that he may restore its serenity to the
Word, which false words of men fill with mourning and sadness; that he
may satisfy the desires of the angels, who await from him the
development of the marvels of nature; that, in fine, his world may be
filled with God, as eternity is.''\footnote{St. Martin.}

Another attempt we will give, by an obscure observer \page{22} of our
own day and country, to draw some lines of the desired image. It was
suggested by seeing the design of Crawford's Orpheus, and connecting
with the circumstance of the American, in his garret at Rome, making
choice of this subject, that of Americans here at home showing such
ambition to represent the character, by calling their prose and verse
``Orphic sayings''---``Orphics.'' We wish we could add that they
have shown that musical apprehension of the progress of Nature through
her ascending gradations which entitled them so to do, but their
attempts are frigid, though sometimes grand; in their strain we are
not warmed by the fire which fertilized the soil of Greece.

Orpheus was a lawgiver by theocratic commission. He understood nature,
and made her forms move to his music. He told her secrets in the form
of hymns, Nature as seen in the mind of God. His soul went forth
toward all beings, yet could remain sternly faithful to a chosen type
of excellence. Seeking what he loved, he feared not death nor hell;
neither could any shape of dread daunt his faith in the power of the
celestial harmony that filled his soul.

It seemed significant of the state of things in this country, that the
sculptor should have represented the seer at the moment when he was
obliged with his hand to shade his eyes.

% NOTE: The format of the poem seems to call for indenting every other
% line, starting with the second. But in the original, the last line
% is not indented. Rather, it is positioned oddly to the left of the
% other lines.

\begin{verse}
Each Orpheus must to the depths descend;\\
\hspace{1.1em} For only thus the Poet can be wise;\\
Must make the sad Persephone his friend,\\
\hspace{1.1em} And buried love to second life arise;\\
\page{23}Again his love must lose through too much love,\\
\hspace{1.1em} Must lose his life by living life too true,\\
For what he sought below is passed above,\\
\hspace{1.1em} Already done is all that he would do;\\
Must tune all being with his single lyre,\\
\hspace{1.1em} Must melt all rocks free from their primal pain,\\
Must search all nature with his one soul's fire,\\
\hspace{1.1em} Must bind anew all forms in heavenly chain.\\
If he already sees what he must do,\\
\hspace{1.1em} Well may he shade his eyes from the far-shining view.
\end{verse}

A better comment could not be made on what is required to perfect Man,
and place him in that superior position for which he was designed,
than by the interpretation of Bacon upon the legends of the Syren
coast. ``When the wise Ulysses passed,'' says he, ``he caused his
mariners to stop their ears with wax, knowing there was in them no
power to resist the lure of that voluptuous song. But he, the much
experienced man, who wished to be experienced in all, and use all to
the service of wisdom, desired to hear the song that he might
understand its meaning. Yet, distrusting his own power to be firm in
his better purpose, he caused himself to be bound to the mast, that he
might be kept secure against his own weakness. But Orpheus passed
unfettered, so absorbed in singing hymns to the gods that he could not
even hear those sounds of degrading enchantment.''

Meanwhile, not a few believe, and men themselves have expressed the
opinion, that the time is come when Eurydice is to call for an
Orpheus, rather than Orpheus for Eurydice; that the idea of Man,
however imperfectly brought out, has been far more so than that of
Woman: \page{24} that she, the other half of the same thought, the
other chamber of the heart of life, needs now take her turn in the
full pulsation, and that improvement in the daughters will best aid in
the reformation of the sons of this age.

It should be remarked that, as the principle of liberty is better
understood, and more nobly interpreted, a broader protest is made in
behalf of Woman. As men become aware that few men have had a fair
chance, they are inclined to say that no women have had a fair chance.
The French Revolution, that strangely disguised angel, bore witness in
favor of Woman, but interpreted her claims no less ignorantly than
those of Man. Its idea of happiness did not rise beyond outward
enjoyment, unobstructed by the tyranny of others. The title it gave
was ``citoyen,'' ``citoyenne;'' and it is not unimportant to Woman
that even this species of equality was awarded her. Before, she could
be condemned to perish on the scaffold for treason, not as a citizen,
but as a subject. The right with which this title then invested a
human being was that of bloodshed and license. The Goddess of Liberty
was impure. As we read the poem addressed to her, not long since, by
Beranger, we can scarcely refrain from tears as painful as the tears
of blood that flowed when ``such crimes were committed in her name.''
Yes! Man, born to purify and animate the unintelligent and the cold,
can, in his madness, degrade and pollute no less the fair and the
chaste. Yet truth was prophesied in the ravings of that hideous fever,
caused by long ignorance and abuse. Europe is conning a valued lesson
\page{25} from the blood-stained page. The same tendencies, further
unfolded, will bear good fruit in this country.

Yet, by men in this country, as by the Jews, when Moses was leading
them to the promised land everything has been done that inherited
depravity could do, to hinder the promise of Heaven from its
fulfilment. The cross, here as elsewhere, has been planted only to be
blasphemed by cruelty and fraud. The name of the Prince of Peace has
been profaned by all kinds of injustice toward the Gentile whom he
said he came to save. But I need not speak of what has been done
towards the Red Man, the Black Man. Those deeds are the scoff of the
world; and they have been accompanied by such pious words that the
gentlest would not dare to intercede with ``Father, forgive them, for
they know not what they do.''

Here, as elsewhere, the gain of creation consists always in the growth
of individual minds, which live and aspire, as flowers bloom and birds
sing, in the midst of morasses; and in the continual development of
that thought, the thought of human destiny, which is given to eternity
adequately to express, and which ages of failure only seemingly
impede. Only seemingly; and whatever seems to the contrary, this
country is as surely destined to elucidate a great moral law, as
Europe was to promote the mental culture of Man.

Though the national independence be blurred by the servility of
individuals; though freedom and equality have been proclaimed only to
leave room for a monstrous display of slave-dealing and slave-keeping;
though the \page{26} free American so often feels himself free, like
the Roman, only to pamper his appetites and his indolence through the
misery of his fellow-beings; still it is not in vain that the verbal
statement has been made, ``All men are born free and equal.'' There it
stands, a golden certainty wherewith to encourage the good, to shame
the bad. The New World may be called clearly to perceive that it
incurs the utmost penalty if it reject or oppress the sorrowful
brother. And, if men are deaf, the angels hear. But men cannot be
deaf. It is inevitable that an external freedom, an independence of
the encroachments of other men, such as has been achieved for the
nation, should be so also for every member of it. That which has once
been clearly conceived in the intelligence cannot fail, sooner or
later, to be acted out. It has become a law as irrevocable as that of
the Medes in their ancient dominion; men will privately sin against
it, but the law, as expressed by a leading mind of the age,

\begin{verse}
``Tutti fatti a sembianza d'un Solo,\\
Figli tutti d'un solo risoatto,\\
In qual'ora, in qual parte del suolo\\
Trascorriamo quest' aura vital,\\
Siam fratelli, siam stretti ad un patto:\\
Maladetto colui che lo infrange,\\
Che s'innalza sul fiacco che piange\\
Che contrista uno spirto immortal.''\footnote{Mansoni.}

``All made in the likeness of the One,\\
\hspace{1.1em} All children of one ransom,\\
In whatever hour, in whatever part of the soil,\\
\hspace{1.1em} We draw this vital air,\\
\page{27}We are brothers; we must be bound by one compact;\\
\hspace{1.1em} Accursed he who infringes it,\\
Who raises himself upon the weak who weep,\\
\hspace{1.1em} Who saddens an immortal spirit.''
\end{verse}

This law cannot fail of universal recognition. Accursed be he who
willingly saddens an immortal spir\-it---doomed to infamy in later,
wiser ages, doomed in future stages of his own being to deadly
penance, only short of death. Accursed be he who sins in ignorance, if
that ignorance be caused by sloth.

We sicken no less at the pomp than the strife of words. We feel that
never were lungs so puffed with the wind of declamation, on moral and
religious subjects, as now. We are tempted to implore these
``word-heroes,'' these word-Catos, word-Christs, to beware of
cant\footnote{Dr. Johnson's one piece of advice should be written on
every door: ``Clear your mind of cant.'' But Byron, to whom it was so
acceptable, in clearing away the noxious vine, shook down the
building. Sterling's emendation is worthy of honor: ``Realize your
cant, not cast it off.''} above all things; to remember that hypocrisy
is the most hopeless as well as the meanest of crimes, and that those
must surely be polluted by it, who do not reserve a part of their
morality and religion for private use. Landor says that he cannot have
a great deal of mind who cannot afford to let the larger part of it
lie fallow; and what is true of genius is not less so of virtue. The
tongue is a valuable member, but should appropriate but a small part
of the vital juices that are needful all over the body. We feel that
the mind may \page{28} ``grow black and rancid in the smoke'' even
``of altars.'' We start up from the harangue to go into our closet and
shut the door. There inquires the spirit, ``Is this rhetoric the bloom
of healthy blood, or a false pigment artfully laid on?'' And yet again
we know where is so much smoke, must be some fire; with so much talk
about virtue and freedom, must be mingled some desire for them; that
it cannot be in vain that such have become the common topics of
conversation among men, rather than schemes for tyranny and plunder,
that the very newspapers see it best to proclaim themselves
``Pilgrims,'' ``Puritans,'' ``Heralds of Holiness.'' The king that
maintains so costly a retinue cannot be a mere boast, or Carabbas
fiction. We have waited here long in the dust; we are tired and
hungry; but the triumphal procession must appear at last.

Of all its banners, none has been more steadily upheld, and under none
have more valor and willingness for real sacrifices been shown, than
that of the champions of the enslaved African. And this band it is,
which, partly from a natural following out of principles, partly
because many women have been prominent in that cause, makes, just now,
the warmest appeal in behalf of Woman.

Though there has been a growing liberality on this subject, yet
society at large is not so prepared for the demands of this party, but
that its members are, and will be for some time, coldly regarded as
the Jacobins of their day.

``Is it not enough,'' cries the irritated trader, ``that you have done
all you could to break up the national \page{29} union, and thus
destroy the prosperity of our country but now you must be trying to
break up family union, to take my wife away from the cradle and the
kitchen-hearth to vote at polls, and preach from a pulpit? Of course,
if she does such things, she cannot attend to those of her own sphere.
She is happy enough as she is. She has more leisure than I
have,---every means of improvement, every indulgence.''

``Have you asked her whether she was satisfied with these
\textit{indulgences?}''

``No, but I know she is. She is too amiable to desire what would make
me unhappy, and too judicious to wish to step beyond the sphere of her
sex. I will never consent to have our peace disturbed by any such
discussions.''

```Consent---you?' it is not consent from you that is in question---it
is assent from your wife.''

``Am not I the head of my house?''

``You are not the head of your wife. God has given her a mind of her
own.''

``I am the head, and she the heart.''

``God grant you play true to one another, then! I suppose I am to be
grateful that you did not say she was only the hand. If the head
represses no natural pulse of the heart, there can be no question as
to your giving your consent. Both will be of one accord, and there
needs but to present any question to get a full and true answer. There
is no need of precaution, of indulgence, nor consent. But our doubt is
whether the heart \textit{does} consent with the head, or only obeys
its decrees with a passiveness that precludes the exercise of its
natural \page{30} powers, or a repugnance that turns sweet qualities
to bitter, or a doubt that lays waste the fair occasions of life. It
is to ascertain the truth that we propose some liberating
measures.''

Thus vaguely are these questions proposed and discussed at present.
But their being proposed at all implies much thought, and suggests
more. Many women are considering within themselves what they need that
they have not, and what they can have if they find they need it. Many
men are considering whether women are capable of being and having more
than they are and have, \textit{and} whether, if so, it will be best
to consent to improvement in their condition.

This morning, I open the Boston `` Daily Mail,'' and find in its
``poet's corner'' a translation of Schiller's ``Dignity of Woman.'' In
the advertisement of a book on America, I see in the table of contents
this sequence, ``Republican Institutions. American Slavery. American
Ladies.''

I open the ``\textit{Deutsche Schnellpost},'' published in New York,
and find at the head of a column, \textit{Judenund Frauen-emancipation
in Ungarn}---``Emancipation of Jews and Women in Hungary.''

The past year has seen action in the Rhode Island legislature, to
secure married women rights over their own property, where men showed
that a very little examination of the subject could teach them much;
an article in the Democratic Review on the same subject more largely
considered, written by a woman, impelled, it is said, by glaring wrong
to a distinguished friend, hav-\page{31}ing shown the defects in the
existing laws, and the state of opinion from which they spring; and an
answer from the revered old man, J. Q. Adams, in some respects the
Phocion of his time, to an address made him by some ladies. To this
last I shall again advert in another place.

These symptoms of the times have come under my view quite
accidentally: one who seeks, may, each month or week, collect more.

The numerous party, whose opinions are already labeled and adjusted
too much to their mind to admit of any new light, strive, by lectures
on some model-woman of bride-like beauty and gentleness, by writing
and lending little treatises, intended to mark out with precision the
limits of Woman's sphere, and Woman's mission, to prevent other than
the rightful shepherd from climbing the wall, or the flock from
using any chance to go astray.

Without enrolling ourselves at once on either side, let us look upon
the subject from the best point of view which to-day offers; no
better, it is to be feared, than a high house-top. A high hill-top, or
at least a cathedral-spire, would be desirable.

It may well be an Anti-Slavery party that pleads for Woman, if we
consider merely that she does not hold property on equal terms with
men; so that, if a husband dies without making a will, the wife,
instead of taking at once his place as head of the family, inherits
only a part of his fortune, often brought him by herself, as if she
were a child, or ward only, not an equal partner.

We will not speak of the innumerable instances in \page{32} which
profligate and idle men live upon the earnings of industrious wives;
or if the wives leave them, and take with them the children, to
perform the double duty of mother and father, follow from place to
place, and threaten to rob them of the children, if deprived of the
rights of a husband, as they call them, planting themselves in their
poor lodgings, frightening them into paying tribute by taking from
them the children, running into debt at the expense of these otherwise
so overtasked helots. Such instances count up by scores within my own
memory. I have seen the husband who had stained himself by a long
course of low vice, till his wife was wearied from her heroic
forgiveness, by finding that his treachery made it useless, and that
if she would provide bread for herself and her children, she must be
separate from his ill fame---I have known this man come to install
himself in the chamber of a woman who loathed him, and say she should
never take food without his company. I have known these men steal
their children, whom they knew they had no means to maintain, take
them into dissolute company, expose them to bodily danger, to frighten
the poor woman, to whom, it seems, the fact that she alone had borne
the pangs of their birth, and nourished their infancy, does not give
an equal right to them. I do believe that this mode of
kid\-nap\-ping---and it is frequent enough in all classes of
so\-ci\-ety---will be by the next age viewed as it is by Heaven now,
and that the man who avails himself of the shelter of men's laws to
steal from a mother her own children, or arrogate any superior right
in them, save that of superior \page{33} virtue, will bear the stigma
he deserves, in common with him who steals grown men from their
mother-land, their hopes, and their homes.

I said, we will not speak of this now; yet I \textit{have} spoken, for
the subject makes me feel too much. I could give instances that would
startle the most vulgar and callous; but I will not, for the public
opinion of their own sex is already against such men, and where cases
of extreme tyranny are made known, there is private action in the
wife's favor. But she ought not to need this, nor, I think, can she
long. Men must soon see that as, on their own ground, Woman is the
weaker party, she ought to have legal protection, which would make
such oppression impossible. But I would not deal with ``atrocious
instances,'' except in the way of illustration, neither demand from
men a partial redress in some one matter, but go to the root of the
whole. If principles could be established, particulars would adjust
themselves aright. Ascertain the true destiny of Woman; give her
legitimate hopes, and a standard within herself; marriage and all
other relations would by degrees be harmonized with these.

But to return to the historical progress of this matter. Knowing that
there exists in the minds of men a tone of feeling toward women as
toward slaves, such as is expressed in the common phrase, ``Tell that
to women and children;'' that the infinite soul can only work through
them in already ascertained limits; that the gift of reason, Man's
highest prerogative, is allotted to them in much lower degree; that
they must be kept from mis-\page{34}chief and melancholy by being
constantly engaged in active labor, which is to be furnished and
directed by those better able to think, \&c., \&c.,---we need not
multiply instances, for who can review the experience of last week
without recalling words which imply, whether in jest or earnest, these
views, or views like these,---know\-ing this, can we wonder that many
reformers think that measures are not likely to be taken in behalf of
women, unless their wishes could be publicly represented by women?

``That can never be necessary,'' cry the other side. ``All men are
privately influenced by women; each has his wife, sister, or female
friends, and is too much biased by these relations to fail of
representing their interests; and, if this is not enough, let them
propose and enforce their wishes with the pen. The beauty of home
would be destroyed, the delicacy of the sex be violated, the dignity
of halls of legislation degraded, by an attempt to introduce them
there. Such duties are inconsistent with those of a mother;'' and then
we have ludicrous pictures of ladies in hysterics at the polls, and
senate-chambers filled with cradles.

But if, in reply, we admit as truth that Woman seems destined by
nature rather for the inner circle, we must add that the arrangements
of civilized life have not been, as yet, such as to secure it to her.
Her circle, if the duller, is not the quieter. If kept from
''excitement,'' she is not from drudgery. Not only the Indian squaw
carries the burdens of the camp, but the favorites of Louis XIV.
accompany him in his journeys, and the \page{35} washerwoman stands at
her tub, and carries home her work at all seasons, and in all states
of health. Those who think the physical circumstances of Woman would
make a part in the affairs of national government unsuitable, are by
no means those who think it impossible for negresses to endure
field-work, even during pregnancy, or for sempstresses to go through
their killing labors.

As to the use of the pen, there was quite as much opposition to
Woman's possessing herself of that help to free agency as there is now
to her seizing on the rostrum or the desk; and she is likely to draw,
from a permission to plead her cause that way, opposite inferences to
what might be wished by those who now grant it.

As to the possibility of her filling with grace and dignity any such
position, we should think those who had seen the great actresses, and
heard the Quaker preachers of modern times, would not doubt that Woman
can express publicly the fulness of thought and creation, without
losing any of the peculiar beauty of her sex. What can pollute and
tarnish is to act thus from any motive except that something needs to
be said or done. Woman could take part in the processions, the songs,
the dances of old religion; no one fancied her delicacy was impaired
by appearing in public for such a cause.

As to her home, she is not likely to leave it more than she now does
for balls, theatres, meetings for promoting missions, revival
meetings, and others to which she flies, in hope of an animation for
her existence commensurate with what she sees enjoyed by men.
Governors of ladies'-fairs are no less engrossed by such a charge,
than \page{36} the governor of a state by his; presidents of
Washingtonian societies no less away from home than presidents of
conventions. If men look straitly to it, they will find that, unless
their lives are domestic, those of the women will not be. A house is
no home unless it contain food and fire for the mind as well as for
the body. The female Greek, of our day, is as much in the street as
the male to cry, ``What news?'' We doubt not it was the same in
Athens of old. The women, shut out from the market-place, made up for
it at the religious festivals. For human beings are not so constituted
that they can live without expansion. If they do not get it in one
way, they must in another, or perish.

As to men's representing women fairly at present, while we hear from
men who owe to their wives not only all that is comfortable or
graceful, but all that is wise, in the arrangement of their lives, the
frequent remark, ``You cannot reason with a woman,''---when from those
of delicacy, nobleness, and poetic culture, falls the contemptuous
phrase ``women and children,'' and that in no light sally of the hour,
but in works intended to give a permanent statement of the best
experiences,---when not one man, in the million, shall I say? no, not
in the hundred million, can rise above the belief that Woman was made
\textit{for Man},---when such traits as these are daily forced upon
the attention, can we feel that Man will always do justice to the
interests of Woman? Can we think that he takes a sufficiently
discerning and religious view of her office and destiny \textit{ever}
to do her justice, except when prompted by sentiment,---accidentally
or \page{37} transiently, that is, for the sentiment will vary
according to the relations in which he is placed? The lover, the poet,
the artist, are likely to view her nobly. The father and the
philosopher have some chance of liberality; the man of the world, the
legislator for expediency, none.

Under these circumstances, without attaching importance, in
themselves, to the changes demanded by the champions of Woman, we hail
them as signs of the times. We would have every arbitrary barrier
thrown down. We would have every path laid open to Woman as freely as
to Man. Were this done, and a slight temporary fermentation allowed to
subside, we should see crystallizations more pure and of more various
beauty. We believe the divine energy would pervade nature to a degree
unknown in the history of former ages, and that no discordant
collision, but a ravishing harmony of the spheres, would ensue.

Yet, then and only then will mankind be ripe for this, when inward and
outward freedom for Woman as much as for Man shall be acknowledged as
a \textit{right}, not yielded as a concession. As the friend of the
negro assumes that one man cannot by right hold another in bondage, so
should the friend of Woman assume that Man cannot by right lay even
well-meant restrictions on Woman. If the negro be a soul, if the woman
be a soul, apparelled in flesh, to one Master only are they
accountable. There is but one law for souls, and, if there is to be an
interpreter of it, he must come not as man, or son of man, but as son
of God.

Were thought and feeling once so far elevated that \page{38} Man
should esteem himself the brother and friend, but nowise the lord and
tutor, of Woman,---were he really bound with her in equal
worship,---arrangements as to function and employment would be of no
consequence. What Woman needs is not as a woman to act or rule, but as
a nature to grow, as an intellect to discern, as a soul to live
freely and unimpeded, to unfold such powers as were given her when we
left our common home. If fewer talents were given her, yet if allowed
the free and full employment of these, so that she may render back to
the giver his own with usury, she will not complain; nay, I dare to
say she will bless and rejoice in her earthly birth-place, her earthly
lot. Let us consider what obstructions impede this good era, and what
signs give reason to hope that it draws near.

