
\author{Seneca}
\authdate{ca. 4--1 {\smaller BCE} -- 65 {\smaller CE}}
\textdate{ca. 52}
\chapter[Seneca -- De Ira, bk. 2.6--2.10]{De Ira [On Anger]\\\smaller
Book 2.6--2.10}

\nfootnote{\fullcite{seneca1928.1.3}}

\page{1:177}``If,'' some one argues, ``virtue is well disposed toward
what is honourable, it is her duty to feel anger toward what is
base.'' What if he should say that virtue must be both low and great?
And yet this is what he does say---he would have her be both exalted
and debased, since joy on account of a right action is splendid and
glorious, while anger on account of another's sin is mean and
narrow-minded. \page{179} And virtue will never be guilty of
simulating vice in the act of repressing it; anger in itself she
considers reprehensible, for it is in no way better, often even worse,
than those shortcomings which provoke anger. The distinctive and
natural property of virtue is to rejoice and be glad; it no more
comports with her dignity to be angry than to be sad. But sorrow is
the companion of anger, and all anger comes round to this as the
result either of remorse or of defeat. Besides, if it is the part of a
wise ma to be angry at sin, the greater this is the more angry will he
be, and he will be angry often; it follows that the wise man will not
only become angry, but will be prone to anger. But if we believe that
neither great anger nor frequent anger has a place in the mind of a
wise man, is there any reason why we should not free him from this
passion altogether? No limit, surely, can be set if the degree of his
anger is to be determined by each man's deed. For either he will be
unjust if he has equal anger toward unequal delinquencies, or he will
be habitually angry if he blazes up every time crimes give him
warrant.

And what is more unworthy of the wise man than that his passion should
depend upon the wickedness of others? Shall great Socrates lose the
power to carry back home the same look he had brought from home? But
if the wise man is to be angered by base deeds, if he is to be
perturbed and saddened by crimes, surely nothing is more woeful than
the wise man's lot; his whole life will be passed in anger and in
grief. For what moment will there be when he will not see something to
disapprove of? Every time he leaves his house, he will have to walk
among criminals and misers and spendthrifts and \page{181}
prof\-li\-gates---men who are happy in being such. Nowhere will he
turn his eyes without finding something to move them to indignation.
He will give out if he forces himself to be angry every time occasion
requires. All these thousands hurrying to the forum at break of
day---how base their cases, and how much baser are their advocates!
One assails his father's will, which it were more fitting that he
respect; another arraigns his mother at the bar; another comes as an
informer of the very crime in which he is more openly the culprit; the
judge, too, is chosen who will condemn the same deeds that he himself
has committed, and the crowd, misled by the fine voice of a pleader,
shows favour to a wicked cause.

By why recount all the different types? Whenever you see the forum
with its thronging multitude, and the polling-places filled with all
the gathered concourse, and the great Circus where the largest part of
the populace displays itself, you may be sure that just as many vices
are gathered there as men. Among those whom you see in civilian garb
there is no peace; for a slight reward any one of them can be led to
compass the destruction of another; no one makes gain save by
another's loss; the prosperous they hate, the unprosperous they
despise; superiors they loathe, and to inferiors are loathsome; they
are goaded on by opposite desires; they desire for the sake of some
little pleasure or plunder to see the whole world lost. They live as
though they were in a gladiatorial school---those with whom they eat,
they likewise fight. It is a community of wild beasts, only that
beasts are gentle toward each other and refrain from tearing their own
kind, while men \page{183} glut themselves with rending one another.
They differ from the dumb animals in this a\-lone---that animals grow
gentle toward those who feed them, while men in their madness prey
upon the very persons by whom they are nurtured.

Never will the wise man cease to be angry if once he begins. Every
place is full of crime and vice; too many crimes are committed to be
cured by any possible restraint. Men struggle in a mighty rivalry of
wickedness. Every day the desire for wrong-doing is greater, the dread
of it less; all regard for what is better and more just is banished,
lust hurls itself wherever it likes, and crimes are now no longer
covert. They stalk before our very eyes, and wickedness has come to
such a public state, has gained such power over the hearts of all,
that innocence is not rare---it is non-existent. For is it only the
casual man or the few who break the law? On every hand, as if at a
given signal, men rise to level all the barriers of right and wrong:

\begin{verse}
No guest from host is safe, nor daughter's sire\\
From daughter's spouse; e'en brothers' love is rare.\\
The husband doth his wife, she him, ensnare;\\
Ferocious stepdames brew their ghastly banes;\\
The soon too soon his father's years arraigns.
\end{verse}

\noindent And yet how few of all the crimes are these! The poet makes
no mention of the battling camps that claim a common blood, of the
parents and the children sundered by a soldier's oath, of the flames a
Roman hand applied to Rome, of the hostile bands of horsemen that
scour the land to find the hiding-places of citizens proscribed, of
springs defiled by poison, of plague the hand of man has made, of the
trench flung around beleaguered parents, of crowded prisons, of
\page{185} fires that burn whole cities to the ground, of baleful
tyrannies and secret plots for regal power and for subversion of the
state, of acts that now are glorified, but still are crimes so long as
power endures to crush them, rape and lechery and the lust that spares
not even human mouths. Add now to these, public acts of perjury
between nations, broken treaties, and all the booty seized when
resistance could not save it from the stronger, the double-dealings,
the thefts and frauds and debts dis\-owned---for such crimes all three
forums supply not courts enough! If you expect the wise man to be as
angry as the shamefulness of crimes compels, he must not be angry
merely, but go mad.

This rather is what you should think---that no one should be angry at
the mistakes of men. For tell me, should one be angry with those who
move with stumbling footsteps in the dark? with those who do not heed
commands because they are deaf? with children because forgetting the
observance of their duties they watch the games and foolish sports of
their playmates? Would you want to be angry with those who become
weary because they are sick or growing old? Among the various ills to
which humanity is prone there is this be\-sides---the darkness that
fills the mind, and not so much the necessity of going astray, as the
love of straying. That you may not be angry with individuals, you must
forgive mankind at large, you must grant indulgence to the human race.
If you are angry with the young and the old because they sin, be angry
with babes as well; they are destined to sin. But who is angry with
children who are still too young to have the power of discrimination?
Yet to be a human being is an even \page{187} greater and truer excuse
for error than to be a child. This is the lot to which we are
born---we are creatures subject to as many ills of the mind as of the
body, and though our power of discernment is neither blunted nor dull,
yet we make poor use of it and become examples of vice to each other.
If any one follows in the footsteps of others who have taken the wrong
road, should he not be excused because it was the public highway that
led him astray? Upon the individual soldier the commander may
unsheathe all his sternness, but he needs must forbear when the while
army deserts. What, then, keeps the wise man from anger? The great
mass of sinners. He understands both how unjust and how dangerous it
is to grow angry at universal sin.

Whenever Heraclitus went forth from his house and saw all around him
so many men who were living a wretched life---no, rather, were dying a
wretched death---he would weep, and all the joyous and happy people he
met stirred his pity; he was gentle-hearted, but too weak, and was
himself one of those who had need of pity. Democritus, on the other
hand, it is said, never appeared in public without laughing; so little
did the serious pursuits of men seem serious to him. Where in all this
is there room for anger? Everything gives cause for either laughter or
tears.

The wise man will have no anger toward sinners. Do you ask why?
Because he knows that no one is born wise but becomes so, knows that
only the fewest in every age turn out wise, because he has fully
grasped the conditions of human life, and no sensible man becomes
angry with nature. Think you a sane man would marvel because apples do
not hang from \page{189} the brambles of the woodland? Would he marvel
because thorns and briars are not covered with some useful fruit? No
one becomes angry with a fault for which nature stands sponsor. And so
the wise man is kindly and just toward errors, he is not the foe, but
the reformer of sinners, and as he issues forth each day his thought
will be: ``I shall meet many who are in bondage to wine, many who are
lustful, many ungrateful, many grasping, many who are lashed by the
frenzy of ambition.'' He will view all these things in as kindly a way
as a physician views the sick. When the skipper finds that his ship
has sprung her seams and in every part is letting in a copious flow of
water, does he then become angry with the seamen and with the ship
herself? No, he rushes rather to the rescue and shuts out a part of
the flood, a part he bales out, and he closes up the visible openings,
the hidden leaks that secretly let water into the hold he tries to
overcome by ceaseless labour, and he does not relax his effort simply
because as much water springs up as is pumped out. The succor against
continuous and prolific evils must be tenacious, aimed not at their
cessation but against their victory.

