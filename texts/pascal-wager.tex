
\author{Blaise Pascal}
\authdate{1623--1662}
\textdate{ca. 1658--1662}
\addon{Pens\'{e}es, excerpt}
\chapter{Pascal's Wager}
\source*{pascal1901}

\page{96}\noindent Let us now speak according to the light of nature.

If there be a God, he is infinitely incomprehensible, since having
neither parts nor limits he has no relation to us. We are then
incapable of knowing either what he is or if he is. This being so, who
will dare to undertake the solution of the question? Not we, who have
no relation to him.

Who then will blame Christians for not being able to give a reason for
their faith; those who profess a religion for which they cannot give a
reason? They declare in putting it forth to the world that it is a
foolishness, \textit{stultitiam}, and then you complain that they do
not prove it. Were they to prove it they would not keep their word, it
is in lacking proof that they are not lacking in sense.---Yes, but
although this excuses those who offer it as such, and takes away from
them the blame of putting it forth without reason, it does not excuse
those who receive it.---Let us then examine this point, and say, ``God
is, or he is not.'' But to which side shall we incline? Reason can
\page{97} determine nothing about it. There is an infinite gulf fixed
between us. A game is playing at the extremity of this infinite
distance in which heads or tails may turn up. What will you wager?
There is no reason for backing either one or the other, you cannot
reasonably argue in favour of either.

Do not then accuse of error those who have already chosen, for you
know nothing about it.---No, but I blame them for having made, not
this choice, but a choice, for again both the man who calls `heads'
and his adversary are equally to blame, they are both in the wrong;
the true course is not to wager at all.---

Yes, but you must wager; this depends not on your will, you are
embarked in the affair. Which will you choose? Let us see. Since you
must choose, let us see which least interests you. You have two things
to lose, truth and good, and two things to stake, your reason and your
will, your knowledge and your happiness; and your nature has two
things to avoid, error and misery. Since you must needs choose, your
reason is no more wounded in choosing one than the other. Here is one
point cleared up, but what of your happiness? Let us weigh the gain
and the loss in choosing heads that God is. Let us weigh the two
cases: if you gain, you gain all; if you lose, you lose nothing. Wager
then unhesitatingly that he is.---You are right. Yes, I must wager,
but I may stake too much.---Let us see. Since there is an equal chance
of gain and loss, if you had only to gain two lives for one, you might
still wager. But were there three of them to gain, you would have to
play, since needs must that you play, and you would be imprudent,
since you must play, not to chance your life to gain three at a game
where the chances of loss or gain are even. But there is an eternity
of life and happiness. And that being so, were there an infinity of
chances of which one only would be for you, you would still be right
to stake one to win two, and you would act foolishly, being obliged to
play, did you refuse to stake one life against three at a game in
which out of an infinity of chances there be one for you, if there
were an infinity of an infinitely happy life to win. But there is here
an in-\page{98}finity of an infinitely happy life to win, a chance of
gain against a finite number of chances of loss, and what you stake is
finite; that is decided. Wherever the infinite exists and there is not
an infinity of chances of loss against that of gain, there is no room
for hesitation, you must risk the whole. Thus when a man is forced to
play he must renounce reason to keep life, rather than hazard it for
infinite gain, which is as likely to happen as the loss of
nothingness.

For it is of no avail to say it is uncertain that we gain, and certain
that we risk, and that the infinite distance between the certainty of
that which is staked and the uncertainty of what we shall gain, equals
the finite good which is certainly staked against an uncertain
infinite. This is not so. Every gambler stakes a certainty to gain an
uncertainty, and yet he stakes a finite certainty against a finite
uncertainty without acting unreasonably. It is false to say there is
infinite distance between the certain stake and the uncertain gain.
There is in truth an infinity between the certainty of gain and the
certainty of loss. But the uncertainty of gain is proportioned to the
certainty of the stake, according to the proportion of chances of gain
and loss, and if therefore there are as many chances on one side as on
the other, the game is even. And thus the certainty of the venture is
equal to the uncertainty of the winnings, so far is it from the truth
that there is infinite distance between them. So that our argument is
of infinite force, if we stake the finite in a game where there are
equal chances of gain and loss, and the infinite is the winnings. This
is demonstrable, and if men are capable of any truths, this is one.

I confess and admit it. Yet is there no means of seeing the hands at
the game?---Yes, the Scripture and the rest, etc.

---Well, but my hands are tied and my mouth is gagged: I am forced to
wager and am not free, none can release me, but I am so made that I
cannot believe. What then would you have me do?

True. But understand at least your incapacity to believe, since your
reason leads you to belief and yet you cannot \page{99} believe. Labour
then to convince yourself, not by increase of the proofs of God, but
by the diminution of your passions. You would fain arrive at faith,
but know not the way; you would heal yourself of unbelief, and you ask
remedies for it. Learn of those who have been bound as you are, but
who now stake all that they possess; these are they who know the way
you would follow, who are cured of a disease of which you would be
cured. Follow the way by which they began, by making believe that they
believed, taking the holy water, having masses said, etc. Thus you
will naturally be brought to believe, and will lose your
acuteness.---But that is just what I fear.---Why? what have you to
lose?

But to show you that this is the right way, this it is that will
lessen the passions, which are your great obstacles, etc.---

What you say comforts and delights me, etc.---If my words please you,
and seem to you cogent, know that they are those of one who has thrown
himself on his knees before and after to pray that Being, infinite,
and without parts, to whom he submits all his own being, that you also
would submit to him all yours, for your own good and for his glory,
and that this strength may be in accord with this weakness.

% NOTE: period after 'argument' not in the text but added for
% consistency; 'honourable' followed by period in text

\textit{The end of this argument}.---Now what evil will happen to you
in taking this side? You will be trustworthy, honourable, humble,
grateful, generous, friendly, sincere, and true. In truth you will no
longer have those poisoned pleasures, glory and luxury, but you will
have other pleasures. I tell you that you will gain in this life, at
each step you make in this path you will see so much certainty of
gain, so much nothingness in what you stake, that you will know at
last that you have wagered on a certainty, an infinity, for which you
have risked nothing.

\vspace{1\baselineskip}

% NOTE: text missing period after 'hell'

\textit{Objection}.---Those who hope for salvation are so far happy,
but they have as a counterpoise the fear of hell.

\textit{Answer}.---Who has most reason to fear hell, the man who is in
ignorance if there be a hell, and who is certain of damnation if there
be; or he who is certainly convinced \page{100} that there is a hell,
and has a hope of being saved if there be?

