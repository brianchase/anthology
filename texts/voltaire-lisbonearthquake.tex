
%\author{Voltaire [François-Marie Arouet]}
\author{Voltaire}
\authdate{1694--1778}
\textdate{1755}
\chapter{The Lisbon Earthquake}
\source{voltaire1904.1}

\page{5}\section*{Author's Preface to \emph{The Lisbon Earthquake}.}

If the question concerning physical evil ever deserves the attention
of men, it is in those melancholy events which put us in mind of the
weakness of our nature; such as plagues, which carry off a fourth of
the inhabitants of the known world; the earthquake which swallowed up
four hundred thousand of the Chinese in 1699, that of Lima and Callao,
and, in the last place, that of Portugal and the kingdom of Fez. The
maxim, ``whatever is, is right,'' appears somewhat extraordinary to
those who have been eye-witnesses of such calamities. All things are
doubtless arranged and set in order by Providence, but it has long
been too evident, that its superintending power has not disposed them
in such a manner as to promote our temporal happiness.

When the celebrated Pope published his ``Essay on Man,'' and expounded
in immortal verse the systems of Leibnitz, Lord Shaftesbury and Lord
Bolingbroke, his system was attacked by a multitude of divines of a
variety of different communions. They were shocked at the novelty of
the propositions, ``whatever is, is right''; and that ``man always
enjoys that measure of happiness which is suited to his being.'' There
are few writings that may not be condemned, if considered in one
light, or approved of, if considered in another. It would be much more
reasonable to attend only to the beauties and improving parts of a
work, than to endeavor to put an odious construction on it; but it is
one of the imperfections of our nature to put a bad interpretation on
whatever has a dubious sense, and to run down whatever has been
successful.

In a word, it was the opinion of many, that the axiom, ``whatever is,
is right,'' was subversive of all our received ideas. If it be true,
said they, that whatever is, is right, it follows that human nature is
not degenerated. If the general order requires that everything should
be as it is, human nature has not been corrupted, and consequently
could have had no occasion for a Redeemer. If this world, such as it
is, be the best of systems possible, we have no room to hope for a
happy future state. If the various evils \page{6} by which man is
overwhelmed, end in general good, all civilized nations have been
wrong in endeavoring to trace out the origin of moral and physical
evil. If a man devoured by wild beasts, causes the well-being of those
beasts, and contributes to promote the orders of the universe; if the
misfortunes of individuals are only the consequence of this general
and necessary order, we are nothing more than wheels which serve to
keep the great machine in motion; we are not more precious in the eyes
of God, than the animals by whom we are devoured.

These are the inferences which were drawn from Mr. Pope's poem; and
these very conclusions increased the sale and success of the work. But
it should have been seen from another point of view. Readers should
have considered the reverence for the Deity, the resignation to His
supreme will, the useful morality, and the spirit of toleration, which
breathe through this excellent poem. This the public has done, and the
work being translated by men equal to the task, has completely
triumphed over critics, though it turned on matters of so delicate a
nature.

It is the nature of over violent censurers to give importance to the
opinions which they attack. A book is railed at on account of its
success, and a thousand errors are imputed to it. What is the
consequence of this? Men, disgusted with these invectives, take for
truths the very errors which these critics think they have discovered.
Cavillers raise phantoms on purpose to combat them, and indignant
readers embrace these very phantoms.

Critics have declared that Pope and Leibnitz maintain the doctrine of
fatality; the partisans of Leibnitz and Pope have said on the other
hand that, if Leibnitz and Pope have taught the doctrine of fatality,
they were in the right, and all this invincible fatality we should
believe.

Pope had advanced that ``whatever is, is right,'' in a sense that
might very well be admitted, and his followers maintain the same
proposition in a sense that may very well be contested.

The author of the poem, ``The Lisbon Earthquake,'' does not write
against the illustrious Pope, whom he always loved and admired; he
agrees with him in almost every particular, but compassionating the
misery of man; he declares against the abuse of the new maxim,
``whatever is, is right.'' He maintains that ancient and sad truth
acknowledged by all men, that there is evil upon earth; he
acknowledges that the words ``whatever is, is right,'' if understood
in a positive sense, and without any hopes of a happy future state,
only insult us in our present misery.

If, when Lisbon, Moquinxa, Tetuan, and other cities were swallowed up
with a great number of their inhabitants in \page{7} the month of
November, 1750, philosophers had cried out to the wretches, who with
difficulty escaped from the ruins, ``all this is productive of general
good; the heirs of those who have perished will increase their
fortune; masons will earn money by rebuilding the houses, beasts will
feed on the carcasses buried under the ruins; it is the necessary
effect of necessary causes; your particular misfortune is nothing, it
contributes to universal good,'' such a harangue would doubtless have
been as cruel as the earthquake was fatal, and all that the author of
the poem upon the destruction of Lisbon has said amounts only to this.

He acknowledges with all mankind that there is evil as well as good on
the earth; he owns that no philosopher has ever been able to explain
the nature of moral and physical evil. He asserts that Bayle, the
greatest master of the art of reasoning that ever wrote, has only
taught to doubt, and that he combats himself; he owns that man's
understanding is as weak as his life is miserable. He lays a concise
abstract of the several different systems before his readers. He says
that Revelation alone can untie the great knot which philosophers have
only rendered more puzzling; and that nothing but the hope of our
existence being continued in a future state can console us under our
present misfortunes; that the goodness of Providence is the only
asylum in which man can take refuge in the darkness of reason, and in
the calamities to which his weak and frail nature is exposed.

P. S.---Readers should always distinguish between the objections which
an author proposes to himself and his answers to those objections, and
should not mistake what he refutes for what he adopts.

\page{8}\section*{The Lisbon Earthquake.\footnote{The great earthquake
occurred on November 1, 1755. The ruin was instantaneous. Between
30,000 and 40,000 lives were lost in the shock and in the fire.}}

\begin{center}
\textsc{An Inquiry into the Maxim, ``Whatever Is, Is Right.''}
\end{center}

\settowidth\versewidth{Will you maintain death to their crimes was due?}
\begin{verse}[\versewidth]\normalsize
Oh wretched man, earth-fated to be cursed;\\
Abyss of plagues, and miseries the worst!\\
Horrors on horrors, griefs on griefs must show,\\
That man's the victim of unceasing woe,\\
And lamentations which inspire my strain,\\
Prove that philosophy is false and vain.\\
Approach in crowds, and meditate awhile\\
Yon shattered walls, and view each ruined pile,\\
Women and children heaped up mountain high,\\
Limbs crushed which under ponderous marble lie;\\
Wretches unnumbered in the pangs of death,\\
Who mangled, torn, and panting for their breath,\\
Buried beneath their sinking roofs expire,\\
And end their wretched lives in torments dire.\\
Say, when you hear their piteous, half-formed cries,\\
Or from their ashes see the smoke arise,\\
Say, will you then eternal laws maintain,\\
Which God to cruelties like these constrain?\\
Whilst you these facts replete with horror view,\\
Will you maintain death to their crimes was due?\\
\page{9}And can you then impute a sinful deed\\
To babes who on their mothers' bosoms bleed?\\
Was then more vice in fallen Lisbon found,\\
Than Paris, where voluptuous joys abound?\\
Was less debauchery to London known,\\
Where opulence luxurious holds her throne?\\
Earth Lisbon swallows; the light sons of France\\
Protract the feast, or lead the sprightly dance.\\
Spectators who undaunted courage show,\\
While you behold your dying brethren's woe;\\
With stoical tranquillity of mind\\
You seek the causes of these ills to find;\\
But when like us Fate's rigors you have felt,\\
Become humane, like us you'll learn to melt.\\
When the earth gapes my body to entomb,\\
I justly may complain of such a doom.\\
Hemmed round on every side by cruel fate,\\
The snares of death, the wicked's furious hate,\\
Preyed on by pain and by corroding grief\\
Suffer me from complaint to find relief.\\
'Tis pride, you cry, seditious pride that still\\
Asserts mankind should be exempt from ill.\\
The awful truth on Tagus' banks explore,\\
Rummage the ruins on that bloody shore,\\
Wretches interred alive in direful grave\\
Ask if pride cries, ``Good Heaven thy creatures save.''\\
If 'tis presumption that makes mortals cry,\\
``Heaven on our sufferings cast a pitying eye.''\\
All's right, you answer, the eternal cause\\
Rules not by partial, but by general laws.\\
\page{10}Say what advantage can result to all,\\
From wretched Lisbon's lamentable fall?\\
Are you then sure, the power which could create\\
The universe and fix the laws of fate,\\
Could not have found for man a proper place,\\
But earthquakes must destroy the human race?\\
Will you thus limit the eternal mind?\\
Should not our God to mercy be inclined?\\
Cannot then God direct all nature's course?\\
Can power almighty be without resource?\\
Humbly the great Creator I entreat,\\
This gulf with sulphur and with fire replete,\\
Might on the deserts spend its raging flame,\\
God my respect, my love weak mortals claim;\\
When man groans under such a load of woe,\\
He is not proud, he only feels the blow.\\
Would words like these to peace of mind restore\\
The natives sad of that disastrous shore?\\
Grieve not, that others' bliss may overflow,\\
Your sumptuous palaces are laid thus low;\\
Your toppled towers shall other hands rebuild;\\
With multitudes your walls one day be filled;\\
Your ruin on the North shall wealth bestow,\\
For general good from partial ills must flow;\\
You seem as abject to the sovereign power,\\
As worms which shall your carcasses devour.\\
No comfort could such shocking words impart,\\
But deeper wound the sad, afflicted heart.\\
When I lament my present wretched state,\\
Allege not the unchanging laws of fate;\\
Urge not the links of the eternal chain,\\
'Tis false philosophy and wisdom vain.\\
\page{11}The God who holds the chain can't be enchained;\footnote{The
universal chain is not, as some have thought, a regular gradation
which connects all beings. There is, in all probability, an immense
distance between man and beast, as well as between man and substances
of a superior nature; there is likewise an infinity between God and
all created beings whatever. There are none of these insensible
gradations in the globes which move round our sun in their several
periods, whether we consider their mass, their distances, or their
satellites.

If we may believe Pope, man is not capable of discovering the reason
why the satellites of Jove are less than Jove himself; he is herein
mistaken, such an error as this may well be overlooked in so fine a
genius. Every smatterer in mathematics could have told Lord
Bolingbroke and Mr. Pope, that if the satellites of Jove had equalled
him in magnitude, they could not have moved round him; but no
mathematician is able to discover the regular gradation in the bodies
of the solar system.

It is not true, that the world could not exist if a single atom was
taken from it: This was justly observed by Mr. Crousaz, a learned
geometrician, in a tract which he wrote against Pope. He seems to have
been right in this point, though he was fully refuted by Mr. Warburton
and Mr. Silhouette.

The concatenation of events was admitted and defended with the utmost
ingenuity by the celebrated philosopher Leibnitz; it is worth
explaining. All bodies and all events depend upon other bodies and
other events. That cannot be denied; but all bodies are not essential
to the support of the universe, and the preservation of its order;
neither are all events necessary in the general series of events. A
drop of water, a grain of sand more or less, can cause no revolution
in the general system. Nature is not confined to any determinate
quantity, or any determinate form. No planet moves in a curve
completely regular; there is nothing in Nature of a figure exactly
mathematical; no fixed quantity is required for any operation: Nature
is never very strict or rigid in her method of proceeding. It is,
therefore, absurd to advance, that the removal of an atom from the
earth might be the cause of its destruction.

% CHECK: an instance of C\ae sar is split across a line; does it print
% correctly?

This holds, in like manner, with regard to events. The cause of every
event is contained in some precedent event; this no philosopher has
ever called in question. If C\ae sar's mother had never gone through
the C\ae sarian operation, C\ae sar had never subverted the
commonwealth; he could never have adopted Octavius, and Octavius could
never have chosen Tiberius for his successor in the empire. The
marriage of Maximilian with the heiress of Burgundy and the Low
Countries, gave rise to a war which lasted two hundred years. But C\ae
sar's spitting on the right or left side, or the Duchess of Burgundy's
dressing her head in this manner or in that, could nave altered
nothing in the general plan of Providence.

It follows, therefore, that there are some events which have
consequences and others which have none. Their chain resembles a
genealogical tree, some branches of which disappear at the first
generation, whilst the race is continued by others. There are many
events which pass away without ever generating others. Thus in every
machine there are some effects indispensably necessary towards
producing motion, and others which are productive of nothing at all.
The wheels of a coach make it go; but whether they raise more or less
dust, the journey is finished alike. Such is the general order of the
world, that the links of the chain would not be in the least
discomposed by a small increase or diminution of the quantity of
matter, or by an inconsiderable deviation from regularity.

The chain is not in an absolute \textit{plenum}; it has been
demonstrated that the celestial bodies perform their revolutions in an
unresisting medium. Every space is not filled. It follows then, that
there is not a progression of bodies from an atom to the most remote
fixed star. There may of consequence be immense intervals between
beings imbued with sensation, as well as between those that are not.
We cannot then be certain, that man must be placed in one of these
links joined to another by an uninterrupted connection. That all
things are linked together means only that all things are regularly
disposed of in their proper order. God is the cause and the regulator
of that order. Homer's Jupiter was the slave of destiny; but,
according to more rational philosophy, God is the master of destiny.
(See Clarke's Treatise ``Upon the Existence of God.'')}\\
By His blest will are all events ordained:\\
\page{12}He's just, nor easily to wrath gives way,\\
Why suffer we beneath so mild a sway:\footnote{\textit{Sub
Deo justo nemo miser nisimereatur}.---St. Augustine. The meaning of
this \textit{ipse dixit} of the Saint is, no one is miserable under
the government of a just God, without deserving to be so.}\\
This is the fatal knot you should untie,\\
Our evils do you cure when you deny?\\
\page{13}Men ever strove into the source to pry,\\
Of evil, whose existence you deny.\\
If he whose hand the elements can wield,\\
To the winds' force makes rocky mountains yield;\\
If thunder lays oaks level with the plain,\\
From the bolts' strokes they never suffer pain.\\
But I can feel, my heart oppressed demands\\
Aid of that God who formed me with His hands.\\
Sons of the God supreme to suffer all\\
Fated alike; we on our Father call.\\
No vessel of the potter asks, we know,\\
Why it was made so brittle, vile, and low?\\
Vessels of speech as well as thought are void;\\
The urn this moment formed and that destroyed,\\
The potter never could with sense inspire,\\
Devoid of thought it nothing can desire.\\
The moralist still obstinate replies,\\
Others' enjoyments from your woes arise,\\
To numerous insects shall my corpse give birth,\\
When once it mixes with its mother earth:\\
Small comfort 'tis that when Death's ruthless power\\
Closes my life, worms shall my flesh devour.\\
Remembrances of misery refrain\\
From consolation, you increase my pain:\\
Complaint, I see, you have with care repressed,\\
And proudly hid your sorrows in your breast.\\
But a small part I no importance claim\\
In this vast universe, this general frame;\\
All other beings in this world below\\
Condemned like me to lead a life of woe,\\
Subject to laws as rigorous as I,\\
Like me in anguish live and like me die.\\
\page{14}The vulture urged by an insatiate maw,\\
Its trembling prey tears with relentless claw:\\
This it finds right, endowed with greater powers\\
The bird of Jove the vulture's self devours.\\
Man lifts his tube, he aims the fatal ball\\
And makes to earth the towering eagle fall;\\
Man in the field with wounds all covered o'er,\\
Midst heaps of dead lies weltering in his gore,\\
While birds of prey the mangled limbs devour,\\
Of Nature's Lord who boasts his mighty power.\\
Thus the world's members equal ills sustain,\\
And perish by each other born to pain:\\
Yet in this direful chaos you'd compose\\
A general bliss from individuals' woes?\\
Oh worthless bliss! in injured reason's sight,\\
With faltering voice you cry, ``What is, is right''?\\
The universe confutes your boasting vain,\\
Your heart retracts the error you maintain.\\
Men, beasts, and elements know no repose\\
From dire contention; earth's the seat of woes:\\
We strive in vain its secret source to find.\\
Is ill the gift of our Creator kind?\\
Do then fell Typhon's cursed laws ordain\\
Our ill, or Arimanius doom to pain?\\
Shocked at such dire chimeras, I reject\\
Monsters which fear could into gods erect.\\
But how conceive a God, the source of love,\\
Who on man lavished blessings from above,\\
Then would the race with various plagues confound,\\
Can mortals penetrate His views profound?\\
Ill could not from a perfect being spring,\\
Nor from another, since God's sovereign king;\\
\page{15}And yet, sad truth! in this our world 'tis found,\\
What contradictions here my soul confound!\\
A God once dwelt on earth amongst mankind,\\
Yet vices still lay waste the human mind;\\
He could not do it, this proud sophist cries,\\
He could, but he declined it, that replies;\\
He surely will, ere these disputes have end,\\
Lisbon's foundations hidden thunders rend,\\
And thirty cities' shattered remnants fly,\\
With ruin and combustion through the sky,\\
From dismal Tagus' ensanguined shore,\\
To where of Cadiz' sea the billows roar.\\
Or man's a sinful creature from his birth,\\
And God to woe condemns the sons of earth;\\
Or else the God who being rules and space,\\
Untouched with pity for the human race,\\
Indifferent, both from love and anger free,\\
Still acts consistent to His first decree:\\
Or matter has defects which still oppose\\
God's will, and thence all human evil flows;\\
Or else this transient world by mortals trod,\\
Is but a passage that conducts to God.\\
Our transient sufferings here shall soon be o'er,\\
And death will land us on a happier shore.\\
But when we rise from this accursed abyss,\\
Who by his merit can lay claim to bliss?\\
Dangers and difficulties man surround,\\
Doubts and perplexities his mind confound.\\
To nature we apply for truth in vain,\\
God should His will to human kind explain.\\
He only can illume the human soul,\\
Instruct the wise man, and the weak console.\\
\page{16}Without Him man of error still the sport,\\
Thinks from each broken reed to find support.\\
Leibnitz can't tell me from what secret cause\\
In a world governed by the wisest laws,\\
Lasting disorders, woes that never end\\
With our vain pleasures real sufferings blend;\\
Why ill the virtuous with the vicious shares?\\
Why neither good nor bad misfortunes spares?\\
I can't conceive that ``what is, ought to be,''\\
In this each doctor knows as much as me.\\
We're told by Plato, that man, in times of yore,\\
Wings gorgeous to his glorious body wore,\\
That all attacks he could unhurt sustain,\\
By death ne'er conquered, ne'er approached by pain.\\
Alas, how changed from such a brilliant state!\\
He crawls 'twixt heaven and earth, then yields to fate.\\
Look round this sublunary world, you'll find\\
That nature to destruction is consigned.\\
Our system weak which nerves and bone compose,\\
Cannot the shock of elements oppose;\\
This mass of fluids mixed with tempered clay,\\
To dissolution quickly must give way.\\
Their quick sensations can't unhurt sustain\\
The attacks of death and of tormenting pain,\\
This is the nature of the human frame,\\
Plato and Epicurus I disclaim.\\
Nature was more to Bayle than either known:\\
What do I learn from Bayle, to doubt alone?\\
Bayle, great and wise, all systems overthrows,\\
Then his own tenets labors to oppose.\\
Like the blind slave to Delilah's commands,\\
Crushed by the pile demolished by his hands.\\
\page{17}Mysteries like these can no man penetrate,\\
Hid from his view remains the book of fate.\\
Man his own nature never yet could sound,\\
He knows not whence he is, nor whither bound.\footnote{It is
self-evident, that man cannot acquire this knowledge without
assistance. The human mind derives all its knowledge from experience;
no experience can give us an insight into what preceded our existence,
into what is to follow it, nor into what supports it at present. In
what manner have we received life? What is the spring upon which it
depends? How is our brain capable of ideas and memory? In what manner
do our limbs obey every motion of the will. Of all this we are
entirely ignorant. Is our globe the only one that is inhabited? Was it
created after other globes, or at the same instant? Does every
particular species of plants proceed from a first plant? Is every
species of animals produced by two first animals? The most profound
philosophers are no more able to solve these questions than the most
ignorant of men. All these questions may be reduced to the vulgar
proverb: Was the hen before the egg, or the egg before the hen? The
proverb is rather low, but it confounds the utmost penetration of
human wisdom, which is utterly at a loss with regard to the first
principles of things without supernatural assistance.}\\
Atoms tormented on this earthly ball,\\
The sport of fate, by death soon swallowed all,\\
But thinking atoms, who with piercing eyes\\
Have measured the whole circuit of the skies;\\
We rise in thought up to the heavenly throne,\\
But our own nature still remains unknown.\\
This world which error and o'erweening pride,\\
Rulers accursed between them still divide,\\
Where wretches overwhelmed with lasting woe,\\
Talk of a happiness they never know,\\
Is with complaining filled, all are forlorn\\
In seeking bliss; none would again be born.\\
If in a life midst sorrows past and fears,\\
With pleasure's hand we wipe away our tears,\\
\page{18}Pleasure his light wings spreads, and quickly flies,\\
Losses on losses, griefs on griefs arise.\\
The mind from sad remembrance of the past,\\
Is with black melancholy overcast;\\
Sad is the present if no future state,\\
No blissful retribution mortals wait,\\
If fate's decrees the thinking being doom\\
To lose existence in the silent tomb.\\
All may be well; that hope can man sustain,\\
All now is well; 'tis an illusion vain.\\
The sages held me forth delusive light,\\
Divine instructions only can be right.\\
Humbly I sigh, submissive suffer pain,\\
Nor more the ways of Providence arraign.\\
In youthful prime I sung in strains more gay,\\
Soft pleasure's laws which lead mankind astray.\\
But times change manners; taught by age and care\\
Whilst I mistaken mortals' weakness share,\\
The light of truth I seek in this dark state,\\
And without murmuring submit to fate.\\
A caliph once when his last hour drew nigh,\\
Prayed in such terms as these to the most high:\\
``Being supreme, whose greatness knows no bound,\\
I bring thee all that can't in Thee be found;\\
Defects and sorrows, ignorance and woe.''\\
Hope he omitted, man's sole bliss below.
\end{verse}

