
\author{Miguel de Unamuno}
\authdate{1864--1936}
\textdate{1910}
\chapter{My Religion}
\source{unamuno1924.14}

\page{154}\noindent A friend writing to me from Chile tells me that he
has met people acquainted with my writings who have asked him: ``What,
in a word, is the religion of this Se\~nor Unamuno?'' I myself have
several times been asked a similar question. And I am going to see if
I cannot---I will not say, answer it, for that is a thing I do not
pretend to be able to do, but endeavour at any rate to elucidate the
meaning of the question.

Individuals as well as peoples characterized by intellectual
inertia---and intellectual inertia is quite compatible with great
productive activity in the sphere of economics and in other kindred
spheres---tend to dogmatism, whether they know it or not, whether they
wish it or not, whether they intend it or not. Intellectual inertia
shuns the critical or sceptical attitude.

I say sceptical, taking the word scepticism in its etymological and
philosophical sense, for the sceptic does not mean him who doubts, but
him who investigates or researches, as opposed to him who asserts and
thinks that he has found. The one is the man who studies a problem and
the other is the man who gives us a formula, correct or incorrect, as
the solution of it.

In the order of pure philosophical speculation it is premature to
demand that an investigator shall produce a definite solution of a
problem while he is engaged in defining the problem itself more
exactly. When a long \page{155} calculation does not work out
correctly, it is no small step forward to rub it all out and begin
afresh. When a house threatens to collapse or becomes completely
uninhabitable, the first thing to do is to pull it down and not to
demand that another shall be built on top of it. The new house may
indeed be built with materials taken from the old one, but only after
the old one has first been demolished. In the meantime, if there is no
other house available, the people can find shelter in a hut or sleep
in the open.

And it is necessary not to lose sight of the fact that in the problems
of practical life we must seldom expect to find definite scientific
solutions. Men live and always have lived upon hazardous hypotheses
and explanations, and sometimes even without them. Men have not waited
to agree as to whether or not the criminal was possessed of free will
before punishing him, and a man does not pause before sneezing to
reflect upon the possible injury that may be caused by the obstructing
particle that provokes him to sneeze.

I think that those men are mistaken who assert that they would live
evilly if they did not believe in the eternal pains of hell, and the
mistake is all to their credit. If they ceased to believe in a
sanction after death, they would not live worse, but they would look
for some other ideal justification for their conduct. The good man is
not good because he believes in a transcendental order, but rather he
believes it because he is good---a proposition which I am sure must
appear obscure or involved to those who suffer from intellectual
inertia.

I am asked, then: ``What is your religion?'' And \page{156} I will
reply: My religion is to seek truth in life and life in truth, even
though knowing full well that I shall never find them so long as I
live; my religion is to wrestle unceasingly and unwearyingly with
mystery; my religion is to wrestle with God from nightfall until the
breaking of the day, as Jacob is said to have wrestled with Him. I
cannot accommodate myself to the doctrine of the Unknowable or to that
of ``thus far and no farther.'' I reject the everlasting
\textit{Ignorabimus}. And at all hazards I seek to scale the
unattainable.

``Be perfect as your Father in heaven is perfect,'' Christ said to us,
and such an ideal of perfection is, without doubt, unattainable. But
He put the unattainable before us as the goal and term of our
endeavours. And He attained to it, say the theologians, by grace. And
I wish to fight my fight careless of victory. Are there not armies and
even peoples who march to certain defeat? Do we not praise those who
die fighting rather than surrender? This, then, is my religion.

Those who put this question to me want me to give them a dogma, a
solution which they can accept without disturbing their mental
inertia. Or rather it is not this that they want, so much as to be
able to label me and put me into one of the divisions in which they
classify minds, so that they can say of me: He is a Lutheran, a
Calvanist, a Catholic, an atheist, a rationalist, a mystic, or any
other of those nicknames whose exact meaning they do not understand
but which dispense them from further thinking. And I do not wish to
have myself labelled, for I, Miguel de Unamuno, like every other man
who aspires to full consciousness, am a unique species. ``There are no
diseases, but only \page{157} persons who are diseased,'' some doctors
say, and I say that there are no opinions, but only opining persons.

In religion there is but little that is capable of rational
resolution, and as I do not possess that little I cannot communicate
it logically, for only the rational is logical and transmissible. I
have, it is true, so far as my affections, my heart and my feelings
are concerned, a strong bent towards Christianity, but without
adhering to the special dogmas of this or that Christian confession. I
count every man a Christian who invokes the name of Christ with
respect and love, and I am repelled by the orthodox, whether Catholic
or Protestant---the latter being usually as intransigent as the
former---who deny the Christianity of those who interpret the Gospel
differently from themselves. I know a Protestant Christian who denies
the Unitarians are Christians.

I frankly confess that the supposed rational proofs---ontological,
cosmological, ethical, etc.---of the existence of God, prove to me
nothing; that all the reasons adduced to show that a God exists appear
to me to be based on sophistry and begging of the question. In this I
am with Kant. And in discussions of this kind, I feel that I am unable
to talk to cobblers in the terms of their craft.

Nobody has succeeded in convincing me rationally of the existence of
God, nor yet of His non-existence; the arguments of atheists appear to
me even more superficial and futile than those of their opponents. And
if I believe in God, or at least believe that I believe in Him, it is,
first of all, because I wish that God may exist, and then, because He
is revealed to me, \page{158} through the channel of the heart, in the
Gospel and in Christ and in history. It is an affair of the heart.

Which means that I am not convinced of it as I am of the fact that two
and two make four.

% 'peace of conscience' is correct!
If it were a question of something that did not touch my peace of
conscience or console me for having been born, perhaps I should pay no
heed to the problem; but as it involves my whole interior life and the
spring of all my actions, I cannot quiet myself by saying: I do not
know nor can I know. I do not know, that is certain; perhaps I can
never know. But I want to know. I want to, and that is enough.

And I shall spend my life wrestling with mystery, and even without
hope of penetrating it, for this wrestling is my sustenance and my
consolation. Yes, my consolation. I have accustomed myself to wrest
hope from despair itself. And let not fools in their superficiality
shriek: Paradox!

I cannot conceive of a man of culture without this preoccupation, and
in point of culture---and culture is not the same as civilization---I
can hope but little from those who live without interest in the
metaphysical aspects of the religious problem, and only study it in
its social or political aspects. I can hope but very little for the
enrichment of the spiritual treasury of mankind from those men or from
those peoples who, whether it be from intellectual inertia, or from
superficiality, or from scientificism, or from any other cause, are
unmoved by the great and eternal disquietudes of the heart. I can hope
nothing from those who say: ``We must not think about these things!''
I can hope even less from those who believe in a heaven and a hell
\page{159} such as those which we believed in when we were children;
and still less can I hope from those who affirm with a fool's gravity:
``All this is but myth and fable; he who dies is buried and there's an
end of it.'' I can hope for something only from those who do not know,
but who are not resigned not to know; from those who fight unrestingly
for the truth and put their life in the fight itself rather than in
the victory.

The greater part of my work has always been to disquiet my neighbors,
to rob them of heart's ease, to vex them if I can. I have said this
already in my commentary upon ``The Life of Don Quixote and Sancho,''
in which I have confessed myself most fully. Let them seek as I seek,
let them wrestle as I wrestle, and between us all we will tear some
shred of secret from God, and at any rate this wrestling will make us
more men, men of more spirit.

In order to accomplish this work---a religious work---among peoples
like those that speak the Castilian tongue who suffer from
intellectual inertia and superficiality, slumbering in the routine of
Catholic dogma or in the dogmatism of free-thought or of
scientificism, it has been necessary for me to appear sometimes
shameless and indecorous, at other times harsh and aggressive, and not
a few times perverse and paradoxical. In our pusillanimous literature
it was a rare thing to hear anyone cry out from the depths of his
heart, to get excited, to exclaim. The shout was almost unknown.
Writers were frightened of making themselves ridiculous. They behaved
and still behave like those who put up with an affront in the street
for fear of the ridicule of being seen with their hat on the
\page{160} ground marched off by the police. But I, no! When I have
felt like shouting I have shouted. Never have I been restrained by
decorum. And this is one of the things for which I have never been
forgiven by my colleagues of the pen, so discreet, so correct, so
disciplined, even when they preach indiscretion and indiscipline.
Literary anarchists are more punctilious about style and syntax than
about anything else. And when they play out of tune they do so
tunefully; their discords resolve themselves into harmonies.

When I have felt a pain I have shouted and shouted in public. The
psalms which are to be found in my \textit{Poes\'ias} are simply the
cries from the heart with which I have sought to make the
heart-strings of the wounded hearts of others vibrate. If they have no
heart-strings or only heart-strings that are too rigid to vibrate,
then my cry will awaken no echo in them and they will declare that it
is not poetry and they will proceed to investigate its acoustic
properties. It is possible also to study acoustically the cry that is
torn from the heart of a man who sees his son suddenly fall down
dead---and he who has neither heart nor sons will understand no more
of it than the acoustics.

These psalms, together with various other pieces in my
\textit{Poes\'ias}, are my religion, a religion that I have sung, not
expressed in logic and reasoning. And I sing it as best I can, with
the voice and ear that God has given me, because I cannot reason it.
And he to whom my verses appear to be more full of reasoning and logic
and method and exegesis than of life, because they are not peopled
with fauns, dryads, satyrs and the like or garbed in the latest
modernist fashion, had better \page{161} leave them alone, for it is
evident that I shall not touch his heart whether I use a violin bow or
a hammer.

What I fly from, I repeat, as from the plague, is any kind of
classification of myself, and when I die I hope I shall still hear
these intellectual sluggards inquiring: ``And this gentleman, what is
he?'' Liberal or progressive fools will take me for a reactionary and
perhaps for a mystic, without understanding of course what they may
mean; and conservative and reactionary fools will take me for a kind
of spiritual anarchist; and both of them will pity me as an
unfortunate gentleman anxious to distinguish himself by singularity,
hoping to be reputed an original, and with a bonnet full of bees. But
no one need worry about what fools think of him, be they progressive
or conservative, liberal or reactionary.

And since man is naturally intractable, and does not habitually thirst
for the truth, and after being preached at for four hours usually
returns to all his inveterate habits, these busy inquirers, if they
chance to read this, will return to me with the question: ``Well, but
what solutions do you offer?'' And I will tell them, once and for all,
that if it is solutions they want, they can go to the shop opposite,
for I do not deal in the article. My earnest desire has been, is and
will be that those who read me should think and meditate on
fundamental things, and it has never been to furnish them with
thoughts ready made. I have always sought to agitate and to suggest
rather than to instruct. It is not bread that I sell, not bread, but
yeast, ferment.

I have friends, and good friends, who advise me to abandon this task
and to concentrate upon what they \page{162} call some objective work,
something which will be, so they express it, definitive, something
constructive, something that will last. They mean something dogmatic.
I declare that I am incapable of it, and I claim my liberty, my holy
liberty, even, if need be, the liberty of contradicting myself. I do
not know whether anything that I have written or may write in the
future is destined to live for years and centuries after I am dead;
but I know that if anyone agitates the surface of a shoreless sea the
waves will go radiating without end, even though at last they dwindle
into ripples. To agitate is something. And if thanks to this agitation
another who comes after me shall create something that will live, then
my work will live in his.

It is a work of supreme mercy to awaken the sleeper and to shake the
sluggard, and it is a work of supreme religious piety to seek truth in
everything and to expose fraud, stupidity and ignorance wherever they
are to be found.

