
\author{Michel de Montaigne}
\authdate{1533--1592}
\textdate{1580}
\chapter{That to Study Philosophy is to Learn to Die}
\source{montaigne1913.19}

\page{64}Cicero says\footnote{Tusc., i. 31.} ``that to study
philosophy is nothing but to prepare one's self to die.'' The reason
of which is, because study and contemplation do in some sort withdraw
from us \page{65} our soul, and employ it separately from the body,
which is a kind of apprenticeship and a resemblance of death; or else,
because all the wisdom and reasoning in the world do in the end
conclude in this point, to teach us not to fear to die. And to say the
truth, either our reason mocks us, or it ought to have no other aim
but our contentment only, nor to endeavour anything but, in sum, to
make us live well, and, as the Holy Scripture says,\footnote{Eccles.
iii. 12, where, however, the exact text is, ``For a man to rejoice and
to do good in his life.''} at our ease. All the opinions of the world
agree in this, that pleasure is our end, though we make use of divers
means to attain it: they would, otherwise, be rejected at the first
motion; for who would give ear to him that should propose affliction
and misery for his end? The controversies and disputes of the
philosophical sects upon this point are merely
verbal---``Transcurramus solertissimas nugas''\footnote{``Let us skip
over those subtle trifles.''---\textsc{Seneca}, \textit{Ep}.,
117.}---there is more in them of opposition and obstinacy than is
consistent with so sacred a profession; but whatsoever personage a man
takes upon himself to perform, he ever mixes his own part with it.

Let the philosophers say what they will, the main thing at which we
all aim, even in virtue itself, is pleasure. It amuses me to rattle in
their ears this word, which they so nauseate to hear; and if it
signify some supreme pleasure and excessive contentment, it is more
due to the assistance of virtue than to any other assistance whatever.
This pleasure, for being more gay, more sinewy, more robust, and more
manly, is only the more seriously voluptuous, and we ought to give it
the name of pleasure, as that which is more favourable, gentle, and
natural, and not that of vigour, from which we have denominated it.
The other, and meaner pleasure, if it could deserve this fair name, it
ought to be by way of competition, and not of privilege. I find it
less exempt from traverses and inconveniences than virtue itself; and,
besides that the enjoyment is more momentary, fluid, and frail, it has
its watchings, fasts, and labours, its sweat and its blood; and,
moreover, has particular to itself so many several sorts of sharp and
wounding passions, and so dull a satiety attending it, as equal it to
the severest penance. And we mistake if we think that these
incommodities serve \page{66} it for a spur and a seasoning to its
sweetness (as in nature one contrary is quickened by another), or say,
when we come to virtue, that like consequences and difficulties
overwhelm and render it austere and inaccessible; whereas, much more
aptly than in voluptuousness, they ennoble, sharpen, and heighten the
perfect and divine pleasure they procure us. He renders himself
unworthy of it who will counterpoise its cost with its fruit, and
neither understands the blessing nor how to use it. Those who preach
to us that the quest of it is craggy, difficult, and painful, but its
fruition pleasant, what do they mean by that but to tell us that it is
always unpleasing? For what human means will ever attain its
enjoyment? The most perfect have been fain to content themselves to
aspire unto it, and to approach it only, without ever possessing it.
But they are deceived, seeing that of all the pleasures we know, the
very pursuit is pleasant. The attempt ever relishes of the quality of
the thing to which it is directed, for it is a good part of, and
consubstantial with, the effect. The felicity and beatitude that
glitters in Virtue, shines throughout all her appurtenances and
avenues, even to the first entry and utmost limits.

Now, of all the benefits that virtue confers upon us, the contempt of
death is one of the greatest, as the means that accommodates human
life with a soft and easy tranquillity, and gives us a pure and
pleasant taste of living, without which all other pleasure would be
extinct. Which is the reason why all the rules centre and concur in
this one article. And although they all in like manner, with common
accord, teach us also to despise pain, poverty, and the other
accidents to which human life is subject, it is not, nevertheless,
with the same solicitude, as well by reason these accidents are not of
so great necessity, the greater part of mankind passing over their
whole lives without ever knowing what poverty is, and some without
sorrow or sickness, as Xenophilus the musician, who lived a hundred
and six years in perfect and continual health; as also because, at the
worst, death can, whenever we please, cut short and put an end to all
other inconveniences. But as to death, it is inevitable:---

\begin{verse}
\poke{``}Omnes eodom cogimur; omnium\\
Versatur urna serius ocius\\
\page{67}Sors exitura, et nos in \ae ternum\\
Exilium impositura cymb\ae,''\footnote{``We are all bound one voyage;
the lot of all, sooner or later, is to come out of the urn. All must
to eternal exile sail away.''---\textsc{Hor}., \textit{Od}., ii. 3,
25.}
\end{verse}

\noindent and, consequently, if it frights us, 'tis a perpetual
torment, for which there is no sort of consolation. There is no way by
which it may not reach us. We may continually turn our heads this way
and that, as in a suspected country, ``qu\ae, quasi saxum Tantalo,
semper impendet.''\footnote{``Ever, like Tantalus' stone, it hangs
over us.''---\textsc{Cicero}, \textit{De Finib}., i. 18.} Our courts
of justice often send back condemned criminals to be executed upon the
place where the crime was committed; but, carry them to fine houses by
the way, prepare for them the best entertainment you can---

\begin{verse}
\vin\vin\vin \poke{``}Non Sicul\ae\ dapes\\
Dulcem elaborabunt saporem:\\
\vin Non avium cithar\ae que cantus\\
\vin\vin Somnum reducent.''\footnote{``Sicilian dainties will not
tickle their palates, nor the melody of birds or harps bring back
sleep.''---\textsc{Hor}., \textit{Od}., iii. 1, 18.}
\end{verse}

\noindent Do you think they can relish it? and that the fatal end of
their journey being continually before their eyes, would not alter and
deprave their palate from tasting these regalios?

\begin{verse}
\poke{``}Audit iter, numeratque dies, spatioque viarum\\
Metitur vitam; torquetur peste futura.''\footnote{``He considers the
route, computes the time of travelling, measuring his life by the
length of the journey, and torments himself by thinking of the blow to
come.''---\textsc{Claudianus}, in \textit{Ruf}., ii. 137.}
\end{verse}

The end of our race is death; 'tis the necessary object of our aim,
which, if it fright us, how is it possible to advance a step without a
fit of ague? The remedy the vulgar use is not to think on't; but from
what brutish stupidity can they derive so gross a blindness? They must
bridle the ass by the tail.

\begin{verse}
\poke{``}Qui capite ipse suo instituit vestigia
retro,''\footnote{``Who in his folly seeks to advance
backwards.''---\textsc{Lucretius}, iv. 474.}
\end{verse}

\page{68}\noindent 'tis no wonder if he be often trapped in the
pitfall. They affright people with the very mention of death, and many
cross themselves, as it were the name of the devil. And because the
making a man's will is in reference to dying, not a man will be
persuaded to take a pen in hand to that purpose till the physician has
passed sentence upon him, and totally given him over, and then betwixt
grief and terror, God knows in how fit a condition of understanding he
is to do it.

The Romans, by reason that this poor syllable \textit{death} sounded
so harshly to their ears, and seemed so ominous, found out a way to
soften and spin it out by a periphrasis, and instead of pronouncing
such a one is dead, said, ``Such a one has lived,'' or ``Such a one
has ceased to live;''\footnote{Plutarch, Life of Cicero, c. 22.} for,
provided there was any mention of life in the case, though past, it
carried yet some sound of consolation. And from them it is that we
have borrowed our expression, ``The late Monsieur such and such a
one.'' Peradventure, as the saying is, the term we have lived is worth
our money. I was born betwixt eleven and twelve o'clock in the
forenoon the last day of February 1533, according to our computation,
beginning the year the 1st of January,\footnote{This was in virtue of
an ordinance of Charles IX. in 1563. Previously the year commenced at
Easter, so that the 1st January 1563 became the first day of the year
1564.} and it is now but just fifteen days since I was complete
nine-and-thirty years old; I make account to live, at least, as many
more.\footnote{Montaigne did not realise his expectation, as he died
in 1592.} In the mean time, to trouble a man's self with the thought
of a thing so far off, were folly. But what? Young and old die upon
the same terms; no one departs out of life otherwise than if he had
but just before entered into it; neither is any man so old and
decrepit, who, having heard of Methuselah, does not think he has yet
twenty years good to come. Fool that thou art, who has assured unto
thee the term of life? Thou dependest upon physicians' tales: rather
consult effects and experience. According to the common course of
things, 'tis long since that thou hast lived by extraordinary favour;
thou hast already outlived the ordinary term of life. And that it is
so, reckon up \page{69} thy acquaintance, how many more have died
before they arrived at thy age than have attained unto it; and of
those who have ennobled their lives by their renown, take but an
account, and I dare lay a wager thou wilt find more who have died
before than after five-and-thirty years of age. It is full both of
reason and piety too, to take example by the humanity of Jesus Christ
Himself; now, He ended His life at three-and-thirty years. The
greatest man, that was no more than a man, Alexander, died also at the
same age. How many several ways has death to surprise us?

\begin{verse}
\poke{``}Quid quisque, vitet, nunquam homini satis\\
Cautum est in horas.''\footnote{``Be as cautious as he may, man can
never foresee the danger that may at any hour befal
him.''---\textsc{Hon}., \textit{O}. ii. 13, 13.}
\end{verse}

To omit fevers and pleurisies, who would ever have imagined that a
duke of Brittany\footnote{John II. died 1305.} should be pressed to
death in a crowd as that duke was, at the entry of Pope Clement, my
neighbour, into Lyons?\footnote{This neighbour, Clement V., was
Bertrand de Got, Archbishop of Bordeaux.} Hast thou not seen one of
our kings\footnote{Henry II., killed in a tournament, July 10, 1559.}
killed at a tilting, and did not one of his ancestors die by the
jostle of a hog?\footnote{Philip, eldest son of Louis le Gros.} \AE
schylus, threatened with the fall of a house, was to much purpose
circumspect to avoid that danger, seeing that he was knocked on the
head by a tortoise falling out of an eagle's talons in the
air.\footnote{Val. Max., ix. 12, ext. 2.} Another was choked with a
grapestone;\footnote{Idem, ibid., ext. 8.} an emperor killed with the
scratch of a comb in combing his head. \AE milius Lepidus with a
stumble at his own threshold,\footnote{Pliny, Nat. Hist., vii. 33.}
and Aufidius with a jostle against the door as he entered the
council-chamber. And betwixt the very thighs of women, Cornelius
Gallus the pr\ae tor; Tigillinus, captain of the watch at Rome;
Ludovico, son of Guido di Gonzaga, Marquis of Mantua; and (of worse
example) Speusippus, a Platonic philosopher,\footnote{As to
Speusippus, Diogenes Laertius (iv. 9) says he killed himself, tired of
old age and infirmity.} and one of \page{70} our Popes. The poor judge
Bebius gave adjournment in a case for eight days, but he himself,
meanwhile, was condemned by death, and his own stay of life expired.
Whilst Caius Julius, the physician, was anointing the eyes of a
patient, death closed his own; and, if I may bring in an example of my
own blood, a brother of mine, Captain St. Martin, a young man,
three-aud-twenty years old, who had already given sufficient testimony
of his valour, playing a match at tennis, received a blow of a ball a
little above his right ear, which, as it gave no manner of sign of
wound or contusion, he took no notice of it, nor so much as sat down
to repose himself, but, nevertheless, died within five or six hours
after, of an apoplexy occasioned by that blow.

These so frequent and common examples passing every day before our
eyes, how is it possible a man should disengage himself from the
thought of death, or avoid fancying that it has us, every moment, by
the throat? What matter is it, you will say, which way it comes to
pass, provided a man does not terrify himself with the expectation?
For my part, I am of this mind, and if a man could by any means avoid
it, though by creeping under a calf's skin, I am one that should not
be ashamed of the shift; all I aim at is, to pass my time at my ease,
and the recreations that will most contribute to it, I take hold of,
as little glorious and exemplary as you will.

\begin{verse}
\poke{``}Pr\ae tulerim\ldots delirus inersque videri,\\
Dum mea delectent mala me, vel denique fallant,\\
Quam sapere, et ringi.''\footnote{``I had rather seem mad or a sluggard,
so that my defects are agreeable to myself, or that I am not painfully
conscious of them, than be wise and captious.''---\textsc{Hor}.,
\textit{Ep}., ii. 2, 126.}
\end{verse}

\noindent But 'tis folly to think of doing anything that way. They go,
they come, they gallop and dance, and not a word of death. All this is
very fine: but withal, when it comes either to themselves, their
wives, their children, or friends, surprising them at unawares and
unprepared, then, what torment, what outcries, what madness and
despair! Did you ever see anything so subdued, so changed, and so
confounded? A man must, therefore, make more early provi-\page{71}sion
for it; and this brutish negligence, could it possibly lodge in the
brain of any man of sense (which I think utterly impossible), sells us
its merchandise too dear. Were it an enemy that could be avoided, I
would then advise to borrow arms even of cowardice itself; but seeing
it is not, and that it will catch you as well flying and playing the
poltroon, as standing to't like an honest man---

\begin{verse}
\poke{``}Nempe et fugacem persequitur virum,\\
Nec parcit imbellis juvent\ae\\
\vin Poplitibus timidoque tergo.''\footnote{``He pursues the flying
poltroon, nor spares the hamstrings of the unwarlike youth who turns
his back.''---\textit{Idem}, \textit{ibid}., iii. 2, 14.}
\end{verse}

And seeing that no temper of arms is of proof to secure us---

\begin{verse}
\poke{``}Ille licet ferro cautus se condat, et \ae re,\\
\vin Mors tamen inclusum protraket inde caput''\footnote{``Let him
hide beneath iron or brass in his fear, death will pull his head out
of his armour.''---\textsc{Propertius}, iii. 18.}
\end{verse}

\noindent ---let us learn bravely to stand our ground, and fight him.
And to begin to deprive him of the greatest advantage he has over us,
let us take a way quite contrary to the common course. Let us disarm
him of his novelty and strangeness, let us converse and be familiar
with him, and have nothing so frequent in our thoughts as death. Upon
all occasions represent him to our imagination in his every shape; at
the stumbling of a horse, at the falling of a tile, at the least prick
with a pin, let us presently consider, and say to ourselves, ``Well,
and what if it had been death itself?'' and, thereupon, let us
encourage and fortify ourselves. Let us evermore, amidst our jollity
and feasting, set the remembrance of our frail condition before our
eyes, never suffering ourselves to be so far transported with our
delights, but that we have some intervals of reflecting upon, and
considering how many several ways this jollity of ours tends to death,
and with how many dangers it threatens it. The Egyptians were wont to
do after this manner, who in the height of their feasting and mirth,
caused a dried skeleton of a man to be brought into the room to serve
for a memento to their guests.

\begin{verse}
\page{72}\poke{``}Omnem crede diem tibi diluxisse supremum:\\
Grata superveniet, qu\ae\ non sperabitur, hora.''\footnote{``Think each
day, when past, is thy last: the next day, as unexpected, will be the
more welcome.''---\textsc{Hor}., \textit{Ep}., i. 4, 13.}
\end{verse}

Where death waits for us is uncertain; let us look for him everywhere.
The premeditation of death is the premeditation of liberty; he who has
learned to die, has unlearned to serve. There is nothing of evil in
life, for him who rightly comprehends that the privation of life is no
evil: to know how to die, delivers us from all subjection and
constraint. Paulus \AE milius answered him whom the miserable King of
Macedon, his prisoner, sent to entreat him that he would not lead him
in his triumph, ``Let him make that request to
himself.''\footnote{Plutarch, Life of Paulus \AE milius, c. 17;
Cicero, Tusc., v. 40.}

In truth, in all things, if nature do not help a little, it is very
hard for art and industry to perform anything to purpose. I am in my
own nature not melancholic, but meditative; and there is nothing I
have more continually entertained myself withal than imaginations of
death, even in the most wanton time of my age:

\begin{verse}
\poke{``}Jucundum quum \ae tas florida ver ageret.''\footnote{``When
my florid age rejoiced in pleasant spring.''---\textsc{Catullus},
lxviii.}
\end{verse}

In the company of ladies, and at games, some have perhaps thought me
possessed with some jealousy, or the uncertainty of some hope, whilst
I was entertaining myself with the remembrance of some one, surprised,
a few days before, with a burning fever of which he died, returning
from an entertainment like this, with his head full of idle fancies of
love and jollity, as mine was then, and that, for aught I knew, the
same destiny was attending me.

\begin{verse}
\poke{``}Jam fuerit, nec post unquam revocare
licebit.''\footnote{``Presently the present will have gone, never to
be recalled.''---\textsc{Lucretius}, iii. 928.}
\end{verse}

\noindent Yet did not this thought wrinkle my forehead any more than
any other. It is impossible but we must feel a sting in such
imaginations as these, at first; but with often turning and re-turning
them in one's mind, they, at last, \page{73} become so familiar as to
be no trouble at all; otherwise, I, for my part, should be in a
perpetual fright and frenzy; for never man was so distrustful of his
life, never man so uncertain as to its duration. Neither health,
which I have hitherto ever enjoyed very strong and vigorous, and very
seldom interrupted, does prolong, nor sickness contract my hopes.
Every minute, methinks, I am escaping, and it eternally runs in my
mind, that what may be done tomorrow, may be done to-day. Hazards
and dangers do, in truth, little or nothing hasten our end; and if we
consider how many thousands more remain and hang over our heads,
besides the accident that immediately threatens us, we shall find that
the sound and the sick, those that are abroad at sea, and those that
sit by the fire, those who are engaged in battle, and those who sit
idle at home, are the one as near it as the other. ``Nemo altero
fragilior est: nemo in crastinum sui certior.''\footnote{``No man is
more fragile than another: no man more certain than another of
to-morrow.''---\textsc{Seneca}, \textit{Ep}., 91.} For anything I have
to do before I die, the longest leisure would appear too short, were
it but an hour's business I had to do.

A friend of mine the other day turning over my tablets, found therein
a memorandum of something I would have done after my decease,
whereupon I told him, as it was really true, that though I was no more
than a league's distance only from my own house, and merry and well,
yet when that thing came into my head, I made haste to write it down
there, because I was not certain to live till I came home. As a man
that am eternally brooding over my own thoughts, and confine them to
my own particular concerns, I am at all hours as well prepared as I am
ever like to be, and death, whenever he shall come, can bring nothing
along with him I did not expect long before. We should always, as near
as we can, be booted and spurred, and ready to go, and, above all
things, take care, at that time, to have no business with any one but
one's self:---

\begin{verse}
\poke{``}Quid brevi fortes jaculamur \ae vo\\
Multa?''\footnote{``Why, for so short a life, teaze ourselves with so
many projects?''---\textsc{Hor}., \textit{Od}., ii. 16, 17.}
\end{verse}

\page{74}\noindent for we shall there find work enough to do, without
any need of addition. One man complains, more than of death, that he
is thereby prevented of a glorious victory; another, that he must die
before he has married his daughter, or educated his children; a third
seems only troubled that he must lose the society of his wife; a
fourth, the conversation of his son, as the principal comfort and
concern of his being. For my part, I am, thanks be to God, at this
instant in such a condition, that I am ready to dislodge, whenever it
shall please Him, without regret for anything whatsoever. I disengage
myself throughout from all worldly relations; my leave is soon taken
of all but myself. Never did any one prepare to bid adieu to the world
more absolutely and unreservedly, and to shake hands with all manner
of interest in it, than I expect to do. The deadest deaths are the
best.

\begin{verse}
\vin \poke{```}Miser, O miser,' aiunt, `omnia ademit\\
Una dies infesta mihi tot pr\ae mia vit\ae.'''\footnote{```Wretch that I
am,' they cry, `one fatal day has deprived me of so many joys of
life.'''---\textsc{Lucretius}, iii. 911.}
\end{verse}

\noindent And the builder,

\begin{verse}
\poke{``}Manent,'' says he, ``opera interrupta, min\ae que\\
Murorum ingentes.''\footnote{``The works remain incomplete, the tall
pinnacles of the walls unmade.''---\textit{\AE neid}, iv. 88, where
\textit{manent} is \textit{pendent}.}
\end{verse}

\noindent A man must design nothing that will require so much time to
the finishing, or, at least, with no such passionate desire to see it
brought to perfection. We are born to action.

\begin{verse}
\poke{``}Quum moriar medium solvar et inter opus.''\footnote{``When I
shall die, let it be doing that I had designed.''---\textsc{Ovid},
\textit{Amor}., ii. 10, 36.}
\end{verse}

\noindent I would always have a man to be doing, and, as much as in
him lies, to extend and spin out the offices of life; and then let
death take me planting my cabbages, indifferent to him, and still less
of my garden's not being finished. I saw one die, who, at his last
gasp, complained of nothing so much as that destiny was about to cut
the thread of a chronicle \page{75} history he was then compiling,
when he was gone no farther than the fifteenth or sixteenth of our
kings.

\begin{verse}
\poke{``}Illud in his rebus non addunt, nec tibi earum\\
Jam desiderium rerum super insidit una.''\footnote{``They do not add,
that dying, we have no longer a desire to possess
things.''---\textsc{Lucretius}, iii. 913.}
\end{verse}

\noindent We are to discharge ourselves from these vulgar and hurtful
humours. To this purpose it was that men first appointed the places of
sepulture adjoining the churches, and in the most frequented places of
the city, to accustom, says Lycurgus,\footnote{Plutarch, in Vita.} the
common people, women, and children, that they should not be startled
at the sight of a corpse, and to the end, that the continual spectacle
of bones, graves, and funeral obsequies should put us in mind of our
frail condition.

\begin{verse}
\poke{``}Quin etiam exhilarare viris eonvivia c\ae de\\
Mos olim, et miscere epulis spectacula dira\\
Certantum ferro, s\ae pe et super ipsa cadentum\\
Pocula, respersis non parco sanguine mensis.''\footnote{``It was
formerly the custom to enliven banquets with slaughter, and to combine
with the repast the dire spectacle of men contending with the sword,
the dying in many cases falling upon the cups, and covering the tables
with blood.''---\textsc{Silius Italicus}, xi. 51.}
\end{verse}

\noindent And as the Egyptians after their feasts were wont to present
the company with a great image of death, by one that cried out to
them, ``Drink and be merry, for such shalt thou be when thou art
dead;'' so it is my custom to have death not only in my imagination,
but continually in my mouth. Neither is there anything of which I am
so inquisitive, and delight to inform myself, as the manner of men's
deaths, their words, looks, and bearing; nor any places in history I
am so intent upon; and it is manifest enough, by my crowding in
examples of this kind, that I have a particular fancy for that
subject. If I were a writer of books, I would compile a register, with
a comment, of the various deaths of men: he who should teach men to
die, would at the same time teach them to live. Dicearchus made one,
to which he gave that title; but it was designed for another and less
profitable end.\footnote{Cicero, De Offic., ii. 5.}

\page{76}Peradventure, some one may object, that the pain and terror
of dying so infinitely exceed all manner of imagination, that the best
fencer will be quite out of his play when it comes to the push. Let
them say what they will: to premeditate is doubtless a very great
advantage; and besides, is it nothing to go so far, at least, without
disturbance or alteration? Moreover, Nature herself assists and
encourages us: if the death be sudden and violent, we have not leisure
to fear; if otherwise, I perceive that as I engage further in my
disease, I naturally enter into a certain loathing and disdain of
life. I find I have much more ado to digest this resolution of dying,
when I am well in health, than when languishing of a fever; and by how
much I have less to do with the commodities of life, by reason that I
begin to lose the use and pleasure of them, by so much I look upon
death with less terror. Which makes me hope, that the further I remove
from the first, and the nearer I approach to the latter, I shall the
more easily exchange the one for the other. And, as I have experienced
in other occurrences, that, as C\ae sar says,\footnote{De Bello Gall.,
vii. 84.} things often appear greater to us at a distance than near at
hand, I have found, that being well, I have had maladies in much
greater horror than when really afflicted with them. The vigour
wherein I now am, the cheerfulness and delight wherein I now live,
make the contrary estate appear in so great a disproportion to my
present condition, that, by imagination, I magnify those
inconveniences by one-half, and apprehend them to be much more
troublesome, than I find them really to be, when they lie the most
heavy upon me; I hope to find death the same.

Let us but observe in the ordinary changes and declinations we daily
suffer, how nature deprives us of the light and sense of our bodily
decay. What remains to an old man of the vigour of his youth and
better days?

\begin{verse}
\poke{``}Heu! senibus vit\ae\ portio quanta manet.''\footnote{``Alas,
to old men how small a portion of life is left!''---\textsc{Maximian},
\textit{vel} \textsc{Pseudo-Gallus}, i. 16.}
\end{verse}

\noindent C\ae sar, to an old weather-beaten soldier of his guards,
who came to ask him leave that he might kill himself, taking \page{77}
notice of his withered body and decrepit motion, pleasantly answered,
``Thou fanciest, then, that thou art yet alive.''\footnote{Seneca,
Ep., 77.} Should a man fall into this condition on the sudden, I do
not think humanity capable of enduring such a change: but nature,
leading us by the hand, an easy and, as it were, an insensible pace
step by step conducts us to that miserable state, and by that means
makes it familiar to us, so that we are insensible of the stroke when
our youth dies in us, though it be really a harder death than the
final dissolution of a languishing body, than the death of old age;
forasmuch as the fall is not so great from an uneasy being to none at
all, as it is from a sprightly and flourishing being to one that is
troublesome and painful. The body, bent and bowed, has less force to
support a burden; and it is the same with the soul, and therefore it
is, that we are to raise her up firm and erect against the power of
this adversary. For, as it is impossible she should ever be at rest,
whilst she stands in fear of it; so, if she once can assure herself,
she may boast (which is a thing as it were surpassing human condition)
that it is impossible that disquiet, anxiety, or fear, or any other
disturbance, should inhabit or have any place in her.

% NOTE: Perhaps 'well-settled' below is 'well settled'. This remains
% unclear even after looking at a scan of the 1905 edition as well as
% a different scans of this edition.
\begin{verse}
\poke{``}Non vultus instantis tyranni\\
\vin Mente quatit solida, neque Auster\\
Dux inquieti turbidus Adri\ae,\\
Nec fulminantis magna Jovis manus.''\footnote{``Not the menacing look
of a tyrant shakes her well-settled soul, nor turbulent Auster, the
prince of the stormy Adriatic, nor yet the strong hand of thundering
Jove, such a temper moves.''---\textsc{Hor}., \textit{Od}., iii. 3,
3.}
\end{verse}

\noindent She is then become sovereign of all her lusts and passions,
mistress of necessity, shame, poverty, and all the other injuries of
fortune. Let us, therefore, as many of us as can, get this advantage;
'tis the true and sovereign liberty here on earth, that fortifies us
wherewithal to defy violence and injustice, and to contemn prisons and
chains.

\begin{verse}
\vin \poke{``}In manicis et\\
Compedibus s\ae vo te sub custode tenebo.\\
\page{78}Ipse Deus, simul atque volam, me solvet. Opinor,\\
Hoc sentit; moriar; mors ultima linea rerum est.''\footnote{``I will
keep thee in fetters and chains, in custody of a surly keeper.---A god
will, when I ask him, set me free. This god I think is death. Death is
the term of all things.''---\textsc{Hor}., \textit{Ep}., i. 16, 76.}
\end{verse}

Our very religion itself has no surer human foundation than the
contempt of death. Not only the argument of reason invites us to
it---for why should we fear to lose a thing, which being lost cannot be
lamented?---but, also, seeing we are threatened by so many sorts of
death, is it not infinitely worse eternally to fear them all, than
once to undergo one of them? And what matters it, when it shall
happen, since it is inevitable? To him that told Socrates, ``The
thirty tyrants have sentenced thee to death;'' ``And nature them,''
said he.\footnote{Socrates was not condemned to death by the thirty
tyrants, but by the A\-the\-ni\-ans.---\textsc{Diogenes Laertius}, ii.
35.} What a ridiculous thing it is to trouble ourselves about taking
the only step that is to deliver us from all trouble! As our birth
brought us the birth of all things, so in our death is the death of
all things included. And therefore to lament that we shall not be
alive a hundred years hence, is the same folly as to be sorry we were
not alive a hundred years ago. Death is the beginning of another life.
So did we weep, and so much it cost us to enter into this, and so did
we put off our former veil in entering into it. Nothing can be a
grievance that is but once. Is it reasonable so long to fear a thing
that will so soon be despatched? Long life, and short, are by death
made all one; for there is no long, nor short, to things that are no
more. Aristotle tells us that there are certain little beasts upon the
banks of the river Hypanis, that never live above a day: they which
die at eight of the clock in the morning, die in their youth, and
those that die at five in the evening, in their
decrepitude:\footnote{Cicero, Tuse., i. 39.} which of us would not
laugh to see this moment of continuance put into the consideration of
weal or woe? The most and the least, of ours, in comparison with
eternity, or yet with the duration of mountains, rivers, stars, trees,
and even of some animals, is no less ridiculous.\footnote{Seneca,
Consol. ad Marciam, c. 20.}

\page{79}But nature compels us to it. ``Go out of this world,'' says
she, ``as you entered into it; the same pass you made from death to
life, without passion or fear, the same, after the same manner, repeat
from life to death. Your death is a part of the order of the universe,
'tis a part of the life of the world.

\begin{verse}
\vin \poke{```}Inter se mortales mutua vivunt\\
\dotfill\\
Et, quasi cursores, vita\"{i} lampada tradunt.'\footnote{``Mortals,
amongst themselves, live by turns, and, like the runners in the games,
give up the lamp, when they have won the race, to the next
comer.''---\textsc{Lucretius}, ii. 75, 78.}
\end{verse}

``Shall I exchange for you this beautiful contexture of things? 'Tis
the condition of your creation; death is a part of you, and whilst you
endeavour to evade it, you evade yourselves. This very being of yours
that you now enjoy is equally divided betwixt life and death. The day
of your birth is one day's advance towards the grave.

\begin{verse}
\poke{```}Prima, qu\ae\ vitam dedit, hora carpsit.'\footnote{``The
first hour that gave us life, took away also an
hour.''---\textsc{Seneca}, \textit{Her. Fur.}, 3 Chor. 874.}
\end{verse}

\begin{verse}
\poke{```}Nascentes morimur, finisque ab origine
pendet.'\footnote{``As we are born, we die, and the end commences with
the beginning.''---\textsc{Manilius}, \textit{Ast}., iv. 16.}
\end{verse}

\noindent ``All the whole time you live, you purloin from life, and
live at the expense of life itself. The perpetual work of your life is
but to lay the foundation of death. You are in death, whilst you are
in life, because you still are after death, when you are no more
alive; or, if you had rather have it so, you are dead after life, but
dying all the while you live; and death handles the dying much more
rudely than the dead, and more sensibly and essentially. If you have
made your profit of life, you have had enough of it; go your way
satisfied.

\begin{verse}
\poke{```}Cur non ut plenus vit\ae\ conviva recedis?'\footnote{``Why
not depart from life, as a sated guest from a
feast?''---\textsc{Lucretius}, iii. 951.}
\end{verse}

\noindent ``If you have not known how to make the best use of it, if
\page{80}it was unprofitable to you, what need you care to lose it, to
what end would you desire longer to keep it?

\begin{verse}
\vin \poke{```}Cur amplius addere qu\ae ris,\\
Rursum quod pereat mal\`e, et ingratum occidat omne?'\footnote{``Why
seek to add longer life, merely to renew ill-spent time, and be again
tormented?''---\textsc{Lucretius}, iii. 914.}
\end{verse}

\noindent ``Life in itself is neither good nor evil; it is the scene of
good or evil, as you make it. And, if you have lived a day, you have
seen all: one day is equal and like to all other days. There is no
other light, no other shade; this very sun, this moon, these very
stars, this very order and disposition of things, is the same your
ancestors enjoyed, and that shall also entertain your posterity.

\begin{verse}
\poke{```}Non alium videre patres, aliumve nepotes\\
Aspicient.'\footnote{``Your grandsires saw no other things; nor will
your posterity.''---\textsc{Manilius}, i. 529.}
\end{verse}

\noindent ``And, come the worst that can come, the distribution and
variety of all the acts of my comedy are performed in a year. If you
have observed the revolution of my four seasons, they comprehend the
infancy, the youth, the virility, and the old age of the world: the
year has played his part, and knows no other art but to begin again;
it will always be the same thing.

\begin{verse}
\poke{```}Versamur ibidem, atque insumus usque.'\footnote{``We are
even turning in the same circle, ever therein
confined.''---\textsc{Lucretius}, iii. 1093.}
\end{verse}

\begin{verse}
\poke{```}Atque in se sua per vestigia volvitur annus.'\footnote{``The
year is even turning round in the same footsteps.''---\textsc{Virgil},
\textit{Georg}., ii. 402.}
\end{verse}

``I am not prepared to create for you any new recreations.

\begin{verse}
\poke{```}Nam tibi pr\ae terea quod machiner, inveniamque\\ Quod
placeat, nihil est; eadem sunt omnia semper.'\footnote{``I can devise,
nor find anything else to please you: 'tis the same thing over and
over again.''---\textsc{Lucretius}, iii. 957.}
\end{verse}

``Give place to others, as others have given place to you. Equality is
the soul of equity. Who can complain of being comprehended in the same
destiny, wherein all are involved? \page{81} Besides, live as long
as you can, you shall by that nothing shorten the space you are to be
dead; 'tis all to no purpose; you shall be every whit as long in the
condition you so much fear, as if you had died at nurse.

\begin{verse}
\vin \poke{```}Licet quot vis vivendo vincere secla,\\
Mora \ae terna tamen nihilominus illa manebit.'\footnote{``Live
triumphing over as many ages as you will, death still will remain
eternal.''---\textsc{Lucretius}, iii. 1103.}
\end{verse}

\noindent ``And yet I will place you in such a condition as you shall
have no reason to be displeased.

\begin{verse}
\poke{```}In vera nescis nullum fore morte alium te,\\ Qui possit
vivus tibi te lugere peremptum,\\ Stansque jacentem.'\footnote{``Know
you not that, when dead, there can be no other living self to lament
you dead, standing on your grave.''---\textit{Idem}, \textit{ibid}.,
898.}
\end{verse}

\noindent ``Nor shall you so much as wish for the life you are so
concerned about.

\begin{verse}
\poke{```}Nec sibi enim quisquam tum se vitamque
requirit.\footnote{``No one then troubles himself about himself, or
about life.''---\textit{Idem}, \textit{ibid}., 932.}\\
\dotfill\\
Nec desiderium nostri nos afficit ullum.'\footnote{``Nor has any
regret about himself.''---\textit{Idem}, \textit{ibid}., 935.}
\end{verse}

``Death is less to be feared than nothing, if there could be anything
less than nothing.

\begin{verse}
\poke{```}Multo\ldots mortem minus ad nos esse
putandum,\footnote{``Death would seem much less to us---if indeed
there could be less in that which we see to be
nothing.''---\textit{Idem}, \textit{ibid}., 939.}\\ Si minus esse
potest, quam quod nihil esse videmus.'
\end{verse}

\noindent ``Neither can it any way concern you, whether you are living
or dead: living, by reason that you are still in being; dead, because
you are no more. Moreover, no one dies before his hour: the time you
leave behind was no more yours, than that was lapsed and gone before
you came into the world; nor does it any more concern you.

\begin{verse}
\page{82}\poke{```}Respice enim, quam nil ad nos anteacta vetustas\\
Temporis \ae terni fuerit.''\footnote{``Consider, how as nothing to us
is the old age of times past.''---\textsc{Lucretius}, iii. 985.}
\end{verse}

``Wherever your life ends, it is all there. The utility of living
consists not in the length of days, but in the use of time; a man may
have lived long, and yet lived but a little. Make use of time while it
is present with you. It depends upon your will, and not upon the
number of days, to have a sufficient length of life. Is it possible
you can imagine never to arrive at the place towards which you are
continually going? and yet there is no journey but hath its end.
And, if company will make it more pleasant or more easy to you, does
not all the world go the self-same way?

\begin{verse}
\poke{```}Omnia te, vit\^a perfuncta, sequentur.'\footnote{``All
things, then, life over, must follow thee.''---\textit{Idem},
\textit{ibid}., 981.}
\end{verse}

\noindent ``Does not all the world dance the same brawl that you do?
Is there anything that does not grow old, as well as you? A thousand
men, a thousand animals, a thousand other creatures, die at the same
moment that you die:---

\begin{verse}
\poke{```}Nam nox nulla diem, neque noctem aurora sequuta est,\\
Qu\ae\ non audierit mistos vagitibus \ae gris\\ Ploratus, mortis
comites et funeris atri.'\footnote{``No night has followed day, no day
has followed night, in which there has not been heard sobs and
sorrowing cries, the companions of death and
funerals.''---\textit{Idem}, v. 579.}
\end{verse}

``To what end should you endeavour to draw back, if there be no
possibility to evade it? you have seen examples enough of those who
have been well pleased to die, as thereby delivered from heavy
miseries; but have you ever found any who have been dissatisfied with
dying? It must, therefore, needs be very foolish to condemn a thing
you have neither experimented in your own person, nor by that of any
other. Why dost thou complain of me and of destiny? Do we do thee any
wrong? Is it for thee to govern us, or for us to govern thee? Though,
peradventure, thy age may not be accomplished, yet thy life is: a man
of low stature is as much a man as a giant; neither men nor their
\page{83} lives are measured by the ell. Chiron refused to be
immortal, when he was acquainted with the conditions under which he
was to enjoy it, by the god of time itself and its duration, his
father Saturn. Do but seriously consider how much more insupportable
and painful an immortal life would be to man than what I have already
given him. If you had not death, you would eternally curse me for
having deprived you of it; I have mixed a little bitterness with it,
to the end, that seeing of what convenience it is, you might not too
greedily and indiscreetly seek and embrace it: and that you might be
so established in this moderation, as neither to nauseate life, nor
have an antipathy for dying, which I have decreed you shall once do, I
have tempered the one and the other betwixt pleasure and pain. It was
I that taught Thales, the most eminent of your sages, that to live and
to die were indifferent; which made him, very wisely, answer him, `Why
then he did not die?' `Because,' said he, `it is
indifferent.'\footnote{Diogenes Laertius, i. 35.} Water, earth, air,
and fire, and the other parts of this creation of mine, are no more
instruments of thy life than they are of thy death. Why dost thou fear
thy last day? it contributes no more to thy dissolution, than every
one of the rest: the last step is not the cause of lassitude: it does
but confess it. Every day travels towards death: the last only arrives
at it.'' These are the good lessons our mother Nature teaches.

I have often considered with myself whence it should proceed, that in
war the image of death, whether we look upon it in ourselves or in
others, should, without comparison, appear less dreadful than at home
in our own houses (for if it were not so, it would be an army of
doctors and whining milksops), and that being still in all places the
same, there should be, notwithstanding, much more assurance in
peasants and the meaner sort of people, than in others of better
quality. I believe, in truth, that it is those terrible ceremonies and
preparations wherewith we set it out, that more terrify us than the
thing itself; a new, quite contrary way of living; the cries of
mothers, wives, and children; the visits of astounded and afflicted
\page{84} friends; the attendance of pale and blubbering servants; a
dark room, set round with burning tapers; our beds environed with
physicians and divines; in sum, nothing but ghostliness and horror
round about us; we seem dead and buried already. Children are afraid
even of those they are best acquainted with, when disguised in a
visor; and so 'tis with us; the visor must be removed as well from
things as from persons; that being taken away, we shall find nothing
underneath but the very same death that a mean servant, or a poor
chambermaid, died a day or two ago, without any manner of
apprehension. Happy is the death that leaves us no leisure to prepare
things for all this foppery.\footnote{Seneca, Ep., 120.}

