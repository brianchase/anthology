
\author{Cesare Beccaria}
\authdate{1738--1794}
\textdate{1764}
\addon{Introduction and Chapters 1, 2, 16, and 28}
\chapter[On Crimes and Punishments, excerpt]{On Crimes and Punishments}
\source{beccaria1872}

\page{11}\section*{Introduction}

In every human society, there is an effort continually tending to
confer on one part the height of power and happiness, and to reduce
the other to the extreme of weakness and misery. The intent of good
laws is to oppose this effort, and to diffuse their influence
universally and equally. But men generally abandon the care of their
most important concerns to the uncertain prudence and discretion of
those, whose interest it is to reject the best and wisest
institutions; and it is not till they have been led into a thousand
mistakes, in matters the most essential to their lives and liberties,
and are weary of suffering, that they can be induced to apply a remedy
to the evils with which \page{12} they are oppressed. It is then they
begin to conceive, and acknowledge the most palpable truths, which,
from their very simplicity, commonly escape vulgar minds, incapable of
analysing objects, accustomed to receive impressions without
distinction, and to be determined rather by the opinions of others,
than by the result of their own examination.

If we look into history we shall find that laws which are, or ought to
be, conventions between men in a state of freedom, have been, for the
most part, the work of the passions of a few, or the consequences of a
fortuitous or temporary necessity; not dictated by a cool examiner of
human nature, who knew how to collect in one point the actions of a
multitude, and had this only end in view, \textit{the greatest
happiness of the greatest number}. Happy are those few nations who
have not waited till the slow succession of human vicissitudes should,
from the extremity of evil, produce a transition to good; but, by
prudent laws, have facilitated the progress from one to the other! And
how great are the obligations due from mankind to that philosopher,
who, from the obscurity of his closet, had the courage to scatter
among the multitude the seeds of useful truths, so long unfruitful!

\page{13}The art of printing has diffused the knowledge of those
philosophical truths, by which the relations between sovereigns and
their subjects, and between nations, are discovered. By this knowledge
commerce is animated, and there has sprung up a spirit of emulation
and industry worthy of rational beings. These are the produce of this
enlightened age; but the cruelty of punishments, and the irregularity
of proceeding in criminal cases, so principal a part of the
legislation, and so much neglected throughout Europe, has hardly ever
been called in question. Errors, accumulated through many centuries,
have never been exposed by ascending to general principles; nor has
the force of acknowledged truths been ever opposed to the unbounded
licentiousness of ill-directed power, which has continually produced
so many authorized examples of the most unfeeling barbarity. Surely,
the groans of the weak, sacrificed to the cruel ignorance and
indolence of the powerful; the barbarous torments lavished and
multiplied with useless severity, for crimes either not proved, or in
their nature impossible; the filth and horrors of a prison, increased
by the most cruel tormentor of the miserable, uncertainty, ought to
have roused the attention of those, whose business is to direct the
opinions of mankind.

\page{14}The immortal \textit{Montesquieu} has but slightly touched on
this subject. Truth, which is eternally the same, has obliged me to
follow the steps of that great man; but the studious part of mankind,
for whom I write, will easily distinguish the superstructure from the
foundation. I shall be happy, if, with him, I can obtain the secret
thanks of the obscure and peaceful disciples of reason and philosophy,
and excite that tender emotion, in which sensible minds sympathise
with him who pleads the cause of humanity.

\page{15}\section*{Chapter I. Of the Origin of Punishments}

Laws are the conditions under which men, naturally independent, united
themselves in society. Weary of living in a continual state of war,
and of enjoying a liberty which became of little value, from the
uncertainty of its duration, they sacrificed one part of it to enjoy
the rest in peace and security. The sum of all these portions of the
liberty of each individual constituted the sover-\page{16}eignty of a
nation; and was deposited in the hands of the sovereign, as the lawful
administrator. But it was not sufficient only to establish this
deposit; it was also necessary to defend it from the usurpation of
each individual, who will always endeavour to take away from the mass,
not only his own portion, but to encroach on that of others. Some
motives, therefore, that strike the senses, were necessary to prevent
the despotism of each individual from plunging society into its former
chaos. Such motives are the punishment established against the
infractors of the laws. I say that motives of this kind are necessary;
because experience shews that, the multitude adopt no established
rules of conduct; and because, society is prevented from approaching
to that dissolution (to which, as well as all other parts of the
physical and moral world, it naturally tends) only by motives that are
the immediate objects of sense, and which, being continually presented
to the mind, are sufficient to counterbalance the effects of the
passions of the individual which oppose the general good. Neither the
power of eloquence, nor the sublimest truths, are sufficient to
restrain, for any length of time, those passions which are excited by
the lively impression of present objects.

\page{17}\section*{Chapter II. Of the Right to Punish}

Every punishment which does not arise from absolute necessity, says
the great \textit{Montesquieu}, is tyrannical. A proposition which may
be made more general, thus. Every act of authority of one man over
another, for which there is not an absolute necessity, is tyrannical.
It is upon this, then, that the sovereign's right to punish crimes is
founded; that is, upon the necessity of defending the public liberty,
intrusted to his care, from the usurpation of individuals; and
punishments are just in proportion as the liberty, preserved by the
sovereign, is sacred and valuable.

Let us consult the human heart, and there we shall find the foundation
of the sovereign's right to punish; for no advantage in moral policy
can be lasting, which is not founded on the indeliable sentiments of
the heart of man. Whatever law deviates from this principle will
always meet with a resistance, which will destroy it in the end; for
the smallest force, continually applied, \page{18} will overcome the
most violent motion communicated to bodies.

No man ever gave up his liberty merely for the good of the public.
Such a chimera exists only in romances. Every individual wishes, if
possible, to be exempt from the compacts that bind the rest of
mankind.

The multiplication of mankind, though slow, being too great for the
means which the earth, in its natural state, offered to satisfy
necessities, which every day became more numerous, obliged men to
separate again, and form new societies. These naturally opposed the
first, and a state of war was transferred from individuals to nations.

Thus it was necessity that forced men to give up a part of their
liberty; it is certain, then, that every individual would chuse to put
into the public stock the smallest portion possible; as much only as
was sufficient to engage others to defend it. The aggregate of these,
the smallest portions possible, forms the right of punishing: all that
extends beyond this is abuse, not justice.

Observe, that by \textit{justice} I understand nothing more than that
bond, which is necessary to keep the interest of individuals united;
without which, men would return to the original state of barbarity.
All punishments, which exceed the necessity of \page{19} preserving
this bond, are in their nature unjust. We should be cautious how we
associate with the word \textit{justice}, an idea of anything real,
such as a physical power, or a being that actually exists. I do not,
by any means, speak of the justice of God, which is of another kind,
and refers immediately to rewards and punishments in a life to come.

\page{58}\section*{Chapter XVI. Of Torture}

The torture of a criminal, during the course of his trial, is a
cruelty, consecrated by custom in most nations. It is used with an
intent either to make him confess his crime, or explain some
contradictions, into which he had been led during his examination; or
discover his accomplices; or for some kind of metaphysical and
incomprehensible purgation of infamy; or, finally, in order to
discover other crimes, of which he is not accused, but of which he may
be guilty.

No man can be judged a criminal until he be found guilty; nor can
society take from him the public protection, until it have been proved
that he has violated the conditions on which it was granted. What
right, then, but that of power, can authorise the punishment of a
citizen, so long \page{59} as there remains any doubt of his guilt?
The dilemma is frequent. Either he is guilty, or not guilty. If
guilty, he should only suffer the punishment ordained by the laws, and
torture becomes useless, as his confession is unnecessary. If he be
not guilty, you torture the innocent; for, in the eye of the law,
every man is innocent, whose crime has not been proved. Besides, it is
confounding all relations, to expect that a man should be both the
accuser and accused; and that pain should be the test of truth, as if
truth resided in the muscles and fibres of a wretch in torture. By
this method, the robust will escape, and the feeble be condemned.
These are the inconveniencies of this pretended test of truth, worthy
only of a cannibal; and which the Romans, in many respects barbarous,
and whose savage virtue has been too much admired, reserved for the
slaves alone.

What is the political intention of punishments? To terrify, and to be
an example to others. Is this intention answered, by thus privately
torturing the guilty and the innocent? It is doubtless of importance,
that no crime should remain unpunished; but it is useless to make a
public example of the author of a crime hid in darkness. A crime
already committed, and for which there can be no remedy, can only be
punished by a \page{60} political society, with an intention that no
hopes of impunity should induce others to commit the same. If it be
true, that the number of those, who, from fear or virtue, respect the
laws, is greater than of those by whom they are violated, the risk of
torturing an innocent person is greater, as there is a greater
probability that, \textit{c\ae teris paribus}, an individual hath
observed, than that he hath infringed the laws.

There is another ridiculous motive for torture, namely, \textit{to
purge a man from infamy}. Ought such an abuse to be tolerated in the
eighteenth century? Can pain, which is a sensation, have any
connection with a moral sentiment, a matter of opinion? Perhaps the
rack may be considered as a refiner's furnace.

It is not difficult to trace this senseless law to its origin; for an
absurdity, adopted by a whole nation, must have some affinity with
other ideas, established and respected by the same nation. This custom
seems to be the offspring of religion, by which mankind, in all
nations and in all ages, are so generally influenced. We are taught by
our infallible church, that those stains of sin, contracted through
human frailty, and which have not deserved the eternal anger of the
Almighty, are to be purged away, in another life, by an \page{61}
incomprehensible fire. Now infamy is a stain, and if the punishments
and fire of purgatory can take away all spiritual stains, why should
not the pain of torture take away those of a civil nature? I imagine
that the confession of a criminal, which in some tribunals is
required, as being essential to his condemnation, has a similar
origin, and has been taken from the mysterious tribunal of penitence,
where the confession of sins is a necessary part of the sacrament.
Thus have men abused the unerring light of revelation; and in the
times of tractable ignorance, having no other, they naturally had
recourse to it on every occasion, making the most remote and absurd
applications. Moreover, infamy is a sentiment regulated neither by the
laws nor by reason, but entirely by opinion. But torture renders the
victim infamous, and therefore cannot take infamy away.

Another intention of torture is, to oblige the supposed criminal to
reconcile the contradictions into which he may have fallen during his
examination; as if the dread of punishment, the uncertainty of his
fate, the solemnity of the court, the majesty of the judge, and the
ignorance of the accused, were not abundantly sufficient to account
for contradictions, which are so common to men even in a state of
tranquillity; and which must \page{62} necessarily be multiplied by
the perturbation of the mind of a man, entirely engaged in the thought
of saving himself from imminent danger.

This infamous test of truth is a remaining monument of that ancient
and savage legislation, in which trials by fire, by boiling water, or
the uncertainty of combats, were called \textit{judgments of God;} as
if the links of that eternal chain, whose beginning is in the breast
of the first cause of all things, could never be disunited by the
institutions of men. The only difference between torture, and trials
by fire and boiling water, is, that the event of the first depends on
the will of the accused; and of the second, on a fact entirely
physical and external: but this difference is apparent only, not real.
A man on the rack, in the convulsions of torture, has it as little in
his power to declare the truth, as, in former times, to prevent,
without fraud, the effect of fire or of boiling water.

Every act of the will is invariably in proportion to the force of the
impression on our senses. The impression of pain, then, may increase
to such a degree, that, occupying the mind entirely, it will compel
the sufferer to use the shortest method of freeing himself from
torment. His answer, therefore, will be an effect as necessary as that
of fire or boiling water; and he will \page{63} accuse himself of
crimes of which he is innocent. So that the very means employed to
distinguish the innocent from the guilty, will most effectually
destroy all difference between them.

It would be superfluous to confirm these reflections by examples of
innocent persons, who from the agony of torture have confessed
themselves guilty: innumerable instances may be found in all nations,
and in every age. How amazing, that mankind have always neglected to
draw the natural conclusion! Lives there a man who, if he have carried
his thoughts ever so little beyond the necessities of life, when he
reflects on such cruelty, is not tempted to fly from society, and
return to his natural state of independence?

% NOTE: the source has 'calcution' instead of 'calculation'

The result of torture, then, is a matter of calculation, and depends
on the constitution, which differs in every individual, and is in
proportion to his strength and sensibility; so that to discover truth
by this method, is a problem which may be better resolved by a
mathematician than a judge, and may be thus stated: \textit{The force
of the muscles, and the sensibility of the nerves of an innocent
person being given, it is required to find the degree of pain
necessary to make him confess himself guilty of a given crime}.

The examination of the accused is intended to \page{64} find out the
truth; but if this be discovered with so much difficulty, in the air,
gesture, and countenance of a man at ease, how can it appear in a
countenance distorted by the convulsions of torture. Every violent
action destroys those small alterations in the features, which
sometimes disclose the sentiments of the heart.

These truths were known to the Roman legislators, amongst whom, as I
have already observed, slaves, only, who were not considered as
citizens, were tortured. They are known to the English, a nation in
which the progress of science, superiority in commerce, riches and
power, its natural consequences, together with the numerous examples
of virtue and courage, leave no doubt of the excellence of its laws.
They have been acknowledged in Sweden, where torture has been
abolished. They are known to one of the wisest monarchs in Europe,
who, having seated philosophy on the throne, by his beneficent
legislation, has made his subjects free, though dependent on the laws;
the only freedom that reasonable men can desire in the present state
of things. In short, torture has not been thought necessary in the
laws of armies, composed chiefly of the dregs of mankind, where its
use should seem most necessary. Strange phenomenon! that a set of men,
hardened by \page{65} slaughter, and familiar with blood, should teach
humanity to the sons of peace.

It appears also, that these truths were known, though imperfectly,
even to those by whom torture has been most frequently practised; for
a confession made during torture is null, if it be not afterwards
confirmed by an oath; which, if the criminal refuses, he is tortured
again. Some civilians, and some nations, permit this infamous
\textit{petitio principii} to be only three times repeated, and others
leave it to the discretion of the judge; and therefore of two men
equally innocent or equally guilty, the most robust and resolute will
be acquitted, and the weakest and most pusillanimous will be
condemned, in consequence of the following excellent method of
reasoning. \textit{I, the judge, must find some one guilty. Thou, who
art a strong fellow, hast been able to resist the force of torment;
therefore I acquit thee. Thou, being weaker, hath yielded to it; I
therefore condemn thee. I am sensible, that the confession which was
extorted from thee, has no weight: but if thou dost not confirm by
oath what thou hast already confessed, I will have thee tormented
again}.

A very strange but necessary consequence of the use of torture, is
that the case of the innocent is worse than that of the guilty. With
regard to \page{66} the first, either he confesses the crime, which he
has not committed, and is condemned; or he is acquitted, and has
suffered a punishment he did not deserve. On the contrary, the person
who is really guilty has the most favourable side of the question; for
if he supports the torture with firmness and resolution, he is
acquitted, and has gained, having exchanged a greater punishment for a
less.

The law by which torture is authorised, says, \textit{Men, be
insensible to pain. Nature has indeed given you an irresistible
self-love, and an unalienable right of self-preservation, but I create
in you a contrary sentiment, an heroical hatred of yourselves. I
command you to accuse yourselves, and to declare the truth, midst the
tearing of your flesh and the dislocation of your bones}.

Torture is used to discover, whether the criminal be guilty of other
crimes besides those of which he is accused: which is equivalent to
the following reasoning: \textit{Thou art guilty of one crime,
therefore it is possible that thou mayst have committed a thousand
others: but the affair being doubtful, I must try it by my criterion
of truth. The laws order thee to be tormented, because thou art
guilty, because thou mayst be guilty, and because I chuse thou
shouldst be guilty}.

\page{67}Torture is used to make the criminal discover his
accomplices; but if it has been demonstrated that it is not a proper
means of discovering truth, how can it serve to discover the
accomplices, which is one of the truths required. Will not the man who
accuses himself, yet more readily accuse others? Besides, is it just
to torment one man for the crime of another? May not the accomplices
be found out by the examination of the witnesses, or of the criminal;
from the evidence, or from the nature of the crime itself; in short,
by all the means that have been used to prove the guilt of the
prisoner? The accomplices commonly fly when their comrade is taken.
The uncertainty of their fate condemns them to perpetual exile, and
frees society from the danger of further injury; whilst the punishment
of the criminal, by deterring others, answers the purpose for which it
was ordained.

\page{97}\section*{Chapter XXVIII. Of the Punishment of Death}

The useless profusion of punishments, which has never made men better,
induces me to inquire, whether the punishment of \textit{death} be
really just or useful in a well-governed state? What \textit{right}, I
ask, have men to cut the throats of their fel\-low-crea\-tures?
Certainly not that on which the sovereignty and laws are founded. The
laws, as I have said before, are only the sum of the smallest portions
of the private liberty of each individual, and represent the general
will, which is the aggregate of that of each individual. Did any one
ever give to others the right of taking away his life? Is it possible,
that in the smallest portions of the liberty of each, sacrificed to
the good of the public, can be obtained the greatest \page{98} of all
good, life? If it were so, how shall it be reconciled to the maxim
which tells us, that a man has no right to kill himself? Which he
certainly must have, if he could give it away to another.

But the punishment of death is not authorised by any right; for I have
demonstrated that no such right exists. It is therefore a war of a
whole nation against a citizen, whose destruction they consider as
necessary or useful to the general good. But if I can further
demonstrate, that it is neither necessary nor useful, I shall have
gained the cause of humanity.

The death of a citizen cannot be necessary but in one case. When,
though deprived of his liberty, he has such power and connections as
may endanger the security of the nation; when his existence may
produce a dangerous revolution in the established form of government.
But even in this case, it can only be necessary when a nation is on
the verge of recovering or losing its liberty; or in times of absolute
anarchy, when the disorders themselves hold the place of laws. But in
a reign of tranquillity; in a form of government approved by the
united wishes of the nation; in a state fortified from enemies
without, and supported by strength within, and opinion, \page{99}
perhaps more efficacious; where all power is lodged in the hands of
the true sovereign; where riches can purchase pleasures and not
authority, there can be no necessity for taking away the life of a
subject.

If the experience of all ages be not sufficient to prove, that the
punishment of death has never prevented determined men from injuring
society; if the example of the Romans; if twenty years reign of
Elizabeth, empress of Russia, in which she gave the fathers of their
country an example more illustrious than many conquests bought with
blood; if, I say, all this be not sufficient to persuade mankind, who
always suspect the voice of reason, and who chuse rather to be led by
authority, let us consult human nature in proof of my assertion.

It is not the intenseness of the pain that has the greatest effect on
the mind, but its continuance; for our sensibility is more easily and
more powerfully affected by weak, but by repeated impressions, than by
a violent but momentary impulse. The power of habit is universal over
every sensible being. As it is by that we learn to speak, to walk, and
to satisfy our necessities, so the ideas of morality are stamped on
our minds by repeated impressions. The death of a crimi-\page{100}nal
is a terrible but momentary spectacle, and therefore a less
efficacious method of deterring others, than the continued example of
a man deprived of his liberty, condemned as a beast of burden, to
repair, by his labour, the injury he has done to society. \textit{If I
commit such a crime}, says the spectator to himself, \textit{I shall
be reduced to that miserable condition for the rest of my life}. A
much more powerful preventive than the fear of death, which men always
behold in distant obscurity.

The terrors of death make so slight an impression, that it has not
force enough to withstand the forgetfulness natural to mankind, even
in the most essential things; especially when assisted by the
passions. Violent impressions surprise us, but their effect is
momentary; they are fit to produce those revolutions which instantly
transform a common man into a Laced\ae monian or a Persian; but in a
free and quiet government they ought to be rather frequent than
strong.

The execution of a criminal is, to the multitude, a spectacle which in
some excites compassion mixed with indignation. These sentiments
occupy the mind much more than that salutary terror which the laws
endeavour to inspire; but in the contemplation of continued suffering,
terror is the only, or at least, the predominant sensation. \page{101}
The severity of a punishment should be just sufficient to excite
compassion in the spectators, as it is intended more for them than for
the criminal.

A punishment, to be just, should have only that degree of severity
which is sufficient to deter others. Now there is no man, who, upon
the least reflection, would put in competition the total and perpetual
loss of his liberty, with the greatest advantages he could possibly
obtain in consequence of a crime. Perpetual slavery, then, has in it
all that is necessary to deter the most hardened and determined, as
much as the punishment of death. I say, it has more. There are many
who can look upon death with intrepidity and firmness; some through
fanaticism, and others through vanity, which attends us even to the
grave; others from a desperate resolution, either to get rid of their
misery, or cease to live: but fanaticism and vanity forsake the
criminal in slavery, in chains and fetters, in an iron cage; and
despair seems rather the beginning than the end of their misery. The
mind, by collecting itself and uniting all its force, can, for a
moment, repel assailing grief; but its most vigorous efforts are
insufficient to resist perpetual wretchedness.

In all nations, where death is used as punishment, every example
supposes a new crime com-\page{102}mitted. Whereas, in perpetual
slavery, every criminal affords a frequent and lasting example: and if
it be necessary that men should often be witnesses of the power of the
laws, criminals should often be put to death; but this supposes a
frequency of crimes; and from hence this punishment will cease to have
its effect, so that it must be useful and useless at the same time.

% NOTE: the source has 'ignoran' rather than 'ignorant' below

I shall be told, that perpetual slavery is as painful a punishment as
death, and therefore as cruel. I answer, that if all the miserable
moments in the life of a slave were collected into one point, it would
be a more cruel punishment than any other; but these are scattered
through his whole life, whilst the pain of death exerts all its force
in a moment. There is also another advantage in the punishment of
slavery, which is, that it is more terrible to the spectator than to
the sufferer himself; for the spectator considers the sum of all his
wretched moments, whilst the sufferer, by the misery of the present,
is prevented from thinking of the future. All evils are increased by
the imagination, and the sufferer finds resources and consolation, of
which the spectators are ignorant; who judge by their own sensibility
of what passes in a mind by habit grown callous to misfortune.

\page{103}Let us, for a moment, attend to the reasoning of a robber or
assassin, who is deterred from violating the laws by the gibbet or the
wheel. I am sensible, that to develop the sentiments of one's own
heart, is an art which education only can teach; but although a
villain may not be able to give a clear account of his principles,
they nevertheless influence his conduct. He reasons thus: ``What are
these laws that I am bound to respect, which make so great a
difference between me and the rich man? He refuses me the farthing I
ask of him, and excuses himself by bidding me have recourse to labour,
with which he is unacquainted. Who made these laws? The rich and the
great, who never deigned to visit the miserable hut of the poor; who
have never seen him dividing a piece of mouldy bread, amidst the cries
of his famished children, and the tears of his wife. Let us break
those ties, fatal to the greatest part of mankind, and only useful to
a few indolent tyrants. Let us attack injustice at its source. I will
return to my natural state of independence. I shall live free and
happy on the fruits of my courage and industry. A day of pain and
repentance may come, but it will be short; and for an hour of grief, I
shall enjoy years of pleasure and liberty. King of a small number, as
determined \page{104} as myself, I will correct the mistakes of
fortune; and shall see those tyrants grow pale and tremble at the
sight of him, whom, with insulting pride, they would not suffer to
rank with dogs and horses.''

Religion then presents itself to the mind of this lawless villain, and
promising him almost a certainty of eternal happiness upon the easy
terms of repentance, contributes much to lessen the horror of the last
scene of the tragedy.

But he who foresees that he must pass a great number of years, even
his whole life, in pain and slavery; a slave to those laws by which he
was protected; in sight of his fellow citizens, with whom he lives in
freedom and society; makes an useful comparison between those evils,
the uncertainty of his success, and the shortness of the time in which
he shall enjoy the fruits of his transgression. The example of those
wretches continually before his eyes, makes a much greater impression
on him than a punishment, which, instead of correcting, makes him more
obdurate.

The punishment of death is pernicious to society, from the example of
barbarity it affords. If the passions, or necessity of war, have
taught men to shed the blood of their fellow creatures, the laws which
are intended to moderate the ferocity of mankind, should not increase
it by examples \page{105} of barbarity, the more horrible, as this
punishment is usually attended with formal pageantry. Is it not
absurd, that the laws, which detect and punish homicide, should, in
order to prevent murder, publicly commit murder themselves? What are
the true and most useful laws? Those compacts and conditions which all
would propose and observe, in those moments when private interest is
silent, or combined with that of the public. What are the natural
sentiments of every person concerning the punishment of death? We may
read them in the contempt and indignation with which every one looks
on the executioner, who is nevertheless an innocent executor of the
public will; a good citizen, who contributes to the advantage of
society; the instrument of the general security within, as good
soldiers are without. What then is the origin of this contradiction?
Why is this sentiment of mankind indelible to the scandal of reason?
It is, that in a secret corner of the mind, in which the original
impressions of nature are still preserved, men discover a sentiment
which tells them, that their lives are not lawfully in the power of
any one, but of that necessity only, which with its iron sceptre rules
the universe.

What must men think, when they see wise \page{106} magistrates and
grave ministers of justice, with indifference and tranquillity,
dragging a criminal to death, and whilst a wretch trembles with agony,
expecting the fatal stroke, the judge, who has condemned him, with the
coldest insensibility, and perhaps with no small gratification from
the exertion of his authority, quits his tribunal to enjoy the
comforts and pleasures of life? They will say, ``Ah! those cruel
formalities of justice are a cloak to tyranny, they are a secret
language, a solemn veil, intended to conceal the sword by which we are
sacrificed to the insatiable idol of despotism. Murder, which they
would represent to us as an horrible crime, we see practiced by them
without repugnance or remorse. Let us follow their example. A violent
death appeared terrible in their descriptions, but we see that it is
the affair of a moment. It will be still less terrible to him, who,
not expecting it, escapes almost all the pain.'' Such is the fatal,
though absurd reasoning of men who are disposed to commit crimes; on
whom the abuse of religion has more influence than religion itself.

If it be objected, that almost all nations in all ages have punished
certain crimes with death, I answer that the force of these examples
vanishes, when opposed to truth, against which prescription \page{107}
is urged in vain. The history of mankind is an immense sea of errors,
in which a few obscure truths may here and there be found.

But human sacrifices have also been common in almost all nations. That
some societies only, either few in number, or for a very short time,
abstained from the punishment of death, is rather favourable to my
argument, for such is the fate of great truths, that their duration is
only as a flash of lightning in the long and dark night of error. The
happy time is not yet arrived, when truth, as falsehood has been
hitherto, shall be the portion of the greatest number.

I am sensible that the voice of one philosopher is too weak to be
heard amidst the clamours of a multitude, blindly influenced by
custom; but there is a small number of sages, scattered on the face of
the earth, who will echo to me from the bottom of their hearts; and if
these truths should happily force their way to the thrones of princes,
be it known to them, that they come attended with the secret wishes of
all mankind, and tell the sovereign who deigns them a gracious
reception, that his fame shall outshine the glory of conquerors, and
that equitable posterity will exalt his peaceful trophies above those
of a Titus, an Antoninus, or a Trajan.

\page{108}How happy were mankind, if laws were now to be first formed!
now that we see on the thrones of Europe benevolent monarchs, friends
to the virtues of peace, to the arts and sciences, fathers of their
people, though crowned yet citizens; the increase of whose authority
augments the happiness of their subjects, by destroying that
intermediate despotism which intercepts the prayers of the people to
the throne. If these humane princes have suffered the old laws to
subsist, it is doubtless because they are deterred by the numberless
obstacles which oppose the subversion of errors established by the
sanction of many ages; and therefore every wise citizen will wish for
the increase of their authority.

