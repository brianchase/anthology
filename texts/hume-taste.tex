
\author{David Hume}
\authdate{1711--1776}
\textdate{1757}
\chapter[Of the Standard of Taste]{Of the Standard of
Taste\source{hume1889.1.23}\footnote{[This Essay was first published
in Edition L.]}}

\page{266}The great variety of Taste, as well as of opinion, which
prevails in the world, is too obvious not to have fallen under every
one's observation. Men of the most confined knowledge are able to
remark a difference of taste in the narrow circle of their
acquaintance, even where the persons have been educated under the same
government, and have early imbibed the same prejudices. But those, who
can enlarge their view to contemplate distant nations and remote ages,
are still more surprized at the great inconsistence and contrariety.
We are apt to call \textit{barbarous} whatever departs widely from our
own taste and apprehension: But soon find the epithet of reproach
retorted on us. And the highest arrogance and self-conceit is at last
startled, on observing an equal assurance on all sides, and scruples,
amidst such a contest of sentiment, to pronounce positively in its own
favour.

As this variety of taste is obvious to the most careless enquirer; so
will it be found, on examination, to be still greater in reality than
in appearance. The sentiments of men often differ with regard to
beauty and deformity of all kinds, even while their general discourse
is the same. There are certain terms in every language, which import
blame, and others praise; and all men, who use the same tongue, must
agree in their application of them. Every voice is united in
applauding elegance, propriety, simplicity, spirit in writing; and in
blaming fustian, affectation, coldness, and a false brilliancy: But
when critics come to particulars, this seeming unanimity vanishes; and
it is found, that they had affixed a very different meaning to their
expressions. In all matters of opinion and science, the case is
opposite: The difference among men is there oftener found to lie in
generals than in particulars; and to be less in reality than in
appearance. An explanation of the terms commonly ends the controversy;
and the disputants are surprized to find, that they had been
quarrelling, while at bottom they agreed in their judgment.

Those who found morality on sentiment, more than on reason, are
inclined to comprehend ethics under the former observation, and to
maintain, that, in all questions, which regard conduct and manners,
the difference among men is really greater than at first sight it
appears. It is indeed \page{267} obvious, that writers of all nations
and all ages concur in applauding justice, humanity, magnanimity,
prudence, veracity; and in blaming the opposite qualities. Even poets
and other authors, whose compositions are chiefly calculated to please
the imagination, are yet found, from \textsc{Homer} down to
\textsc{Fenelon}, to inculcate the same moral precepts, and to bestow
their applause and blame on the same virtues and vices. This great
unanimity is usually ascribed to the influence of plain reason; which,
in all these cases, maintains similar sentiments in all men, and
prevents those controversies, to which the abstract sciences are so
much exposed. So far as the unanimity is real, this account may be
admitted as satisfactory: But we must also allow that some part of the
seeming harmony in morals may be accounted for from the very nature of
language. The word \textit{virtue}, with its equivalent in every
tongue, implies praise; as that of \textit{vice} does blame: And no
one, without the most obvious and grossest impropriety, could affix
reproach to a term, which in general acceptation is understood in a
good sense; or bestow applause, where the idiom requires
disapprobation. \textsc{Homer's} general precepts, where he delivers
any such, will never be controverted; but it is obvious, that, when he
draws particular pictures of manners, and represents heroism in
\textsc{Achilles} and prudence in \textsc{Ulysses}, he intermixes a
much greater degree of ferocity in the former, and of cunning and
fraud in the latter, than \textsc{Fenelon} would admit of. The sage
\textsc{Ulysses} in the \textsc{Greek} poet seems to delight in lies
and fictions, and often employs them without any necessity or even
advantage: But his more scrupulous son, in the \textsc{French} epic
writer, exposes himself to the most imminent perils, rather than
depart from the most exact line of truth and veracity.

The admirers and followers of the \textsc{Alcoran} insist on the
excellent moral precepts interspersed throughout that wild and absurd
performance. But it is to be supposed, that the \textsc{Arabic} words,
which correspond to the \textsc{English}, equity, justice, temperance,
meekness, charity, were such as, from the constant use of that tongue,
must always be taken in a good sense; and it would have argued the
greatest ignorance, not of morals, but of language, to have mentioned
them with any epithets, besides those of applause and approbation. But
would we know, whether the pretended prophet had really attained a
just sentiment of morals? Let us attend to his \page{268} narration;
and we shall soon find, that he bestows praise on such instances of
treachery, inhumanity, cruelty, revenge, bigotry, as are utterly
incompatible with civilized society. No steady rule of right seems
there to be attended to; and every action is blamed or praised, so far
only as it is beneficial or hurtful to the true believers.

The merit of delivering true general precepts in ethics is indeed very
small. Whoever recommends any moral virtues, really does no more than
is implied in the terms themselves. That people, who invented the word
\textit{charity}, and used it in a good sense, inculcated more clearly
and much more efficaciously, the precept, \textit{be charitable}, than
any pretended legislator or prophet, who should insert such a
\textit{maxim} in his writings. Of all expressions, those, which,
together with their other meaning, imply a degree either of blame or
approbation, are the least liable to be perverted or mistaken.

It is natural for us to seek a \textit{Standard of Taste}; a rule, by
which the various sentiments of men may be reconciled; at least, a
decision, afforded, confirming one sentiment, and condemning another.

There is a species of philosophy, which cuts off all hopes of success
in such an attempt, and represents the impossibility of ever attaining
any standard of taste. The difference, it is said, is very wide
between judgment and sentiment. All sentiment is right; because
sentiment has a reference to nothing beyond itself, and is always
real, wherever a man is conscious of it. But all determinations of the
understanding are not right; because they have a reference to
something beyond themselves, to wit, real matter of fact; and are not
always conformable to that standard. Among a thousand different
opinions which different men may entertain of the same subject, there
is one, and but one, that is just and true; and the only difficulty is
to fix and ascertain it. On the contrary, a thousand different
sentiments, excited by the same object, are all right: Because no
sentiment represents what is really in the object. It only marks a
certain conformity or relation between the object and the organs or
faculties of the mind; and if that conformity did not really exist,
the sentiment could never possibly have being. Beauty is no quality in
things themselves: It exists merely in the mind which contemplates
them; and each mind perceives a different beauty. One person may even
perceive deformity, \page{269} where another is sensible of beauty;
and every individual ought to acquiesce in his own sentiment, without
pretending to regulate those of others. To seek the real beauty, or
real deformity, is as fruitless an enquiry, as to pretend to ascertain
the real sweet or real bitter. According to the disposition of the
organs, the same object may be both sweet and bitter; and the proverb
has justly determined it to be fruitless to dispute concerning tastes.
It is very natural, and even quite necessary, to extend this axiom to
mental, as well as bodily taste; and thus common sense, which is so
often at variance with philosophy, especially with the sceptical kind,
is found, in one instance at least, to agree in pronouncing the same
decision.

But though this axiom, by passing into a proverb, seems to have
attained the sanction of common sense; there is certainly a species of
common sense which opposes it, at least serves to modify and restrain
it. Whoever would assert an equality of genius and elegance between
\textsc{Ogilby} and \textsc{Milton}, or \textsc{Bunyan} and
\textsc{Addison}, would be thought to defend no less an extravagance,
than if he had maintained a mole-hill to be as high as
\textsc{Teneriffe}, or a pond as extensive as the ocean. Though there
may be found persons, who give the preference to the former authors;
no one pays attention to such a taste; and we pronounce without
scruple the sentiment of these pretended critics to be absurd and
ridiculous. The principle of the natural equality of tastes is then
totally forgot, and while we admit it on some occasions, where the
objects seem near an equality, it appears an extravagant paradox, or
rather a palpable absurdity, where objects so disproportioned are
compared together.

It is evident that none of the rules of composition are fixed by
reasonings \textit{a priori}, or can be esteemed abstract conclusions
of the understanding, from comparing those habitudes and relations of
ideas, which are eternal and immutable. Their foundation is the same
with that of all the practical sciences, experience; nor are they any
thing but general observations, concerning what has been universally
found to please in all countries and in all ages. Many of the beauties
of poetry and even of eloquence are founded on falsehood and fiction,
on hyperboles, metaphors, and an abuse or perversion of terms from
their natural meaning. To check the sallies of the imagination, and to
reduce every expression to \page{270} geometrical truth and exactness,
would be the most contrary to the laws of criticism; because it would
produce a work, which, by universal experience, has been found the
most insipid and disagreeable. But though poetry can never submit to
exact truth, it must be confined by rules of art, discovered to the
author either by genius or observation. If some negligent or irregular
writers have pleased, they have not pleased by their transgressions of
rule or order, but in spite of these transgressions: They have
possessed other beauties, which were conformable to just criticism;
and the force of these beauties has been able to overpower censure,
and give the mind a satisfaction superior to the disgust arising from
the blemishes. \textsc{Ariosto} pleases; but not by his monstrous and
improbable fictions, by his bizarre mixture of the serious and comic
styles, by the want of coherence in his stories, or by the continual
interruptions of his narration. He charms by the force and clearness
of his expression, by the readiness and variety of his inventions, and
by his natural pictures of the passions, especially those of the gay
and amorous kind: And however his faults may diminish our
satisfaction, they are not able entirely to destroy it. Did our
pleasure really arise from those parts of his poem, which we
denominate faults, this would be no objection to criticism in general:
It would only be an objection to those particular rules of criticism,
which would establish such circumstances to be faults, and would
represent them as universally blameable. If they are found to please,
they cannot be faults; let the pleasure, which they produce, be ever
so unexpected and unaccountable.

But though all the general rules of art are founded only on experience
and on the observation of the common sentiments of human nature, we
must not imagine, that, on every occasion, the feelings of men will be
conformable to these rules. Those finer emotions of the mind are of a
very tender and delicate nature, and require the concurrence of many
favourable circumstances to make them play with facility and
exactness, according to their general and established principles. The
least exterior hindrance to such small springs, or the least internal
disorder, disturbs their motion, and confounds the operation of the
whole machine. When we would make an experiment of this nature, and
would try the force of any beauty or deformity, we must choose with
care a \page{271} proper time and place, and bring the fancy to a
suitable situation and disposition. A perfect serenity of mind, a
recollection of thought, a due attention to the object; if any of
these circumstances be wanting, our experiment will be fallacious, and
we shall be unable to judge of the catholic and universal beauty. The
relation, which nature has placed between the form and the sentiment,
will at least be more obscure; and it will require greater accuracy to
trace and discern it. We shall be able to ascertain its influence not
so much from the operation of each particular beauty, as from the
durable admiration, which attends those works, that have survived all
the caprices of mode and fashion, all the mistakes of ignorance and
envy.

The same \textsc{Homer}, who pleased at \textsc{Athens} and
\textsc{Rome} two thousand years ago, is still admired at
\textsc{Paris} and at \textsc{London}. All the changes of climate,
government, religion, and language, have not been able to obscure his
glory. Authority or prejudice may give a temporary vogue to a bad poet
or orator; but his reputation will never be durable or general. When
his compositions are examined by posterity or by foreigners, the
enchantment is dissipated, and his faults appear in their true
colours. On the contrary, a real genius, the longer his works endure,
and the more wide they are spread, the more sincere is the admiration
which he meets with. Envy and jealousy have too much place in a narrow
circle; and even familiar acquaintance with his person may diminish
the applause due to his performances: But when these obstructions are
removed, the beauties, which are naturally fitted to excite agreeable
sentiments, immediately display their energy; and while the world
endures, they maintain their authority over the minds of men.

It appears then, that, amidst all the variety and caprice of taste,
there are certain general principles of approbation or blame, whose
influence a careful eye may trace in all operations of the mind. Some
particular forms or qualities, from the original structure of the
internal fabric, are calculated to please, and others to displease;
and if they fail of their effect in any particular instance, it is
from some apparent defect or imperfection in the organ. A man in a
fever would not insist on his palate as able to decide concerning
flavours; nor would one, affected with the jaundice, pretend to give a
verdict with regard to colours. In each creature, there is a
\page{272} sound and a defective state; and the former alone can be
supposed to afford us a true standard of taste and sentiment. If, in
the sound state of the organ, there be an entire or a considerable
uniformity of sentiment among men, we may thence derive an idea of the
perfect beauty; in like manner as the appearance of objects in
day-light, to the eye of a man in health, is denominated their true
and real colour, even while colour is allowed to be merely a phantasm
of the senses.

Many and frequent are the defects in the internal organs, which
prevent or weaken the influence of those general principles, on which
depends our sentiment of beauty or deformity. Though some objects, by
the structure of the mind, be naturally calculated to give pleasure,
it is not to be expected, that in every individual the pleasure will
be equally felt. Particular incidents and situations occur, which
either throw a false light on the objects, or hinder the true from
conveying to the imagination the proper sentiment and perception.

One obvious cause, why many feel not the proper sentiment of beauty,
is the want of that \textit{delicacy} of imagination, which is
requisite to convey a sensibility of those finer emotions. This
delicacy every one pretends to: Every one talks of it; and would
reduce every kind of taste or sentiment to its standard. But as our
intention in this essay is to mingle some light of the understanding
with the feelings of sentiment, it will be proper to give a more
accurate definition of delicacy, than has hitherto been attempted. And
not to draw our philosophy from too profound a source, we shall have
recourse to a noted story in \textsc{Don Quixote}.

It is with good reason, says \textsc{Sancho} to the squire with the
great nose, that I pretend to have a judgment in wine: This is a
quality hereditary in our family. Two of my kinsmen were once called
to give their opinion of a hogshead, which was supposed to be
excellent, being old and of a good vintage. One of them tastes it;
considers it; and after mature reflection pronounces the wine to be
good, were it not for a small taste of leather, which he perceived in
it. The other, after using the same precautions, gives also his
verdict in favour of the wine; but with the reserve of a taste of
iron, which he could easily distinguish. You cannot imagine how much
they were both ridiculed for their judgment. But who laughed in the
end? On emptying the hogshead, there was found at the bottom, an old
key with a leathern thong tied to it.

\page{273}The great resemblance between mental and bodily taste will
easily teach us to apply this story. Though it be certain, that beauty
and deformity, more than sweet and bitter, are not qualities in
objects, but belong entirely to the sentiment, internal or external;
it must be allowed, that there are certain qualities in objects, which
are fitted by nature to produce those particular feelings. Now as
these qualities may be found in a small degree, or may be mixed and
confounded with each other, it often happens, that the taste is not
affected with such minute qualities, or is not able to distinguish all
the particular flavours, amidst the disorder, in which they are
presented. Where the organs are so fine, as to allow nothing to escape
them; and at the same time so exact as to perceive every ingredient in
the composition: This we call delicacy of taste, whether we employ
these terms in the literal or metaphorical sense. Here then the
general rules of beauty are of use; being drawn from established
models, and from the observation of what pleases or displeases, when
presented singly and in a high degree: And if the same qualities, in a
continued composition and in a smaller degree, affect not the organs
with a sensible delight or uneasiness, we exclude the person from all
pretensions to this delicacy. To produce these general rules or avowed
patterns of composition is like finding the key with the leathern
thong; which justified the verdict of \textsc{Sancho's} kinsmen, and
confounded those pretended judges who had condemned them. Though the
hogshead had never been emptied, the taste of the one was still
equally delicate, and that of the other equally dull and languid: But
it would have been more difficult to have proved the superiority of
the former, to the conviction of every by-stander. In like manner,
though the beauties of writing had never been methodized, or reduced
to general principles; though no excellent models had ever been
acknowledged; the different degrees of taste would still have
subsisted, and the judgment of one man been preferable to that of
another; but it would not have been so easy to silence the bad critic,
who might always insist upon his particular sentiment, and refuse to
submit to his antagonist. But when we show him an avowed principle of
art; when we illustrate this principle by examples, whose operation,
from his own particular taste, he acknowledges to be conformable to
the principle; when we prove, that the same principle may be applied
to the pre-\page{274}sent case, where he did not perceive or feel its
influence: He must conclude, upon the whole, that the fault lies in
himself, and that he wants the delicacy, which is requisite to make
him sensible of every beauty and every blemish, in any composition or
discourse.

It is acknowledged to be the perfection of every sense or faculty, to
perceive with exactness its most minute objects, and allow nothing to
escape its notice and observation. The smaller the objects are, which
become sensible to the eye, the finer is that organ, and the more
elaborate its make and composition. A good palate is not tried by
strong flavours; but by a mixture of small ingredients, where we are
still sensible of each part, notwithstanding its minuteness and its
confusion with the rest. In like manner, a quick and acute perception
of beauty and deformity must be the perfection of our mental taste;
nor can a man be satisfied with himself while he suspects, that any
excellence or blemish in a discourse has passed him unobserved. In
this case, the perfection of the man, and the perfection of the sense
or feeling, are found to be united. A very delicate palate, on many
occasions, may be a great inconvenience both to a man himself and to
his friends: But a delicate taste of wit or beauty must always be a
desirable quality; because it is the source of all the finest and most
innocent enjoyments, of which human nature is susceptible. In this
decision the sentiments of all mankind are agreed. Wherever you can
ascertain a delicacy of taste, it is sure to meet with approbation;
and the best way of ascertaining it is to appeal to those models and
principles, which have been established by the uniform consent and
experience of nations and ages.

But though there be naturally a wide difference in point of delicacy
between one person and another, nothing tends further to encrease and
improve this talent, than \textit{practice} in a particular art, and
the frequent survey or contemplation of a particular species of
beauty. When objects of any kind are first presented to the eye or
imagination, the sentiment, which attends them, is obscure and
confused; and the mind is, in a great measure, incapable of
pronouncing concerning their merits or defects. The taste cannot
perceive the several excellencies of the performance; much less
distinguish the particular character of each excellency, and ascertain
its quality and degree. If it pronounce the whole in \page{275}
general to be beautiful or deformed, it is the utmost that can be
expected; and even this judgment, a person, so unpractised, will be
apt to deliver with great hesitation and reserve. But allow him to
acquire experience in those objects, his feeling becomes more exact
and nice: He not only perceives the beauties and defects of each part,
but marks the distinguishing species of each quality, and assigns it
suitable praise or blame. A clear and distinct sentiment attends him
through the whole survey of the objects; and he discerns that very
degree and kind of approbation or displeasure, which each part is
naturally fitted to produce. The mist dissipates, which seemed
formerly to hang over the object: The organ acquires greater
perfection in its operations; and can pronounce, without danger of
mistake, concerning the merits of every performance. In a word, the
same address and dexterity, which practice gives to the execution of
any work, is also acquired by the same means, in the judging of it.

So advantageous is practice to the discernment of beauty, that, before
we can give judgment on any work of importance, it will even be
requisite, that that very individual performance be more than once
perused by us, and be surveyed in different lights with attention and
deliberation. There is a flutter or hurry of thought which attends the
first perusal of any piece, and which confounds the genuine sentiment
of beauty. The relation of the parts is not discerned: The true
characters of style are little distinguished: The several perfections
and defects seem wrapped up in a species of confusion, and present
themselves indistinctly to the imagination. Not to mention, that there
is a species of beauty, which, as it is florid and superficial,
pleases at first; but being found incompatible with a just expression
either of reason or passion, soon palls upon the taste, and is then
rejected with disdain, at least rated at a much lower value.

It is impossible to continue in the practice of contemplating any
order of beauty, without being frequently obliged to form
\textit{comparisons} between the several species and degrees of
excellence, and estimating their proportion to each other. A man, who
has had no opportunity of comparing the different kinds of beauty, is
indeed totally unqualified to pronounce an opinion with regard to any
object presented to him. By comparison alone we fix the epithets of
praise or blame, and \page{276} learn how to assign the due degree of
each. The coarsest daubing contains a certain lustre of colours and
exactness of imitation, which are so far beauties, and would affect
the mind of a peasant or Indian with the highest admiration. The most
vulgar ballads are not entirely destitute of harmony or nature; and
none but a person, familiarized to superior beauties, would pronounce
their numbers harsh, or narration uninteresting. A great inferiority
of beauty gives pain to a person conversant in the highest excellence
of the kind, and is for that reason pronounced a deformity: As the
most finished object, with which we are acquainted, is naturally
supposed to have reached the pinnacle of perfection, and to be
entitled to the highest applause. One accustomed to see, and examine,
and weigh the several performances, admired in different ages and
nations, can alone rate the merits of a work exhibited to his view,
and assign its proper rank among the productions of genius.

But to enable a critic the more fully to execute this undertaking, he
must preserve his mind free from all \textit{prejudice}, and allow
nothing to enter into his consideration, but the very object which is
submitted to his examination. We may observe, that every work of art,
in order to produce its due effect on the mind, must be surveyed in a
certain point of view, and cannot be fully relished by persons, whose
situation, real or imaginary, is not conformable to that which is
required by the performance. An orator addresses himself to a
particular audience, and must have a regard to their particular
genius, interests, opinions, passions, and prejudices; otherwise he
hopes in vain to govern their resolutions, and inflame their
affections. Should they even have entertained some prepossessions
against him, however unreasonable, he must not overlook this
disadvantage; but, before he enters upon the subject, must endeavour
to conciliate their affection, and acquire their good graces. A critic
of a different age or nation, who should peruse this discourse, must
have all these circumstances in his eye, and must place himself in the
same situation as the audience, in order to form a true judgment of
the oration. In like manner, when any work is addressed to the public,
though I should have a friendship or enmity with the author, I must
depart from this situation; and considering myself as a man in
general, forget, if possible, my individual being and my peculiar
circumstances. \page{277} A person influenced by prejudice, complies
not with this condition; but obstinately maintains his natural
position, without placing himself in that point of view, which the
performance supposes. If the work be addressed to persons of a
different age or nation, he makes no allowance for their peculiar
views and prejudices; but, full of the manners of his own age and
country, rashly condemns what seemed admirable in the eyes of those
for whom alone the discourse was calculated. If the work be executed
for the public, he never sufficiently enlarges his comprehension, or
forgets his interest as a friend or enemy, as a rival or commentator.
By this means, his sentiments are perverted; nor have the same
beauties and blemishes the same influence upon him, as if he had
imposed a proper violence on his imagination, and had forgotten
himself for a moment. So far his taste evidently departs from the true
standard; and of consequence loses all credit and authority.

It is well known, that in all questions, submitted to the
understanding, prejudice is destructive of sound judgment, and
perverts all operations of the intellectual faculties: It is no less
contrary to good taste; nor has it less influence to corrupt our
sentiment of beauty. It belongs to \textit{good sense} to check its
influence in both cases; and in this respect, as well as in many
others, reason, if not an essential part of taste, is at least
requisite to the operations of this latter faculty. In all the nobler
productions of genius, there is a mutual relation and correspondence
of parts; nor can either the beauties or blemishes be perceived by
him, whose thought is not capacious enough to comprehend all those
parts, and compare them with each other, in order to perceive the
consistence and uniformity of the whole. Every work of art has also a
certain end or purpose, for which it is calculated; and is to be
deemed more or less perfect, as it is more or less fitted to attain
this end. The object of eloquence is to persuade, of history to
instruct, of poetry to please by means of the passions and the
imagination. These ends we must carry constantly in our view, when we
peruse any performance; and we must be able to judge how far the means
employed are adapted to their respective purposes. Besides, every kind
of composition, even the most poetical, is nothing but a chain of
propositions and reasonings; not always, indeed, the justest and most
exact, but still plausible and specious, \page{278} however disguised
by the colouring of the imagination. The persons introduced in tragedy
and epic poetry, must be represented as reasoning, and thinking, and
concluding, and acting, suitably to their character and circumstances;
and without judgment, as well as taste and invention, a poet can never
hope to succeed in so delicate an undertaking. Not to mention, that
the same excellence of faculties which contributes to the improvement
of reason, the same clearness of conception, the same exactness of
distinction, the same vivacity of apprehension, are essential to the
operations of true taste, and are its infallible concomitants. It
seldom, or never happens, that a man of sense, who has experience in
any art, cannot judge of its beauty; and it is no less rare to meet
with a man who has a just taste without a sound understanding.

Thus, though the principles of taste be universal, and, nearly, if not
entirely the same in all men; yet few are qualified to give judgment
on any work of art, or establish their own sentiment as the standard
of beauty. The organs of internal sensation are seldom so perfect as
to allow the general principles their full play, and produce a feeling
correspondent to those principles. They either labour under some
defect, or are vitiated by some disorder; and by that means, excite a
sentiment, which may be pronounced erroneous. When the critic has no
delicacy, he judges without any distinction, and is only affected by
the grosser and more palpable qualities of the object: The finer
touches pass unnoticed and disregarded. Where he is not aided by
practice, his verdict is attended with confusion and hesitation. Where
no comparison has been employed, the most frivolous beauties, such as
rather merit the name of defects, are the object of his admiration.
Where he lies under the influence of prejudice, all his natural
sentiments are perverted. Where good sense is wanting, he is not
qualified to discern the beauties of design and reasoning, which are
the highest and most excellent. Under some or other of these
imperfections, the generality of men labour; and hence a true judge in
the finer arts is observed, even during the most polished ages, to be
so rare a character: Strong sense, united to delicate sentiment,
improved by practice, perfected by comparison, and cleared of all
prejudice, can alone entitle critics to this valuable character; and
the joint \page{279} verdict of such, wherever they are to be found,
is the true standard of taste and beauty.

But where are such critics to be found? By what marks are they to be
known? How distinguish them from pretenders? These questions are
embarrassing; and seem to throw us back into the same uncertainty,
from which, during the course of this essay, we have endeavoured to
extricate ourselves.

But if we consider the matter aright, these are questions of fact, not
of sentiment. Whether any particular person be endowed with good sense
and a delicate imagination, free from prejudice, may often be the
subject of dispute, and be liable to great discussion and enquiry: But
that such a character is valuable and estimable will be agreed in by
all mankind. Where these doubts occur, men can do no more than in
other disputable questions, which are submitted to the understanding:
They must produce the best arguments, that their invention suggests to
them; they must acknowledge a true and decisive standard to exist
somewhere, to wit, real existence and matter of fact; and they must
have indulgence to such as differ from them in their appeals to this
standard. It is sufficient for our present purpose, if we have proved,
that the taste of all individuals is not upon an equal footing, and
that some men in general, however difficult to be particularly pitched
upon, will be acknowledged by universal sentiment to have a preference
above others.

But in reality the difficulty of finding, even in particulars, the
standard of taste, is not so great as it is represented. Though in
speculation, we may readily avow a certain criterion in science and
deny it in sentiment, the matter is found in practice to be much more
hard to ascertain in the former case than in the latter. Theories of
abstract philosophy, systems of profound theology, have prevailed
during one age: In a successive period, these have been universally
exploded: Their absurdity has been detected: Other theories and
systems have supplied their place, which again gave place to their
successors: And nothing has been experienced more liable to the
revolutions of chance and fashion than these pretended decisions of
science. The case is not the same with the beauties of eloquence and
poetry. Just expressions of passion and nature are sure, after a
little time, to gain \page{280} public applause, which they maintain
for ever. \textsc{Aristotle}, and \textsc{Plato}, and
\textsc{Epicurus}, and \textsc{Descartes}, may successively yield to
each other: But \textsc{Terence} and \textsc{Virgil} maintain an
universal, undisputed empire over the minds of men. The abstract
philosophy of \textsc{Cicero} has lost its credit: The vehemence of
his oratory is still the object of our admiration.

% NOTE: this paragraph's last sentence has 'favorite' in the original
% text; changed to 'favourite' to conform with spelling elsewhere in
% the essay

Though men of delicate taste be rare, they are easily to be
distinguished in society, by the soundness of their understanding and
the superiority of their faculties above the rest of mankind. The
ascendant, which they acquire, gives a prevalence to that lively
approbation, with which they receive any productions of genius, and
renders it generally predominant. Many men, when left to themselves,
have but a faint and dubious perception of beauty, who yet are capable
of relishing any fine stroke, which is pointed out to them. Every
convert to the admiration of the real poet or orator is the cause of
some new conversion. And though prejudices may prevail for a time,
they never unite in celebrating any rival to the true genius, but
yield at last to the force of nature and just sentiment. Thus, though
a civilized nation may easily be mistaken in the choice of their
admired philosopher, they never have been found long to err, in their
affection for a favourite epic or tragic author.

But notwithstanding all our endeavours to fix a standard of taste, and
reconcile the discordant apprehensions of men, there still remain two
sources of variation, which are not sufficient indeed to confound all
the boundaries of beauty and deformity, but will often serve to
produce a difference in the degrees of our approbation or blame. The
one is the different humours of particular men; the other, the
particular manners and opinions of our age and country. The general
principles of taste are uniform in human nature: Where men vary in
their judgments, some defect or perversion in the faculties may
commonly be remarked; proceeding either from prejudice, from want of
practice, or want of delicacy; and there is just reason for approving
one taste, and condemning another. But where there is such a diversity
in the internal frame or external situation as is entirely blameless
on both sides, and leaves no room to give one the preference above the
other; in that case a certain degree of diversity in judgment is
unavoidable, and we seek in vain \page{281} for a standard, by which
we can reconcile the contrary sentiments.

A young man, whose passions are warm, will be more sensibly touched
with amorous and tender images, than a man more advanced in years, who
takes pleasure in wise, philosophical reflections concerning the
conduct of life and moderation of the passions. At twenty,
\textsc{Ovid} may be the favourite author; \textsc{Horace} at forty;
and perhaps \textsc{Tacitus} at fifty. Vainly would we, in such cases,
endeavour to enter into the sentiments of others, and divest ourselves
of those propensities, which are natural to us. We choose our
favourite author as we do our friend, from a conformity of humour and
disposition. Mirth or passion, sentiment or reflection; whichever of
these most predominates in our temper, it gives us a peculiar sympathy
with the writer who resembles us.

One person is more pleased with the sublime; another with the tender;
a third with raillery. One has a strong sensibility to blemishes, and
is extremely studious of correctness: Another has a more lively
feeling of beauties, and pardons twenty absurdities and defects for
one elevated or pathetic stroke. The ear of this man is entirely
turned towards conciseness and energy; that man is delighted with a
copious, rich, and harmonious expression. Simplicity is affected by
one; ornament by another. Comedy, tragedy, satire, odes, have each its
partizans, who prefer that particular species of writing to all
others. It is plainly an error in a critic, to confine his approbation
to one species or style of writing, and condemn all the rest. But it
is almost impossible not to feel a predilection for that which suits
our particular turn and disposition. Such preferences are innocent and
unavoidable, and can never reasonably be the object of dispute,
because there is no standard, by which they can be decided.

For a like reason, we are more pleased, in the course of our reading,
with pictures and characters, that resemble objects which are found in
our own age or country, than with those which describe a different set
of customs. It is not without some effort, that we reconcile ourselves
to the simplicity of ancient manners, and behold princesses carrying
water from the spring, and kings and heroes dressing their own
victuals. We may allow in general, that the representation of such
manners is no fault in the author, nor \page{282} deformity in the
piece; but we are not so sensibly touched with them. For this reason,
comedy is not easily transferred from one age or nation to another. A
\textsc{Frenchman} or \textsc{Englishman} is not pleased with the
\textsc{Andria} of \textsc{Terence}, or \textsc{Clitia} of
\textsc{Machiavel}; where the fine lady, upon whom all the play turns,
never once appears to the spectators, but is always kept behind the
scenes, suitably to the reserved humour of the ancient \textsc{Greeks}
and modern \textsc{Italians}. A man of learning and reflection can
make allowance for these peculiarities of manners; but a common
audience can never divest themselves so far of their usual ideas and
sentiments, as to relish pictures which no wise resemble them.

But here there occurs a reflection, which may, perhaps, be useful in
examining the celebrated controversy concerning ancient and modern
learning; where we often find the one side excusing any seeming
absurdity in the ancients from the manners of the age, and the other
refusing to admit this excuse, or at least, admitting it only as an
apology for the author, not for the performance. In my opinion, the
proper boundaries in this subject have seldom been fixed between the
contending parties. Where any innocent peculiarities of manners are
represented, such as those above mentioned, they ought certainly to be
admitted; and a man, who is shocked with them, gives an evident proof
of false delicacy and refinement. The poet's \textit{monument more
durable than brass}, must fall to the ground like common brick or
clay, were men to make no allowance for the continual revolutions of
manners and customs, and would admit of nothing but what was suitable
to the prevailing fashion. Must we throw aside the pictures of our
ancestors, because of their ruffs and fardingales? But where the ideas
of morality and decency alter from one age to another, and where
vicious manners are described, without being marked with the proper
characters of blame and disapprobation; this must be allowed to
disfigure the poem, and to be a real deformity. I cannot, nor is it
proper I should, enter into such sentiments; and however I may excuse
the poet, on account of the manners of his age, I never can relish the
composition. The want of humanity and of decency, so conspicuous in
the characters drawn by several of the ancient poets, even sometimes
by \textsc{Homer} and the \textsc{Greek} tragedians, diminishes
considerably the merit of their noble performances, and gives
\page{283} modern authors an advantage over them. We are not
interested in the fortunes and sentiments of such rough heroes: We are
displeased to find the limits of vice and virtue so much confounded:
And whatever indulgence we may give to the writer on account of his
prejudices, we cannot prevail on ourselves to enter into his
sentiments, or bear an affection to characters, which we plainly
discover to be blameable.

The case is not the same with moral principles, as with speculative
opinions of any kind. These are in continual flux and revolution. The
son embraces a different system from the father. Nay, there scarcely
is any man, who can boast of great constancy and uniformity in this
particular. Whatever speculative errors may be found in the polite
writings of any age or country, they detract but little from the value
of those compositions. There needs but a certain turn of thought or
imagination to make us enter into all the opinions, which then
prevailed, and relish the sentiments or conclusions derived from them.
But a very violent effort is requisite to change our judgment of
manners, and excite sentiments of approbation or blame, love or
hatred, different from those to which the mind from long custom has
been familiarized. And where a man is confident of the rectitude of
that moral standard, by which he judges, he is justly jealous of it,
and will not pervert the sentiments of his heart for a moment, in
complaisance to any writer whatsoever.

Of all speculative errors, those, which regard religion, are the most
excusable in compositions of genius; nor is it ever permitted to judge
of the civility or wisdom of any people, or even of single persons, by
the grossness or refinement of their theological principles. The same
good sense, that directs men in the ordinary occurrences of life, is
not hearkened to in religious matters, which are supposed to be placed
altogether above the cognizance of human reason. On this account, all
the absurdities of the pagan system of theology must be overlooked by
every critic, who would pretend to form a just notion of ancient
poetry; and our posterity, in their turn, must have the same
indulgence to their forefathers. No religious principles can ever be
imputed as a fault to any poet, while they remain merely principles,
and take not such strong possession of his heart, as to lay him under
the imputation of \textit{bigotry} or \textit{superstition}. Where
that happens, they confound the sentiments of morality, and alter
\page{284} the natural boundaries of vice and virtue. They are
therefore eternal blemishes, according to the principle above
mentioned; nor are the prejudices and false opinions of the age
sufficient to justify them.

It is essential to the \textsc{Roman} catholic religion to inspire a
violent hatred of every other worship, and to represent all pagans,
mahometans, and heretics as the objects of divine wrath and vengeance.
Such sentiments, though they are in reality very blameable, are
considered as virtues by the zealots of that communion, and are
represented in their tragedies and epic poems as a kind of divine
heroism. This bigotry has disfigured two very fine tragedies of the
\textsc{French} theatre, \textsc{Polieucte} and \textsc{Athalia};
where an intemperate zeal for particular modes of worship is set off
with all the pomp imaginable, and forms the predominant character of
the heroes. `What is this,' says the sublime \textsc{Joad} to
\textsc{Josabet}, finding her in discourse with \textsc{Mathan}, the
priest of \textsc{Baal}, `Does the daughter of \textsc{David} speak to
this traitor? Are you not afraid, lest the earth should open and pour
forth flames to devour you both? Or lest these holy walls should fall
and crush you together? What is his purpose? Why comes that enemy of
God hither to poison the air, which we breathe, with his horrid
presence?' Such sentiments are received with great applause on the
theatre of \textsc{Paris}; but at \textsc{London} the spectators would
be full as much pleased to hear \textsc{Achilles} tell
\textsc{Agamemnon}, that he was a dog in his forehead, and a deer in
his heart, or \textsc{Jupiter} threaten \textsc{Juno} with a sound
drubbing, if she will not be quiet.

\textsc{Religious} principles are also a blemish in any polite
composition, when they rise up to superstition, and intrude themselves
into every sentiment, however remote from any connection with
religion. It is no excuse for the poet, that the customs of his
country had burthened life with so many religious ceremonies and
observances, that no part of it was exempt from that yoke. It must for
ever be ridiculous in \textsc{Petrarch} to compare his mistress
\textsc{Laura}, to \textsc{Jesus Christ}. Nor is it less ridiculous in
that agreeable libertine, \textsc{Boccace}, very seriously to give
thanks to \textsc{God Almighty} and the ladies, for their assistance
in defending him against his enemies.

